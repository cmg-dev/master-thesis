%--------------------------------------------------------------
%--------------------------------------------------------------
%--------------------------------------------------------------
\newcommand\confidentialoverlay{
  % Taken from the TikZ documentation.
  % NB: This requires \usepackage{tikz}!
  \begin{tikzpicture}[remember picture,overlay]
    \node [rotate=60,scale=10,text opacity=0.1]
      at (current page.center) {Vertraulich};
  \end{tikzpicture} 
 
} 
%--------------------------------------------------------------
%--------------------------------------------------------------
%--------------------------------------------------------------
\newcommand{\myvec}[1]{\hat{\mathbf{#1}}}% Vector notation

%--------------------------------------------------------------
%- This can be used for aligning equations --------------------
%--------------------------------------------------------------
\newcommand{\phantomeq}[2]{
\begin{equation}
	\phantom{#1}
	#2
\end{equation}
}% Vector notation

%--------------------------------------------------------------
%- seraches for input in the "extern" folder ------------------
%--------------------------------------------------------------
\newcommand{\externInput}[1]{\input{extern/#1}}

%--------------------------------------------------------------
%- seraches for input in the "extern" folder ------------------
%--------------------------------------------------------------
\newcommand{\Index}[1]{#1\index{#1}}

%--------------------------------------------------------------
%- seraches for input in the "intern" folder ------------------
%--------------------------------------------------------------
\newcommand{\internInput}[1]{\input{intern/#1}}

%--------------------------------------------------------------
%- seraches for input in the "common" folder ------------------
%--------------------------------------------------------------
\newcommand{\commonInput}[1]{\input{common/#1}}

\newcommand{\cpp}{%
  \mbox{\emph{\textrm{C\hspace{-1.5pt}\raisebox{1.75pt}{\scriptsize +}%
  \hspace{-2pt}\raisebox{.75pt}{\scriptsize +}}}}%
}

\newcommand{\amedogmbh}{%
  amedo STS
}

%--------------------------------------------------------------
%- seraches for input in the "common" folder ------------------
%--------------------------------------------------------------
\newcommand{\Gitter}[4]{
    \draw[very thin,color=gray] (#1,#3) grid (#2,#4);
}
\newcommand{\Koordinatenkreuz}[6]{
    \draw[->, >=latex, color=gray!50!black] (#1,0) -- (#2,0) node[right] {#5};
    \draw[->, >=latex, color=gray!50!black] (0,#3) -- (0,#4) node[left] {#6};
}
\newcommand{\KoordinatenkreuzOhneLabelsVerschobenKeinPfeil}[5]{
    \draw[-] (#1,0) -- (#2,0);
    \draw[-] (#5,#3) -- (#5,#4);

}
\newcommand{\ZeigerdiagrammText}[4]{
\begin{tikzpicture}[scale=1, samples=100, >=latex]

    \def\Alpha{#1}
    \def\Phase{#2}
    \def\AmplitudeSpannung{#3}
    \def\AmplitudeStrom{#4}
    \def\SpannungsWert{{\AmplitudeSpannung*sin(\Alpha)}}
    \def\StromWert{{\AmplitudeStrom*sin(\Alpha+\Phase)}}
    %%%%%%%%%%%%%%%%%%%%%%%%%%%%%%%%%%%%%%%%%%%%%%%%%%%%%%%%%%
    \def\FarbeSpannung{gray!90!white}
    \def\FarbeStrom{black!90!white}
    \def\FarbeWinkelZeichnung{gray}
    %%%%%%%%%%%%%%%%%%%%%%%%%%%%%%%%%%%%%%%%%%%%%%%%%%%%%%%%%%
    \def\Beta{\Alpha+\Phase}
    \def\AlphaRad{\Alpha*3.141592654/180}
    \def\PhaseRad{\Phase*3.141592654/180}
    %%%%%%%%%%%%%%%%%%%%%%%%%%%%%%%%%%%%%%%%%%%%%%%%%%%%%%%%%%
%    \Gitter{-.1}{7}{-2-5}{2.5}
    \Koordinatenkreuz{-.2}{7}{-2.5}{2.5}{$\omega t$}{}
    \draw (1.570795,0) node[above=4pt]{$\frac{\pi}{2}$};
    \draw (3.14159,0) node[above=4pt]{${\pi}$};
    \draw (4.71238898,0) node[above=4pt]{$\frac{3\pi}{2}$};
    \draw (6.283185307,0) node[above=4pt]{${2\pi}$};
    \draw (-2.5,0) circle (2.3cm);
    \KoordinatenkreuzOhneLabelsVerschobenKeinPfeil{-5.2}{-.3}{-2.6}{2.6}{-2.5}
    %%%%%%%%%%%%%%%%%%%%%%%%%%%%%%%%%%%%%%%%%%%%%%%%%%%%%%%%%%

    % voltage
    \draw[color=\FarbeSpannung, very thick] plot[id=voltage, domain=0:7] function{\AmplitudeSpannung*sin(x)} node[right] {\textsf{Reference}};
    % voltage circle
    \draw[color=\FarbeSpannung, loosely dashed] (-2.5,0) circle (\AmplitudeSpannung cm);
    % angle
    \draw[color=\FarbeWinkelZeichnung!50!black, thick] (\AlphaRad, \SpannungsWert)--(\AlphaRad,\StromWert) node[below=18pt] {$\alpha$};
    % angle in the circle
%    \filldraw[fill=\FarbeWinkelZeichnung!20,draw=\FarbeWinkelZeichnung!50!black] (-2.5,0) -- (-3,0) arc (0:\Alpha:1) -- cycle node[right] {$\alpha$};
    % voltage pointer
    \draw[<-,color=\FarbeSpannung, very thick] (\Alpha:\AmplitudeSpannung)++(-2.5,0) --(-2.5,0);
    \draw[color=\FarbeSpannung,  dashed] (\Alpha:\AmplitudeSpannung)++(-2.5,0) -- (\AlphaRad,\SpannungsWert);
    % current
    \draw[color=\FarbeStrom, very thick] plot[id=current, domain=0:7] function{\AmplitudeStrom*sin(x+\PhaseRad)} node[right] {\textsf{Incomming}};		
    % current circle
    \draw[color=\FarbeStrom, loosely dashed]    (-2.5,0) circle (\AmplitudeStrom cm);
    % current pointer
    \draw[<-,color=\FarbeStrom, very thick] (\Beta:\AmplitudeStrom)++(-2.5,0) --(-2.5,0);
    \draw[color=\FarbeStrom,  dashed](\Beta:\AmplitudeStrom)++(-2.5,0) -- (\AlphaRad,\StromWert);
    % phase difference
    \ifthenelse{\Phase<0}{
        \draw[snake=brace] (pi/2 ,2.3)--(pi/2-\PhaseRad ,2.3) node[pos=0.5, above=5pt] {$\Theta$};
    }
    {
        \draw[snake=brace] (pi/2-\PhaseRad ,2.3)--(pi/2 ,2.3) node[pos=0.5, above=5pt] {$\Theta$};
    }
    \draw[<->] (0,-2.2) -- (2*pi,-2.2) node[pos=0.5, below]{$\lambda$};
    \draw (2*pi,.2) -- (2*pi,-2.5);
    
%    \draw[snake=brace, mirror snake] (0 ,-2.3)--(pi*2 ,-2.3) node[above=10pt, left=10pt]  {$\Theta$} ;
    
    % angular velocity \omega
    \draw[->, xshift=-2.5cm]  (120:2cm) arc (120:170:2) node[below] {$\omega$};
\end{tikzpicture}
}
%\renewenvironment{itemize}[1]{\begin{compactitem}#1}{\end{compactitem}}
%\renewenvironment{enumerate}[1]{\begin{compactenum}#1}{\end{compactenum}}
%\renewenvironment{description}[0]{\begin{compactdesc}}{\end{compactdesc}}
