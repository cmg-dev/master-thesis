%\begin{figure}[h]
%	\begin{center}
%		\caption[dasd]{2D-Übersicht auf die Szene mit drei Antennen, einem Tag und einer Landmarke. Die Position von $\{A_1,A_3,A_3\}$, sowie der Landmarke, zum Koordinatenursprung sind bekannt. Die Vektoren $r_1,r_2,r_3$ sind die gemessene Entfernung zu einer Antenne. Die Landmarke wird im späteren Verlauf eine Antenne sein, die ihrerseits ein gemessene Entfernung $r_0$ produziert. Der Schnittpunkt aller Kreise ist die Lösung der gemessenen Entfernung und der geom. Anordnung, die sich für die Position des Tags ergibt.} 
%		\begin{tikzpicture}[
    scale=1,
    axis/.style={thick, ->, >=stealth'},
%    vector/.style={thick, ->,-latex, >=stealth'},
%    antenna/.style={thick},
     important line/.style={thick},
     antenna/.style={thick, cyan!70},
%    dashed line/.style={dashed, thin},
%    pile/.style={thick, ->, >=stealth', shorten <=2pt, shorten
%    >=2pt},
%    every node/.style={color=black},
%    main node/.style={circle,fill=blue!20,draw},
%    help lines/.style={gray,very thin}
    ]
    % axis
    \draw[axis] (-.1,0)  -- (1,0) node(xline)[right] {$x$};
    \draw[axis] (0,-.1) -- (0,1) node(yline)[above] {$y$};

	\draw[gray, very thin, dotted] (0,0) grid (15,6);

	\coordinate (A1_start) at (4,3);
	\coordinate (A1_end) at (4,4);
	\coordinate (A2_start) at (7,5);
	\coordinate (A2_end) at (8,5);
	\coordinate (A3_start) at (8,1);
	\coordinate (A3_end) at (8,2);

	\coordinate (A1_end_) at ($(A1_start)!1!-10:(A1_end)$);
	\coordinate (A2_end_) at ($(A2_start)!1!-10:(A2_end)$);
	\coordinate (A3_end_) at ($(A3_start)!1!-35:(A3_end)$);
	
	\coordinate (Tag_0) at (6,2);
	\coordinate (REF_0) at (12,5);
	\coordinate (Int1) at ($(A1_start)!.5!(A1_end_)$);
	\coordinate (Int2) at ($(A2_start)!.5!(A2_end_)$);
	\coordinate (Int3) at ($(A3_start)!.5!(A3_end_)$);
	
	\begin{scope}
		\node [draw,orange!50,dashed] at (Int1) [circle through={(Tag_0)}] {};
		\node [draw,orange!50,dashed] at (Int2) [circle through={(Tag_0)}] {};
		\node [draw,orange!50,dashed] at (Int3) [circle through={(Tag_0)}] {};
	\end{scope}
	
	\draw[antenna] (A1_start) node[font=\scriptsize,black,below] {$A_1$} -- ($(A1_start)!1!-10:(A1_end)$);
	\draw[antenna] (A2_start) node[font=\scriptsize,black,above] {$A_2$}-- ($(A2_start)!1!-10:(A2_end)$);
	\draw[antenna] (A3_start) node[font=\scriptsize,black,below] {$A_3$}-- ($(A3_start)!1!-35:(A3_end)$);
	
	\node [green!60!black!90, right,font=\scriptsize ] at (REF_0) {$\text{Landmarke}@(x_0,y_0,z_0)$};

	\draw[latex-latex] (Tag_0) -- node[sloped,above,midway] {$r_1$}(Int1);
	\draw[latex-latex] (Tag_0) -- node[sloped,above,midway] {$r_2$}(Int2);
	\draw[latex-latex] (Tag_0) -- node[sloped,above,midway] {$r_2$}(Int3);
	\draw[-latex,dashed,green!60!black!90] (REF_0) -- node[sloped,above,midway] {$r_0$}(Tag_0);
	
	\draw[ -latex,violet!60,font=\scriptsize,dotted] (REF_0) -- node[sloped,above,midway] {$d_{10}$}(Int1);
	\draw[ -latex,violet!60,font=\scriptsize,dotted] (REF_0) -- node[sloped,above,midway] {$d_{20}$}(Int2);
	\draw[ -latex,violet!60,font=\scriptsize,dotted] (REF_0) -- node[sloped,above,midway] {$d_{30}$}(Int3);
		
	\fill[red!70] (Tag_0) circle [radius=2pt];
	\node[font=\scriptsize,black,below] at (Tag_0) {$Tag$} ;
	\fill[green!60!black!90] (REF_0) circle [radius=2pt];
	
\end{tikzpicture}



%		
%	\end{center}
%\end{figure}
%
Folgende Nomenklatur und Symbole gelten für diesen Abschnitt:
\begin{itemize}
	\item	$N:=$Anzahl der Antennen $N=\{1,..,8\}$
	\item	$k$ ist der Index der Antennen, es gilt $k = \{1,2,..,N-1\}$
	\item	$r_{k}$ := Abstand vom Tag zur Antenne
	\item	$d_{k0}$ := Abstand zur Landmarke
	\item	fette Großbuchstaben stehen für Matrizen (bspw. $\mathbf{A}$)
	\item	fette Kleinbuchstaben stehen für Vektoren (bspw. $\mathbf{x}$)
\end{itemize}
%
Wie gezeigt werden konnte \footnote{Wochenbericht KW 20, Anhang B} ergibt sich für den Fall der Trilateration und der Annahme, dass vier Antennen Messwerte liefern, die Gleichung:
\begin{equation}\label{eq:final_trilateration_model}
0=
\left(
	\begin{array}{ccc}
		x_k-x_0 & y_k-y_0 & z_k-z_0 
	\end{array}
\right)
\left(
   \begin{array}{c}
	   x-x_0\\
	   y-y_0\\
	   z-z_0
   \end{array}
\right)
-
\left(
	\begin{array}{c}
		c_{k0}
	\end{array}
\right) 
\end{equation}
%
Dabei ist:
\begin{equation}\label{eq:c_k0}
	c_{k0}=\frac{1}{2}[d_{k0}^2+r_{0}^2-r_k^2]
\end{equation}
%
Ziel dieser Erweiterung ist es, einen Zusammenhang zwischen diesem Modell und der Wellenzahl zu erzeugen. Folgender Ansatz wird gewählt:
	\begin{equation}\label{eq:r_0_theta} r(\Theta)=\frac{\lambda}{2}\left(\frac{\Theta}{2\pi}+n\right),\\\lambda=\frac{c}{f}, n:= \text{Wellenzahl}
\end{equation}
%
%
Weiterhin ist $\Theta$ die gemessene Phase, die das PRPS-System liefert und $n$ die gesuchte Wellenzahl.\\
Durch einsetzen von \eqref{eq:r_0_theta} in \eqref{eq:c_k0}, erhalten wir:
\begin{equation}\label{eq:c_k0_extended}
	c_{k0}(\Theta_0, \Theta_k, n_0, n_k) =\frac{1}{2}\left[d_{k0}^2+\frac{\lambda^2}{4}\left(\frac{\Theta_0}{2\pi}+n_0\right)^2-\frac{\lambda^2}{4}\left(\frac{\Theta_k}{2\pi}+n_k\right)^2\right]
\end{equation}
%
Wir stellen Gleichung~\eqref{eq:c_k0_extended} um:
\begin{align}
%	
	c_{k0}(\Theta_0, \Theta_k, n_0, n_k) &= \frac{1}{2}\left\{d_{k0}^2+\frac{\lambda^2}{4}\left[\left(\frac{\Theta_0}{2\pi}\right)^2+2\frac{\Theta_0}{2\pi}n_0+n_0^2 \right.\right.\nonumber\\
	&\phantom{=}\; 
	\left.\left.-\left(\frac{\Theta_k}{2\pi}\right)^2-2\frac{\Theta_k}{2\pi}n_k-n_k^2\right]\right\}\\
%    
    &=\frac{1}{2}\left\{d_{k0}^2+\frac{\lambda^2}{4}\left[\left(\frac{\Theta_0}{2\pi}\right)^2-\left(\frac{\Theta_k}{2\pi}\right)^2 \right.\right.\nonumber\\
    &\phantom{=}\;
   	\left.\left.+2\frac{\Theta_0}{2\pi}n_0-2\frac{\Theta_k}{2\pi}n_k+n_0^2-n_k^2\right]\right\}\\
%	
	&=\frac{1}{2}d_{k0}^2+\frac{\lambda^2}{8}\left[\frac{1}{(2\pi)^2}\left(\Theta_0^2-\Theta_k^2\right) \right.\nonumber\\
	&\phantom{=}\;
	\left. +\frac{1}{\pi}\left(\Theta_0n_0-\Theta_kn_k\right)+\left(n_0^2-n_k^2\right)\right]\label{c_k0_rearragend}
\end{align}
%
Führen wir nun:
\phantomeq{c_{k0}(\Theta_0, \Theta_k, n_0, n_k)}{a_{0k} := \frac{1}{2}d_{k0}^2\nonumber}
\phantomeq{c_{k0}(\Theta_0, \Theta_k, n_0, n_k)}{a_1 := \frac{\lambda^2}{8}\nonumber}
\phantomeq{c_{k0}(\Theta_0, \Theta_k, n_0, n_k)}{a_2 := a_1\frac{1}{\pi}\nonumber}
\phantomeq{c_{k0}(\Theta_0, \Theta_k, n_0, n_k)}{a_{3k0} := a_1\frac{1}{(2\pi)^2}(\Theta_0^2-\Theta_k^2)\nonumber}
%
in Gleichung~\eqref{c_k0_rearragend} ein, erhalten die finale Form der Gleichung:
\begin{equation}
c_{k0}(\Theta_0, \Theta_k, n_0, n_k) = a_{0k}+a_1(n_0^2-n_k^2)+a_2(\Theta_0n_0-\Theta_kn_k)-a_{3k0}\label{c_k0_final_form}   
\end{equation}
%
Die Einführung hat zum Einen den praktischen Nutzen die Gleichung übersichtlicher zu machen. Zum Anderen sind mit Blick auf eine spätere Softwareimplementation so Rechenschritte zu sparen. Das sollte sich positiv auf den späteren Berechnungsaufwand auswirken.
%
Im Weiteren erkennt man durch scharfes hinsehen das in Gleichung~\eqref{c_k0_final_form}, für $\Theta_k=\text{const.}$ \& $\Theta_0=\text{const.}$ gilt. Das resultiert aus der Tatsache, dass . Es ermöglicht uns zu schreiben:
\begin{equation}
c_{k0}(\Theta_0, \Theta_k, n_0, n_k) = c_{k0}(n_0, n_k)
\end{equation}
%
Im engeren Sinne einer mathematischen Funktion sollten wir die Parameter alle als Argument aufnehmen. Im späteren Gebrauch wird diese Gleichung in der Optimierung eingesetzt werden. jedoch soll diese Form darstellen, das
Für unser Gleichungssystem ergibt sich:
\begin{equation}\label{eq:wavenumber_trilateration_model}
0=
\left(
	\begin{array}{ccc}
		x_k-x_0 & y_k-y_0 & z_k-z_0 
	\end{array}
\right)
\left(
   \begin{array}{c}
	   x-x_0\\
	   y-y_0\\
	   z-z_0
   \end{array}
\right)
-
\left(
	\begin{array}{c}
		c_{k0}(n_0, n_k)
	\end{array}
\right)
\end{equation}
%
Betrachten wir den vormals\footnote{Wochenbericht KW 20, Anhang B} beschriebenen Fall und setzen $N'=4$, d.h. wir verwenden 4 Antennen. 1 ist die Referenz-Antenne und 3 Weitere sind Messwertgeber.
%
\begin{equation}\label{eq:wavenumber_trilateration_model_explicit}
0=
\underbrace{\left(
	\begin{array}{ccc}
		x_1-x_0 & y_1-y_0 & z_1-z_0 \\
		x_2-x_0 & y_2-y_0 & z_2-z_0 \\
		x_3-x_0 & y_3-y_0 & z_3-z_0 
	\end{array}
\right)}_{\textbf{A}}
\underbrace{\left(
   \begin{array}{c}
	   x-x_0\\
	   y-y_0\\
	   z-z_0
   \end{array}
\right)}_{\textbf{x}}
-
\underbrace{\left(
	\begin{array}{c}
		c_{10}(n_0, n_1) \\
		c_{20}(n_0, n_2) \\
		c_{30}(n_0, n_3)
	\end{array}
\right)}_{\textbf{b}}
\end{equation}
%
\begin{equation}
\mathbf{b}=
\left(
	\begin{array}{c}
		a_{01}+a_1( n_0^2-n_1^2)+a_2(\Theta_0n_0-\Theta_1n_1)-a_{310} \\
		a_{02}+a_1(n_0^2-n_2^2)+a_2(\Theta_0n_0-\Theta_2n_2)-a_{320} \\
		a_{03}+a_1(n_0^2-n_3^2)+a_2(\Theta_0n_0-\Theta_3n_3)-a_{330}
	\end{array}
\right)
\end{equation}
%
Das Ergebnis ist ein um $\Theta$ und $n$ erweitertes Gleichungssystem. Zusätzlich enthält  es mehrere geometrische Konstanten ($a_{0k}, k=\{1,..,N-1\}$), mehrere Phasen-Konstanten ($a_{3k0}, k=\{1,..,N-1\}$), sowie zwei allgemeine ($a_1$ und $a_2$). Allgemeiner formuliert ergibt sich:
%
\begin{multline}\label{final_equation}
0=
\left(
	\begin{array}{ccc}
		x_k-x_0 & y_k-y_0 & z_k-z_0 
	\end{array}
\right)
\left(
   \begin{array}{c}
	   x-x_0\\
	   y-y_0\\
	   z-z_0
   \end{array}
\right) \\
-
\left(
	\begin{array}{c}
		a_{0k}+a_1(n_0^2-n_k^2)+a_2(\Theta_0k_0-\Theta_kn_k)-a_{3k0}
	\end{array}
	\right)
\end{multline}
%
Aus Gleichung~\eqref{final_equation} ist durch eine geeignete Wahl von $N'={4,..,N}]$ sofort ersichtlich wie viele Veränderliche sich für eine gewählte Konstellation an Antennen ergeben. Für $k$ gilt in diesem Fall $k=\{1,..,N'-1\}$.
%
Beispielsweise ergibt sich für das Modell aus Gleichung~\eqref{eq:wavenumber_trilateration_model_explicit} mit $N'=4$, insgesamt 7 ($\mathbf{x},n_0,n_1,n_2,n_3$) Variablen. Analog würde sich für ein Modell mit allen 8 Antennen, 11 Variablen ergeben.
