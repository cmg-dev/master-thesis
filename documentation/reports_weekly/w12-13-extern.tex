\documentclass[a4paper,12pt,fleqn]{article}
\usepackage[T1]{fontenc}
\usepackage{ucs}
\usepackage[utf8x]{inputenc}
\usepackage{ngerman}
\usepackage[ngerman]{babel}
\usepackage{lastpage}
\usepackage[pdftex]{color,graphicx}
\usepackage{listings}
\usepackage{pdflscape}
\usepackage{longtable}
\usepackage[inner=2cm,outer=2cm,top=1cm,bottom=1.5cm,includeheadfoot]{geometry}
\usepackage{fancyhdr}
\usepackage{url}
\usepackage{draftwatermark}
\usepackage{booktabs}
\usepackage{blindtext} 
\usepackage{framed} 
\usepackage{xcolor} 
\colorlet{shadecolor}{black} 
\usepackage{latexsym}

\usepackage{bbm}
\usepackage{enumitem}

\SetWatermarkText{Vertraulich}
\SetWatermarkScale{4}
\SetWatermarkLightness{0.9}

\usepackage{pgfgantt}
\usepackage{amsmath,amssymb,amsfonts,amstext}
\usepackage{floatflt}
\usepackage{tikz}
\usetikzlibrary[arrows,snakes,backgrounds,shapes]
\usetikzlibrary{through}
\usetikzlibrary{calc}
\usepackage{caption}
\usepackage{subcaption}

% highlighting
\usepackage{xcolor,soul}

%---- PageLayout
\pagestyle{fancy}

\setlength{\headsep}{10mm}

\usepackage{eso-pic}

%----------------------------------------------------------------------------
% HEADER --------------------------------------------------------------------
%----------------------------------------------------------------------------
\fancyhead[R]{
  \includegraphics[width=100pt,keepaspectratio]{img/amedo2012.png}
}

\fancyhead[C]{ Wochenbericht KW 28/29 }

\fancyhead[L]{
  \begin{tabular}[b]{l}
  Christoph Gnip\\
  Projekt: PRPS-Evolution
  \end{tabular}
}

%Linie oben
\renewcommand{\headrulewidth}{0.5pt}
%----------------------------------------------------------------------------

%----------------------------------------------------------------------------
%----------------------------------------------------------------------------
%----------------------------------------------------------------------------
\fancyfoot[L]{Stand: \today}
\fancyfoot[C]{ EXTERN }
\fancyfoot[R]{\thepage{} von \pageref{LastPage}}

% Linie unten
\renewcommand{\footrulewidth}{0.5pt}
%----------------------------------------------------------------------------

% Import Macros  ------------------------------------------------------------
%--------------------------------------------------------------
%--------------------------------------------------------------
%--------------------------------------------------------------
\newcommand\confidentialoverlay{
  % Taken from the TikZ documentation.
  % NB: This requires \usepackage{tikz}!
  \begin{tikzpicture}[remember picture,overlay]
    \node [rotate=60,scale=10,text opacity=0.1]
      at (current page.center) {Vertraulich};
  \end{tikzpicture} 
 
} 
%--------------------------------------------------------------
%--------------------------------------------------------------
%--------------------------------------------------------------
\newcommand{\myvec}[1]{\hat{\mathbf{#1}}}% Vector notation

%--------------------------------------------------------------
%- This can be used for aligning equations --------------------
%--------------------------------------------------------------
\newcommand{\phantomeq}[2]{
\begin{equation}
	\phantom{#1}
	#2
\end{equation}
}% Vector notation

%--------------------------------------------------------------
%- seraches for input in the "extern" folder ------------------
%--------------------------------------------------------------
\newcommand{\externInput}[1]{\input{extern/#1}}

%--------------------------------------------------------------
%- seraches for input in the "intern" folder ------------------
%--------------------------------------------------------------
\newcommand{\internInput}[1]{\input{intern/#1}}

%--------------------------------------------------------------
%- seraches for input in the "common" folder ------------------
%--------------------------------------------------------------
\newcommand{\commonInput}[1]{\input{common/#1}}

\newcommand{\cpp}{%
  \mbox{\emph{\textrm{C\hspace{-1.5pt}\raisebox{1.75pt}{\scriptsize +}%
  \hspace{-2pt}\raisebox{.75pt}{\scriptsize +}}}}%
}

\newcommand{\amedogmbh}{%
  amedo GmbH
}

%\renewenvironment{itemize}[1]{\begin{compactitem}#1}{\end{compactitem}}
%\renewenvironment{enumerate}[1]{\begin{compactenum}#1}{\end{compactenum}}
%\renewenvironment{description}[0]{\begin{compactdesc}}{\end{compactdesc}}


%----------------------------------------------------------------------------
% Start the Document --------------------------------------------------------
%----------------------------------------------------------------------------
\begin{document}

\setlength{\headheight}{36pt}

\begin{titlepage}

\input{extern/title/title_w12-13.tex}

\end{titlepage}

%- Section 1 ----------------------------------------------------------------
\section[Allgemeines]{Allgemeines}
%
Dieser Wochenbericht fasst die Projektwochen 12 und 13 zusammen. Die Ergebnisse dieses Zeitraums können so besser dargestellt werden.
%
%- Section 2 ----------------------------------------------------------------
\section[Fortschritt]{Projektfortschritt}
%
Diese Woche wurde eine Tiefpassfilterung für den FPGA-entworfen und mittels Matlab/ Simulink getestet. Die Entwicklung an der Software wurde weitergeführt. So kann nun eine diskret-/kontinuierliche Optimierung durchgeführt werden. Die Software wurde auf Shark 3.0b umgestellt.


%- Section 2.1 --------------------------------------------------------------
\subsection{Programmierung}
%
Das Programm wurde in mehrere Libraries aufgeteilt. Zum Einen erhöhen sich Wartbarkeit und Wiederverwendbarkeit und zum Anderen können diese vor-compiliert werden und reduzieren die Compilezeit erheblich. Einen Überblick über die wesentlichsten Merkmale gibt der Anhang~\ref{documentation_libs}.\\
Die Verwendung von \cpp11 erleichtert eine Menge der Entwicklung. Die anfängliche Einarbeitung in diesen Standard hat sich bereits mehr als ausgezahlt. Im Besonderen sind zur Zeit die Möglichkeiten für Thread-Programmierung und Verwaltung großer Datenmengen zu nennen. Diese sind anderen aktuellen Programmiersprachen, wie Java oder C$\sharp$, überlegen und beschleunigen die Entwicklung.
%
%- Section 2.2 --------------------------------------------------------------
\subsection{Umstellung auf Shark 3.0b}
%
Die Motivation für die Umstellung auf Shark 3.0b ist ein erheblich einfacheres Interface für die Definition verwendeter Zielfunktionen. Das Beispielprogramm\footnote{\url{http://image.diku.dk/shark/sphinx_pages/build/html/rest_sources/tutorials/algorithms/cma.html}} für den CMA-ES Algorithmus zeigt den erheblich reduzierten Aufwand für die Implementation der Fitnessfunktion.

%
%- Section 2.3 --------------------------------------------------------------
\subsection{Neue Modelle von Susanne Winter}
%
Mit der Implementation wurde begonnen, es wurden jedoch noch keine Ergebnisse generiert und ausgewertet. Diese liegen innerhalb der nächsten zwei Wochen vor.
%
%- Section 2.4 --------------------------------------------------------------
\subsection{Auslegung der FIR-Filter}
%
In Absprache mit M.Hüther wurden in dieser Woche zwei Entwürfe für FIR-Filter gemacht. Diese werden im FPGA implementiert und realisieren eine Tiefpassfilterung der Daten. Der Entwurf wurde mittels Matlab erstellt, das verwendete Simulink Modell und sie Ergebnisse sind in Anhang~\ref{FirFilterResult} gezeigt.

%
%- Section 3 -----------------------------------------------------------------
\section{Probleme}
\label{Problems}
%
Es kam bei der Umstellung auf die neue Shark Version 3.0 (beta) zu erheblichen Problemen. Die Bibliothek ließ sich zunächst nicht compilieren und der Vorgang brach mit mehreren hundert Fehlermeldungen ab. Die Fehler gingen von einer veralteten Version der boost Library zum Einen und der Verwendung der Verwendeten Compiler-Version (gcc 4.8) zum Anderen aus. Eine Deinstallation der veralteten Boost-Library und anschließender Rekompilierung der neuesten Version (boost 1.54) mit dem gcc 4.8 hat das Problem behoben.
%
%- Appendix ------------------------------------------------------------------
%
%
%
\begin{appendix}

%----------------------------------------------------------------------------
%----------------------------------------------------------------------------
%----------------------------------------------------------------------------
\newpage

\begin{center}
	\huge{Anhänge}
\end{center}

\normalsize

%----------------------------------------------------------------------------
%----------------------------------------------------------------------------
%----------------------------------------------------------------------------
\section{Filter Entwurf - Ergebnisse}
%\begin{landscape}
\label{FirFilterResult}
\begin{figure} [h]
         \centering
         \caption{ lorem ipsum }
         \label{fig:1}
         \centering
         \includegraphics[width=.8\textwidth]{common/img/AmpGefiltert_small.png}

\end{figure}
%---------------------------------------------------------------------------------------
\vspace{.5cm}
%---------------------------------------------------------------------------------------
\begin{figure} [h]
         \centering
         \caption{ Spektrum des Messsignals, vor und nach der Filterung  }
         \label{fig:2}
	     \centering
	     \includegraphics[width=.6\textwidth]{common/img/SpektrumAmp.PNG} \\
\vspace{.2cm}
Die Grafik zeigt das Spektrum des Messsignals der Amplitude. Im linken Bild ist das ungefilterte Signal und im Rechten das gefilterte.
%
\end{figure}
%---------------------------------------------------------------------------------------
\vspace{.5cm}
%---------------------------------------------------------------------------------------
\begin{figure} [h]
         \centering
         \caption{ Frequenzgänge der entworfenen Filter. Beide ähneln sich in den Parametern, verfügen jedoch über etwas unterschiedliche Eckfrequenzen. Als Entwurfsmethode wurde die sog. "Least-squares"-Methode verwendet. Diese Methode liefert gute Ergebnisse im Hinblick auf aöglichst kleine Sidelobes und eine geringe Anzahl an Taps. }
         \label{fig:3}
%         
         \begin{subfigure}[t]{0.5\textwidth}
                 \centering
                 \includegraphics[width=\textwidth]{common/img/filter.png}
                 \vspace{.1cm}
                 \caption{Erstes Filter mit den Parametern wpass~=~0.1 und wstop~=~0.15. Das Ergebnis ist ein schmalbandigeres Filter. }
                 \label{fig:Filter1_A}\textit{}
         \end{subfigure}
%         
\qquad
         \begin{subfigure}[t]{0.5\textwidth}
                 \centering
                 \includegraphics[width=\textwidth]{common/img/filter2.png}
                 \vspace{.1cm}
                 \caption{ Zweites Filter mit den Parametern wpass~=~0.1 und wstop~=~0.2. Der Durchlas bereich ist etwas breit, dafür sind die Sidelobes besser gedämpft }
                 \label{fig:Filter2_B}
         \end{subfigure}
%
\end{figure}
%---------------------------------------------------------------------------------------
%\end{landscape}
%----------------------------------------------------------------------------
%----------------------------------------------------------------------------
%----------------------------------------------------------------------------
\newpage
\section{Konditionen vs. Berechnete Ergebnisse}
\label{cond_vs_results}
\begin{figure} [h]
         \centering
         \caption{Analyse der Konditionszahlen gegen die Berechnungsergebnisse eines Messpunktes; Auffällig ist, dass die Lösungen für Referenzantenne 0 und 7 praktisch \textbf{keine} korrekte Lösung liefern. }
         \label{fig:CondNumberAnalyze}
%         
         \begin{subfigure}[t]{0.4\textwidth}
                 \centering
                 \includegraphics[width=\textwidth]{common/img/ConditionPlot_scaled.png}
                 \caption{Farbkodierte Matrix der Kondition aller möglichen Kombinationen (skaliert auf den höchsten vorkommenden Wert)\\
                 B=> grün~:=~gute -, rot=~schlechte Konditionierung}
                 \label{fig:ConditionMatrix}\textit{}
         \end{subfigure}
%         
\qquad
         \begin{subfigure}[t]{0.4\textwidth}
                 \centering
                 \includegraphics[width=\textwidth]{common/img/60_Results_scaled.png}
                 \caption{ Farbkodierte Ergebnisse der möglichen Kombinationen aus 6 von 8 Antennen; Die Position des Ergebnisses in der Matrix entspricht der in der Konditionsmatrix ddddddddd ddddddd dddddd dddddddddddddddddd.}
                 \label{fig:Results}
         \end{subfigure}
%
\end{figure}


%----------------------------------------------------------------------------
%----------------------------------------------------------------------------
%----------------------------------------------------------------------------
\newpage
\section{Ausgabe des Programms}
\label{output_program}
\begin{figure} [h]
         \centering
         \caption{Ausgabe des Programms; Die Ausführungszeit in diesem Beispiel beträgt ca. 2 Minuten, dabei werden 12.000 Lösungen bestimmt und gespeichert.  }
         \label{fig:Program_Output}
         \includegraphics[width=0.8\textwidth]{common/img/Program_Output_12k_results-via-parallel.png}
%
\end{figure}



%----------------------------------------------------------------------------
%----------------------------------------------------------------------------
%----------------------------------------------------------------------------
\newpage
\begin{landscape}
	\section{Projektlaufplan KW 27}
	\label{sec:projectplan}
	\scalebox{.75}{
		\begin{ganttchart}[vgrid={draw=none,*1{gray, dashed}},
				hgrid=true,
				today=24,
				title height=1,
				y unit title=0.6cm,
				y unit chart=0.8cm,
				group right shift=0,
				group top shift=.3,
				group height=.3,
				milestone width=.8,
				group peaks={}{}{.2},
				incomplete/.style={fill=black!15}, %
				bar/.style={fill=white}, %
				today label={Heute},
				today rule/.style={dashed, thick}]{44}


\gantttitle{\textbf{2013}}{44} \\
\gantttitlelist{16,...,37}{2} \\
%-------------------------------------------------------------
\ganttgroup{Projekt Evaluation}{3}{14} \\
\ganttbar[progress=100, progress label font=\small\color{black!75},
	progress label anchor/.style={right=4pt}]{Installation der Umgebungen}{3}{6} \\
	
\ganttbar[progress=100, progress label font=\small\color{black!75},
	progress label anchor/.style={right=4pt},
	bar label font=\normalsize\color{black},
	name=rech]{Recherche}{3}{7} \\
	
\ganttmilestone[name=ms1]{Vorstellung der Ergebnisse}{7} \\
	
\ganttbar[progress=90, progress label font=\small\color{black!75},
	progress label anchor/.style={right=4pt},
	bar label font=\normalsize\color{black},
	name=pflichten]
	{Pflichtenheft}{5}{8} \\
	
\ganttmilestone[name=ms2]{Pflichtenheft fertig}{8} \\

\ganttbar[progress=100, progress label font=\small\color{black!75},
	progress label anchor/.style={right=4pt},
	bar label font=\normalsize\color{black},
	name=bNumVerf]
	{Einarbeitung num. Verfahren}{5}{16} \\

\ganttbar[progress=95, progress label font=\small\color{black!75},
	progress label anchor/.style={right=34pt},
	bar label font=\normalsize\color{black},
	name=bCMAES]
	{speziell CMA-ES}{7}{10} \\

\ganttmilestone[name=ms3]{Beurteilung num. Verfahren}{16} \\

\ganttlinkedbar[progress=100, progress label font=\small\color{black!75},
	progress label anchor/.style={right=34pt},
	bar label font=\normalsize\color{black}]
	{Shark Einarbeitung}{17}{18} \\

\ganttlinkedmilestone[name=ms7]{Abschluss Evaluation}{18} \\
	
%-------------------------------------------------------------
\ganttgroup{Erstellung Prototyp}{15}{26} \\
\ganttgroup{(optional)}{15}{18} \\
\ganttbar[progress=25, progress label font=\small\color{black!75},
	progress label anchor/.style={right=4pt},
	bar label font=\normalsize\color{black}]
	{(Entwurf digi. Filter)}{15}{15} \\

\ganttlinkedbar[progress=10, progress label font=\small\color{black!75},
	progress label anchor/.style={right=4pt},
	bar label font=\normalsize\color{black},
	name=bImpFPGA]
	{(Implementation FPGA)}{16}{18} \\

\ganttmilestone[name=ms4]{(Verifikation dig. Filter)}{18} \\
	
\ganttbar[progress=90, progress label font=\small\color{black!75},
	progress label anchor/.style={right=4pt},
	bar label font=\normalsize\color{black},
	name=bImplAlgo]
	{Implementation Algorithmus}{15}{26} \\

\ganttlinkedmilestone[name=ms5]{Implementation Done}{26} \\

%-------------------------------------------------------------
\ganttgroup{Verifikation}{27}{34} \\
\ganttbar[progress=10, progress label font=\small\color{black!75},
	progress label anchor/.style={right=4pt},
	bar label font=\normalsize\color{black},
	name=bVerf]
	{Durchf\"uhrung Verifikation}{27}{34} \\

\ganttlinkedmilestone[name=ms6]{Verifikation Done}{34} \\

%-------------------------------------------------------------
\ganttgroup{Projektdokumentation}{35}{42} \\

\ganttbar[progress=0, progress label font=\small\color{black!75},
	progress label anchor/.style={right=4pt},
	bar label font=\normalsize\color{black},
	name=thesis]
	{Thesis schreiben}{35}{42} \\
	
\ganttmilestone[name=msthesis,milestone label font=\color{red}, 
	milestone/.style={fill=red}]{Abgabe}{42}

%\ganttlink{ms7}{bImplAlgo}
\ganttlink{bImpFPGA}{ms4}
\ganttlink{bNumVerf}{ms3}
\ganttlink{bCMAES}{ms3}
\ganttlink{rech}{ms1}
\ganttlink{pflichten}{ms2}
\ganttlink{thesis}{msthesis}

	\end{ganttchart}
		}
\end{landscape}

%----------------------------------------------------------------------------

\end{appendix}


\newpage
%- Bibliography --------------------------------------------------------------
\bibliographystyle{ieeetr}
\bibliography{../bib/mathesis_collection1}

\end{document}