\documentclass[a4paper,12pt,fleqn]{article}
\usepackage[T1]{fontenc}
\usepackage{ucs}
\usepackage[utf8x]{inputenc}
\usepackage{ngerman}
\usepackage[ngerman]{babel}
\usepackage{lastpage}
\usepackage[pdftex]{color,graphicx}
\usepackage{listings}
\usepackage{pdflscape}
\usepackage{longtable}
\usepackage[inner=2cm,outer=2cm,top=1cm,bottom=1.5cm,includeheadfoot]{geometry}
\usepackage{fancyhdr}
\usepackage{url}
\usepackage{draftwatermark}

\SetWatermarkText{Vertraulich}
\SetWatermarkScale{4}
\SetWatermarkLightness{0.9}

\usepackage{pgfgantt}
\usepackage{amsmath,amssymb,amsfonts,amstext}
\usepackage{floatflt}
\usepackage{tikz}
\usetikzlibrary{arrows, snakes}
\usetikzlibrary{through}
\usetikzlibrary{calc}

% highlighting
\usepackage{xcolor,soul}

%---- PageLayout
\pagestyle{fancy}

\setlength{\headsep}{10mm}

\usepackage{eso-pic}

%----------------------------------------------------------------------------
% HEADER --------------------------------------------------------------------
%----------------------------------------------------------------------------
\fancyhead[R]{
  \includegraphics[width=100pt,keepaspectratio]{img/amedo2012.png}
}

\fancyhead[C]{ Wochenbericht KW 20 }

\fancyhead[L]{
  \begin{tabular}[b]{l}
  Christoph Gnip\\
  Projekt: PRPS-Evolution
  \end{tabular}
}

%Linie oben
\renewcommand{\headrulewidth}{0.5pt}
%----------------------------------------------------------------------------


%----------------------------------------------------------------------------
%----------------------------------------------------------------------------
%----------------------------------------------------------------------------
\fancyfoot[L]{Stand: \today}
\fancyfoot[C]{ }
\fancyfoot[R]{\thepage{} von \pageref{LastPage}}

% Linie unten
\renewcommand{\footrulewidth}{0.5pt}
%----------------------------------------------------------------------------

% Import Macros  ------------------------------------------------------------
%--------------------------------------------------------------
%--------------------------------------------------------------
%--------------------------------------------------------------
\newcommand\confidentialoverlay{
  % Taken from the TikZ documentation.
  % NB: This requires \usepackage{tikz}!
  \begin{tikzpicture}[remember picture,overlay]
    \node [rotate=60,scale=10,text opacity=0.1]
      at (current page.center) {Vertraulich};
  \end{tikzpicture} 
 
} 
%--------------------------------------------------------------
%--------------------------------------------------------------
%--------------------------------------------------------------
\newcommand{\myvec}[1]{\hat{\mathbf{#1}}}% Vector notation

%--------------------------------------------------------------
%- This can be used for aligning equations --------------------
%--------------------------------------------------------------
\newcommand{\phantomeq}[2]{
\begin{equation}
	\phantom{#1}
	#2
\end{equation}
}% Vector notation

%--------------------------------------------------------------
%- seraches for input in the "extern" folder ------------------
%--------------------------------------------------------------
\newcommand{\externInput}[1]{\input{extern/#1}}

%--------------------------------------------------------------
%- seraches for input in the "intern" folder ------------------
%--------------------------------------------------------------
\newcommand{\internInput}[1]{\input{intern/#1}}

%--------------------------------------------------------------
%- seraches for input in the "common" folder ------------------
%--------------------------------------------------------------
\newcommand{\commonInput}[1]{\input{common/#1}}

\newcommand{\cpp}{%
  \mbox{\emph{\textrm{C\hspace{-1.5pt}\raisebox{1.75pt}{\scriptsize +}%
  \hspace{-2pt}\raisebox{.75pt}{\scriptsize +}}}}%
}

\newcommand{\amedogmbh}{%
  amedo GmbH
}

%\renewenvironment{itemize}[1]{\begin{compactitem}#1}{\end{compactitem}}
%\renewenvironment{enumerate}[1]{\begin{compactenum}#1}{\end{compactenum}}
%\renewenvironment{description}[0]{\begin{compactdesc}}{\end{compactdesc}}


%----------------------------------------------------------------------------
% Start the Document --------------------------------------------------------
%----------------------------------------------------------------------------
\begin{document}

\setlength{\headheight}{36pt}

\begin{titlepage}

\input{title/title_w4.tex}

\end{titlepage}

%- Section 1 ----------------------------------------------------------------
\section[Allgemeines]{Allgemeines}
Eine aktualisierte Version des Projektplans befindet sich in Anhang~\ref{sec:projectplan}.

%- Section 2 ----------------------------------------------------------------
\section[Fortschritt]{Projektfortschritt}
In dieser Woche wurden die Arbeiten an der Modellbildung für die num. Verfahren begonnen. Die Überlegungen sind im Anhang~\ref{sec:iso_phase_modell} und im Anhang~\ref{sec:trilateration_modell} zu finden.
Am Ende der Woche ist es gelungen, Messdaten von allen Antennen zu erhalten. Dadurch ist sichergestellt, dass auf Daten zurück gegriffen werden kann, wenn die Modellbildung weit genug fortgeschritten ist. 

%- Section 2.2 --------------------------------------------------------------
\subsection{Modellbildung}
Die Tauglichkeit des in Anhang~\ref{sec:iso_phase_modell} Modells ist aus jetziger Sicht in Frage zu stellen. Es kann voraussichtlich nicht mit der aktuellen Anordnung des Messaufbaus verwendet werden.\\
Daher wurde im Anschluss ein allgemeineres Modell erarbeitet, dass auf den Grundlagen der bereits in Verwendung befindlichen Trilateration basiert und um eine Linearisierung erweitert wurde um es effizienter optimieren zu können.\\
Die betriebenen Überlegungen für das Modell führten zu Vereinfachungen an dem Kalibrieraufbau. Das Kalibrierstück, das für die freie Antennenpositionierung verwendet wird, wurde nach den Überlegungen die zu \eqref{eq:final_trilateration_model}  führten (Linearisierung), mit einem zusätzlichen Referenzpunkt (Landmarke) in dem Kalibrieraufbau angebracht. Eine erneute Einmessung wird in der kommenden Woche erfolgen.\\
Im nächsten Schritt wird das Modell um die physikalischen Grundlagen (Wellenzahl und Phase) erweitert. Ausgehend von Gleichung~\eqref{eq:c_k0} werden die bekannten Zusammenhänge für $r_0$ und $r_k$. Diese Anstrengungen sollten, nach dem aktuellen Kenntnisstand, zu einen für die Minimierung geeigneten Modell führen. Wie viele Parameter für die Optimierung übrig bleiben wird sich in der nächsten Wochen zeigen.

%- Section 2.3 --------------------------------------------------------------
\subsection{Kalibierung}
Mit V. Trösken wurden die Modelle besprochen. Die Erkenntnisse aus dem Modell fließen in einer Erweiterung für das in der letzten Woche angefertigten Kalibrierstücks ein. Der neue Aufbau wird einen vierten Messpunkt mit bekannter Geometrie erhalten, der die Verwendung des im Anhang~\ref{sec:trilateration_modell} vorgestellten Modells erlaubt.\\
Die in dieser Woche durchgeführte Vermessung wird in der nächsten Woche mit dem neuen Aufbau wiederholt und im Anschluss wird die Anordnung berechnet.\\
Der neue Aufbau steht Anfang der nächsten Woche zur Verfügung.

%- Section 3 -----------------------------------------------------------------
\section[Probleme]{Probleme}
Durch die Feiertage läuft die Entwicklung zur Zeit langsamer als ursprünglich geplant. Es wird voraussichtlich einer Anpassung des Projektplans notwendig sein um diesem Umstand gerecht zu werden.

%- Appendix ------------------------------------------------------------------
%
%
%
%

\begin{appendix}
%----------------------------------------------------------------------------
%----------------------------------------------------------------------------
%----------------------------------------------------------------------------
\newpage
\huge{Anhänge}
\normalsize

\section{Iso-Phasen-Modell}
\label{sec:iso_phase_modell}

Auf den folgenden Seiten befindet sich eine Modellierung des physikalischen Zusammenhangs. Ziel ist die Entwicklung eines Modells, dass sich für die Verwendung in einer Minimierungsstrategie eignet.
%----------------------------------------------------------------------------
\begin{figure}[h]
	\begin{center}
		% Intersection of
% Author: Rasmus Pank Roulund

\begin{tikzpicture}[
    scale=10,
    axis/.style={very thick, ->, >=stealth'},
    vector/.style={thick, ->, >=stealth'},
    antenna/.style={thick},
    important line/.style={thick},
    dashed line/.style={dashed, thin},
    pile/.style={thick, ->, >=stealth', shorten <=2pt, shorten
    >=2pt},
    every node/.style={color=black}
    ]
    % axis
    \draw[axis] (-0.05,0)  -- (0.2,0) node(xline)[right] {$x$};
    \draw[axis] (0,-0.05) -- (0,0.2) node(yline)[above] {$z$};
    % Lines
    \draw[vector] (0,0) coordinate (Xor) -- (.40,.25)
        coordinate (A) node[right, text width=5em]{};
%
    \draw[antenna,rotate around={40:(A)}] (A) -- (.40,.32) coordinate (b) node[right, text width=5em] {};

    \draw[antenna,rotate around={220:(A)}] (A) -- (.40,.32);

    \draw[axis,rotate around={40:(A)},gray] (A)  -- ( .40,.30) node(xline)[above] {$x'$};
    \draw[axis,rotate around={40:(A)},gray] (A) -- ( .45,.25) node(yline)[above] {$z'$};
        
%        
%    \draw[important line] (0.9,0.5) coordinate (C) -- (D) node[right, text width=5em]
%%    \draw[important line] (D) -- (0.5,0.9) coordinate (F) node[right, text width=5em]
%%         {$\mathit{NX}=x$};
%
%	\fill[red] (-.075,-.2) coordinate (out) circle (.2pt)
%        node[below left] {$B$};

%    % Intersection of lines
%    \fill[red] (intersection cs:
%       first line={(A) -- (B)},
%       second line={(C) -- (D)}) coordinate (E) circle (.4pt)
%       node[above,] {$A$};
%    % The E point is placed more or less randomly
%    \fill[red]  (E) +(-.075cm,-.2cm) coordinate (out) circle (.4pt)
%        node[below left] {$B$};
%    % Line connecting out and ext balances
%    \draw [pile] (out) -- (intersection of A--B and out--[shift={(0:1pt)}]out)
%        coordinate (extbal);
%    \fill[red] (extbal) circle (.4pt) node[above] {$C$};
%    % line connecting  out and int balances
%    \draw [pile] (out) -- (intersection of C--D and out--[shift={(0:1pt)}]out)
%        coordinate (intbal);
%    \fill[red] (intbal) circle (.4pt) node[above] {$D$};
%    % line between out og all balanced out :)
%    \draw[pile] (out) -- (E);
\end{tikzpicture}

%%% Local Variables:
%%% mode: latex
%%% TeX-master: t
%%% End:
		 \caption[Kurzeintrag]{Entwurf des Modells für die Berechnung. Gezeigt ist, ein Tag und zwei Antennen. Die graue Linie stellt eine Iso-Fläche der Phase dar. Von diesen Flächen befinden sich $N$ viele mit einem Abstand von $\frac{\lambda}{2}$ (der Übersicht halber nicht eingezeichnet).} 
	\end{center}
\end{figure}

Aus einfachen Überlegungen ergibt sich:

\begin{equation}\label{eq:Tri1}
(l_1+\Delta)^2 = (l_1+\delta)^2+h^2
\end{equation}
\begin{equation}\label{eq:Tri2}
(l_2+\Delta)^2 = (l_2-\delta)^2+h^2
\end{equation}
%
aus \eqref{eq:Tri1}:
\begin{align}
h^2 &= (l_1+\Delta)^2 - (l_1+\delta)^2\\
	&= l_1^2 + 2l_1\Delta + \Delta^2 -[ l_1^2 + 2 l_1 \delta + \delta^2]\nonumber\\
	&= l_1^2 + 2l_1\Delta + \Delta^2 - l_1^2 - 2 l_1 \delta - \delta^2	\nonumber\\
	&= 2l_1\Delta + \Delta^2 - 2 l_1 \delta - \delta^2 \label{eq:Tri3}
\end{align}
%
analog dazu lässt sich aus~\eqref{eq:Tri2} ableiten:
%
\begin{align}
h^2 &= 2l_2\Delta + \Delta^2 + 2 l_2 \delta - \delta^2 \label{eq:Tri4}
\end{align}
%
Wir suchen einen Ausdruck um $\Delta$ zu eliminieren, \eqref{eq:Tri3}=\eqref{eq:Tri4}:
%
\begin{align}
	2l_1\Delta + \Delta^2 - 2 l_1 \delta - \delta^2 &=  2l_2\Delta + \Delta^2 + 2 l_2 \delta - \delta^2 \nonumber \\
	2l_1\Delta + \Delta^2 - 2l_2\Delta - \Delta^2&= 2 l_2 \delta - \delta^2 +2l_1\delta +\delta^2 \nonumber \\
	\Delta(l_1 - l_2  ) &= \delta (l_1+l_2) \nonumber \\
	\Delta &= \delta\frac{(l_1 + l_2)}{(l_1 - l_2)} \label{eq:Tri5}
\end{align}

\eqref{eq:Tri5} in \eqref{eq:Tri3} einsetzen:
\begin{align}
	h^2 &= 2l_1\delta\left(\frac{(l_1 + l_2)}{(l_1 - l_2)}\right) + \delta^2\left(\frac{(l_1 + l_2)}{(l_1 - l_2)}\right)^2 - 2l_1 \delta - \delta^2 \nonumber \\
	&= \underbrace{\left(\frac{(l_1 + l_2)}{(l_1 - l_2)} - 1 \right)^2}_\text{$a_0$} \delta^2 + \underbrace{\left(\frac{(l_1 + l_2)}{(l_1 - l_2)}  - 1 \right)}_\text{$a_1$} 2l_1\delta
\end{align}

Nun können wir für $\delta$ den Ausdruck $x'$ einführen. und erhalten:
\begin{equation}
	h^2 = a_0 x'^2 + a_1 x' \label{eq:TriFinal}
\end{equation}
%
Gleichung~\eqref{eq:TriFinal} drückt aus, dass sich die Berechnung noch in dem Koordinatensystem der beiden gewählten Antennen liegt. Es ist noch eine geeignete Koordinatensystem-Transformation zu wählen um auf ein vernünftiges Referenz-Koordinatensystem zu kommen.
Weiterhin gelten die Überlegungen nur für diesen Zweidimensionalen Fall. Eine Anwendung auf die dritte Dimension sollte über die Erweiterung von $h^2 = y^2+z^2$ möglich sein.

%----------------------------------------------------------------------------

%----------------------------------------------------------------------------
%----------------------------------------------------------------------------
%----------------------------------------------------------------------------
\newpage
%----------------------------------------------------------------------------
\section{Trilaterationsmodell}
\label{sec:trilateration_modell}

\begin{figure}
	\begin{center}
		\caption[Antennen-Szene mit einem Tag]{2D-Übersicht auf die Szene mit drei Antennen, einem Tag und einer Landmarke. Die Position von $\{A_1,A_3,A_3\}$, sowie der Landmarke, zum Koordinatenursprung sind bekannt. Die Vektoren $r_1,r_2,r_3$ sind die gemessene Entfernung zu einer Antenne. Die Landmarke wird im späteren Verlauf eine Antenne sein, die ihrerseits ein gemessene Entfernung $r_0$ produziert. Der Schnittpunkt aller Kreise ist die Lösung der gemessenen Entfernung und der geom. Anordnung, die sich für die Position des Tags ergibt.} 
		\label{fig:TrilaterationScene}
		\begin{tikzpicture}[
    scale=1,
    axis/.style={thick, ->, >=stealth'},
%    vector/.style={thick, ->,-latex, >=stealth'},
%    antenna/.style={thick},
     important line/.style={thick},
     antenna/.style={thick, cyan!70},
%    dashed line/.style={dashed, thin},
%    pile/.style={thick, ->, >=stealth', shorten <=2pt, shorten
%    >=2pt},
%    every node/.style={color=black},
%    main node/.style={circle,fill=blue!20,draw},
%    help lines/.style={gray,very thin}
    ]
    % axis
    \draw[axis] (-.1,0)  -- (1,0) node(xline)[right] {$x$};
    \draw[axis] (0,-.1) -- (0,1) node(yline)[above] {$y$};

	\draw[gray, very thin, dotted] (0,0) grid (15,6);

	\coordinate (A1_start) at (4,3);
	\coordinate (A1_end) at (4,4);
	\coordinate (A2_start) at (7,5);
	\coordinate (A2_end) at (8,5);
	\coordinate (A3_start) at (8,1);
	\coordinate (A3_end) at (8,2);

	\coordinate (A1_end_) at ($(A1_start)!1!-10:(A1_end)$);
	\coordinate (A2_end_) at ($(A2_start)!1!-10:(A2_end)$);
	\coordinate (A3_end_) at ($(A3_start)!1!-35:(A3_end)$);
	
	\coordinate (Tag_0) at (6,2);
	\coordinate (REF_0) at (12,5);
	\coordinate (Int1) at ($(A1_start)!.5!(A1_end_)$);
	\coordinate (Int2) at ($(A2_start)!.5!(A2_end_)$);
	\coordinate (Int3) at ($(A3_start)!.5!(A3_end_)$);
	
	\begin{scope}
		\node [draw,orange!50,dashed] at (Int1) [circle through={(Tag_0)}] {};
		\node [draw,orange!50,dashed] at (Int2) [circle through={(Tag_0)}] {};
		\node [draw,orange!50,dashed] at (Int3) [circle through={(Tag_0)}] {};
	\end{scope}
	
	\draw[antenna] (A1_start) node[font=\scriptsize,black,below] {$A_1$} -- ($(A1_start)!1!-10:(A1_end)$);
	\draw[antenna] (A2_start) node[font=\scriptsize,black,above] {$A_2$}-- ($(A2_start)!1!-10:(A2_end)$);
	\draw[antenna] (A3_start) node[font=\scriptsize,black,below] {$A_3$}-- ($(A3_start)!1!-35:(A3_end)$);
	
	\node [green!60!black!90, right,font=\scriptsize ] at (REF_0) {$\text{Landmarke}@(x_0,y_0,z_0)$};

	\draw[latex-latex] (Tag_0) -- node[sloped,above,midway] {$r_1$}(Int1);
	\draw[latex-latex] (Tag_0) -- node[sloped,above,midway] {$r_2$}(Int2);
	\draw[latex-latex] (Tag_0) -- node[sloped,above,midway] {$r_2$}(Int3);
	\draw[-latex,dashed,green!60!black!90] (REF_0) -- node[sloped,above,midway] {$r_0$}(Tag_0);
	
	\draw[ -latex,violet!60,font=\scriptsize,dotted] (REF_0) -- node[sloped,above,midway] {$d_{10}$}(Int1);
	\draw[ -latex,violet!60,font=\scriptsize,dotted] (REF_0) -- node[sloped,above,midway] {$d_{20}$}(Int2);
	\draw[ -latex,violet!60,font=\scriptsize,dotted] (REF_0) -- node[sloped,above,midway] {$d_{30}$}(Int3);
		
	\fill[red!70] (Tag_0) circle [radius=2pt];
	\node[font=\scriptsize,black,below] at (Tag_0) {$Tag$} ;
	\fill[green!60!black!90] (REF_0) circle [radius=2pt];
	
\end{tikzpicture}



%		
	\end{center}
\end{figure}
%
Folgende Nomenklatur und Symbole gelten für diesen Abschnitt:
\begin{itemize}[itemsep=0mm]
	\item	$r_{k}$ := Abstand vom Tag zur Antenne
	\item	$d_{kJ}$ := Abstand zur Landmarke
	\item	$N_0:=$ Menge der verfügbaren Antennen $N=\{1,..,8\}$
	\item	$N:=$ Menge der Antennen für die Optimierung verfügbar sind\footnote{d.h. ein Messergebnis liefern}($N \subseteq N_0$)
	\item	$N':=$ Menge der Antennen für die Optimierung ($N' \subseteq N$)
%	; Dabei ist $|N'| \geq 3$
%	\item	Es gilt $|N'| \geq |N| \geq |N_0|$   
	\item	$j$ ist der Index der Referenzantenne, es gilt $j = \{1,2,..,8\}$
	\item	$k$ ist der Index der Antennen einer Messung, es gilt $k = 1,2,..,|N'|-1$
\end{itemize}
%
Wir starten mit der Überlegung über den geometrischen Zusammenhang zwischen der Antennenposition von Antenne $k$ zu der Position des Tags $r_k$:
\begin{align}
	\label{eq:base_vactor}
	r_{k}^2 &= (x-x_k)^2+(y-y_k)^2+(z-z_k)^2
\end{align}
%
Diese Gleichung stellt die Euklidische Vektornorm dar und entspricht der Strecke Antenne-Tag. Für die Ermittelung einer Postion (mit drei Raumkoordinaten) sind drei Antennen Notwendig. Daraus ergibt sich:
%
\begin{itemize}[itemsep=0mm]
\item 3 Gleichunge n
\item 3 Unbekannte
\item Quadratisches Gleichungssystem
\end{itemize}
%
Das Gleichungssystem sieht wie folgt aus:
%
\begin{align}
	r_{1}^2 &= (x-x_1)^2+(y-y_1)^2+(z-z_1)^2 \nonumber\\
	r_{2}^2 &= (x-x_2)^2+(y-y_2)^2+(z-z_2)^2 \nonumber\\
	r_{3}^2 &= (x-x_3)^2+(y-y_3)^2+(z-z_3)^2 \nonumber
%	
\end{align}
%
Es ist trivial und wird in verschiedenen Beispielen gezeigt\footnote{z.B. \url{http://en.wikipedia.org/w/index.php?title=Trilateration&oldid=553215995}}, dass man die Koordinaten aus dem quadratischen Gleichungssystem unmittelbar berechnen kann. Es muss jedoch ein quadratisches Gleichungssystem gelöst werden, was zu den bekannten Problematiken führt, insbesondere der Ausschluss mehrdeutiger Ergebnisse. Der Messaufbau der \amedogmbh erlaubt die Verwendung von mehr als 3 Messwertgebern. Diese zusätzliche Informationen lassen sich für eine Linearisierung des Gleichungssystems verwenden. Dieser Ansatz wird für ein Modell im Rahmen dieser Arbeit verwendet und wird im Folgenden beschrieben.\\
%
Von den Antennen sind die Raumkoordinaten ($x,y,z-Koordinaten$) bekannt, bzw. wurden durch Kalibrierung \ref{sec:calibration} in einem vorherigen Schritt bestimmt. Wir können zusätzlich zu notieren:
%
\begin{equation}\label{eq:d_k0}
	d_{kj}^2= (x_k-x_0)^2+(y_k-y_0)^2+(z_k-z_0)^2
\end{equation}
%
Linearisierung des Modells. Dazu wird Gleichung~\ref{eq:base_vactor} in mehreren Schritten umgebaut. Zuerst wird eine neutrale Erweiterung durchgeführt und die Terme geschickt zusammengefasst. Das führt zu:
%
\begin{align}
	r_{k}^2 &= (x-x_k)^2+(y-y_k)^2+(z-z_k)^2 \nonumber \\
	&=(x-x_k+x_0-x_0)^2+(y-y_k+y_0-y_0)^2+(z-z_k+z_0-z_0)^2 \nonumber \\
	&=((x-x_0)-(x_k-x_0))^2+((y-y_0)-(y_k-y_0))^2+((z-z_0)-(z_k-z_0))^2 \nonumber \\ 
	%2 bin. Form
	&=(x-x_0)^2-2(x-x_0)(x_k-x_0)+(x_k-x_0)^2\underbrace{+\dots{}+\dots{}}_\text{y-\& z-Terme analog}
	\label{eq:tri_temp1}
%
\end{align}
%
Um Platz zu sparen sind die y- und z-Terme nicht explizit notiert. Sie ergeben sich durch einfaches Ersetzen der Indizes und werden im Finalen Modell eingefügt. Durch Umstellen von \eqref{eq:tri_temp1} erhalten wir:
\begin{align}
(x-x_0)(x_k-x_0)+\dots{}+\dots{}&=-\frac{1}{2}[r_k^2-(x_k-x_0)^2 -(x-x_0)^2 +\dots{} +\dots{}]\nonumber\\
(x-x_0)(x_k-x_0)+\dots{}+\dots{}&=\phantom{-}\frac{1}{2}[(x_k-x_0)^2 +(x-x_0)^2 +\dots{}+\dots{}-r_k^2]\nonumber
%
\end{align}
%
\begin{multline}\label{eq:rk_final}
(x-x_0)(x_k-x_0)+(y-y_0)(y_k-y_0)+(z-z_0)(z_k-z_0)= \\\frac{1}{2}[(x_k-x_0)^2+(x-x_0)^2-(y_k-y_0)^2+(y-y_0)^2
\\-(z_k-z_0)^2 +(z-z_0)^2-r_k^2]
\end{multline}
%
Vergleich von \eqref{eq:rk_final} mit \eqref{eq:d_k0} bringt: 
%
\begin{multline}
(x-x_0)(x_k-x_0)+(y-y_0)(y_k-y_0)+(z-z_0)(z_k-z_0)= \\\frac{1}{2}[\underbrace{(x_k-x_0)^2+(z_k-z_0)^2+(y_k-y_0)^2}_\text{\boldmath{$d_{kj}^2$}}
\\+\underbrace{(x-x_0)^2+(y-y_0)^2 +(z-z_0)^2}_\text{\boldmath{$r_j^2$}}-r_k^2]
\end{multline}
%
\begin{equation}
(x-x_0)(x_k-x_0)+(y-y_0)(y_k-y_0)+(z-z_0)(z_k-z_0)=\frac{1}{2}[d_{kj}^2+r_{j}^2-r_k^2]\label{eq:rk_final_simplyfied}
\end{equation}
mit 
\begin{equation}\label{eq:c_kj}
\mathbf{c_{kj}}=\frac{1}{2}[d_{kj}^2+r_{j}^2-r_k^2]
\end{equation}
können wir das lineare Gleichungssystem abschließend schreiben:
%
\begin{equation}
%\label{eq:final_trilateration_model}
\mathbf{0}=
\left(
	\begin{array}{ccc}
		x_1-x_j & y_1-y_j & z_1-z_j \\
		x_2-x_j & y_2-y_j & z_2-z_j \\
		x_3-x_j & y_3-y_j & z_3-z_j
	\end{array}
\right)
\left(
   \begin{array}{c}
	   x-x_j\\
	   y-y_j\\
	   z-z_j
   \end{array}
\right)
-
\left(
	\begin{array}{c}
		c_{1j}\\
		c_{2j}\\
		c_{3j}
	\end{array}
\right)
\end{equation}
%
Das Gleichungssystem entspricht ist linear und hat die allg. Form: $\mathbf{0} = \mathbf{Ax}+\mathbf{b}$ es lässt sich mit bekannten Methoden lösen.




%----------------------------------------------------------------------------

%----------------------------------------------------------------------------
%----------------------------------------------------------------------------
%----------------------------------------------------------------------------
\begin{landscape}
	\section{Projektlaufplan KW 20}
	\label{sec:projectplan}
	\scalebox{.75}{
		\input{documents/project_plan.tex}
		}
\end{landscape}

%----------------------------------------------------------------------------

\end{appendix}


\newpage
%- Bibliography --------------------------------------------------------------
\bibliographystyle{ieeetr}
\bibliography{../bib/mathesis_collection1}

\end{document}