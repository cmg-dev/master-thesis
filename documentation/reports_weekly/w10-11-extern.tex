\documentclass[a4paper,12pt,fleqn]{article}
\usepackage[T1]{fontenc}
\usepackage{ucs}
\usepackage[utf8x]{inputenc}
\usepackage{ngerman}
\usepackage[ngerman]{babel}
\usepackage{lastpage}
\usepackage[pdftex]{color,graphicx}
\usepackage{listings}
\usepackage{pdflscape}
\usepackage{longtable}
\usepackage[inner=2cm,outer=2cm,top=1cm,bottom=1.5cm,includeheadfoot]{geometry}
\usepackage{fancyhdr}
\usepackage{url}
\usepackage{draftwatermark}
\usepackage{booktabs}
\usepackage{blindtext} 
\usepackage{framed} 
\usepackage{xcolor} 
\colorlet{shadecolor}{black} 
\usepackage{latexsym}

\usepackage{bbm}
\usepackage{enumitem}

\SetWatermarkText{Vertraulich}
\SetWatermarkScale{4}
\SetWatermarkLightness{0.9}

\usepackage{pgfgantt}
\usepackage{amsmath,amssymb,amsfonts,amstext}
\usepackage{floatflt}
\usepackage{tikz}
\usetikzlibrary[arrows,snakes,backgrounds,shapes]
\usetikzlibrary{through}
\usetikzlibrary{calc}
\usepackage{caption}
\usepackage{subcaption}

% highlighting
\usepackage{xcolor,soul}

%---- PageLayout
\pagestyle{fancy}

\setlength{\headsep}{10mm}

\usepackage{eso-pic}

%----------------------------------------------------------------------------
% HEADER --------------------------------------------------------------------
%----------------------------------------------------------------------------
\fancyhead[R]{
  \includegraphics[width=100pt,keepaspectratio]{img/amedo2012.png}
}

\fancyhead[C]{ Wochenbericht KW 26/27 }

\fancyhead[L]{
  \begin{tabular}[b]{l}
  Christoph Gnip\\
  Projekt: PRPS-Evolution
  \end{tabular}
}

%Linie oben
\renewcommand{\headrulewidth}{0.5pt}
%----------------------------------------------------------------------------

%----------------------------------------------------------------------------
%----------------------------------------------------------------------------
%----------------------------------------------------------------------------
\fancyfoot[L]{Stand: \today}
\fancyfoot[C]{ EXTERN }
\fancyfoot[R]{\thepage{} von \pageref{LastPage}}

% Linie unten
\renewcommand{\footrulewidth}{0.5pt}
%----------------------------------------------------------------------------

% Import Macros  ------------------------------------------------------------
%--------------------------------------------------------------
%--------------------------------------------------------------
%--------------------------------------------------------------
\newcommand\confidentialoverlay{
  % Taken from the TikZ documentation.
  % NB: This requires \usepackage{tikz}!
  \begin{tikzpicture}[remember picture,overlay]
    \node [rotate=60,scale=10,text opacity=0.1]
      at (current page.center) {Vertraulich};
  \end{tikzpicture} 
 
} 
%--------------------------------------------------------------
%--------------------------------------------------------------
%--------------------------------------------------------------
\newcommand{\myvec}[1]{\hat{\mathbf{#1}}}% Vector notation

%--------------------------------------------------------------
%- This can be used for aligning equations --------------------
%--------------------------------------------------------------
\newcommand{\phantomeq}[2]{
\begin{equation}
	\phantom{#1}
	#2
\end{equation}
}% Vector notation

%--------------------------------------------------------------
%- seraches for input in the "extern" folder ------------------
%--------------------------------------------------------------
\newcommand{\externInput}[1]{\input{extern/#1}}

%--------------------------------------------------------------
%- seraches for input in the "intern" folder ------------------
%--------------------------------------------------------------
\newcommand{\internInput}[1]{\input{intern/#1}}

%--------------------------------------------------------------
%- seraches for input in the "common" folder ------------------
%--------------------------------------------------------------
\newcommand{\commonInput}[1]{\input{common/#1}}

\newcommand{\cpp}{%
  \mbox{\emph{\textrm{C\hspace{-1.5pt}\raisebox{1.75pt}{\scriptsize +}%
  \hspace{-2pt}\raisebox{.75pt}{\scriptsize +}}}}%
}

\newcommand{\amedogmbh}{%
  amedo GmbH
}

%\renewenvironment{itemize}[1]{\begin{compactitem}#1}{\end{compactitem}}
%\renewenvironment{enumerate}[1]{\begin{compactenum}#1}{\end{compactenum}}
%\renewenvironment{description}[0]{\begin{compactdesc}}{\end{compactdesc}}


%----------------------------------------------------------------------------
% Start the Document --------------------------------------------------------
%----------------------------------------------------------------------------
\begin{document}

\setlength{\headheight}{36pt}

\begin{titlepage}

\input{extern/title/title_w10.tex}

\end{titlepage}

%- Section 1 ----------------------------------------------------------------
\section[Allgemeines]{Allgemeines}
%
In diesem Wochenbericht werden die Projektwochen 10 und 11 zusammengefasst. Das ermöglicht eine übersichtlichere Darstellung der Arbeitsergebnisse dieser Wochen, da es vor Allem um die Erstellung von Software ging. 
%
%- Section 2 ----------------------------------------------------------------
\section[Fortschritt]{Projektfortschritt}
%
Das Programm ermöglicht es nun automatisiert alle Antennenkonfigurationen mit allen benötigten Informationen zu erstellen und eine Lösung durch evolutionäre Strategien zu finden. Weiterhin wird automatisiert die Kalibierung des Messaufbaus anhand des in dieser Arbeit entwickelten Modells bestimmt.

%- Section 2.1 --------------------------------------------------------------
\subsection{Programmierung}
%
Das Programm wurde in mehrere Libraries aufgeteilt. Zum Einen erhöhen sich Wartbarkeit und Wiederverwendbarkeit und zum Anderen können diese vor-compiliert werden und reduzieren die Compilezeit erheblich. Einen Überblick über die wesentlichsten Merkmale gibt der Anhang~\ref{documentation_libs}.\\
Die Verwendung von \cpp11 erleichtert eine Menge der Entwicklung. Die anfängliche Einarbeitung in diesen Standard hat sich bereits mehr als ausgezahlt. Im Besonderen sind zur Zeit die Möglichkeiten für Thread-Programmierung und Verwaltung großer Datenmengen zu nennen. Diese sind anderen aktuellen Programmiersprachen, wie Java oder C$\sharp$, überlegen und beschleunigen die Entwicklung.\\
%
%- Section 2.2 --------------------------------------------------------------
\subsection{Neue Modelle von Susanne Winter}
%
In der KW 26 hat Frau S. Winter ihre Ergebnisse vorgestellt und eine Dokumentation übergeben, mit der es möglich ist ihre Ansätze nachzuvollziehen und zu implementieren. Ihre Ansätze unterscheiden sich von denen im Rahmen dieser Arbeit entwickelten grundsätzlich.\\
Ihre Ansätze werden in jedem Fall umgesetzt und sollte es der Rahmen dieser Arbeit es Zeitlich erlauben wird die Performance gegen diese Modelle gebenchmarkt werden.
%
%- Section 3 -----------------------------------------------------------------
\section{Probleme}
\label{Problems}
%

%- Appendix ------------------------------------------------------------------
%
%
%
\begin{appendix}

%----------------------------------------------------------------------------
%----------------------------------------------------------------------------
%----------------------------------------------------------------------------
\newpage

\begin{center}
	\huge{Anhänge}
\end{center}

\normalsize

%----------------------------------------------------------------------------
%----------------------------------------------------------------------------
%----------------------------------------------------------------------------
\section{Dokumentation Libraries}
\label{documentation_libs}
\begin{enumerate}
%
\item libCalibration
	\begin{enumerate}
	\item Enthält die Calibration-Klasse
	\item Bestimmt aus den Entfernungsmessungen von dem Kalibrierphantom die Position der Antennen.
	\end{enumerate}
%	
\item libNormalizer
	\begin{enumerate}
	\item Sammelt verschiedene Methoden für die Normierung der gemessenen Werte
	\end{enumerate}
%	
\item libPermutate
	\begin{enumerate}
	\item Erstellt alle möglichen Permutationen die für den Antennenaufbau möglich sind
	\item Stellt diese Permutationen für spätere Berechnungen komfortabel zur Verfügung
	\end{enumerate}
%
\item libPRPSSystem
	\begin{enumerate}
	\item Ließt die Systemkenndaten (z.B. Messfrequenz etc.) aus der entsprechenden Eingabedatei und berechnet die sich daraus ableitenden Faktoren (z.B. Wellenlänge etc.)
	\end{enumerate}
%
\item libSolve
	\begin{enumerate}
	\item Das eigentliche Herzstück des Projekts
	\item Ermöglicht das Lösen mehrerer unterschiedlicher Modelle
	\item Implementiert unterschiedliche Lösungsstrategien, wie z.B. (1+1)-ES, ($\mu+\lambda$)-ES etc.
	\item Ist Threadsicher um die Performance erheblich zu verbessern
	\item Die Dimension des Problem lässt sich leicht anpassen
	\end{enumerate}
%
\end{enumerate}

%----------------------------------------------------------------------------
%----------------------------------------------------------------------------
%----------------------------------------------------------------------------
\newpage
\section{Konditionen vs. Berechnete Ergebnisse}
\label{cond_vs_results}
\begin{figure} [h]
         \centering
         \caption{Analyse der Konditionszahlen gegen die Berechnungsergebnisse eines Messpunktes; Auffällig ist, dass die Lösungen für Referenzantenne 0 und 7 praktisch \textbf{keine} korrekte Lösung liefern. }
         \label{fig:CondNumberAnalyze}
%         
         \begin{subfigure}[t]{0.4\textwidth}
                 \centering
                 \includegraphics[width=\textwidth]{common/img/ConditionPlot_scaled.png}
                 \caption{Farbkodierte Matrix der Kondition aller möglichen Kombinationen (skaliert auf den höchsten vorkommenden Wert)\\
                 B=> grün~:=~gute -, rot=~schlechte Konditionierung}
                 \label{fig:ConditionMatrix}\textit{}
         \end{subfigure}
%         
\qquad
         \begin{subfigure}[t]{0.4\textwidth}
                 \centering
                 \includegraphics[width=\textwidth]{common/img/60_Results_scaled.png}
                 \caption{ Farbkodierte Ergebnisse der möglichen Kombinationen aus 6 von 8 Antennen; Die Position des Ergebnisses in der Matrix entspricht der in der Konditionsmatrix ddddddddd ddddddd dddddd dddddddddddddddddd.}
                 \label{fig:Results}
         \end{subfigure}
%
\end{figure}


%----------------------------------------------------------------------------
%----------------------------------------------------------------------------
%----------------------------------------------------------------------------
\newpage
\begin{landscape}
	\section{Projektlaufplan KW 26}
	\label{sec:projectplan}
	\scalebox{.75}{
		\input{common/documents/project_plan_KW26.tex}
		}
\end{landscape}

%----------------------------------------------------------------------------

\end{appendix}


\newpage
%- Bibliography --------------------------------------------------------------
\bibliographystyle{ieeetr}
\bibliography{../bib/mathesis_collection1}

\end{document}