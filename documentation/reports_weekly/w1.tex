\documentclass[a4paper,12pt,fleqn]{scrartcl}

\usepackage[T1]{fontenc}
\usepackage{ucs}
\usepackage[utf8x]{inputenc}
\usepackage{graphicx}
\usepackage{ngerman}
\usepackage[ngerman]{babel}

% Title Page
\title{Wochenbericht KW dasdasdasdddXX}
\author{Christoph Gnip}
\date{14.4.2013}

\begin{document}
\maketitle

Start der Masterarbeit.

\section[Allgemeines]{Aufzaehlungen und Nummerierungen}

Aufsetzen der in diesem Projekt verwendeten Entwicklungsumgebung(en). Die
Entwicklungsumgebungen, die im Rahmen dieser Arbeit verwendet werden, wurden im
Pflichtenheft zu dieser Arbeit vorgestellt und diskutiert.

Folgende Umgebungen wurden in dieser Woche aufgesetzt:
Installation einer Virtuellen Maschine; Neuinstallation einer Ubuntu 12.04.
Installation  der Shark-Library;
Kompilation der Shark-Library.

Einarbeitung in den CMA-ES Algorithmus.


\section[Aufzhlungen]{Aufzaehlungen und Nummerierungen}
\begin{itemize}
  \item Ein Tagesordnungspunkt.
  \item Ein weiterer Tagesordnungspunkt.
  \item Noch ein Tagesordnungspunkt!
  \item Gemeinsames Mittagessen.% Es folgt ein beispielhaft
                                % eingeschobener Absatz

    Folgende drei Menüs stehen zur Auswahl:
       \begin{enumerate}
        \item Vegetarischer Gänsebraten an blauen Bohnen.
        \item Flusspferd mit Kroketten und Hirse-Plinsen.
        \item Getreidekeimling mit Hefen in Ingwerbrot,
          Buchweizengrütze mit gehobelten Perserklee und

          -- als Wunderwaffe für den Älteren Herren --
          flambierte Kürbiskerne.
       \end{enumerate}
  \item Letzter Tagesordnungspunkt.
  \item Wecken und Verabschiedung.
\end{itemize}

\end{document}
s