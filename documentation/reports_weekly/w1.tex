\documentclass[a4paper,12pt,fleqn]{scrartcl}
\usepackage[T1]{fontenc}
\usepackage{ucs}
\usepackage[utf8x]{inputenc}
\usepackage{ngerman}
\usepackage[ngerman]{babel}
\usepackage{lastpage}
\usepackage[pdftex]{color,graphicx}
\usepackage{listings}
\usepackage{pdflscape}
\usepackage{longtable}
\usepackage[inner=2cm,outer=2cm,top=1cm,bottom=1.5cm,includeheadfoot]{geometry}
\usepackage{fancyhdr}

%---- Title
\title{Wochenbericht KW 17}
\author{Christoph Gnip}
\date{ 22.4.-28.4.2013 }
%---- End Title

%---- PageLayout
\pagestyle{fancy}

\setlength{\headsep}{10mm}

\usepackage{eso-pic}

%----------------------------------------------------------------------------
% HEADER --------------------------------------------------------------------
%----------------------------------------------------------------------------
\fancyhead[R]{
  \includegraphics[width=100pt,keepaspectratio]{img/amedo2012.png}
}

\fancyhead[C]{ Wochenbericht KW17 }

\fancyhead[L]{
  \begin{tabular}[b]{l}
  Christoph Gnip\\
  Projekt: RFID-Evolution
  \end{tabular}
}

%Linie oben
\renewcommand{\headrulewidth}{0.5pt}
%----------------------------------------------------------------------------


%----------------------------------------------------------------------------
%----------------------------------------------------------------------------
%----------------------------------------------------------------------------
\fancyfoot[L]{Stand: \today}
\fancyfoot[R]{\thepage{} von \pageref{LastPage}}

% Linie unten
\renewcommand{\footrulewidth}{0.5pt}
%----------------------------------------------------------------------------

% Import Macros  ------------------------------------------------------------
%--------------------------------------------------------------
%--------------------------------------------------------------
%--------------------------------------------------------------
\newcommand\confidentialoverlay{
  % Taken from the TikZ documentation.
  % NB: This requires \usepackage{tikz}!
  \begin{tikzpicture}[remember picture,overlay]
    \node [rotate=60,scale=10,text opacity=0.1]
      at (current page.center) {Vertraulich};
  \end{tikzpicture} 
 
} 
%--------------------------------------------------------------
%--------------------------------------------------------------
%--------------------------------------------------------------
\newcommand{\myvec}[1]{\hat{\mathbf{#1}}}% Vector notation

%--------------------------------------------------------------
%- This can be used for aligning equations --------------------
%--------------------------------------------------------------
\newcommand{\phantomeq}[2]{
\begin{equation}
	\phantom{#1}
	#2
\end{equation}
}% Vector notation

%--------------------------------------------------------------
%- seraches for input in the "extern" folder ------------------
%--------------------------------------------------------------
\newcommand{\externInput}[1]{\input{extern/#1}}

%--------------------------------------------------------------
%- seraches for input in the "intern" folder ------------------
%--------------------------------------------------------------
\newcommand{\internInput}[1]{\input{intern/#1}}

%--------------------------------------------------------------
%- seraches for input in the "common" folder ------------------
%--------------------------------------------------------------
\newcommand{\commonInput}[1]{\input{common/#1}}

\newcommand{\cpp}{%
  \mbox{\emph{\textrm{C\hspace{-1.5pt}\raisebox{1.75pt}{\scriptsize +}%
  \hspace{-2pt}\raisebox{.75pt}{\scriptsize +}}}}%
}

\newcommand{\amedogmbh}{%
  amedo GmbH
}

%\renewenvironment{itemize}[1]{\begin{compactitem}#1}{\end{compactitem}}
%\renewenvironment{enumerate}[1]{\begin{compactenum}#1}{\end{compactenum}}
%\renewenvironment{description}[0]{\begin{compactdesc}}{\end{compactdesc}}


%----------------------------------------------------------------------------
% Start the Document --------------------------------------------------------
%----------------------------------------------------------------------------
\begin{document}

\setlength{\headheight}{36pt}

\begin{titlepage}

% make the title
\maketitle

%this has to be called here, after \maketitle
\thispagestyle{empty}

%----include the time table for this week
%Table generated by Excel2LaTeX from sheet 'Tabelle1'
\begin{table}[right]
    Anwesenheit
    \hfill
    \begin{tabular}{|l|l|l|l|l|l|l|}
    \textbf{Mo} & \textbf{Di} & \textbf{Mi} & \textbf{Do} & \textbf{Fr} &
\textbf{Sa} & \textbf{So} \\

    Bo    & Bo    & Bo    & Ge    & Ge    & x     & x \\
    \end{tabular}%
\end{table}%



\end{titlepage}

%- Section 1 -------------------------------------------------------------------
\section[Allgemeines]{Allgemeines}
Arbeiten an der Masterarbeit aufgenommen.
\subsection{Organisation der Arbeit}
\begin{table}[right]
    \renewcommand{\arraystretch}{1.5}
    \begin{tabular}{lp{11cm}}
      Titel der Arbeit:  [DE] & Entwicklung eines Systems zur
Entfernungsabschätzung für Phasen basiertes UHF RFID Tracking durch Verwendung
evolutionärer Berechnungsverfahren \\
      Titel der Arbeit:  [EN] & Development of a Distance Estimation System for
Phase-Based UHF RFID Tracking by Utilizing Methodes of Evolutionary Computation
\\
      Interne Projektbezeichnung: & PRPS-Evo      \\
      Zeitraum: & 22.April-2.September (20 Wochen)
    \end{tabular}
\end{table}
%
Am Donnerstag, den 18.4., fand ein Treffen mit Herrn Prof. Dr. Frank Bärmann statt. Ihm wurde die Masterthesis vorgestellt und er erklärt
sich als Erstbetreuer für die Arbeit einverstanden.
\newline
Die Anmeldung erfolgt am 2.5. Der Termin verspätet sich aufgrund der Öffnungszeiten des Prüfungsamts.


%- Section 2 ----------------------------------------------------------------
\section[Fortschritt]{Projektfortschritt}
In dieser Woche haben die Arbeiten an der Masterarbeit begonnen. Der Rahmen der Arbeit (Muss-, Soll- und Wunschkriterien)
wurde detaillierter im Pflichtenheft abgesteckt. Die Fertigstellung des Pflichtenhefts wird Anfang KW 19 erwartet.

Weiterhin wurden in dieser Woche die in diesem Projekt verwendeten Entwicklungsumgebung(en) aufgesetzt und bedarfsgerecht
installiert. Die Entwicklungsumgebungen, die im Rahmen dieser Arbeit verwendet werden, werden im Pflichtenheft ausführlicher vorgestellt und diskutiert.

Folgende Umgebungen wurden in dieser Woche aufgesetzt:

%- Section 2.1 --------------------------------------------------------------
\subsection{Projektverwaltung}
Für die Verwaltung dieser Arbeit wird das Versionsverwaltungssystem Git verwendet. Es wird mit GitHub \cite{github} webbasierter
Hosting-Dienst verwendet um die Datensicherung zu gewährleisten. Da GitHub im Rahmen einer Abschlussarbeit verwendet wird, erlaubt
es der Betreiber das Repository den öffentlich Zugang zu verweigern. Somit geht das mit der Geheimhaltung konform. Git wird bereits
in anderen Projekten der Amedo GmbH eingesetzt.

%- Section 2.2 --------------------------------------------------------------
\subsection{Entwicklung}
\begin{itemize}
  \item Installation der virtuellen Maschine, die als Produktivumgebung in
      diesem Projekt verwendet wird, als Betriebssystem wird Ubuntu in der Version 12.04 verwendet
  \item Installation aller zur Kompilierung notwendigen Software (u.A. GCC)
  \item Übersetzung und Installation der Shark-Library auf der frischen
      Linux-Maschine
\end{itemize}

%- Section 2.3 --------------------------------------------------------------
\subsection{Dokumentation}
\begin{itemize}
  \item Für die Erstellung der Dokumentation und der Thesis selbst, wird
      \LaTeX{} verwendet.

\end{itemize}

%- Section 2.4 --------------------------------------------------------------
\subsection{Projektfortschritt}
\begin{enumerate}
 \item Einarbeitung in den CMA-ES Algorithmus. Der Algorithmus wurde erfolgreich
      in Matlab implementiert und erste Versuche (Benchmarks) wurden mit verschiedenen Populationsgrößen durchgeführt.
      Die Ergebnisse des Benchmarks sind hier im Einzelnen nicht vorgestellt, da sie nur der Einarbeitung dienen.
      Ein Auszug aus dem Quellcode ist im \ Anhang~\ref{lst:cmaes-mat-code} gezeigt.
 \item Bereits in dieser Woche ist es gelungen die Beispiel-Programme der Shark-Library zu erstellen.

\end{enumerate}

\cite{KALMANandSMOOTHING}
Bereits in dieser Woche ist es gelungen die Beispiel-Programme der Shark-Library
zu erstellen.


%- Section 3 -------------------------------------------------------------------
\section[Probleme]{Probleme}
Die Einrichtung der \LaTeX{} Umgebung dauerte länger als erwartet. Es war ein
Tag Arbeit vorgesehen, aber es wurden zwei benötigt.
Unter Windows gab es Stabilitätsprobleme mit dem verwendeten Editor Kile und dem KBibTeX-Tool.
Die beobachteten Instabilitäten traten nicht in der Ubuntu-Umgebung auf. Ein entsprechender Bug-Report wurde eingereicht.

%- Appendix --------------------------------------------------------------------
\begin{appendix}

\newpage
\huge{Anhänge}
\normalsize
\thispagestyle{plain}
\section{Verwendete Abkürzungen}
\dots{}

\newpage
\thispagestyle{plain}
\section{Matlab CMA-ES Code}
\label{lst:cmaes-mat-code}[blablabla]

\newpage
\thispagestyle{plain}
\section{Tabellen}
  \label{tbl:population-compare}

%   \begin{table}[position specifier]
    \begin{table}[h]
      \centering

    \begin{tabular}{lll}
	\hline
	\textbf{Populationsgröße} & \textbf{Ausführungszeit} & \textbf{Funktion}\\\hline
	x & x & x \\
	y & y & y \\

      \end{tabular}
    \caption{Performancevergleich für unterschiedliche Populationsgrößen}
    \end{table}

\end{appendix}


\newpage
%- Bibliography ----------------------------------------------------------------
\bibliography{../bib/mathesis_collection1}{}
\bibliographystyle{plain}

\end{document}