\documentclass[a4paper,12pt,fleqn]{scrartcl}
\usepackage[T1]{fontenc}
\usepackage{ucs}
\usepackage[utf8x]{inputenc}
\usepackage{ngerman}
\usepackage[ngerman]{babel}
\usepackage{lastpage}
\usepackage[pdftex]{color,graphicx}
\usepackage{listings}

\usepackage[inner=2cm,outer=2cm,top=1cm,bottom=1.5cm,includeheadfoot]{geometry}
\usepackage{fancyhdr}

%---- Title Page
\title{Wochenbericht KW 17}
\author{Christoph Gnip}
\date{ 22.4.-28.4.2013 }

%---- End Title Page

%---- PageLayout
\pagestyle{fancy}

\setlength{\headsep}{10mm}

\usepackage{eso-pic}

% HEADER -------------------------------------------------------------
\fancyhead[R]{
  \includegraphics[width=100pt,keepaspectratio]{img/amedo2012.png}
}

\fancyhead[L]{
  \begin{tabular}[b]{l}
  Christoph Gnip\\
  Projekt: RFID-Evolution
  \end{tabular}
}

%Linie oben
\renewcommand{\headrulewidth}{0.5pt}

% HEADER -------------------------------------------------------------
%Fußzeile links bzw. innen
\fancyfoot[L]{Stand: \today}
%Fußzeile rechts bzw. außen
\fancyfoot[R]{\thepage{} von \pageref{LastPage}}
%Linie unten
\renewcommand{\footrulewidth}{0.5pt}

% Import Macros  ----------------------------------------------------------
%--------------------------------------------------------------
%--------------------------------------------------------------
%--------------------------------------------------------------
\newcommand\confidentialoverlay{
  % Taken from the TikZ documentation.
  % NB: This requires \usepackage{tikz}!
  \begin{tikzpicture}[remember picture,overlay]
    \node [rotate=60,scale=10,text opacity=0.1]
      at (current page.center) {Vertraulich};
  \end{tikzpicture} 
 
} 
%--------------------------------------------------------------
%--------------------------------------------------------------
%--------------------------------------------------------------
\newcommand{\myvec}[1]{\hat{\mathbf{#1}}}% Vector notation

%--------------------------------------------------------------
%- This can be used for aligning equations --------------------
%--------------------------------------------------------------
\newcommand{\phantomeq}[2]{
\begin{equation}
	\phantom{#1}
	#2
\end{equation}
}% Vector notation

%--------------------------------------------------------------
%- seraches for input in the "extern" folder ------------------
%--------------------------------------------------------------
\newcommand{\externInput}[1]{\input{extern/#1}}

%--------------------------------------------------------------
%- seraches for input in the "intern" folder ------------------
%--------------------------------------------------------------
\newcommand{\internInput}[1]{\input{intern/#1}}

%--------------------------------------------------------------
%- seraches for input in the "common" folder ------------------
%--------------------------------------------------------------
\newcommand{\commonInput}[1]{\input{common/#1}}

\newcommand{\cpp}{%
  \mbox{\emph{\textrm{C\hspace{-1.5pt}\raisebox{1.75pt}{\scriptsize +}%
  \hspace{-2pt}\raisebox{.75pt}{\scriptsize +}}}}%
}

\newcommand{\amedogmbh}{%
  amedo GmbH
}

%\renewenvironment{itemize}[1]{\begin{compactitem}#1}{\end{compactitem}}
%\renewenvironment{enumerate}[1]{\begin{compactenum}#1}{\end{compactenum}}
%\renewenvironment{description}[0]{\begin{compactdesc}}{\end{compactdesc}}


\begin{document}
\setlength{\headheight}{36pt}
%- the Title page --------------------------------------------------------
\begin{titlepage}
% \AddToShipoutPicture*{\BackgroundPic}

% make the title
\maketitle

%this has to be called here, after \maketitle
\thispagestyle{empty}

%----include the time table for this week
%Table generated by Excel2LaTeX from sheet 'Tabelle1'
\begin{table}[right]
    Anwesenheit
    \hfill
    \begin{tabular}{|l|l|l|l|l|l|l|}
    \textbf{Mo} & \textbf{Di} & \textbf{Mi} & \textbf{Do} & \textbf{Fr} &
\textbf{Sa} & \textbf{So} \\

    Bo    & Bo    & Bo    & Ge    & Ge    & x     & x \\
    \end{tabular}%
\end{table}%



\end{titlepage}
%-------------------------------------------------------------------------------
% the document
%-------------------------------------------------------------------------------
%- Section 1 -------------------------------------------------------------------
\section[Allgemeines]{Allgemeines}
Arbeiten an der Masterarbeit aufgenommen.
\subsection{Organisation der Arbeit}
\begin{table}[right]
    \renewcommand{\arraystretch}{1.5}
    \begin{tabular}{lp{11cm}}
      Titel der Arbeit:  [DE] & Entwicklung eines Systems zur
Entfernungsabschätzung für Phasen basiertes UHF RFID Tracking durch Verwendung
evolutionärer Berechnungsverfahren \\
      Titel der Arbeit:  [EN] & Development of a Distance Estimation System for
Phase-Based UHF RFID Tracking by Utilizing Methodes of Evolutionary Computation
\\
      Interne Projektbezeichnung: & PRPS-Evo      \\
      Zeitraum: & 22.April-2.September (20 Wochen)
    \end{tabular}
\end{table}
%
Am Donnerstag, den 18.4., fand ein Treffen mit Herrn Prof.
Dr. Frank Bärmann statt. Ihm wurde die Masterthesis vorgestellt und er erklärt
sich als Erstbetreuer für die Arbeit einverstanden.
\newline
%
Die Anmeldung erfolgt am 25.4.
\newline
%
Aufsetzen der in diesem Projekt verwendeten Entwicklungsumgebung(en). Die
Entwicklungsumgebung, die im Rahmen dieser Arbeit verwendet werden, wurden im
Pflichtenheft zu dieser Arbeit vorgestellt und diskutiert.
\newline
%
Folgende Umgebungen wurden in dieser Woche aufgesetzt:
\begin{itemize}
  \item Installation der virtuellen Maschine, die als Produktivumgebung in
diesem
Projekt verwendet wird, als Betriebssystem wird Ubuntu 12.04 verwendet
  \item Übersetzung und Installation der Shark-Library auf der frischen
  Linux-Maschine
  \item Für die Erstellung der Dokumentation und der Thesis selbst, wird
\LaTeX{} verwendet.
  Dazu wurde die Umgebung aufgesetzt. Sie steht sowohl auf der Linux-Maschine
als auch auf dem Windowshost zur Verfügung.
\end{itemize}

%- Section 2 -------------------------------------------------------------------
\section[Fortschritte]{Fortschritte}
\begin{enumerate}
 \item Einarbeitung in den CMA-ES Algorithmus. Der Algorithmus wurde erfolgreich
in Matlab
      implementiert und erste Versuche (Benchmarks) wurden mit verschiedenen Populationsgrößen durchgeführt. Die Ergebnisse des Benchmarks sind im \ref{tbl:population-compare} zu finden.
      Ein Auszug aus dem Quellcode ist im \ Anhang~\ref{lst:cmaes-mat-code} gezeigt.
 \item Bereits in dieser Woche ist es gelungen die Beispiel-Programme der Shark-Library zu erstellen.

%  \caption{Close-up of a gull}
%   \label{gull}
% \end{figure}
% Figure~\ref{gull} shows a photograph of a gull.

\end{enumerate}

\cite{KALMANandSMOOTHING} Bereits in dieser Woche ist es gelungen die Beispiel-Programme der Shark-Library
zu erstellen.

%- Section 3 -------------------------------------------------------------------
\section[Probleme]{Probleme}
Die Einrichtung der \LaTeX{} Umgebung dauerte länger als erwartet. Es war ein
Tag Arbeit
vorgesehen, aber 2 benötigt.
Unter Windows gab es Stabilitätsprobleme mit dem verwendeten Editor Kile und dem KBibTeX-Tool. Die beobachteten instabilitäten traten nicht in der Ubuntu-Umgebung auf.
Ein entsprechender Bug-Report wurde eingereicht.

%- Appendix --------------------------------------------------------------------
\begin{appendix}

\newpage
\huge{Anhänge}
\normalsize
\thispagestyle{plain}
\section{Verwendete Abkürzungen}
\dots{}

\newpage
\thispagestyle{plain}
\section{Matlab CMA-ES Code}
\label{lst:cmaes-mat-code}[blablabla]

\newpage
\thispagestyle{plain}
\section{Tabellen}
  \label{tbl:population-compare}

%   \begin{table}[position specifier]
    \begin{table}[h]
      \centering

    \begin{tabular}{lll}
	\hline
	\textbf{Populationsgröße} & \textbf{Ausführungszeit} & \textbf{Funktion}\\\hline
	x & x & x \\
	y & y & y \\

      \end{tabular}
    \caption{Performancevergleich für unterschiedliche Populationsgrößen}
    \end{table}

\end{appendix}


\newpage
%- Bibliography ----------------------------------------------------------------
\bibliography{../bib/mathesis_collection1}{}
\bibliographystyle{plain}

\end{document}