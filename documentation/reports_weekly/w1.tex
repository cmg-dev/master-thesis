\documentclass[a4paper,10pt,fleqn]{scrartcl}

\usepackage[T1]{fontenc}
\usepackage{ucs}
\usepackage[utf8x]{inputenc}
\usepackage{graphicx}
\usepackage{ngerman}
\usepackage[ngerman]{babel}

% Title Page
\title{Wochenbericht KW 16}
\author{Christoph Gnip}
\date{ 15.4.-21.4.2013 }

\begin{document}
\maketitle

%Table generated by Excel2LaTeX from sheet 'Tabelle1'
\begin{table}[right]
    Anwesenheit
    \hfill
    \begin{tabular}{|l|l|l|l|l|l|l|}
    \textbf{Mo} & \textbf{Di} & \textbf{Mi} & \textbf{Do} & \textbf{Fr} &
\textbf{Sa} & \textbf{So} \\

    Bo    & Bo    & Bo    & Ge    & Ge    & x     & x \\
    \end{tabular}%
\end{table}%



\section[Allgemeines]{Allgemeines}

Arbeiten an der Masterarbeit aufgenommen. Projektnummernzuweisung:
\{PROJEC TNUMBER\}.

Am Donnerstag fand ein Treffen mit dem Erstbetreuer der Masterarbeit Herrn Prof.
Dr. Frank Bärmann statt. Herr Bärmann erklärt sich als Erstbetreuer für die
Arbeit einverstanden.

Aufsetzen der in diesem Projekt verwendeten Entwicklungsumgebung(en). Die
Entwicklungsumgebung, die im Rahmen dieser Arbeit
verwendet werden, wurden im
Pflichtenheft zu dieser Arbeit vorgestellt und diskutiert.
\newline
Folgende Umgebungen wurden in dieser Woche aufgesetzt:
\begin{itemize}
 \item Installation der virtuellen Maschine, die als Produktivumgebung in
diesem Projekt verwendet wird, als Betriebssystem wird Ubuntu 12.04 verwendet
 \item Übersetzung und Installation der Shark-Library auf der frischen
Linux-Maschine
 \item Für die Erstellung der Dokumentation/ Thesis wird \LaTeX verwendet.
Dazu wurde die Umgebung aufgesetzt. Sie steht sowohl auf der Linux-Maschine als
auch auf dem Windowshost zur Verfügung.
\end{itemize}


\section[Fortschritte]{Fortschritte}
\begin{enumerate}
 \item Einarbeitung in den CMA-ES Algorithmus. Der Algorithmus wurde in Matlab
implementiert und erfolgreich ausgeführt. Ein Auszug aus dem Quellcode ist im
{Anhang A} gezeigt.
 \item

\end{enumerate}

Bereits in dieser Woche ist es gelungen die ersten kleineren Programme mit der
Shark-Library zu erstellen.

\section[Probleme]{Probleme}

Keine Probleme in dieser Woche.

\end{document}