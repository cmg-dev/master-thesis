\documentclass[a4paper,12pt,fleqn]{article}
\usepackage[T1]{fontenc}
\usepackage{ucs}
\usepackage[utf8x]{inputenc}
\usepackage{ngerman}
\usepackage[ngerman]{babel}
\usepackage{lastpage}
\usepackage[pdftex]{color,graphicx}
\usepackage{listings}
\usepackage{pdflscape}
\usepackage{longtable}
\usepackage[inner=2cm,outer=2cm,top=1cm,bottom=1.5cm,includeheadfoot]{geometry}
\usepackage{fancyhdr}
\usepackage{url}
\usepackage{draftwatermark}
\usepackage{booktabs}
\usepackage{blindtext} 
\usepackage{framed} 
\usepackage{xcolor} 
\colorlet{shadecolor}{black} 

\usepackage{enumitem}

\SetWatermarkText{Vertraulich}
\SetWatermarkScale{4}
\SetWatermarkLightness{0.9}

\usepackage{pgfgantt}
\usepackage{amsmath,amssymb,amsfonts,amstext}
\usepackage{floatflt}
\usepackage{tikz}
\usetikzlibrary{arrows, snakes}
\usetikzlibrary{through}
\usetikzlibrary{calc}

% highlighting
\usepackage{xcolor,soul}

%---- PageLayout
\pagestyle{fancy}

\setlength{\headsep}{10mm}

\usepackage{eso-pic}

%----------------------------------------------------------------------------
% HEADER --------------------------------------------------------------------
%----------------------------------------------------------------------------
\fancyhead[R]{
  \includegraphics[width=100pt,keepaspectratio]{img/amedo2012.png}
}

\fancyhead[C]{ Wochenbericht KW 22 }

\fancyhead[L]{
  \begin{tabular}[b]{l}
  Christoph Gnip\\
  Projekt: PRPS-Evolution
  \end{tabular}
}

%Linie oben
\renewcommand{\headrulewidth}{0.5pt}
%----------------------------------------------------------------------------


%----------------------------------------------------------------------------
%----------------------------------------------------------------------------
%----------------------------------------------------------------------------
\fancyfoot[L]{Stand: \today}
\fancyfoot[C]{ EXTERN }
\fancyfoot[R]{\thepage{} von \pageref{LastPage}}

% Linie unten
\renewcommand{\footrulewidth}{0.5pt}
%----------------------------------------------------------------------------

% Import Macros  ------------------------------------------------------------
%--------------------------------------------------------------
%--------------------------------------------------------------
%--------------------------------------------------------------
\newcommand\confidentialoverlay{
  % Taken from the TikZ documentation.
  % NB: This requires \usepackage{tikz}!
  \begin{tikzpicture}[remember picture,overlay]
    \node [rotate=60,scale=10,text opacity=0.1]
      at (current page.center) {Vertraulich};
  \end{tikzpicture} 
 
} 
%--------------------------------------------------------------
%--------------------------------------------------------------
%--------------------------------------------------------------
\newcommand{\myvec}[1]{\hat{\mathbf{#1}}}% Vector notation

%--------------------------------------------------------------
%- This can be used for aligning equations --------------------
%--------------------------------------------------------------
\newcommand{\phantomeq}[2]{
\begin{equation}
	\phantom{#1}
	#2
\end{equation}
}% Vector notation

%--------------------------------------------------------------
%- seraches for input in the "extern" folder ------------------
%--------------------------------------------------------------
\newcommand{\externInput}[1]{\input{extern/#1}}

%--------------------------------------------------------------
%- seraches for input in the "intern" folder ------------------
%--------------------------------------------------------------
\newcommand{\internInput}[1]{\input{intern/#1}}

%--------------------------------------------------------------
%- seraches for input in the "common" folder ------------------
%--------------------------------------------------------------
\newcommand{\commonInput}[1]{\input{common/#1}}

\newcommand{\cpp}{%
  \mbox{\emph{\textrm{C\hspace{-1.5pt}\raisebox{1.75pt}{\scriptsize +}%
  \hspace{-2pt}\raisebox{.75pt}{\scriptsize +}}}}%
}

\newcommand{\amedogmbh}{%
  amedo GmbH
}

%\renewenvironment{itemize}[1]{\begin{compactitem}#1}{\end{compactitem}}
%\renewenvironment{enumerate}[1]{\begin{compactenum}#1}{\end{compactenum}}
%\renewenvironment{description}[0]{\begin{compactdesc}}{\end{compactdesc}}


%----------------------------------------------------------------------------
% Start the Document --------------------------------------------------------
%----------------------------------------------------------------------------
\begin{document}

\setlength{\headheight}{36pt}

\begin{titlepage}


%- the Title page --------------------------------------------------------
\begin{center}
%\vspace*{2.5cm}
{\Huge \textbf{Wochenbericht KW 22}\par}
\vspace{1cm}
{\Huge 27.5. - 2.6.2013\par}
\vspace{1cm}
{\Huge Projektwoche: 6\par}

\vspace{2cm}

\large{Erstellt durch}\\
\Large{\textbf{Christoph Gnip}}


%\vspace{4cm}
\vfill

{\normalsize Fachbereich Elektrotechnik und angewandte Naturwissenschaften\\
Westfälische Hochschule\\[2ex]Mai 2013}

\end{center}
\newpage

\end{titlepage}

%- Section 1 ----------------------------------------------------------------
\section[Allgemeines]{Allgemeines}
Der Projektplan wurde angepasst um dem aktuellen Projektverlauf besser zu entsprechen. Einige Arbeiten wurden vorgezogen und ein paar neue Pakete kommen hinzu. Der Plan ist im Anhang~\ref{sec:projectplan} zu finden.
%
%- Section 2 ----------------------------------------------------------------
\section[Fortschritt]{Projektfortschritt}
Auf Basis des in der letzten Woche entwickelten Modells konnte bereits unter Verwendung des Excel Solver eine Bestimmung der Wellenzahl durchgeführt werden. Das Ergebnis des Excel Solvers ist eine Entfernung, basierend auf den von dem PRPS-System gemessenen Phasenwerten als Eingangsparametern. Die ersten Versuche waren vielversprechend und es konnte bereits eines Abweichung von $~5\%$ erreicht werden.
%
%- Section 2.2 --------------------------------------------------------------
\subsection{Matrix Konditionierung}
Im Laufe der ersten Experimente mit dem Solver und dem Modell der letzten Woche lieferten manche Konfigurationen von Antennen wesentliche bessere Ergebnisse als andere. So war z.B. die Konfiguration aus den Antennen $\{0,2,3,4\}$ die mit dem geringsten Fehler ($\sim5\%$). Andere Konfigurationen lieferten ein schlechteres Ergebnis mit einem Fehler von $\sim20-30\%$.\\
Es gilt nun herauszufinden, woher die Unterschiede stammen. Nach den Arbeiten dieser Woche, kann bereits eine mögliche Herkunft angegeben werden. Da die Konfiguration der Antennen in der Koeffizientenmatrix $\mathbf{A}$ des Modells eingeht, scheint es sinnvoll eine Betrachtung des Fehlerfortpflanzung zu erstellen. Das lineare Gleichungssystem ermöglicht dies relativ einfach durch die Bestimmung der Konditionszahl des Matrix (Details siehe Anhang~\ref{seq:ConditionNumber}).\\
Wie in Anhang~\ref{seq:ConditionNumber} beschrieben ist die Berechnung nicht trivial und die Implementation wird in der nächsten Woche durchgeführt.
%
%- Section 3 -----------------------------------------------------------------
\section{Probleme}
\label{sec:Problems}

%- Appendix ------------------------------------------------------------------
%
%
%
%

\begin{appendix}

%----------------------------------------------------------------------------
%----------------------------------------------------------------------------
%----------------------------------------------------------------------------
\newpage

\begin{center}
	\huge{Anhänge}
\end{center}

\normalsize

%----------------------------------------------------------------------------
%----------------------------------------------------------------------------
%----------------------------------------------------------------------------
\section{Konditionszahl eine Matrix}
\label{seq:ConditionNumber}
{
\small
Folgende Nomenklatur und Symbole gelten für diesen Abschnitt:
\begin{itemize}[itemsep=0mm]
	\item	fette Großbuchstaben stehen für Matrizen (bspw. $\mathbf{A}$)
	\item	fette Kleinbuchstaben stehen für Vektoren (bspw. $\mathbf{x}$)
	\item	$\mathbf{0} := \text{Nullvektor}$)
\end{itemize}
%
Gegeben ist ein lineares Gleichungssystem der Form:
$$ \mathbf{A}\mathbf{x}-\mathbf{b} =\mathbf{0} $$
Eine numerische Lösung für in der Regel zu einer von $\mathbf{0}$ verschiedenen Lösung so das wir:
$$ \mathbf{A}\mathbf{\tilde{x}}-\mathbf{b} =\mathbf{r} $$
schreiben. Man nennt $\mathbf{r}$ den Residuumvektor. Es ist offensichtlich, dass ein kleines Residuum nicht hinreichend ist um von einem kleinen relaitven Fehler auszugehen.\\
Aus $\mathbf{A}\mathbf{x}-\mathbf{b} =\mathbf{0}$ und $\mathbf{A}\mathbf{\tilde{x}}-\mathbf{b} =\mathbf{r}$ folgt $$ \mathbf{A}\Delta\mathbf{x}=\mathbf{r}$$
und damit:
$ 
\lVert \mathbf{b} \rVert=\lVert \mathbf{Ax} \rVert \leq \lVert \mathbf{A} \rVert \lVert \mathbf{x} \rVert
$, 
$
\lVert \Delta\mathbf{x} \rVert=\lVert -\mathbf{A^{-1}r} \rVert \leq \lVert \mathbf{A^{-1}} \rVert \lVert \mathbf{r} \rVert
$
Wir können nun für den relativen Fehler schreiben:
$$
\frac{\lVert \Delta\mathbf{x} \rVert}{\lVert \mathbf{x} \rVert} \leq 
\frac{\lVert \mathbf{A^{-1}} \rVert \lVert \mathbf{r} \rVert}{\lVert \mathbf{b} \rVert / \lVert \mathbf{A} \rVert} =
\lVert \mathbf{A} \rVert \lVert \mathbf{A^{-1}} \rVert \frac{\lVert \mathbf{r} \rVert}{\lVert \mathbf{b} \rVert}
$$
Der Term $\lVert \mathbf{A} \rVert \lVert \mathbf{A^{-1}} \rVert := \text{cond}(\mathbf{A})$ heißt Konditionszahl. Auch der Begriff Konditionsmaß ist gebräuchlich und bezieht sich auf die gewählte Marixnorm.
Es kann gezeigt werden, dass $\text{cond}(\mathbf{A}) \gg 1$  für eine schlechte Konditionierung der Matrix steht. Wird im Folgenden von einer speziellen Matrixnorm gesprochen schreiben wir $\text{cond}(\mathbf{A})$ zu 
$$ 
\text{cond}_k(\mathbf{A}) = \lVert \mathbf{A} \rVert_k \lVert \mathbf{A^{-1}} \rVert_k
$$ \\

\textbf{Konditionszahl der Spektralnorm}\\
%
Für die Spektralnorm einer Matrix berechnet sich die Konditionszahl zu:
$$ 
\text{cond}_2(\mathbf{A}) = \lVert \mathbf{A} \rVert_2 \lVert \mathbf{A^{-1}} \rVert_2=
\sqrt{\frac{\mu_{max}}{\mu_{min}}}
$$
Nach \{\} kann man folgende Aussage über die Konditionszahl treffen:\\
"Wird ein lineares Gleichungsysten $Ax=b$ mit $t$-stelliger dezimaler Gleitpunktarithmetik gelöst und beträgt die Konditionszahl $\text{cond}(A) \approx10^\alpha$, so sind auf Grund der im allgemeinen unvermeidbaren Fehler in den Eingabedaten $A$ und $b$ nur $t-\alpha-1$ Dezimalstellen der berechneten Lösung $\tilde{x}$ (bezogen auf die betragsgrößte Komponente) sicher."
%\begin{equation}\label{eq:final_trilateration_model}
%0=   \\m\ma\mat\math\mathb\mathbf
%\left(
%	\begin{array}{ccc}
%		x_k-x_0 & y_k-y_0 & z_k-z_0 
%	\end{array}
%\right)
%\left(
%   \begin{array}{c}
%	   x-x_0\\
%	   y-y_0\\
%	   z-z_0
%   \end{array}
%\right)
%-
%\left(
%	\begin{array}{c}
%		c_{k0}
%	\end{array}
%\right) 
%\end{equation}
%%
%Dabei ist:
%\begin{equation}\label{eq:c_k0}
%	c_{k0}=\frac{1}{2}[d_{k0}^2+r_{0}^2-r_k^2]
%\end{equation}
%%
%Ziel dieser Erweiterung ist es, einen Zusammenhang zwischen diesem Modell und der Wellenzahl zu erzeugen. Folgender Ansatz wird gewählt:
%	\begin{equation}\label{eq:r_0_theta} r(\Theta)=\frac{\lambda}{2}\left(\frac{\Theta}{2\pi}+n\right),\\\lambda=\frac{c}{f}, n:= \text{Wellenzahl}
%\end{equation}
%%
%%
%Weiterhin ist $\Theta$ die gemessene Phase, die das PRPS-System liefert und $n$ die gesuchte Wellenzahl.\\
%Durch einsetzen von \eqref{eq:r_0_theta} in \eqref{eq:c_k0}, erhalten wir:
%\begin{equation}\label{eq:c_k0_extended}
%	c_{k0}(\Theta_0, \Theta_k, n_0, n_k) =\frac{1}{2}\left[d_{k0}^2+\frac{\lambda^2}{4}\left(\frac{\Theta_0}{2\pi}+n_0\right)^2-\frac{\lambda^2}{4}\left(\frac{\Theta_k}{2\pi}+n_k\right)^2\right]
%\end{equation}
%%
%Wir stellen Gleichung~\eqref{eq:c_k0_extended} um:
%\begin{align}
%%	
%	c_{k0}(\Theta_0, \Theta_k, n_0, n_k) &= \frac{1}{2}\left\{d_{k0}^2+\frac{\lambda^2}{4}\left[\left(\frac{\Theta_0}{2\pi}\right)^2+2\frac{\Theta_0}{2\pi}n_0+n_0^2 \right.\right.\nonumber\\
%	&\phantom{=}\; 
%	\left.\left.-\left(\frac{\Theta_k}{2\pi}\right)^2-2\frac{\Theta_k}{2\pi}n_k-n_k^2\right]\right\}\\
%%    
%    &=\frac{1}{2}\left\{d_{k0}^2+\frac{\lambda^2}{4}\left[\left(\frac{\Theta_0}{2\pi}\right)^2-\left(\frac{\Theta_k}{2\pi}\right)^2 \right.\right.\nonumber\\
%    &\phantom{=}\;
%   	\left.\left.+2\frac{\Theta_0}{2\pi}n_0-2\frac{\Theta_k}{2\pi}n_k+n_0^2-n_k^2\right]\right\}\\
%%	
%	&=\frac{1}{2}d_{k0}^2+\frac{\lambda^2}{8}\left[\frac{1}{(2\pi)^2}\left(\Theta_0^2-\Theta_k^2\right) \right.\nonumber\\
%	&\phantom{=}\;
%	\left. +\frac{1}{\pi}\left(\Theta_0n_0-\Theta_kn_k\right)+\left(n_0^2-n_k^2\right)\right]\label{c_k0_rearragend}
%\end{align}
%%
%Führen wir nun:
%\phantomeq{c_{k0}(\Theta_0, \Theta_k, n_0, n_k)}{a_{0k} := \frac{1}{2}d_{k0}^2\nonumber}
%\phantomeq{c_{k0}(\Theta_0, \Theta_k, n_0, n_k)}{a_1 := \frac{\lambda^2}{8}\nonumber}
%\phantomeq{c_{k0}(\Theta_0, \Theta_k, n_0, n_k)}{a_2 := a_1\frac{1}{\pi}\nonumber}
%\phantomeq{c_{k0}(\Theta_0, \Theta_k, n_0, n_k)}{a_{3k0} := a_1\frac{1}{(2\pi)^2}(\Theta_0^2-\Theta_k^2)\nonumber}
%%
%in Gleichung~\eqref{c_k0_rearragend} ein, erhalten die finale Form der Gleichung:
%\begin{equation}
%c_{k0}(\Theta_0, \Theta_k, n_0, n_k) = a_{0k}+a_1(n_0^2-n_k^2)+a_2(\Theta_0n_0-\Theta_kn_k)-a_{3k0}\label{c_k0_final_form}   
%\end{equation}
%%
%Die Einführung der Konstanten macht zum Einen die Gleichung übersichtlicher. Zum Anderen können so, mit Blick auf eine spätere Softwareimplementation, Rechenschritte gespart werden. Das sollte sich positiv auf den späteren Berechnungsaufwand auswirken.\\
%%
%Im Weiteren erkennt man durch scharfes hinsehen das in Gleichung~\eqref{c_k0_final_form}, für $\Theta_k=\text{const.}$ \& $\Theta_0=\text{const.}$ gilt. Das resultiert aus der Tatsache, dass . Es ermöglicht uns zu schreiben:
%\begin{equation}
%c_{k0}(\Theta_0, \Theta_k, n_0, n_k) = c_{k0}(n_0, n_k)
%\end{equation}
%%
%Im engeren Sinne einer mathematischen Funktion sollten wir die Parameter alle als Argument aufnehmen. Diese Form soll darstellen, welche Größen von Interesse sind. Im späteren Gebrauch wird diese Gleichung in der Optimierung eingesetzt werden.
%Für unser Gleichungssystem aus\eqref{eq:final_trilateration_model} ergibt sich:
%\begin{equation}\label{eq:wavenumber_trilateration_model}
%0=
%\left(
%	\begin{array}{ccc}
%		x_k-x_0 & y_k-y_0 & z_k-z_0 
%	\end{array}
%\right)
%\left(
%   \begin{array}{c}
%	   x-x_0\\
%	   y-y_0\\
%	   z-z_0
%   \end{array}
%\right)
%-
%\left(
%	\begin{array}{c}
%		c_{k0}(n_0, n_k)
%	\end{array}
%\right)
%\end{equation}
%%
%Betrachten wir nun \eqref{eq:wavenumber_trilateration_model} und setzen $N'=4$, d.h. wir verwenden 4 Antennen. Wir beschreiben die Konfiguration wie folgt: Antenne 0 ist die Referenz-Antenne und Antenne 0-3 sind Messwertgeber für die Phaseninformation. 
%%
%\begin{equation}\label{eq:wavenumber_trilateration_model_explicit}
%0=
%\underbrace{\left(
%	\begin{array}{ccc}
%		x_1-x_0 & y_1-y_0 & z_1-z_0 \\
%		x_2-x_0 & y_2-y_0 & z_2-z_0 \\
%		x_3-x_0 & y_3-y_0 & z_3-z_0 
%	\end{array}
%\right)}_{\textbf{A}}
%\underbrace{\left(
%   \begin{array}{c}
%	   x-x_0\\
%	   y-y_0\\
%	   z-z_0
%   \end{array}
%\right)}_{\textbf{x}}
%-
%\underbrace{\left(
%	\begin{array}{c}
%		c_{10}(n_0, n_1) \\
%		c_{20}(n_0, n_2) \\
%		c_{30}(n_0, n_3)
%	\end{array}
%\right)}_{\textbf{b}}
%\end{equation}
%%
%\begin{equation}
%\mathbf{b}=
%\left(
%	\begin{array}{c}
%		a_{01}+a_1( n_0^2-n_1^2)+a_2(\Theta_0n_0-\Theta_1n_1)-a_{310} \\
%		a_{02}+a_1(n_0^2-n_2^2)+a_2(\Theta_0n_0-\Theta_2n_2)-a_{320} \\
%		a_{03}+a_1(n_0^2-n_3^2)+a_2(\Theta_0n_0-\Theta_3n_3)-a_{330}
%	\end{array}
%\right)
%\end{equation}
%%
%Das Ergebnis ist ein um $\Theta$ und $n$ erweitertes Gleichungssystem. Zusätzlich enthält  es mehrere geometrische Konstanten ($a_{0k}, k=\{1,..,N-1\}$), mehrere Phasen-Konstanten ($a_{3k0}, k=\{1,..,N-1\}$), sowie zwei allgemeine ($a_1$ und $a_2$). Allgemeiner formuliert ergibt sich:
%%
%\begin{multline}\label{eq:final_equation}
%0=
%\left(
%	\begin{array}{ccc}
%		x_k-x_0 & y_k-y_0 & z_k-z_0 
%	\end{array}
%\right)
%\left(
%   \begin{array}{c}
%	   x-x_0\\
%	   y-y_0\\
%	   z-z_0
%   \end{array}
%\right) \\
%-
%\left(
%	\begin{array}{c}
%		a_{0k}+a_1(n_0^2-n_k^2)+a_2(\Theta_0k_0-\Theta_kn_k)-a_{3k0}
%	\end{array}
%	\right)
%\end{multline}
%%
%Aus Gleichung~\eqref{eq:final_equation} ist durch eine geeignete Wahl von $N'=\{4,..,N\}$ sofort ersichtlich wie viele Veränderliche sich für eine gewählte Konstellation an Antennen ergeben. Für $k$ gilt in diesem Fall $k=\{1,..,N'-1\}$.\\
%%
%Beispielsweise ergibt sich für das Modell aus Gleichung~\eqref{eq:final_equation} mit $N'=4$, insgesamt 7 Variablen ($\mathbf{x},n_0,n_1,n_2,n_3$) . Analog würde sich für ein Modell mit allen 8 Antennen, 11 Variablen ($\mathbf{x},n_0,..,n_7$) ergeben.
}

%----------------------------------------------------------------------------
%----------------------------------------------------------------------------
%----------------------------------------------------------------------------
\newpage

%- Section 1 ----------------------------------------------------------------
\section{Evolutionsstrategien}
\label{seq:EvolutionaryStrategies}
Folgende Information entstammen im Wesentlichen aus \cite{kost2003optimierung},\cite{bronstejn2012taschenbuch}\&\cite{Hansen:1} und sind auf den folgenden Seiten lediglich zusammengefasst und neu arrangiert um eine Einarbeitung in die Thematik zu ermöglichen. 

%- Section .1 -----------------------------------------------------------------
\subsection{Allgemeines}
Nach dem Vorbild natürlicher Evolution entworfene stochastische Optimierungsverfahren werden Evolutionsstrategie bezeichnet. Sie verwenden die Prinzipien der Mutation, Rekombination und Selektion analog zu der nat. Evolution.\\
Wie in der Natur auch werden Nachkommen aus der Menge der verfügbaren Eltern gebildet. Dabei bezeichnet im Folgenden:
\begin{itemize}
\item $\mu$ die Anzahl der Eltern (=> Größe der Population)
\item $\lambda$ die Anzahl der Eltern die bei Rekombination neue Kinder erzeugt; Die Anzahl der erzeugten Nachkommen einer neuen Generation
\item $\mathbf{x}_p$ Elternpunkt (Parent)
\item $\mathbf{x}_c$ Nachkomme einer Generation (Child)
\item $X_p^1$ Die Menge aller Eltern der ersten Generation $X_p = \{\mathbf{x}_{p_1}^1,..,\mathbf{x}_{p_\mu}^1\}$
\item $X_p^k$ Die Menge aller Eltern der k-ten Generation $X_p = \{\mathbf{x}_{p_1}^k,..,\mathbf{x}_{p_\mu}^k\}$
\end{itemize}
\textit{Anmerkung: Die Verwendung des Sysmbols $\lambda$ ist in diesem Kontext nicht eindeutig und wird im Späteren durch ein geeigneteres Symbol im Rahmen dieser Arbeit ersetzt werden.}

%
%- Section .2 -----------------------------------------------------------------
\subsection[Mutation]{Mutation}
Ein Nachkomme $\mathbf{x}_C$ wird aus seinem Elternteil $\mathbf{x}_P$ und einer zufälligen Variation $\mathbf{d}$ gebildet.
\begin{equation} \label{eq:Mutation_Child}
	\mathbf{x}_c = \mathbf{x}_P + \mathbf{d}
\end{equation}
Dabei ist $\mathbf{d}$ ein bei jeder Mutation neu zu bestimmender $(0,\sigma^2)-normalverteilte$ Zufallszahl $Z(0,\sigma^2)$:
\begin{equation}\label{eq:wavenumber_trilateration_model}
\mathbf{d}=
\left(
	\begin{array}{c}
		d_1 \\
		\vdots\\
		d_n 
	\end{array}
\right)
=
\left(
	\begin{array}{c}
		Z(0,\sigma_1^2) \\
		\vdots\\
		Z(0,\sigma_n^2) 
	\end{array}
\right)
=
\left(
	\begin{array}{c}
		Z(0,1) \sigma_1 \\
		\vdots\\
		Z(0,1) \sigma_n 
	\end{array}
\right)
\end{equation}
%
Die Normalverteilung der Variation ist nützlich, da kleine Änderungen wahrscheinlicher sind als große. Die maximale Größe der Variation wird durch die Standardabweichung $\sigma_i$ bestimmt.
%
%- Section .3 -----------------------------------------------------------------
\subsection[Rekombination]{Rekombination}
Durch Rekombination zweier oder mehr Eltern aus der Menge aller $\mu$-Eltern $X_{\varrho} \subset X_E$. Die Wahl der Eltern sollte zufällig erfolgen um Inzuchtprobleme zu verhindern.\\
Zwei Arten der Rekombination sind denkbar:\\

Die \textit{intermediär Rekombination} erstellt einen Nachkommen durch das gewichtete Mittel von $\varrho$ Eltern.
%
\begin{equation}
\mathbf{x}_c = \Sigma^\varrho_{i=1} \alpha_i\mathbf{x}_{p_i},\\ \Sigma^\varrho_{i=1} \alpha_i = 1,\\ 2\leq\varrho\leq\mu
\end{equation} 
%

Bei der \textit{diskreten Rekombination} vom $\varrho$-Eltern wird die \textit{i}-te Komponente $x_{ic}$ eines Nachkommen $\mathbf{x}_c$ mit der \textit{i}-te Komponente eines zufällig gewählten Elternpunktes gleichgesetzt.
%
\begin{equation}
\mathbf{x}_{ic} = \mathbf{x}_{ip_j},\\ j\in\{1,...,\varrho\},\\i=1,...,n
\end{equation} 
%
%- Section .4 -----------------------------------------------------------------
\subsection[Selektion]{Selektion}
Die durch Rekombination und/oder Mutation erzeugten Nachkommen werden in dem Schritt Ausgewählt um einen Evolutionsfortschritt zu erreichen. Dies erfolgt anhand des Vergleichs mit dem Zielfunktionswert $f(\mathbf{x})$. Das beste Individuum oder die besten werden für die nachfolgende Generation ausgewählt. Dabei gibt es Strategien bei denen nur die Nachkommen an der Auswahl beteiligt sind und welche bei denen Eltern und Kinder teilnehmen.

%- Section .5 -----------------------------------------------------------------
\subsection{Evolutionsalgorithmus}
%
Der eigentliche Evolutionsalgorithmus ist in Abbildung~\ref{fig:es_flowchart} dargestellt. Er enthält im wesentlichen die in den vorherigen Abschnitten beschriebenen Schritte. Der prinzipielle Ablauf ist für alle Evolutionsalgorithmen gleich. Eine Unterscheidung der Verfahren kann durch verschiedene Parameter beschrieben werden. Wesentlich dabei sind die Populationsgröße $\mu$, die Anzahl an der Rekombination beteiligten Eltern $\varrho$, die gewählte Selektionsstrategie sowie die Anzahl der Nachkommen $\lambda$. Im Folgenden sind zuerst einige Beispiele für die Nomenklatur der Selektionsstrategie aufgeführt, die im Anschluss genauer beschrieben werden.\\
Für Strategien die nur auf Mutation für die Erzeugung von Nachkommen setzten sind folgende Nomenklaturen gebräuchlich:
\begin{itemize}
\item $(\mu+\lambda)$ Elternelemente werden in der Selektion berücksichtigt
\item $(\mu,\lambda)$ Ausschließlich Nachkommen nehmen an der Selektion teil
\end{itemize}
%
Wird die Rekombination eingesetzt kann auch die Anzahl der beteiligten Elternelemente angegeben werden:
\begin{itemize}
\item $({\mu}/{\varrho}+\lambda)$ \& $({\mu}/{\varrho},\lambda)$ Angabe der Anzahl beteiligter Eltern bei der Rekombination.
\end{itemize}
%
Mithilfe dieser Klassifikation werden die Algorithmen im Folgenden stets angegeben.
%
%------------------------------------------------------------------------------
%------------------------------------------------------------------------------
%------------------------------------------------------------------------------
\begin{figure}[h]
	\begin{center}
		\caption[Kurzeintrag]{lorem ipsum}
%		\input{img/antenna_alignment.tex}
		
	\end{center}
\end{figure}
%------------------------------------------------------------------------------

%- Section .6 -----------------------------------------------------------------
\subsection{Strategien mit mehreren Populationen}
Es ist möglich die Strategien auf die Ebene von Populationen zu erweitern. Das bedeutet, man lässt ganze Populationen miteinander in Wettstreit treten und nur diejenige überleben, die die besten Ergebnisse liefern. Das mündet in einem zweistufigen Evolutionsprozess. Man kann die Notation um diesen Umstand erweitern und erhält so:
$$
[\mu_2/\varrho_2,^{+}\lambda_2(\mu_1/\varrho_1,^{+}\lambda_1)]
$$
Sprich aus $\mu_2$-Elternpopulationen werden durch Rekombination mit jeweils $\varrho_2$ Populationen, $\lambda_2$ Nachkommenpopulationen generiert. Innerhalb der Populationen erfolgt die Optimierung anhand einer $({\mu_1}/{\varrho_1}+\lambda_1)$ oder $({\mu_1}/{\varrho_1},\lambda_1)$-Strategie. Nun kann nach einer bestimmten Zahl von Generationen die besten Populationen für die nächste Generation ausgewählt werden. Auch hier stehen verschiedene Auswahlkriterien zur Verfügung. Man kann z.B. die Population anhand des Zielfunktionswert des besten Individuums wählen oder den Mittelwert über alle Individuen wählen.


%----------------------------------------------------------------------------
%----------------------------------------------------------------------------
%----------------------------------------------------------------------------
\newpage
\begin{landscape}
	\section{Projektlaufplan KW 22}
	\label{sec:projectplan}
	\scalebox{.75}{
		\input{common/documents/project_plan_KW22.tex}
		}
\end{landscape}

%----------------------------------------------------------------------------

\end{appendix}


\newpage
%- Bibliography --------------------------------------------------------------
\bibliographystyle{ieeetr}
%\bibliography{../bib/mathesis_collection1}

\end{document}