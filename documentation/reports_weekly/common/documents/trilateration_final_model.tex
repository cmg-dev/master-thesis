{
\small
Folgende Nomenklatur und Symbole gelten für diesen Abschnitt:
\begin{itemize}[itemsep=0mm]
	\item	$N_0:=$Anzahl der verfügbaren Antennen $N=\{1,..,8\}$
	\item	$N:=$Anzahl der Antennen die für die Optimierung verwendet werden können ($N \subseteq N_0$)    
	\item	$N':=$Anzahl der Antennen die für die Optimierung verwendet werden ($N' \subseteq N$); Dabei ist $|N'| \geq 3$
	\item	Es gilt $|N'| \geq |N| \geq |N_0|$   
	\item	$k$ ist der Index der Antennen einer Messung, es gilt $k = 1,2,..,|N'|-1$
	\item	fette Großbuchstaben stehen für Matrizen (bspw. $\mathbf{A}$)
	\item	fette Kleinbuchstaben stehen für Vektoren (bspw. $\mathbf{x}$)
\end{itemize}
%
Das bisher verwendete Modell stellen wir um und erhalten:
\begin{align}
%
\mathbf{0}&=\mathbf{A}\mathbf{x}-\mathbf{b}\\
\mathbf{A}&=
\left(
	\begin{array}{ccccc}
		x_k-x_0 & y_k-y_0 & z_k-z_0 & \sum_{i=1,j=k}^{k}(-a_1\delta_{ij}) \sum_{i=1,j=k}^{k}(-a_2\delta_{ij})
	\end{array}
\right)\nonumber\\
%
\mathbf{x}&=
\left(
   \begin{array}{c}
	   x-x_0\\
	   y-y_0\\
	   z-z_0\\
	   n_0^2-n_k^2\\
	   \Theta_0n_0-\Theta_kn_k
   \end{array}
\right)\nonumber\\
%
\mathbf{b}&=
	\begin{array}{c}
		a_{0k}-a_{3k0} 
	\end{array}
	= c_{k0}'\nonumber
\end{align}
%
Dabei steht $\delta_{ij}$ für den bekannten Kronecker-Operator und bedeutet:
\begin{equation*}
\delta_{ij} = \begin{cases}1 ~\text{für}~ i=j\\ 0 ~\text{für}~ i\neq j\end{cases}
\end{equation*}
%
Im expliziten sehen die Matrix $\mathbf{A}$ und der Vektor $\mathbf{b}$, für denn Fall $N'=3$ und $k=\{1,2,3\}$, wie folgt aus:
%
\begin{equation}
\mathbf{A}=
\left(
	\begin{array}{ccccccccc}
		x_1-x_0 & y_1-y_0 & z_1-z_0 & -a_1 & 0 & 0 & -a_2 & 0 & 0 \\
		x_2-x_0 & y_2-y_0 & z_2-z_0 & 0 & -a_1 & 0 & 0 & -a_2 & 0 \\
		x_3-x_0 & y_3-y_0 & z_3-z_0 & 0 & 0 & -a_1 & 0 & 0 & -a_2
	\end{array}
\right)
\end{equation}
%
\begin{equation}
\mathbf{x}=
\left(
	\begin{array}{c}
		x-x_0	\\
		y-y_0	\\
		z-z_0	\\
		n_0^2-n_1^2	\\
		(\dots)	\\
		n_0^2-n_k^2	\\
		\Theta_0n_0-\Theta_1n_1	\\
		(\dots)	\\
		\Theta_0n_0-\Theta_kn_k	
	\end{array}
\right)
\end{equation}
Das Ergebnis ist eine $3\times9$ und eine $1\times9$ Matrix. Es ist möglich diesem Modell eine beliebige Anzahl an Antennen hinzuzufügen. Fügt man eine Antenne zur Berechnung hinzufügen würde sich die Matrix $\mathbf{A}$ um zwei Spalten und eine Zeile erweitern, der Vektor $\mathbf{x}$ analog um 2 Zeilen.

}