\begin{enumerate}
%
\item libCalibration
	\begin{enumerate}
	\item Enthält die Calibration-Klasse
	\item Bestimmt aus den Entfernungsmessungen von dem Kalibrierphantom die Position der Antennen.
	\end{enumerate}
%	
\item libNormalizer
	\begin{enumerate}
	\item Sammelt verschiedene Methoden für die Normierung der gemessenen Werte
	\end{enumerate}
%	
\item libPermutate
	\begin{enumerate}
	\item Erstellt alle möglichen Permutationen die für den Antennenaufbau möglich sind
	\item Stellt diese Permutationen für spätere Berechnungen komfortabel zur Verfügung
	\end{enumerate}
%
\item libPRPSSystem
	\begin{enumerate}
	\item Ließt die Systemkenndaten (z.B. Messfrequenz etc.) aus der entsprechenden Eingabedatei und berechnet die sich daraus ableitenden Faktoren (z.B. Wellenlänge etc.)
	\end{enumerate}
%
\item libSolve
	\begin{enumerate}
	\item Das eigentliche Herzstück des Projekts
	\item Ermöglicht das Lösen mehrerer unterschiedlicher Modelle
	\item Implementiert unterschiedliche Lösungsstrategien, wie z.B. (1+1)-ES, ($\mu+\lambda$)-ES etc.
	\item Ist Threadsicher um die Performance erheblich zu verbessern
	\item Die Dimension des Problem lässt sich leicht anpassen
	\end{enumerate}
%
\end{enumerate}