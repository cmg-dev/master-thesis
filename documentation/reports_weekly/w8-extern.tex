\documentclass[a4paper,12pt,fleqn]{article}
\usepackage[T1]{fontenc}
\usepackage{ucs}
\usepackage[utf8x]{inputenc}
\usepackage{ngerman}
\usepackage[ngerman]{babel}
\usepackage{lastpage}
\usepackage[pdftex]{color,graphicx}
\usepackage{listings}
\usepackage{pdflscape}
\usepackage{longtable}
\usepackage[inner=2cm,outer=2cm,top=1cm,bottom=1.5cm,includeheadfoot]{geometry}
\usepackage{fancyhdr}
\usepackage{url}
\usepackage{draftwatermark}
\usepackage{booktabs}
\usepackage{blindtext} 
\usepackage{framed} 
\usepackage{xcolor} 
\colorlet{shadecolor}{black} 

\usepackage{enumitem}

\SetWatermarkText{Vertraulich}
\SetWatermarkScale{4}
\SetWatermarkLightness{0.9}

\usepackage{pgfgantt}
\usepackage{amsmath,amssymb,amsfonts,amstext}
\usepackage{floatflt}
\usepackage{tikz}
\usetikzlibrary[arrows,snakes,backgrounds,shapes]
\usetikzlibrary{through}
\usetikzlibrary{calc}

% highlighting
\usepackage{xcolor,soul}

%---- PageLayout
\pagestyle{fancy}

\setlength{\headsep}{10mm}

\usepackage{eso-pic}

%----------------------------------------------------------------------------
% HEADER --------------------------------------------------------------------
%----------------------------------------------------------------------------
\fancyhead[R]{
  \includegraphics[width=100pt,keepaspectratio]{img/amedo2012.png}
}

\fancyhead[C]{ Wochenbericht KW 24 }

\fancyhead[L]{
  \begin{tabular}[b]{l}
  Christoph Gnip\\
  Projekt: PRPS-Evolution
  \end{tabular}
}

%Linie oben
\renewcommand{\headrulewidth}{0.5pt}
%----------------------------------------------------------------------------


%----------------------------------------------------------------------------
%----------------------------------------------------------------------------
%----------------------------------------------------------------------------
\fancyfoot[L]{Stand: \today}
\fancyfoot[C]{ EXTERN }
\fancyfoot[R]{\thepage{} von \pageref{LastPage}}

% Linie unten
\renewcommand{\footrulewidth}{0.5pt}
%----------------------------------------------------------------------------

% Import Macros  ------------------------------------------------------------
%--------------------------------------------------------------
%--------------------------------------------------------------
%--------------------------------------------------------------
\newcommand\confidentialoverlay{
  % Taken from the TikZ documentation.
  % NB: This requires \usepackage{tikz}!
  \begin{tikzpicture}[remember picture,overlay]
    \node [rotate=60,scale=10,text opacity=0.1]
      at (current page.center) {Vertraulich};
  \end{tikzpicture} 
 
} 
%--------------------------------------------------------------
%--------------------------------------------------------------
%--------------------------------------------------------------
\newcommand{\myvec}[1]{\hat{\mathbf{#1}}}% Vector notation

%--------------------------------------------------------------
%- This can be used for aligning equations --------------------
%--------------------------------------------------------------
\newcommand{\phantomeq}[2]{
\begin{equation}
	\phantom{#1}
	#2
\end{equation}
}% Vector notation

%--------------------------------------------------------------
%- seraches for input in the "extern" folder ------------------
%--------------------------------------------------------------
\newcommand{\externInput}[1]{\input{extern/#1}}

%--------------------------------------------------------------
%- seraches for input in the "intern" folder ------------------
%--------------------------------------------------------------
\newcommand{\internInput}[1]{\input{intern/#1}}

%--------------------------------------------------------------
%- seraches for input in the "common" folder ------------------
%--------------------------------------------------------------
\newcommand{\commonInput}[1]{\input{common/#1}}

\newcommand{\cpp}{%
  \mbox{\emph{\textrm{C\hspace{-1.5pt}\raisebox{1.75pt}{\scriptsize +}%
  \hspace{-2pt}\raisebox{.75pt}{\scriptsize +}}}}%
}

\newcommand{\amedogmbh}{%
  amedo GmbH
}

%\renewenvironment{itemize}[1]{\begin{compactitem}#1}{\end{compactitem}}
%\renewenvironment{enumerate}[1]{\begin{compactenum}#1}{\end{compactenum}}
%\renewenvironment{description}[0]{\begin{compactdesc}}{\end{compactdesc}}


%----------------------------------------------------------------------------
% Start the Document --------------------------------------------------------
%----------------------------------------------------------------------------
\begin{document}

\setlength{\headheight}{36pt}

\begin{titlepage}

\input{extern/title/title_w8.tex}

\end{titlepage}

%- Section 1 ----------------------------------------------------------------
\section[Allgemeines]{Allgemeines}
%
%- Section 2 ----------------------------------------------------------------
\section[Fortschritt]{Projektfortschritt}
%
In dieser Woche wurden die Algorithmen verifiziert. Die Permutationen und die Koeffizienten werden korrekt berechnet. Es wurde mit der Umsetzung der Excel Referenzimplementation in ein C++ Modul begonnen.
%
%- Section 2.2 --------------------------------------------------------------
\subsection{Umsetzung Algorithmus}
In dieser Woche wurde begonnen den Algorithmus in ein C++ Modul zu portieren. Um den Compilevorgang zu automatisieren und Plattformunabhängigkeit zu gewährleisten, wurde ein CMake-Skript erstellt. Das Skript steuert erstellt das eigentliche Makefile, das im Anschluss ausgeführt werden kann. Im CMakefile, kann auch eine Referenz auf deinen Cross-Compiler angegeben werden, sodass gegen die Architektur eines beliebigen Zielsystems compiliert werden kann. Die Umsetzung konnte in dieser Woche nicht abgeschlossen werden.
%
\subsection{Besprechung am Mittwoch}
Am Mittwoch fand die Besprechung mit Fr. Susanne Winter statt. Ihr wurde das Projekt in groben Zügen vorgestellt. Sie wird im Rahmen des Projektes für Detailfragen zur Implementierung und Eignung der Modelle für eine Evolutionäre Optimierung zur Verfügung stehen. Dadurch ist zu erwarten, dass die Implementationszeit verkürzt werden kann.
%
%- Section 3 -----------------------------------------------------------------
\section{Probleme}
\label{Problems}
Auf dem neuem FPGA-Board wird es voraussichtlich nicht möglich sein, vier AD-Wandler zu betreiben. Der Grund dafür ist, dass nicht so viele LVDS-Leitungen wie vom Hersteller beworben verwendet werden dürfen. Die Spannungsversorgung war nicht ausreichend dimensioniert. Dieses Problem wurde von S. Gnip durch eine Anpassung der verbauten IC's behoben. Diese aufgetretenen Probleme gefährden dieses Projekt nicht.
%
%- Appendix ------------------------------------------------------------------
%
%
%
\begin{appendix}

%----------------------------------------------------------------------------
%----------------------------------------------------------------------------
%----------------------------------------------------------------------------
\newpage

\begin{center}
	\huge{Anhänge}
\end{center}

\normalsize

%----------------------------------------------------------------------------
%----------------------------------------------------------------------------
%----------------------------------------------------------------------------
\newpage
\begin{landscape}
	\section{Projektlaufplan KW 24}
	\label{sec:projectplan}
	\scalebox{.75}{
		\input{common/documents/project_plan_KW22.tex}
		}
\end{landscape}

%----------------------------------------------------------------------------

\end{appendix}


\newpage
%- Bibliography --------------------------------------------------------------
\bibliographystyle{ieeetr}
\bibliography{../bib/mathesis_collection1}

\end{document}