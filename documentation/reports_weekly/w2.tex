\documentclass[a4paper,12pt,fleqn]{article}
\usepackage[T1]{fontenc}
\usepackage{ucs}
\usepackage[utf8x]{inputenc}
\usepackage{ngerman}
\usepackage[ngerman]{babel}
\usepackage{lastpage}
\usepackage[pdftex]{color,graphicx}
\usepackage{listings}
\usepackage{pdflscape}
\usepackage{longtable}
\usepackage[inner=2cm,outer=2cm,top=1cm,bottom=1.5cm,includeheadfoot]{geometry}
\usepackage{fancyhdr}
\usepackage{url}
\usepackage{draftwatermark}

\SetWatermarkText{Vertraulich}
\SetWatermarkScale{4}
\SetWatermarkLightness{0.9}

\usepackage{amsmath,amssymb,amsfonts,amstext}

% highlighting
\usepackage{xcolor,soul}

%---- PageLayout
\pagestyle{fancy}

\setlength{\headsep}{10mm}

\usepackage{eso-pic}

%----------------------------------------------------------------------------
% HEADER --------------------------------------------------------------------
%----------------------------------------------------------------------------
\fancyhead[R]{
  \includegraphics[width=100pt,keepaspectratio]{img/amedo2012.png}
}

\fancyhead[C]{ Wochenbericht KW 18 }

\fancyhead[L]{
  \begin{tabular}[b]{l}
  Christoph Gnip\\
  Projekt: RFID-Evolution
  \end{tabular}
}

%Linie oben
\renewcommand{\headrulewidth}{0.5pt}
%----------------------------------------------------------------------------


%----------------------------------------------------------------------------
%----------------------------------------------------------------------------
%----------------------------------------------------------------------------
\fancyfoot[L]{Stand: \today}
\fancyfoot[C]{ }
\fancyfoot[R]{\thepage{} von \pageref{LastPage}}

% Linie unten
\renewcommand{\footrulewidth}{0.5pt}
%----------------------------------------------------------------------------

% Import Macros  ------------------------------------------------------------
%--------------------------------------------------------------
%--------------------------------------------------------------
%--------------------------------------------------------------
\newcommand\confidentialoverlay{
  % Taken from the TikZ documentation.
  % NB: This requires \usepackage{tikz}!
  \begin{tikzpicture}[remember picture,overlay]
    \node [rotate=60,scale=10,text opacity=0.1]
      at (current page.center) {Vertraulich};
  \end{tikzpicture} 
 
} 
%--------------------------------------------------------------
%--------------------------------------------------------------
%--------------------------------------------------------------
\newcommand{\myvec}[1]{\hat{\mathbf{#1}}}% Vector notation

%--------------------------------------------------------------
%- This can be used for aligning equations --------------------
%--------------------------------------------------------------
\newcommand{\phantomeq}[2]{
\begin{equation}
	\phantom{#1}
	#2
\end{equation}
}% Vector notation

%--------------------------------------------------------------
%- seraches for input in the "extern" folder ------------------
%--------------------------------------------------------------
\newcommand{\externInput}[1]{\input{extern/#1}}

%--------------------------------------------------------------
%- seraches for input in the "intern" folder ------------------
%--------------------------------------------------------------
\newcommand{\internInput}[1]{\input{intern/#1}}

%--------------------------------------------------------------
%- seraches for input in the "common" folder ------------------
%--------------------------------------------------------------
\newcommand{\commonInput}[1]{\input{common/#1}}

\newcommand{\cpp}{%
  \mbox{\emph{\textrm{C\hspace{-1.5pt}\raisebox{1.75pt}{\scriptsize +}%
  \hspace{-2pt}\raisebox{.75pt}{\scriptsize +}}}}%
}

\newcommand{\amedogmbh}{%
  amedo GmbH
}

%\renewenvironment{itemize}[1]{\begin{compactitem}#1}{\end{compactitem}}
%\renewenvironment{enumerate}[1]{\begin{compactenum}#1}{\end{compactenum}}
%\renewenvironment{description}[0]{\begin{compactdesc}}{\end{compactdesc}}


%----------------------------------------------------------------------------
% Start the Document --------------------------------------------------------
%----------------------------------------------------------------------------
\begin{document}

\setlength{\headheight}{36pt}

\begin{titlepage}

\input{title/title_w2.tex}

\end{titlepage}

%- Section 1 ----------------------------------------------------------------
\section[Allgemeines]{Allgemeines}
Die Anmeldung der Arbeit erfolgte am 2.Mai. in dieser Woche. Der vom Prüfungsamt mitgeteilte Abgabetermin ist der 9. September 2013.

%- Section 2 ----------------------------------------------------------------
\section[Fortschritt]{Projektfortschritt}
In dieser Woche wurde hauptsächlich die Arbeiten an der Recherche fortgeführt. Zusätzlich wurde das Pflichtenheft geschrieben, es befindet sich in der abschließenden redaktionellen Phase und wird voraussichtlich termingerecht fertiggestellt. Aufgrund der zeitlichen Abfolge fließen bereits die Ergebnisse der Recherche in die Planung im Pflichtenheft ein. Im Besonderen war es so möglich einen detaillierteren Projektlaufplan zu erstellen.

%- Section 2.1 --------------------------------------------------------------
\subsection{Dokumentation}
Die Wochenberichte der KW 17 und KW 18 werden dem Erstbetreuer Herrn Prof. Dr. Bärmann erst in der KW 19 übergeben werden. Der Grund dafür ist der in der KW 18 liegende Feiertag, in Zukunft soll eine solche Verzögerung vermieden werden.

%- Section 2.2 --------------------------------------------------------------
\subsection{Recherche}
In dieser Woche wurden die Recherchearbeiten abgeschlossen. Abschließend konnte noch nicht alle Paper etc. gesichtet werden. Im Folgenden werden die gewonnenen Erkenntnisse vorgestellt.

Das bereits in der letzten Woche vorgestellte Paper \cite{KALMANandSMOOTHING} wurde intensiver untersucht. Es präsentiert eine Methode die Positionsgenauigkeit erheblich zu verbessern. Dabei wurden, wie auch bei dem von der Amedo STS verwendeten Aufbau, eine Phasenmessung vorgenommen und die Unsicherheiten mittels Kalmann-Filterung und anschließender Glättung minimiert. Das eigentliche Problem, die Unbestimmtheit der Wellenzahl beim Start der Messung sowie der Verlust des Tags durch nicht gelesene Antennen, wurde in diesem Paper \textbf{nicht} gelöst. Lediglich eine Approximation wurde mittels RSSI-Messung (\textit{Receive Signal Strength Indicator}) vorgenommen.

In der am Institut für Neuroinformatik eingereichte Bachelorarbeit \cite{Muz1}, wurde mit dem System der Amedo STS eine Methode präsentiert mittels Lernverfahren eine genaue Positionsvorhersage zu erstellen. Im Rahmen dieser Arbeit wurden verschiedene Lernmethoden (zwei im Wesentlichen) verwendet. Zum einen ein einfacher Nearest-Neighbour-Algorithmus und ein Ansatz mittels Support-Vector-Machines (SVM). Die Gewinnung der Daten für die Lernsets, war für diesen Ansatz sehr aufwändig und ist so nicht praktisch einsetzbar. Allerdings legen die vorgeschlagenen Methoden nahe, eine 

In der Veröffentlichung \cite{Wil1} wird ein mathematisches Modell präsentiert, das sich für die Erstellung einer Fitness-Funktion für den CMA-ES-Algorithmus eignen sollte. Dieses Modell berücksichtigt die Mehrdeutigkeit der Phasenlage und erlaubt eine Abschätzung der tatsächlichen Entfernung. Die Eignung dieses Ansatzes wird noch genauer untersucht und die Ergebnisse später vorgestellt.

Andere Veröffentlichungen, die in dieser Woche gesichtet wurden, werden hier aufgrund mangelnder Relevanz nicht vorgestellt. Ein großer Teil der gesichteten Literatur stammt von dem WPNC (\textit{Workshop on Positioning, Navigation and Communication}) aus den Jahren 2011 und 2013. Es wird aufgrund der Menge an Veröffentlichungen in diesem Bericht davon Abstand genommen, einzelne Paper vorzustellen, zumal nur wenige sich mit der von der Amedo STS verwendeten Kombination aus UHF, Pasenmessung und genaue Positionierung beschäftigen. Es lässt sich Zusammenfassend sagen, dass Kalmann-Filterung und Mapping im Bereich RFID der Stand der Technik sind. Die Phasenmessung ist aufgrund der Vielzahl an Unbekannten nicht weit verbreitet und die verbreitete Methode über RSSI ermöglicht keine genaue Positionsmessung.

%- Section 3 -----------------------------------------------------------------
\section[Probleme]{Probleme}
Der in diese Woche gefallene Feiertag, es werden noch weitere Feiertage im Mai folgen, diese werden nun durch eine vorausschauende Planung berücksichtigt.

%- Appendix ------------------------------------------------------------------
\input{append/appendix_w2.tex}

\newpage
%- Bibliography --------------------------------------------------------------
\bibliographystyle{ieeetr}
\bibliography{../bib/mathesis_collection1}

\end{document}