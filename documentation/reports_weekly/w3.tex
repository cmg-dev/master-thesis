\documentclass[a4paper,12pt,fleqn]{article}
\usepackage[T1]{fontenc}
\usepackage{ucs}
\usepackage[utf8x]{inputenc}
\usepackage{ngerman}
\usepackage[ngerman]{babel}
\usepackage{lastpage}
\usepackage[pdftex]{color,graphicx}
\usepackage{listings}
\usepackage{pdflscape}
\usepackage{longtable}
\usepackage[inner=2cm,outer=2cm,top=1cm,bottom=1.5cm,includeheadfoot]{geometry}
\usepackage{fancyhdr}
\usepackage{url}
\usepackage{draftwatermark}

\SetWatermarkText{Vertraulich}
\SetWatermarkScale{4}
\SetWatermarkLightness{0.9}

\usepackage{pgfgantt}
\usepackage{amsmath,amssymb,amsfonts,amstext}

% highlighting
\usepackage{xcolor,soul}

%---- PageLayout
\pagestyle{fancy}

\setlength{\headsep}{10mm}

\usepackage{eso-pic}

%----------------------------------------------------------------------------
% HEADER --------------------------------------------------------------------
%----------------------------------------------------------------------------
\fancyhead[R]{
  \includegraphics[width=100pt,keepaspectratio]{img/amedo2012.png}
}

\fancyhead[C]{ Wochenbericht KW 19 }

\fancyhead[L]{
  \begin{tabular}[b]{l}
  Christoph Gnip\\
  Projekt: PRPS-Evolution
  \end{tabular}
}

%Linie oben
\renewcommand{\headrulewidth}{0.5pt}
%----------------------------------------------------------------------------


%----------------------------------------------------------------------------
%----------------------------------------------------------------------------
%----------------------------------------------------------------------------
\fancyfoot[L]{Stand: \today}
\fancyfoot[C]{ }
\fancyfoot[R]{\thepage{} von \pageref{LastPage}}

% Linie unten
\renewcommand{\footrulewidth}{0.5pt}
%----------------------------------------------------------------------------

% Import Macros  ------------------------------------------------------------
%--------------------------------------------------------------
%--------------------------------------------------------------
%--------------------------------------------------------------
\newcommand\confidentialoverlay{
  % Taken from the TikZ documentation.
  % NB: This requires \usepackage{tikz}!
  \begin{tikzpicture}[remember picture,overlay]
    \node [rotate=60,scale=10,text opacity=0.1]
      at (current page.center) {Vertraulich};
  \end{tikzpicture} 
 
} 
%--------------------------------------------------------------
%--------------------------------------------------------------
%--------------------------------------------------------------
\newcommand{\myvec}[1]{\hat{\mathbf{#1}}}% Vector notation

%--------------------------------------------------------------
%- This can be used for aligning equations --------------------
%--------------------------------------------------------------
\newcommand{\phantomeq}[2]{
\begin{equation}
	\phantom{#1}
	#2
\end{equation}
}% Vector notation

%--------------------------------------------------------------
%- seraches for input in the "extern" folder ------------------
%--------------------------------------------------------------
\newcommand{\externInput}[1]{\input{extern/#1}}

%--------------------------------------------------------------
%- seraches for input in the "intern" folder ------------------
%--------------------------------------------------------------
\newcommand{\internInput}[1]{\input{intern/#1}}

%--------------------------------------------------------------
%- seraches for input in the "common" folder ------------------
%--------------------------------------------------------------
\newcommand{\commonInput}[1]{\input{common/#1}}

\newcommand{\cpp}{%
  \mbox{\emph{\textrm{C\hspace{-1.5pt}\raisebox{1.75pt}{\scriptsize +}%
  \hspace{-2pt}\raisebox{.75pt}{\scriptsize +}}}}%
}

\newcommand{\amedogmbh}{%
  amedo GmbH
}

%\renewenvironment{itemize}[1]{\begin{compactitem}#1}{\end{compactitem}}
%\renewenvironment{enumerate}[1]{\begin{compactenum}#1}{\end{compactenum}}
%\renewenvironment{description}[0]{\begin{compactdesc}}{\end{compactdesc}}


%----------------------------------------------------------------------------
% Start the Document --------------------------------------------------------
%----------------------------------------------------------------------------
\begin{document}

\setlength{\headheight}{36pt}

\begin{titlepage}


%- the Title page --------------------------------------------------------
\begin{center}
%\vspace*{2.5cm}
{\Huge \textbf{Wochenbericht KW 18}\par}
\vspace{1cm}
{\Huge 29.4. - 5.5.2013\par}
\vspace{1cm}
{\Huge Projektwoche: 2\par}

\vspace{2cm}

\large{Erstellt durch}\\
\Large{\textbf{Christoph Gnip}}


%\vspace{4cm}
\vfill

{\normalsize Fachbereich Elektrotechnik und angewandte Naturwissenschaften\\
Westfälische Hochschule\\[2ex]Mai 2013}


\end{center}
\newpage

\end{titlepage}

%- Section 1 ----------------------------------------------------------------
\section[Allgemeines]{Allgemeines}
Ein Projektlaufplan wurde erstellt um den aktuellen Stand des Projektes übersichtlich darstellen zu können. Eine erste Version des Plans befindet sich in Anhang~\ref{sec:projectplan}. Die erste gültige Version wird mit Veröffentlichung des Pflichtenhefts erscheinen.

%- Section 2 ----------------------------------------------------------------
\section[Fortschritt]{Projektfortschritt}
Die Einarbeitung in die numerischen Verfahren wurde weitergeführt. Eine Übersicht über die Eignung und Fähigkeiten der Verfahren, die bisher untersucht wurden, wird der nächste Wochenbericht enthalten. Die Arbeiten an einem mathematischen Modell, dass in den num. Verfahren verwendet werden kann ist in dieser Woche fortgeführt worden. Aus zeitlichen Gründen kann im Rahmen dieses Wochenberichts nicht näher auf Details der Modelle eingegangen werden.

%- Section 2.2 --------------------------------------------------------------
\subsection{Datenaufnahme}
Da es für die weiteren Entwicklungen von großem Vorteil ist auf reale Daten zugreifen zu können, wurden in dieser Woche Anstrengungen intensiviert diese Daten mit dem neuen System generieren zu können. Der in dieser Woche erreichte Stand der FPGA- und PC-Software ermöglichen dies prinzipiell. Dazu waren einige Anpassungen der PC-Software notwendig. Erste Messkurven konnten visualisiert werden und stehen für die weitere Auswertung bereit. Auf der Basis dieser Daten können nun Überlegungen für die weitere Messdatenaufbereitung angestellt werden. Dies erfolgt voraussichtlich in KW 21.

%- Section 2.3 --------------------------------------------------------------
\subsection{Antennenpositionierung}
Da es bei dem verwendeten Messaufbau um ein System mit frei positionierbaren Antennen handelt

%- Section 3 -----------------------------------------------------------------
\section[Probleme]{Probleme}
Am Freitag wurden sicherheitsrelevante Updates der Serverinfrastruktur aufgespielt. Diese betreffen auch dieses Projekt. Dieser Punkt war zeitlich nicht eingeplant und erforderte ein administratives Eingreifen meinerseits. Durch diesen Umstand und durch den Feiertag waren es lediglich drei effektive Werktage in dieser Woche.

%- Appendix ------------------------------------------------------------------
%
%
%
%

\begin{appendix}
%----------------------------------------------------------------------------
%----------------------------------------------------------------------------
%----------------------------------------------------------------------------
\newpage
\huge{Anhänge}
\normalsize
\newpage

\begin{landscape}
	\section{Projektlaufplan}
	\label{sec:projectplan}
	\scalebox{.75}{
		\input{documents/project_plan.tex}
		}
\end{landscape}

% \dots{}
%----------------------------------------------------------------------------

\end{appendix}


\newpage
%- Bibliography --------------------------------------------------------------
\bibliographystyle{ieeetr}
\bibliography{../bib/mathesis_collection1}

\end{document}