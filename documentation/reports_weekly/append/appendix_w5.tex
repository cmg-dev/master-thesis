%
%
%
%

\begin{appendix}
%----------------------------------------------------------------------------
%----------------------------------------------------------------------------
%----------------------------------------------------------------------------
\newpage
\begin{center}
	\huge{Anhänge}
\end{center}

\normalsize

%----------------------------------------------------------------------------
\section{Erweiterung des Trilaterationsmodells}
\label{sec:extended_trilateration_model}

{
\small
Folgende Nomenklatur und Symbole gelten für diesen Abschnitt:
%
Wie gezeigt werden konnte\footnote{Wochenbericht KW 20, Anhang B} ergibt sich für den Fall der Trilateration und der Annahme, dass vier Antennen Messwerte liefern, die Gleichung:
\begin{equation}\label{eq:final_trilateration_model}
0=
\left(
	\begin{array}{ccc}
		x_k-x_0 & y_k-y_0 & z_k-z_0 
	\end{array}
\right)
\left(
   \begin{array}{c}
	   x-x_0\\
	   y-y_0\\
	   z-z_0
   \end{array}
\right)
-
\left(
	\begin{array}{c}
		c_{kj}
	\end{array}
\right) 
\end{equation}
%
Dabei ist:
\begin{equation}\label{eq:c_kj}
	c_{kj}=\frac{1}{2}[d_{kj}^2+r_{j}^2-r_k^2]
\end{equation}
%
Ziel dieser Erweiterung ist es, einen Zusammenhang zwischen diesem Modell und der Wellenzahl zu erzeugen. Folgender Ansatz wird gewählt:
	\begin{equation}\label{eq:r_0_varrho} r(\varrho)=\frac{\lambda}{2}\left(\frac{\varrho}{2\pi}+n\right),\\\lambda=\frac{c}{f}, n:= \text{Wellenzahl}
\end{equation}
%
%
Weiterhin ist $\varrho$ die gemessene Phase, die das PRPS-System liefert und $n$ die gesuchte Wellenzahl.\\
Durch einsetzen von \eqref{eq:r_0_varrho} in \eqref{eq:c_kj}, erhalten wir:
\begin{equation}\label{eq:c_k0_extended}
	c_{kj}(\varrho_0, \varrho_k, n_0, n_k) =\frac{1}{2}\left[d_{kj}^2+\frac{\lambda^2}{4}\left(\frac{\varrho_j}{2\pi}+n_0\right)^2-\frac{\lambda^2}{4}\left(\frac{\varrho_k}{2\pi}+n_k\right)^2\right]
\end{equation}
%
Wir stellen Gleichung~\eqref{eq:c_k0_extended} um:
\begin{align}
%	
	c_{kj}(\varrho_0, \varrho_k, n_0, n_k) &= \frac{1}{2}\left\{d_{kj}^2+\frac{\lambda^2}{4}\left[\left(\frac{\varrho_j}{2\pi}\right)^2+2\frac{\varrho_j}{2\pi}n_0+n_0^2 \right.\right.\nonumber\\
	&\phantom{=}\; 
	\left.\left.-\left(\frac{\varrho_k}{2\pi}\right)^2-2\frac{\varrho_k}{2\pi}n_k-n_k^2\right]\right\}\\
%    
    &=\frac{1}{2}\left\{d_{kj}^2+\frac{\lambda^2}{4}\left[\left(\frac{\varrho_j}{2\pi}\right)^2-\left(\frac{\varrho_k}{2\pi}\right)^2 \right.\right.\nonumber\\
    &\phantom{=}\;
   	\left.\left.+2\frac{\varrho_j}{2\pi}n_0-2\frac{\varrho_k}{2\pi}n_k+n_0^2-n_k^2\right]\right\}\\
%	
	&=\frac{1}{2}d_{kj}^2+\frac{\lambda^2}{8}\left[\frac{1}{(2\pi)^2}\left(\varrho_0^2-\varrho_k^2\right) \right.\nonumber\\
	&\phantom{=}\;
	\left. +\frac{1}{\pi}\left(\varrho_0n_0-\varrho_kn_k\right)+\left(n_0^2-n_k^2\right)\right]\label{c_k0_rearragend}
\end{align}
%
Führen wir nun:
\phantomeq{c_{kj}(\varrho_0, \varrho_k, n_0, n_k)}{a_{0k} := \frac{1}{2}d_{kj}^2\nonumber}
\phantomeq{c_{kj}(\varrho_0, \varrho_k, n_0, n_k)}{a_1 := \frac{\lambda^2}{8}\nonumber}
\phantomeq{c_{kj}(\varrho_0, \varrho_k, n_0, n_k)}{a_2 := a_1\frac{1}{\pi}\nonumber}
\phantomeq{c_{kj}(\varrho_0, \varrho_k, n_0, n_k)}{a_{3kj} := a_1\frac{1}{(2\pi)^2}(\varrho_j^2-\varrho_k^2)\nonumber}
%
in Gleichung~\eqref{c_k0_rearragend} ein, erhalten die finale Form der Gleichung:
\begin{equation}
c_{kj}(\varrho_0, \varrho_k, n_0, n_k) = a_{0k}+a_1(n_0^2-n_k^2)+a_2(\varrho_0n_0-\varrho_kn_k)-a_{3kj}\label{c_k0_final_form}   
\end{equation}
%
Die Einführung der Konstanten macht zum Einen die Gleichung übersichtlicher. Zum Anderen können so, mit Blick auf eine spätere Softwareimplementation, Rechenschritte gespart werden. Das sollte sich positiv auf den späteren Berechnungsaufwand auswirken.\\
%
Im Weiteren erkennt man durch scharfes hinsehen das in Gleichung~\eqref{c_k0_final_form}, für $\varrho_k=\text{const.}$ \& $\varrho_0=\text{const.}$ gilt. Das resultiert aus der Tatsache, dass . Es ermöglicht uns zu schreiben:
\begin{equation}
c_{kj}(\varrho_0, \varrho_k, n_0, n_k) = c_{kj}(n_0, n_k)
\end{equation}
%
Im engeren Sinne einer mathematischen Funktion sollten wir die Parameter alle als Argument aufnehmen. Diese Form soll darstellen, welche Größen von Interesse sind. Im späteren Gebrauch wird diese Gleichung in der Optimierung eingesetzt werden.
Für unser Gleichungssystem aus\eqref{eq:final_trilateration_model} ergibt sich:
\begin{equation}\label{eq:wavenumber_trilateration_model}
0=
\left(
	\begin{array}{ccc}
		x_k-x_0 & y_k-y_0 & z_k-z_0 
	\end{array}
\right)
\left(
   \begin{array}{c}
	   x-x_0\\
	   y-y_0\\
	   z-z_0
   \end{array}
\right)
-
\left(
	\begin{array}{c}
		c_{kj}(n_0, n_k)
	\end{array}
\right)
\end{equation}
%
Betrachten wir nun \eqref{eq:wavenumber_trilateration_model}, wählen $N'=4$ (d.h. wir verwenden 4 Antennen) und setzen $j=0$. Wir beschreiben die Konfiguration wie folgt: Antenne 0 ist die Referenz-Antenne und Antenne 0-3 sind Messwertgeber für die Phaseninformation. 
%
\begin{equation}\label{eq:wavenumber_trilateration_model_explicit}
0=
\underbrace{\left(
	\begin{array}{ccc}
		x_1-x_0 & y_1-y_0 & z_1-z_0 \\
		x_2-x_0 & y_2-y_0 & z_2-z_0 \\
		x_3-x_0 & y_3-y_0 & z_3-z_0 
	\end{array}
\right)}_{\textbf{A}}
\underbrace{\left(
   \begin{array}{c}
	   x-x_0\\
	   y-y_0\\
	   z-z_0
   \end{array}
\right)}_{\textbf{x}}
-
\underbrace{\left(
	\begin{array}{c}
		c_{10}(n_0, n_1) \\
		c_{20}(n_0, n_2) \\
		c_{30}(n_0, n_3)
	\end{array}
\right)}_{\textbf{b}}
\end{equation}
%
\begin{equation}
\mathbf{b}=
\left(
	\begin{array}{c}
		a_{01}+a_1( n_0^2-n_1^2)+a_2(\varrho_0n_0-\varrho_1n_1)-a_{310} \\
		a_{02}+a_1(n_0^2-n_2^2)+a_2(\varrho_0n_0-\varrho_2n_2)-a_{320} \\
		a_{03}+a_1(n_0^2-n_3^2)+a_2(\varrho_0n_0-\varrho_3n_3)-a_{330}
	\end{array}
\right)
\end{equation}
%
Das Ergebnis ist ein um $\varrho$ und $n$ erweitertes Gleichungssystem. Zusätzlich enthält  es mehrere geometrische Konstanten ($a_{0k}, k=\{1,..,N-1\}$), mehrere Phasen-Konstanten ($a_{3k0}, k=\{1,..,N-1\}$), sowie zwei allgemeine ($a_1$ und $a_2$). Allgemeiner formuliert ergibt sich:
%
\begin{multline}\label{eq:final_equation}
0=
\left(
	\begin{array}{ccc}
		x_k-x_0 & y_k-y_0 & z_k-z_0 
	\end{array}
\right)
\left(
   \begin{array}{c}
	   x-x_0\\
	   y-y_0\\
	   z-z_0
   \end{array}
\right) \\
-
\left(
	\begin{array}{c}
		a_{0k}+a_1(n_0^2-n_k^2)+a_2(\varrho_0k_0-\varrho_kn_k)-a_{3kj}
	\end{array}
	\right)
\end{multline}
%
Aus Gleichung~\eqref{eq:final_equation} ist durch eine geeignete Wahl von $N'=\{4,..,N\}$ sofort ersichtlich wie viele Veränderliche sich für eine gewählte Konstellation an Antennen ergeben. Für $k$ gilt in diesem Fall $k=\{1,..,N'-1\}$.\\
%
Beispielsweise ergibt sich für das Modell aus Gleichung~\eqref{eq:final_equation} mit $N'=4$, insgesamt 7 Variablen ($\mathbf{x},n_0,n_1,n_2,n_3$) . Analog würde sich für ein Modell mit allen 8 Antennen, 11 Variablen ($\mathbf{x},n_0,..,n_7$) ergeben.
}
%----------------------------------------------------------------------------
%----------------------------------------------------------------------------
%----------------------------------------------------------------------------
\newpage
	\section{Berechnungen Messaufbau}
%
Die Form der Matrix ist Klug gewählt so lässt sich direkt sagen: Das System hat eine Lösung, da $det(\mathbf{A})\neq 0$
	% Table generated by Excel2LaTeX from sheet 'Alle'
\begin{table}[ht]
  \caption{Dargestellt ist ein Ergebnis der Berechnung der unbekannten Antennen-Koordinaten. Die Berechnung wurde Analytisch mithilfe von Excel durchgeführt}
  \vspace{0.5cm}
    \begin{tabular}{r|rrr|r|r|r|r|rrr}
    \multicolumn{11}{l}{\textit{\textbf{Lineare Modell für die Koordinatenberechnung}}} \\
          & \multicolumn{1}{c}{\textbf{0,77}} & \multicolumn{1}{c}{\textbf{0}} & \multicolumn{1}{c|}{\textbf{0}} &       & \multicolumn{1}{c|}{\textit{\textbf{x-x\_0}}} &       & \multicolumn{1}{c|}{\textbf{c\_10}} &       &       &  \\
          & \multicolumn{1}{c}{\textbf{0}} & \multicolumn{1}{c}{\textbf{0,77}} & \multicolumn{1}{c|}{\textbf{0}} & \multicolumn{1}{c|}{*} & \multicolumn{1}{c|}{\textit{\textbf{y-y\_0}}} & =     & \multicolumn{1}{c|}{\textbf{c\_20}} &       &       &  \\
          & \multicolumn{1}{c}{\textbf{0}} & \multicolumn{1}{c}{\textbf{0}} & \multicolumn{1}{c|}{\textbf{0,8}} &       & \multicolumn{1}{c|}{\textit{\textbf{z-z\_0}}} &       & \multicolumn{1}{c|}{\textbf{c\_30}} &       &       &  \\
    \multicolumn{1}{r}{} &       &       & \multicolumn{1}{r}{} & \multicolumn{1}{r}{} & \multicolumn{1}{r}{} & \multicolumn{1}{r}{} & \multicolumn{1}{r}{} &       &       &  \\
    \multicolumn{11}{l}{\textit{\textbf{Lösung des Modells}}} \\
    \multicolumn{1}{r}{} &       &       & \multicolumn{1}{r}{\textbf{1}} & \multicolumn{1}{r}{\textbf{2}} & \multicolumn{1}{r}{\textbf{3}} & \multicolumn{1}{r}{\textbf{4}} & \multicolumn{1}{r}{\textbf{5}} & \textbf{6} & \textbf{7} & \textbf{8}\\
\cline{2-11}    \multicolumn{1}{r}{} & x-x\_0 &       & \multicolumn{1}{r}{0,479} & \multicolumn{1}{r}{-0,775} & \multicolumn{1}{r}{1,521} & \multicolumn{1}{r}{-0,924} & \multicolumn{1}{r}{1,921} & -0,556 & 1,063 & 0,454 \\
    \multicolumn{1}{r}{} & y-y\_0 & =     & \multicolumn{1}{r}{-1,012} & \multicolumn{1}{r}{-1,045} & \multicolumn{1}{r}{-1,053} & \multicolumn{1}{r}{-0,197} & \multicolumn{1}{r}{0,031} & 1,092 & 1,072 & 1,355 \\
    \multicolumn{1}{r}{} & z-z\_0 &       & \multicolumn{1}{r}{0,607} & \multicolumn{1}{r}{1,346} & \multicolumn{1}{r}{1,379} & \multicolumn{1}{r}{1,326} & \multicolumn{1}{r}{1,396} & 1,436 & 1,358 & 0,670 \\
    \multicolumn{1}{r}{} &       &       & \multicolumn{1}{r}{} & \multicolumn{1}{r}{} & \multicolumn{1}{r}{} & \multicolumn{1}{r}{} & \multicolumn{1}{r}{} &       &       &  \\
    \multicolumn{1}{r}{} & \textbf{c\_k0} & \textbf{=} & \multicolumn{8}{l}{$1/2[r_0^2-r_k^2+d_{k0}^2]$} \\
    \multicolumn{1}{r}{} &       &       & \multicolumn{1}{r}{\textbf{}} & \multicolumn{1}{r}{\textbf{}} & \multicolumn{1}{r}{\textbf{}} & \multicolumn{1}{r}{\textbf{}} & \multicolumn{1}{r}{\textbf{}} & \textbf{} & \textbf{} & \textbf{} \\
    \multicolumn{1}{r}{} & c\_10 &       & \multicolumn{1}{r}{0,369} & \multicolumn{1}{r}{-0,596} & \multicolumn{1}{r}{1,171} & \multicolumn{1}{r}{-0,711} & \multicolumn{1}{r}{1,479} & -0,428 & 0,818 & 0,349 \\
    \multicolumn{1}{r}{} & c\_20 & =     & \multicolumn{1}{r}{-0,779} & \multicolumn{1}{r}{-0,804} & \multicolumn{1}{r}{-0,811} & \multicolumn{1}{r}{-0,152} & \multicolumn{1}{r}{0,024} & 0,841 & 0,826 & 1,043 \\
    \multicolumn{1}{r}{} & c\_30 &       & \multicolumn{1}{r}{0,485} & \multicolumn{1}{r}{1,077} & \multicolumn{1}{r}{1,103} & \multicolumn{1}{r}{1,061} & \multicolumn{1}{r}{1,117} & 1,149 & 1,087 & 0,536 \\
    \multicolumn{1}{r}{} &       &       & \multicolumn{1}{r}{} & \multicolumn{1}{r}{} & \multicolumn{1}{r}{} & \multicolumn{1}{r}{} & \multicolumn{1}{r}{} &       &       &  \\
    \multicolumn{11}{l}{\textit{\textbf{Inertialkoordinaten von x\_0}}} \\
    \multicolumn{1}{r}{} & x\_0  & =     & \multicolumn{1}{r}{0} & \multicolumn{1}{r}{} & \multicolumn{1}{r}{} & \multicolumn{1}{r}{} & \multicolumn{1}{r}{} &       &       &  \\
    \multicolumn{1}{r}{} & y\_0  & =     & \multicolumn{1}{r}{0} & \multicolumn{1}{r}{} & \multicolumn{1}{r}{} & \multicolumn{1}{r}{} & \multicolumn{1}{r}{} &       &       &  \\
    \multicolumn{1}{r}{} & z\_0  & =     & \multicolumn{1}{r}{0} & \multicolumn{1}{r}{} & \multicolumn{1}{r}{} & \multicolumn{1}{r}{} & \multicolumn{1}{r}{} &       &       &  \\
    \multicolumn{1}{r}{} &       &       & \multicolumn{1}{r}{} & \multicolumn{1}{r}{} & \multicolumn{1}{r}{} & \multicolumn{1}{r}{} & \multicolumn{1}{r}{} &       &       &  \\
    \multicolumn{11}{l}{\textit{\textbf{Ermittelte Koordinaten}}} \\
    \multicolumn{1}{r}{\textit{\textbf{}}} &       &       & \multicolumn{1}{r}{\textbf{1}} & \multicolumn{1}{r}{\textbf{2}} & \multicolumn{1}{r}{\textbf{3}} & \multicolumn{1}{r}{\textbf{4}} & \multicolumn{1}{r}{\textbf{5}} & \textbf{6} & \textbf{7} & \textbf{8}\\
\cline{2-11}    \multicolumn{1}{r}{} & \textbf{x\_i} & \textbf{=} & \multicolumn{1}{r}{0,479} & \multicolumn{1}{r}{-0,775} & \multicolumn{1}{r}{1,521} & \multicolumn{1}{r}{-0,924} & \multicolumn{1}{r}{1,921} & -0,556 & 1,063 & 0,454\\
    \multicolumn{1}{r}{} & \textbf{y\_i} & \textbf{=} & \multicolumn{1}{r}{-1,012} & \multicolumn{1}{r}{-1,045} & \multicolumn{1}{r}{-1,053} & \multicolumn{1}{r}{-0,197} & \multicolumn{1}{r}{0,031} & 1,092 & 1,072 & 1,355 \\
    \multicolumn{1}{r}{} & \textbf{z\_i} & \textbf{=} & \multicolumn{1}{r}{0,607} & \multicolumn{1}{r}{1,346} & \multicolumn{1}{r}{1,379} & \multicolumn{1}{r}{1,326} & \multicolumn{1}{r}{1,396} & 1,436 & 1,358 & 0,670 \\
    \multicolumn{1}{r}{} &       &       & \multicolumn{1}{r}{} & \multicolumn{1}{r}{} & \multicolumn{1}{r}{} & \multicolumn{1}{r}{} & \multicolumn{1}{r}{} &       &       &  \\
    \multicolumn{11}{l}{\textit{\textbf{Fehlerbetrachtung}}} \\
    \multicolumn{1}{r}{} &       &       & \multicolumn{1}{r}{\textbf{1}} & \multicolumn{1}{r}{\textbf{2}} & \multicolumn{1}{r}{\textbf{3}} & \multicolumn{1}{r}{\textbf{4}} & \multicolumn{1}{r}{\textbf{5}} & \textbf{6} & \textbf{7} & \textbf{8}\\
\cline{2-11}    \multicolumn{1}{r}{} & \multicolumn{2}{r}{gemessen [m]} & \multicolumn{1}{r}{1,259} & \multicolumn{1}{r}{1,894} & \multicolumn{1}{r}{2,334} & \multicolumn{1}{r}{1,661} & \multicolumn{1}{r}{2,399} & 1,851 & 2,055 & 1,574\\
    \multicolumn{1}{r}{} & \multicolumn{2}{r}{berechnet [m]} & \multicolumn{1}{r}{1,274} & \multicolumn{1}{r}{1,872} & \multicolumn{1}{r}{2,307} & \multicolumn{1}{r}{1,628} & \multicolumn{1}{r}{2,375} & 1,887 & 2,031 & 1,578 \\
    \multicolumn{1}{r}{} & \multicolumn{2}{r}{Fehler ABS [m]} & \multicolumn{1}{r}{0,015} & \multicolumn{1}{r}{0,022} & \multicolumn{1}{r}{0,027} & \multicolumn{1}{r}{0,033} & \multicolumn{1}{r}{0,024} & 0,036 & 0,024 & 0,004 \\
    \multicolumn{1}{r}{} & \multicolumn{2}{r}{\textbf{Fehler \%}} & \multicolumn{1}{r}{\textbf{1,14\%}} & \multicolumn{1}{r}{\textbf{1,19\%}} & \multicolumn{1}{r}{\textbf{1,15\%}} & \multicolumn{1}{r}{\textbf{2,01\%}} & \multicolumn{1}{r}{\textbf{1,01\%}} & \textbf{1,93\%} & \textbf{1,19\%} & \textbf{0,26\%} \\
    \end{tabular}%
  \label{tab:addlabel}%
\end{table}%

	\label{sec:coordinate_Measurements}
%----------------------------------------------------------------------------
%----------------------------------------------------------------------------
%----------------------------------------------------------------------------
\newpage
	\section{Berechnungen Messaufbau}
%
	% Table generated by Excel2LaTeX from sheet 'numerisch'
\begin{table}[h]
  \centering
  \caption{Hier gezeigt wird das Ergebnis der Berechnung der Antennenposition. Die Datengrundlage ist dieselbe wie die in Anhang~\ref{sec:coordinate_Measurements}. Die Ergebnisse sind identisch und wurden mit dem Excel Solver bestimmt. Dabei wurde die quadratische Fehlersumme minimiert. ($min(\Sigma\epsilon^2)$).}
    \vspace{0.5cm}
    \begin{tabular}{r|rrr|r|r|r|r|rrr}
    \multicolumn{11}{l}{\textit{\textbf{Lineares Modell für die Koordinatenberechnung}}} \\
          & \multicolumn{1}{c}{{0,77}} & \multicolumn{1}{c}{{0}} & \multicolumn{1}{c|}{{0}} &       & \multicolumn{1}{c|}{${x-x_0}$} &       & \multicolumn{1}{c|}{${c_{10}}$} &       &       &  \\
          & \multicolumn{1}{c}{{0}} & \multicolumn{1}{c}{{0,77}} & \multicolumn{1}{c|}{{0}} & \multicolumn{1}{c|}{*} & \multicolumn{1}{c|}{${y-y_0}$} & \multicolumn{1}{c|}{=} & \multicolumn{1}{c|}{${c_{20}}$} &       &       &  \\
          & \multicolumn{1}{c}{{0}} & \multicolumn{1}{c}{{0}} & \multicolumn{1}{c|}{{0,8}} &       & \multicolumn{1}{c|}{${z-z_0}$} &       & \multicolumn{1}{c|}{${c_{30}}$} &       &       &  \\
    \multicolumn{1}{r}{} & \multicolumn{1}{c}{\textbf{}} & \multicolumn{1}{c}{\textbf{}} & \multicolumn{1}{c}{\textbf{}} & \multicolumn{1}{r}{} & \multicolumn{1}{c}{\textit{\textbf{}}} & \multicolumn{1}{r}{} & \multicolumn{1}{c}{\textbf{}} &       &       &  \\
    \multicolumn{1}{r}{} & \multicolumn{3}{c}{\textbf{A}} & \multicolumn{1}{c}{*} & \multicolumn{1}{c}{\textit{\textbf{x}}} & \multicolumn{1}{c}{=} & \multicolumn{1}{c}{\textbf{b}} & \textbf{} &       &  \\
    \multicolumn{11}{r}{} \\
    \multicolumn{11}{l}{\textit{\textbf{Vorhersage vom Solver}}} \\
    \multicolumn{1}{r}{} &       &       & \multicolumn{1}{r}{\textbf{1}} & \multicolumn{1}{r}{\textbf{2}} & \multicolumn{1}{r}{\textbf{3}} & \multicolumn{1}{r}{\textbf{4}} & \multicolumn{1}{r}{\textbf{5}} & \textbf{6} & \textbf{7} & \textbf{8} \\
\cline{2-11}    \multicolumn{1}{r}{} & $c_{10}'$ &       & \multicolumn{1}{r}{0,369} & \multicolumn{1}{r}{-0,596} & \multicolumn{1}{r}{1,171} & \multicolumn{1}{r}{-0,711} & \multicolumn{1}{r}{1,479} & -0,428 & 0,818 & 0,349 \\
    \multicolumn{1}{r}{} & $c_{20}'$ &       & \multicolumn{1}{r}{-0,779} & \multicolumn{1}{r}{-0,804} & \multicolumn{1}{r}{-0,811} & \multicolumn{1}{r}{-0,152} & \multicolumn{1}{r}{0,024} & 0,841 & 0,826 & 1,043 \\
    \multicolumn{1}{r}{} & {$c_{30}'$} &       & \multicolumn{1}{r}{0,485} & \multicolumn{1}{r}{1,077} & \multicolumn{1}{r}{1,103} & \multicolumn{1}{r}{1,061} & \multicolumn{1}{r}{1,117} & 1,149 & 1,087 & 0,536 \\
    \multicolumn{1}{r}{} & $min(\Sigma\epsilon^2)$ &       & \multicolumn{1}{r}{0,000} & \multicolumn{1}{r}{0,000} & \multicolumn{1}{r}{0,000} & \multicolumn{1}{r}{0,000} & \multicolumn{1}{r}{0,000} & 0,000 & 0,000 & 0,000 \\
    \multicolumn{1}{r}{} &       &       & \multicolumn{1}{r}{} & \multicolumn{1}{r}{} & \multicolumn{1}{r}{} & \multicolumn{1}{r}{} & \multicolumn{1}{r}{} &       &       &  \\
    \multicolumn{11}{l}{\textit{\textbf{Lösung des Modells}}} \\
    \multicolumn{1}{r}{} &       &       & \multicolumn{1}{r}{\textbf{1}} & \multicolumn{1}{r}{\textbf{2}} & \multicolumn{1}{r}{\textbf{3}} & \multicolumn{1}{r}{\textbf{4}} & \multicolumn{1}{r}{\textbf{5}} & \textbf{6} & \textbf{7} & \textbf{8} \\
\cline{2-11}    \multicolumn{1}{r}{} &       &       & \multicolumn{1}{r}{0,479} & \multicolumn{1}{r}{-0,775} & \multicolumn{1}{r}{1,521} & \multicolumn{1}{r}{-0,924} & \multicolumn{1}{r}{1,921} & -0,556 & 1,063 & 0,454\\
    \multicolumn{1}{r}{} & $\mathbf{x_k}$ & =     & \multicolumn{1}{r}{-1,012} & \multicolumn{1}{r}{-1,045} & \multicolumn{1}{r}{-1,053} & \multicolumn{1}{r}{-0,197} & \multicolumn{1}{r}{0,031} & 1,092 & 1,072 & 1,355 \\
    \multicolumn{1}{r}{} &       &       & \multicolumn{1}{r}{0,607} & \multicolumn{1}{r}{1,346} & \multicolumn{1}{r}{1,379} & \multicolumn{1}{r}{1,326} & \multicolumn{1}{r}{1,396} & 1,436 & 1,358 & 0,670 \\
    \multicolumn{1}{r}{} &       &       & \multicolumn{1}{r}{} & \multicolumn{1}{r}{} & \multicolumn{1}{r}{} & \multicolumn{1}{r}{} & \multicolumn{1}{r}{} &       &       &  \\
    \multicolumn{1}{r}{} &       &       & \multicolumn{1}{r}{0,369} & \multicolumn{1}{r}{-0,596} & \multicolumn{1}{r}{1,171} & \multicolumn{1}{r}{-0,711} & \multicolumn{1}{r}{1,479} & -0,428 & 0,818 & 0,349 \\
    \multicolumn{1}{r}{} & $\mathbf{b_k}$ & =     & \multicolumn{1}{r}{-0,779} & \multicolumn{1}{r}{-0,804} & \multicolumn{1}{r}{-0,811} & \multicolumn{1}{r}{-0,152} & \multicolumn{1}{r}{0,024} & 0,841 & 0,826 & 1,043 \\
    \multicolumn{1}{r}{} &       &       & \multicolumn{1}{r}{0,485} & \multicolumn{1}{r}{1,077} & \multicolumn{1}{r}{1,103} & \multicolumn{1}{r}{1,061} & \multicolumn{1}{r}{1,117} & 1,149 & 1,087 & 0,536 \\
    \multicolumn{1}{r}{} &       &       & \multicolumn{1}{r}{} & \multicolumn{1}{r}{} & \multicolumn{1}{r}{} & \multicolumn{1}{r}{} & \multicolumn{1}{r}{} &       &       &  \\
    \multicolumn{11}{l}{\textit{\textbf{Inertialkoordinaten von x\_0}}} \\
    \multicolumn{1}{r}{} & x\_0  & =     & \multicolumn{1}{r}{0} & \multicolumn{1}{r}{} & \multicolumn{1}{r}{} & \multicolumn{1}{r}{} & \multicolumn{1}{r}{} &       &       &  \\
    \multicolumn{1}{r}{} & y\_0  & =     & \multicolumn{1}{r}{0} & \multicolumn{1}{r}{} & \multicolumn{1}{r}{} & \multicolumn{1}{r}{} & \multicolumn{1}{r}{} &       &       &  \\
    \multicolumn{1}{r}{} & z\_0  & =     & \multicolumn{1}{r}{0} & \multicolumn{1}{r}{} & \multicolumn{1}{r}{} & \multicolumn{1}{r}{} & \multicolumn{1}{r}{} &       &       &  \\
    \multicolumn{1}{r}{} &       &       & \multicolumn{1}{r}{} & \multicolumn{1}{r}{} & \multicolumn{1}{r}{} & \multicolumn{1}{r}{} & \multicolumn{1}{r}{} &       &       &  \\
    \multicolumn{11}{l}{\textit{\textbf{Ermittelte Koordinaten}}} \\
    \multicolumn{1}{r}{\textit{\textbf{}}} &       &       & \multicolumn{1}{r}{\textbf{1}} & \multicolumn{1}{r}{\textbf{2}} & \multicolumn{1}{r}{\textbf{3}} & \multicolumn{1}{r}{\textbf{4}} & \multicolumn{1}{r}{\textbf{5}} & \textbf{6} & \textbf{7} & \textbf{8} \\
\cline{2-11}    \multicolumn{1}{r}{} & ${x_k}$ & \textbf{=} & \multicolumn{1}{r}{0,479} & \multicolumn{1}{r}{-0,775} & \multicolumn{1}{r}{1,521} & \multicolumn{1}{r}{-0,924} & \multicolumn{1}{r}{1,921} & -0,556 & 1,063 & 0,454 \\
    \multicolumn{1}{r}{} & ${y_k}$ & \textbf{=} & \multicolumn{1}{r}{-1,012} & \multicolumn{1}{r}{-1,045} & \multicolumn{1}{r}{-1,053} & \multicolumn{1}{r}{-0,197} & \multicolumn{1}{r}{0,031} & 1,092 & 1,072 & 1,355 \\
    \multicolumn{1}{r}{} & ${z_k}$ & \textbf{=} & \multicolumn{1}{r}{0,607} & \multicolumn{1}{r}{1,346} & \multicolumn{1}{r}{1,379} & \multicolumn{1}{r}{1,326} & \multicolumn{1}{r}{1,396} & 1,436 & 1,358 & 0,670 \\
    \multicolumn{1}{r}{} &       &       & \multicolumn{1}{r}{} & \multicolumn{1}{r}{} & \multicolumn{1}{r}{} & \multicolumn{1}{r}{} & \multicolumn{1}{r}{} &       &       &  \\
    \multicolumn{11}{l}{\textit{\textbf{Fehlerbetrachtung}}} \\
    \multicolumn{1}{r}{} &       &       & \multicolumn{1}{r}{\textbf{1}} & \multicolumn{1}{r}{\textbf{2}} & \multicolumn{1}{r}{\textbf{3}} & \multicolumn{1}{r}{\textbf{4}} & \multicolumn{1}{r}{\textbf{5}} & \textbf{6} & \textbf{7} & \textbf{8} \\
\cline{2-11}    \multicolumn{1}{r}{} & \multicolumn{2}{r}{gemessen [m]} & \multicolumn{1}{r}{1,259} & \multicolumn{1}{r}{1,894} & \multicolumn{1}{r}{2,334} & \multicolumn{1}{r}{1,661} & \multicolumn{1}{r}{2,399} & 1,851 & 2,055 & 1,574 \\
    \multicolumn{1}{r}{} & \multicolumn{2}{r}{berechnet [m]} & \multicolumn{1}{r}{1,274} & \multicolumn{1}{r}{1,872} & \multicolumn{1}{r}{2,307} & \multicolumn{1}{r}{1,628} & \multicolumn{1}{r}{2,375} & 1,887 & 2,031 & 1,578 \\
    \multicolumn{1}{r}{} & \multicolumn{2}{r}{Fehler ABS [m]} & \multicolumn{1}{r}{0,015} & \multicolumn{1}{r}{0,022} & \multicolumn{1}{r}{0,027} & \multicolumn{1}{r}{0,033} & \multicolumn{1}{r}{0,024} & 0,036 & 0,024 & 0,004 \\
    \multicolumn{1}{r}{} & \multicolumn{2}{r}{\textbf{Fehler \%}} & \multicolumn{1}{r}{\textbf{1,14\%}} & \multicolumn{1}{r}{\textbf{1,19\%}} & \multicolumn{1}{r}{\textbf{1,15\%}} & \multicolumn{1}{r}{\textbf{2,01\%}} & \multicolumn{1}{r}{\textbf{1,01\%}} & \textbf{1,93\%} & \textbf{1,19\%} & \textbf{0,26\%} \\
    \end{tabular}%
  \label{tab:addlabel}%
\end{table}%

	\label{sec:coordinate_Measurements_Numerical}
%----------------------------------------------------------------------------
%----------------------------------------------------------------------------
%----------------------------------------------------------------------------
\newpage
\begin{landscape}
	\section{Projektlaufplan KW 21}
	\label{sec:projectplan}
	\scalebox{.75}{
		\begin{ganttchart}[vgrid={draw=none,*1{gray, dashed}},
				hgrid=true,
				today=12,
				title height=1,
				y unit title=0.6cm,
				y unit chart=0.8cm,
				group right shift=0,
				group top shift=.3,
				group height=.3,
				milestone width=.8,
				group peaks={}{}{.2},
				incomplete/.style={fill=black!15}, %
				bar/.style={fill=white}, %
				today label={Heute},
				today rule/.style={dashed, thick}]{44}


\gantttitle{\textbf{2013}}{44} \\
\gantttitlelist{16,...,37}{2} \\
%-------------------------------------------------------------
\ganttgroup{Projekt Evaluation}{3}{14} \\
\ganttbar[progress=100, progress label font=\small\color{black!75},
	progress label anchor/.style={right=4pt}]{Installation der Umgebungen}{3}{6} \\
	
\ganttbar[progress=100, progress label font=\small\color{black!75},
	progress label anchor/.style={right=4pt},
	bar label font=\normalsize\color{black},
	name=rech]{Recherche}{3}{7} \\
	
\ganttmilestone[name=ms1]{Vorstellung der Ergebnisse}{7} \\
	
\ganttbar[progress=90, progress label font=\small\color{black!75},
	progress label anchor/.style={right=4pt},
	bar label font=\normalsize\color{black},
	name=pflichten]
	{Pflichtenheft}{5}{8} \\
	
\ganttmilestone[name=ms2]{Pflichtenheft fertig}{8} \\

\ganttbar[progress=70, progress label font=\small\color{black!75},
	progress label anchor/.style={right=4pt},
	bar label font=\normalsize\color{black},
	name=bNumVerf]
	{Einarbeitung num. Verfahren}{5}{12} \\

\ganttbar[progress=20, progress label font=\small\color{black!75},
	progress label anchor/.style={right=34pt},
	bar label font=\normalsize\color{black},
	name=bCMAES]
	{speziell CMA-ES}{7}{10} \\

\ganttmilestone[name=ms3]{Beurteilung num. Verfahren}{12} \\

\ganttlinkedbar[progress=10, progress label font=\small\color{black!75},
	progress label anchor/.style={right=34pt},
	bar label font=\normalsize\color{black}]
	{Shark Einarbeitung}{13}{14} \\

\ganttlinkedmilestone[name=ms7]{Abschluss Evaluation}{14} \\
	
%-------------------------------------------------------------
\ganttgroup{Erstellung Prototyp}{15}{26} \\
\ganttgroup{(optional)}{15}{18} \\
\ganttbar[progress=25, progress label font=\small\color{black!75},
	progress label anchor/.style={right=4pt},
	bar label font=\normalsize\color{black}]
	{(Entwurf digi. Filter)}{15}{15} \\

\ganttlinkedbar[progress=10, progress label font=\small\color{black!75},
	progress label anchor/.style={right=4pt},
	bar label font=\normalsize\color{black},
	name=bImpFPGA]
	{(Implementation FPGA)}{16}{18} \\

\ganttmilestone[name=ms4]{(Verifikation dig. Filter)}{18} \\
	
\ganttbar[progress=0, progress label font=\small\color{black!75},
	progress label anchor/.style={right=4pt},
	bar label font=\normalsize\color{black},
	name=bImplAlgo]
	{Implementation Algorithmus}{15}{26} \\

\ganttlinkedmilestone[name=ms5]{Implementation Done}{26} \\

%-------------------------------------------------------------
\ganttgroup{Verifikation}{27}{34} \\
\ganttbar[progress=0, progress label font=\small\color{black!75},
	progress label anchor/.style={right=4pt},
	bar label font=\normalsize\color{black},
	name=bVerf]
	{Durchf\"uhrung Verifikation}{27}{34} \\

\ganttlinkedmilestone[name=ms6]{Verifikation Done}{34} \\

%-------------------------------------------------------------
\ganttgroup{Projektdokumentation}{35}{42} \\

\ganttbar[progress=0, progress label font=\small\color{black!75},
	progress label anchor/.style={right=4pt},
	bar label font=\normalsize\color{black},
	name=thesis]
	{Thesis schreiben}{35}{42} \\
	
\ganttmilestone[name=msthesis,milestone label font=\color{red}, 
	milestone/.style={fill=red}]{Abgabe}{42}

%\ganttlink{ms7}{bImplAlgo}
\ganttlink{bImpFPGA}{ms4}
\ganttlink{bNumVerf}{ms3}
\ganttlink{bCMAES}{ms3}
\ganttlink{rech}{ms1}
\ganttlink{pflichten}{ms2}
\ganttlink{thesis}{msthesis}

%
%\gantttitle{17}{1}
%\gantttitle{18}{1}
%\gantttitle{19}{1}
%\gantttitle{20}{1}
%\gantttitle{21}{1}
%\gantttitle{KW22}{1}
%\gantttitle{KW23}{1}
%\gantttitle{KW24}{1}
%\gantttitle{2011}{12} \\
%\ganttbar
%[progress=100, progress label font=\small\color{black!75},
%progress label anchor/.style={right=4pt},
%bar label font=\normalsize\color{black},
%name=pp]
%{Preliminary Project}{1}{4} \\
%\ganttset{progress label text={}, link/.style={black, -to}}
%\ganttgroup{Objective 1}{5}{16} \\
%\ganttbar[progress=4, name=T1A]{Task A}{5}{10} \\
%\ganttlinkedbar[progress=0]{Task B}{11}{16} \\
%\ganttgroup{Objective 2}{5}{16} \\
%\ganttbar[progress=15, name=T2A]{Task A}{5}{13} \\
%\ganttlinkedbar[progress=0]{Task B}{14}{16} \\
%\ganttgroup{Objective 3}{9}{12} \\
%\ganttbar[progress=0]{Task A}{9}{12}
%\ganttset{link/.style={OliveGreen}}
%\ganttlink[link mid=.4]{pp}{T1A}
%\ganttlink[link mid=.159]{pp}{T2A}

%	[canvas/
%	.style={fill=none, draw=black!5, line width=.75pt},
%		hgrid style/.style={draw=black!5, line width=.75pt},
%		vgrid={*1{draw=black!5, line width=.75pt}},
%		today=7.1,
%		today rule/.style={draw=black!64,
%		dash pattern=on 3.5pt off 4.5pt, line width=1.5pt},
%		today label={\small\bfseries TODAY},
%		title/.style={draw=none, fill=none},
%		title label font=\bfseries\footnotesize,
%		title label anchor/.style={below=7pt},
%		include title in canvas=false,
%		bar label font=\mdseries\small\color{black!70},
%		bar label anchor/.style={left=2cm},
%		bar/.style={draw=none, fill=white},
%		bar incomplete/.style={fill=barblue},
%		progress label font=\mdseries\footnotesize\color{black!70},
%		group incomplete/.style={fill=groupblue},
%		group left shift=0,
%		group right shift=0,
%		group height=.5,
%		group peaks={0}{}{},
%		group label anchor/.style={left=.6cm},
%		link/.style={-latex, line width=1.5pt},
%		link label font=\scriptsize\bfseries,
%		link label anchor/.style={below left=-2pt and 0pt}
%	]{13}
	
%	\gantttitle[title label anchor/.style={below left=7pt and -3pt}]%
%	{WEEKS:\quad1}{1}
%	\gantttitlelist{1,...,13}{1} \\
%	\ganttgroup[progress=57, progress label font=\bfseries\small]%
%	{WBS 1 Summary Element 1}{1}{10} \\
%	\ganttbar[progress=75, name=WBS1A]%
%	{\textbf{WBS 1.1} Activity A}{1}{8} \\
%	\ganttbar[progress=67, name=WBS1B]%
%	{\textbf{WBS 1.2} Activity B}{1}{3} \\
%	\ganttbar[progress=50, name=WBS1C]%
%	{\textbf{WBS 1.3} Activity C}{4}{10} \\
%	\ganttbar[progress=0, name=WBS1D]%
%	{\textbf{WBS 1.4} Activity D}{4}{10} \\[grid]
%	\ganttgroup[progress=0, progress label font=\bfseries\small]%
%	{WBS 2 Summary Element 2}{4}{10} \\
%	\ganttbar[progress=0]{\textbf{WBS 2.1} Activity E}{4}{5} \\
%	\ganttbar[progress=0]{\textbf{WBS 2.2} Activity F}{6}{8} \\
%	\ganttbar[progress=0]{\textbf{WBS 2.3} Activity G}{9}{10}
%	\ganttlink[link type=s-s]{WBS1A}{WBS1B}
%	\ganttlink[link type=f-s]{WBS1B}{WBS1C}
%	\ganttlink[link type=f-f, link label anchor/.style={left}]{WBS1C}{WBS1D}
%	
	\end{ganttchart}
		}
\end{landscape}

%----------------------------------------------------------------------------

\end{appendix}
