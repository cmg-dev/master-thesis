%- Section 1 ----------------------------------------------------------------
\section[Allgemeines]{Evolutionsstrategien}
Nach dem Vorbild natürlicher Evolution entworfene stochastische Optimierungsverfahren werden Evolutionsstrategie bezeichnet. Sie verwenden die Prinzipien der Mutation, Rekombination und Selektion analog zu der nat. Evolution.\\
Wie in der Natur auch werden Nachkommen aus der Menge der verfügbaren Eltern gebildet. Dabei bezeichnet im Folgenden:
\begin{itemize}
\item $\mu$ die Anzahl der Eltern (=> Größe der Population)
\item $\lambda$ die Anzahl der Eltern die bei Rekombination neue Kinder erzeugt; Die Anzahl der erzeugten Nachkommen einer neuen Generation
\item $\mathbf{x}_p$ Elternpunkt (Parent)
\item $\mathbf{x}_c$ Nachkomme einer Generation (Child)
\item $X_p^1$ Die Menge aller Eltern der ersten Generation $X_p = \{\mathbf{x}_{p_1}^1,..,\mathbf{x}_{p_\mu}^1\}$
\item $X_p^k$ Die Menge aller Eltern der k-ten Generation $X_p = \{\mathbf{x}_{p_1}^k,..,\mathbf{x}_{p_\mu}^k\}$
\end{itemize}
%
%- Section 1.1 ----------------------------------------------------------------
\subsection[Mutation]{Mutation}
Ein Nachkomme $\mathbf{x}_C$ wird aus seinem Elternteil $\mathbf{x}_P$ und einer zufälligen Variation $\mathbf{d}$ gebildet.
\begin{equation} \label{eq:Mutation_Child}
	\mathbf{x}_c = \mathbf{x}_P + \mathbf{d}
\end{equation}
Dabei ist $\mathbf{d}$ ein bei jeder Mutation neu zu bestimmender $(0,\sigma^2)-normalverteilte$ Zufallszahl $Z(0,\sigma^2)$:
\begin{equation}\label{eq:wavenumber_trilateration_model}
\mathbf{d}=
\left(
	\begin{array}{c}
		d_1 \\
		\vdots\\
		d_n 
	\end{array}
\right)
=
\left(
	\begin{array}{c}
		Z(0,\sigma_1^2) \\
		\vdots\\
		Z(0,\sigma_n^2) 
	\end{array}
\right)
=
\left(
	\begin{array}{c}
		Z(0,1) \sigma_1 \\
		\vdots\\
		Z(0,1) \sigma_n 
	\end{array}
\right)
\end{equation}
%
Die Normalverteilung der Variation ist nützlich, da kleine Änderungen wahrscheinlicher sind als große. Die maximale Größe der Variation wird durch die Standardabweichung $\sigma_i$ bestimmt.
%
%- Section 1.2 ----------------------------------------------------------------
\subsection[Rekombination]{Rekombination}
Durch Rekombination zweier oder mehr Eltern aus der Menge aller $\mu$-Eltern $X_{\varrho} \subset X_E$. Die Wahl der Eltern sollte zufällig erfolgen um Inzuchtprobleme zu verhindern.\\
Zwei Arten der Rekombination sind denkbar:\\

Die \textit{intermediär Rekombination} erstellt einen Nachkommen durch das gewichtete Mittel von $\varrho$ Eltern.
%
\begin{equation}
\mathbf{x}_c = \Sigma^\varrho_{i=1} \alpha_i\mathbf{x}_{p_i},\\ \Sigma^\varrho_{i=1} \alpha_i = 1,\\ 2\leq\varrho\leq\mu
\end{equation} 
%

Bei der \textit{diskreten Rekombination} vom $\varrho$-Eltern wird die \textit{i}-te Komponente $x_{ic}$ eines Nachkommen $\mathbf{x}_c$ mit der \textit{i}-te Komponente eines zufällig gewählten Elternpunktes gleichgesetzt.
%
\begin{equation}
\mathbf{x}_{ic} = \mathbf{x}_{ip_j},\\ j\in\{1,...,\varrho\},\\i=1,...,n
\end{equation} 
%
%- Section 1.3 ----------------------------------------------------------------
\subsection[Selektion]{Selektion}
Die durch Rekombination und/oder Mutation erzeugten Nachkommen werden in dem Schritt Ausgewählt um einen Evolutionsfortschritt zu erreichen. Dies erfolgt anhand des Vergleichs mit dem Zielfunktionswert $f(\mathbf{x})$. Das beste Individuum oder die besten werden für die nachfolgende Generation ausgewählt. Dabei gibt es Strategien bei denen nur die Nachkommen an der Auswahl beteiligt sind und welche bei denen Eltern und Kinder teilnehmen.

%- Section 2.3 ----------------------------------------------------------------
\subsection{Evolutionsalgorithmus}
%
Der eigentliche Evolutionsalgorithmus ist in Abbildung~\ref{fig:es_flowchart} dargestellt. Er enthält im wesentlichen die in den vorherigen Abschnitten beschriebenen Schritte. Der prinzipielle Ablauf ist für alle Evolutionsalgorithmen gleich. Eine Unterscheidung der Verfahren kann durch verschiedene Parameter beschrieben werden. Wesentlich dabei sind die Populationsgröße $\mu$, die Anzahl an der Rekombination beteiligten Eltern $\varrho$, die gewählte Selektionsstrategie sowie die Anzahl der Nachkommen $\lambda$. Im Folgenden sind zuerst einige Beispiele für die Nomenklatur der Selektionsstrategie aufgeführt, die im Anschluss genauer beschrieben werden.\\
Für Strategien die nur auf Mutation für die Erzeugung von Nachkommen setzten sind folgende Nomenklaturen gebräuchlich:
\begin{itemize}
\item $(\mu+\lambda)$ Elternelemente werden in der Selektion berücksichtigt
\item $(\mu,\lambda)$ Ausschließlich Nachkommen nehmen an der Selektion teil
\end{itemize}
%
Wird die Rekombination eingesetzt kann auch die Anzahl der beteiligten Elternelemente angegeben werden:
\begin{itemize}
\item $({\mu}/{\varrho}+\lambda)$ \& $({\mu}/{\varrho},\lambda)$ Angabe der Anzahl von beteiligten Eltern bei der Rekombination.
\end{itemize}
%
Mithilfe dieser Klassifikation werden die Algorithmen im Folgenden stets angegeben.

\begin{figure}[h]
	\begin{center}
		\caption[Kurzeintrag]{lorem ipsum}
		\vspace{0.5cm}
		\begin{tikzpicture}[auto]
		\scriptsize
			\tikzstyle{decision} = [diamond, draw=black, thick, fill=black!20, text width=5em, text badly centered, inner sep=1pt]
%			
			\tikzstyle{block} = [rectangle, draw=black, thick, fill=gray!20, text width=15em, text centered, rounded corners, minimum height=4em]
%	
			\tikzstyle{line} = [draw, thick, -latex',shorten >=1pt];
			\tikzstyle{commentline} = [draw, dashed, green!50,-latex',shorten >=1pt];
%	
			\tikzstyle{cloud} = [ dotted, draw=green!50, thick, ellipse,,fill=green!20, minimum height=2em, text width= 10em];
%	
			\matrix [column sep=5mm,row sep=7mm]
			{
				% row 1
				& \node [block] (start) { Start }; & \\
				% row 2
				&\node [block] (init) {Erstelle Startpopulation $X_p^1$ bestehend aus $\mu$-Individuen }; & 
				\node [cloud] (comment1) {Initialisierung}; \\
				% row 4
				& \node [block] (identify) {Erzeuge eine Menge von $\lambda$ Nachkommen $X_c^k$ aus der aktuellen Elterngeneration $X_p^k$ durch Rekombination \& Mutation}; & \\
				% row 5
				\node [block] (update) {Nächste Stufe der Evolution; k++}; &
				\node [block] (evaluate) {Durch Selektion die besten $\mu$ Nachkommen für die Generation $X_p^{k+1}$ auswählen}; & \\
				% row 6
				& \node [decision] (decide) {$\Delta \geq \Delta_{min}$}; & 
				\node [cloud] (criteria) {Abbruchkriterium; Muss geeignet gewählt werden, bspw. max. Anzahl der Generationen oder Erreichen des Optimums};\\
				% row 7
				& \node [block] (stop) {Ende}; & \\
			};
% Arrows
			\tikzstyle{every path}=[line]
			\path (init) -- (identify);
			\path (identify) -- (evaluate);
			\path (evaluate) -- (decide);
			\path (update) |- (identify);
			\path (decide) -| node [near start] {Ja} (update);
			\path (decide) -- node [midway] {Nein} (stop);
			\path (start) -- (init);
			
			\tikzstyle{every path}=[commentline]
			\path (criteria) -- (decide);
			\path (comment1) -- (init);
			
		\end{tikzpicture}
	\end{center}
\end{figure}

