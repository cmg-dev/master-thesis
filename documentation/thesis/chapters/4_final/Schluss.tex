%
\section{Ausblick}
\label{sec:Calibration_Optimaztion}
%
Wie in den Ergebnissen gezeigt, ist die Lösung nicht immer zufriedenstellend gefunden worden. Die Ergebnisse dieser Arbeit weisen relativ große Abweichungen in der gefundenen z-Koordinate auf. Das wird aus den in den Ergebnissen präsentieren Grafiken deutlich. Wie in der Analyse der Komplexität, Kapitel~\ref{sec:Komplexity2}, geschildert ist der Suchraum sehr komplex. Es ist möglich durch vorhandene Informationen die Problemdimension zu reduzieren. Ein darauf basierendes Modell wurde in dieser Arbeit nicht untersucht. Die Umsetzung eines solchen Modells ist mit dem in dieser Arbeit geschaffenen Rahmen jedoch ohne großen Aufwand möglich. Es muss lediglich eine neue Objektfunktion erstellt werden.

Ungünstige Umstände (schlechte Konditionierung, geringe Anzahl an Antennen etc.) können zu den unzureichenden Ergebnissen beitragen. Zur Zeit wird in einem solchen Fall ein statistischer Ansatz durchgeführt. Dazu werden mehrere Lösungsversuche durchgeführt und anschließend der Median-Wert der Variablen für die Abschätzung genutzt. Das Verbessert das Ergebnis zu Lasten der Ausführungszeit. Die Zeit bis ein Resultat vorliegt kann sich zum Teil auf mehrere Sekunden vergrößern. Das liegt außerhalb der Anforderungen. Hier muss durch weitere empirische Analysen ein gutes Mittelmaß gefunden werden.\\
%

Es ist im Weiteren auch nicht Sinnvoll die in dieser Arbeit gefundene Implementation als vollständig richtig anzunehmen. Obwohl auf die Verifikation einzelner Programmteile genau geachtet wurde (siehe z.B. Kapitel~\ref{sec:calibration}) konnte die Implementation des Modells mit keiner Referenz verglichen werden. Alle Eingaben in das Modell, wie die statischen Matrizen und der Ergebnisvektor, wurden von Hand berechnet und konnten so geprüft werden. Allerdings gibt es schlicht keine andere Umsetzung des Modells. Es kann ratsam sein, die Implementation einem Review zu unterziehen.
%

Im Rahmen dieser Arbeit konnte eine automatische, auf Messungen basierende Kalibrierung nicht mehr erprobt werden. Da hier vorgestellte Konzept der Entfernungsfindung eignet sich dazu ein solches System zu implementieren. Dazu könnte ein Phantom mit einer bekannten geometrischen Proportion mit Tags ausgestattet werden. Die Antennen würden die Phasenwerte ausgeben und die Positionen der Antennen würde optimiert werden können. Das bedeutet die Berechnung des umgekehrten Falls zur Entfernungsfindung eines Tags.\\
%

\begin{figure}[ht!]
         \centering
         \caption[Anforderungsspinne]{ Grafische Übersicht der Ergebnisse gegen die gestellten Anforderungen (blau) an diese Arbeit. Der Grüne Bereich stellt die erreichten Ziele dar. Wie bereits Diskutiert werden die Anforderungen an die Genauigkeit nicht gut erfüllt. Auch die Performance ist nicht zufriedenstellend erreicht, die Gründe wurden bereits angegeben. In den anderen Bereichen zeigt sich eine gute Deckung mit den Anforderungen. }
         \vspace{2mm}
         \label{fig:Requirements_reached}
         \input{diagrams/spider_requirements_reached.tex}
\end{figure}
%

\subsubsection{Weitere Anwendungsmöglichkeiten}
%
Das Modell, das in dieser Arbeit entwickelt wurde, ermöglicht eine Anwendung Abseits der Entfernungsfindung. Mit diesem Ansatz kann zur Lösung eines anderen Problems beitragen, das durch die Freie Anordnung der Antennen entsteht. Wenn mehrere Tags im Raum identifiziert werden, muss die Steuerung der Antennenumschaltung\footnote{Diese bestimmt von welcher Antenne gerade versucht wird die Tags zu lesen - es können nicht gleichzeitig alle Antennen gelesen werden} zur Zeit alle Antennen in einem Round-Robin-Verfahren umschalten. Nach einer Umschaltung kann es dazu kommen, dass keine der Antennen einen der zuvor identifizierten Tags "sieht". Da die Umschaltung zur Zeit zufällig zur nächsten Antenne übergeht, werden auch Antennen genommen die keine günstige Positionsberechnung erlauben. Die mögliche Lösung dafür ist die Bestimmung der Kondition für jede Konfiguration aus vier Antennen dazu zu verwenden eine intelligente Steuerung der Umschaltung zu ermöglichen. Dazu würden aus dem gewählten Antennenaufbau die Antennenkombinationen ihrer Konditionszahl nach Aufsteigend\footnote{zur Erinnerung, gute Kondition = kleine Konditionszahl} sortiert und diese Antennenkombinationen von der Antennenumschaltung bevorzugt. Dadurch werden stets die am besten konditionierte Kombinationen gewählt.
%

