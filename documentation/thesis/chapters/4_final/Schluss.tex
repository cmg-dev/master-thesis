%
\section{Ausblick}
\label{sec:Calibration_Optimaztion}
%
Wie in den Ergebnissen gezeigt, ist die Lösung nicht immer zufriedenstellend gefunden worden. Das wird aus den in den Ergebnissen präsentieren Grafiken deutlich. Wie in der Analyse der Komplexität, Kapitel~\ref{sec:Komplexity2}, geschildert, ist der Suchraum sehr komplex. Es ist ggf. möglich durch vorhandene Informationen die Problemdimension zu reduzieren. Ein darauf basierendes Modell wurde in dieser Arbeit nicht untersucht. Die Umsetzung eines solchen Modells ist mit dem in dieser Arbeit geschaffenen Rahmen jedoch ohne großen Aufwand möglich. Dazu muss lediglich eine neue Objektfunktion erstellt werden, wie in Kapitel~\ref{sec:Shark_model} beschrieben.\\

Ungünstige Umstände (z.B: schlechte Konditionierung, geringe Anzahl an Antennen etc.) können zu den unzureichenden Ergebnissen beitragen. Zur Zeit wird in einem solchen Fall ein statistischer Ansatz durchgeführt. Dazu werden mehrere Lösungsversuche durchgeführt und anschließend der Median-Wert der Variablen für die Abschätzung genutzt, siehe Kapitel~\ref{sec:Results1}. Das Verbessert das Ergebnis zu Lasten der Ausführungszeit. Die Zeit bis ein Resultat vorliegt kann sich zum Teil auf mehrere Sekunden vergrößern. Das liegt außerhalb der Anforderungen. Hier muss durch weitere empirische Analysen ein gutes Mittelmaß gefunden werden.\\
%

Es ist im Weiteren auch nicht sinnvoll die in dieser Arbeit gefundene Implementation als fehlerfrei zu betrachten. Obwohl auf die Verifikation einzelner Komponenten genau geachtet wurde (siehe z.B. Kapitel~\ref{sec:calibration}) und die Ergebnisse akzeptabel sind, konnte die Implementation des Modells mit keiner Referenz verglichen werden. Alle Eingaben in das Modell, wie die statischen Matrizen und der Ergebnisvektor, wurden von Hand berechnet und konnten so geprüft werden. Allerdings gibt es schlicht keine andere Umsetzung des Modells. Es ist ratsam, die Implementation einem Review zu unterziehen.\\
%

Im Rahmen dieser Arbeit konnte eine automatische, auf Messungen basierende Kalibrierung nicht mehr erprobt werden. Das hier vorgestellte Konzept der Entfernungsfindung eignet sich dazu, ein solches System zu implementieren. Dazu könnte ein Phantom mit einer bekannten geometrischen Proportion mit Tags ausgestattet werden. Die Antennen würden die Phasenwerte ausgeben und die Positionen der Antennen würde optimiert werden können. Das bedeutet die Berechnung des umgekehrten Falls zur Entfernungsfindung eines Tags.\\
%

\subsubsection{Weitere Anwendungsmöglichkeiten}
%
Das Modell, das in dieser Arbeit entwickelt wurde, ermöglicht eine Anwendung abseits der Entfernungsfindung. Der hier vorgestellte Ansatz kann zur Lösung eines anderen Problems beitragen, das durch die freie Anordnung der Antennen entsteht. Wenn mehrere Tags im Raum identifiziert werden, muss die Steuerung der Antennenumschaltung\footnote{Diese bestimmt von welcher Antenne gerade versucht wird die Tags zu lesen - es können nicht gleichzeitig alle Antennen gelesen werden} zur Zeit alle Antennen in einem Round-Robin-Verfahren\footnote{Umschalten nach einer festen Zeit und in einem vorgegebenen Muster} umschalten. Nach einer Umschaltung kann es dazu kommen, dass keine der Antennen einen der zuvor identifizierten Tags \textit{sieht}. Da die Umschaltung zur Zeit zufällig zur nächsten Antenne übergeht, werden auch Antennen genommen, die keine günstige Positionsberechnung erlauben. Die mögliche Lösung dafür ist, eine intelligente Steuerung der Umschaltung zu implementieren, indem die Kondition für jede Konfiguration aus vier Antennen verwendet wird. Ordnete man die sich auf dem Antennenaufbau ergebenen Antennenkombinationen ihrer Konditionszahl nach Aufsteigend\footnote{zur Erinnerung, gute Kondition = kleine Konditionszahl} an, können diese Antennenkombinationen von der Antennenumschaltung bevorzugt angesteuert werden. Dadurch werden stets die am besten konditionierte Kombinationen gewählt und ein möglichst sicheres Berechnungsergebnis gefunden.
%

\section{Zusammenfassung}
%
Evolutionäre Strategien und besonders die CMA-ES sind dazu geeignet die komplexe Aufgabe der Entfernungsabschätzung zu lösen. Das Entwickelte Verfahren eignet sich um das Problem der unbekannten Wellenzahl ohne Annahmen und Einschränkungen zu lösen. Das ist bisher einzigartig.\\

Die eingesetzten Verfahren entsprechen dem Stand der Technik in der evolutionären Optimierung. Mit Hilfe dieser Verfahren ist wurde eine funktionierende Entfernungsabschätzung für dem Messaufbau des PRPS entwickelt. Es ist gelungen Parameter zu identifizieren mit denen sich die evolutionäre Optimierung steuern lässt. Der Einfluss der Steuerparameter auf die Performance wurde untersucht und quantifiziert.\\

Es wurde ein System entwickelt das die Konditionszahl der Modell-Matrizen berechnet. Anwendungsmöglichkeiten für die Kondition wurden besprochen. Auf der Basis dieses Modells kann eine selbst-optimierende Kalibrierung erstellt werden. Zur Untersuchung der Fitnesslandschaft der Objektfunktion ist ein eigenes Stück Software entwickelt worden. Es erlaubt die Untersuchung von Fitnessebenen in der Fitnesslandschaft der Modellfunktionen. Die Ergebnisse des Werkzeugs wurden vorgestellt- \\

Das entwickelte Modell basiert auf der Trilateration und ermöglicht eine Anwendung in der Kalibrierung von Messaufbauten. Eine entsprechende Lösung befindet sich bereits im Einsatz. Die in dieser Arbeit eingesetzte Software-Bibliothek '\textit{Shark}' ist ein leistungsfähiges und flexibles Tool. Es wurde erfolgreich in das Software-Ökosystem der \amedogmbh integriert und steht für weitere Projekte zur Verfügung. Weiterhin wurde das Softwareerstellungswerkzeug \textit{CMake} in dieser Arbeit erfolgreich erprobt.\\

Eine Weiterentwicklung, basierend auf den Ergebnissen und Erkenntnissen dieser Masterarbeit, ist geplant. Ansatzpunkte bieten dabei die Reduzierung der Problemdimension, die Verbesserung der Performance und einer Analyse der beschriebenen, weiteren Anwendungsmöglichkeiten. Die Verbesserung der Performance ist direkt gekoppelt an die Reduzierung der Dimension des Problems. Auch die Entwicklung der selbst-optimierende Kalibrierung wird weiterverfolgt werden. Wie beschrieben ist eine Validierung der Objektfunktion sinnvoll und wird in kommenden Schritten durchgeführt werden.
