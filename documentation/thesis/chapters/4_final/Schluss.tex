%
\section{Verbesserungen}
%
Wie in den Ergebnissen gezeigt, können unter gewissen Umständen (schlechte Konditionierung, geringe Anzahl an Antennen) unzureichende Ergebnisse erzielt werden. Zur Zeit wird dieser Umstand durch eine häufigere Anzahl an Lösungsversuchen, sowie Verwendung verschiedener Antennenkombinationen kompensiert. Dadurch erhöht sich jedoch die Ausführungszeit, bzw. die Zeit bis ein Ergebnis Vorliegt auf zum Teil mehrere Sekunden. Das liegt außerhalb der Anforderungen. \\
%
Im Rahmen dieser Arbeit konnte eine automatische Kalibrierung nicht mehr entwickelt und erprobt werden. Die vorgestellten Ergebnisse sind prinzipiell dazu geeignet auf dieser Basis ein solches System zu entwickeln.\\


Die Ergebnisse dieser Arbeit weisen noch relativ große Abweichungen in der berechneten z-Koordinate auf. Warum diese auftreten konnte nicht mehr abschließend geklärt werden.

%
\section{Ausblick}
\label{sec:Calibration_Optimaztion}
%
Das Modell, das in dieser Arbeit entwickelt wurde, ermöglicht eine Anwendung Abseits der Positionsberechnung. Es erlaubt ein anderes Problem zu Lösung, dass durch die Freie Anordnung der Antennen entsteht. Wenn mehrere Tags im Raum identifiziert werden, muss die Steuerung der Antennenumschaltung (diese bestimmt von welcher Antenne gerade gelesen wird - es können nicht gleichzeitig alle Antennen gelesen werden) zur Zeit alle Antennen in einem Round-Robin-Verfahren nach einer gewissen Zeit umschalten. Nach einer Umschaltung kann es dazu kommen, dass keine der Antennen einen der zuvor identifizierten Tags "sieht". Da die Umschaltung zur Zeit zufällig zur nächsten Antenne spring, werden auch Antennen genommen, die keine günstige Positionsberechnung erlauben.\\
Die Bestimmung der Kondition für jede Konfiguration aus vier Antennen kann dazu verwendet werden eine gewisse Intelligenz beizutragen. Dazu würden aus dem gewählten Antennenaufbau die Antennenkombinationen ihrer Konditionszahl nach Aufsteigend\footnote{zur Erinnerung, gute Kondition = kleine Konditionszahl}] sortiert und diese Antennenkombinationen von der Antennenumschaltung bevorzugt. Dadurch werden stets gut konditionierte Kombinationen gewählt.