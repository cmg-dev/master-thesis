%
Die in dieser Arbeit bearbeitete Aufgabestellung ist sehr komplex. Die erreichten Ergebnisse erfüllen die Anforderungen nicht ganz. Der Ansatz das Problem über evolutionäre Verfahren zu lösen funktioniert zuverlässig. Auch auf limitierten Ressourcen (wenig Rechenleistung) konnten brauchbare Ergebnisse in einer akzeptablen Zeit gefunden werden. Die Performance kann mit den hier beschriebenen Methoden (Threading) weiter gesteigert werden. Das erhöht die Sicherheit des Messergebnisses. In Abbildung~\ref{fig:Requirements_reached} wird die Anforderungsspinne (Abbildung~\ref{fig:Requirements}) mit den erreichten Resultaten in Deckung gebracht.\\
%
\begin{figure}[ht]
         \centering
         \caption[Anforderungsspinne]{ Grafische Übersicht der Ergebnisse gegen die gestellten Anforderungen (blau) an diese Arbeit. Der grüne Bereich stellt die erreichten Ziele dar. Wie bereits Diskutiert, werden die Anforderungen an die Genauigkeit nicht gut erfüllt. Auch die Performance ist nicht zufriedenstellend erreicht, die Gründe wurden bereits angegeben. In den anderen Bereichen zeigt sich eine gute Deckung mit den Anforderungen. }
         \vspace{2mm}
         \label{fig:Requirements_reached}
         \input{diagrams/spider_requirements_reached.tex}
         \vspace{5mm}
\end{figure}
%

Wie in Kapitel~\ref{sec:Results1} beschrieben liefern sowohl reale Messdaten als auch künstliche, ideale Eingabewerte korrekte Ergebnisse, wenn man die Evolutionsparameter passend wählt. Brauchbare Konfigurationen wurden präsentiert. Es ist nicht nachgewiesen, dass man daraus eine Allgemeingültigkeit für die Zukunft ableiten kann. Es wurde in dieser Arbeit nur ein Messaufbau vermessen. Die Lösung konnte ohne wesentliche Einschränkungen und Registrierung durchgeführt werden. Der in dem verwendeten Aufbau abgedeckte Messbereich umfasse in etwa $30 m^3$. Damit ist erwiesen, dass sich die Methode für ein großes Messvolumen eignet\\
%

Die dabei verwendete Kalibrierroutine wurde ausführlich vorgestellt (Kapitel~\ref{sec:calibrationResults}). Das Softwaremodul für die Kalibrierung komm bereits in anderes Software der \amedogmbh zum Einsatz. Die Integration in die bestehende Softwarestrukturen war einfach. \\
%

Die Software wurde mit \cpp11 entwickelt und ist somit auf dem Stand der Technik. Sie lässt sich einfach auf eine andere Plattform portieren, die das Buildtool \textit{CMake} unterstützt. \\

%
Über die Ziele dieser Arbeit hinaus konnten Vorschläge gemacht werden, die Problemstellung einer automatischen Antennenumschaltung zu lösen. 