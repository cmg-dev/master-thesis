Mit der Entwicklung der minimal-invasiven Chirurgie, einer Operationsmethode bei der durch sehr kleine Einschnitte in den Körper mit besonders filigranen Operationsinstrumenten operiert wird. Die großen Vorteile gegenüber herkömmlichen Operationstechniken begründen die weite Verbreitung und häufigen Einsatz dieser Technik.\\
Mit fortschreitender Miniaturisierung der Instrumente geht die optische Kontrolle über das Operationsgebiet sowie Instrumentarium verloren. Diese ist jedoch unabdingbar für einen Erfolg der Operation, daher werden Aufwändige Bildgebende Verfahren intraoperativ angewendet. Beispielsweise werden bei kardiologischen Interventionen eine permanente Lagekontrolle der Katheter mittels Röntgentechnik durchgeführt. Oder es werden Bilder mittels Magnetresonanztomografie erzeugt. Nicht nur das eine Gewinnung dieser Bilddaten schwierig (MRT) oder gar schädlich (Röntgen) ist, oft muss der Patient dafür samt Instrumentarium umgelagert werden. Diese Techniken sind mit einem enormen Aufwand verbunden und stellen einen Belastung für Patient und Operateur dar.\\
Abhilfe versprechen Navigationssysteme die in der Lage sind den Arzt mit den benötigten Informationen zu versorgen. Alle verfügbaren System bieten unterschiedliche Vor- und Nachteile. Alle diese Systeme erlauben außerdem eine softwaregestützte Planung und assistierte Durchführung der Operation. Das macht diese Systeme im Zuge der stets Steigenden Ansprüche an der Qualitätsmanagement interessant. Die Anforderungen an ein solches System sind somit sehr hoch. Sie müssen über eine entsprechende Technik verfügen und gleichzeitig muss der Umgang mit ihnen leicht sein. Natürlich dürfen die Systeme möglichst wenig kosten.\\

\subsubsection{Stand der Technik}
Es befinden sich Navigationssystem unterschiedlicher Hersteller am Markt. Sie beruhen auf unterschiedlichsten Messprinzipien und leiden unter den daraus entstehenden Nachteilen. Die wichtigsten Technischen Unterschiede sind im Folgenden tabellarisch zusammengefasst:
%
\begin{table} [H]
	\begin{center}
		\begin{tabular}{lcccccc}
			\textbf{Arbeitsweise} & \textbf{Genauigkeit} & \textbf{Frequenz} &\textbf{Volumen} & \textbf{LOS} & \textbf{IV} & \textbf{Identifikation}\\
			Optisch & gut & mittel & mittel & Ja & Nein & Nein \\
			Magnetisch & ausreichend & hoch & klein & Nein & Ja & Nein \\
			Ultraschall & gut & gering & mittel & (Nein) & Nein & Nein \\
			Funk (UHF) & sehr gut\footnote{Abhängig vom Messprinzip} & hoch & groß & Nein & Ja & Ja \\
%			
		\end{tabular}
	\end{center}
	\caption[Übersicht Navigationsverfahren]{Grobe Übersicht und Einteilung verschiedener Navigationsverfahren}
	\label{tab:overview_tracking}
\end{table}
%
Die Tabelle~\ref{tab:overview_tracking} teilt die unterschiedlichen Systeme anhand ihres physikalischen Prinzips ein. Herausgestellt werden vor Allem die Messung betreffende Aspekte der Verfahren. Vor. und Nachteile der Systeme lassen sich aus der Auflistung ableiten. Der in der Praxis gravierendste Nachteil der meisten Verfahren ist die Identifikation. Dieser Punkt beschreibt die Fähigkeit des Systems verschiedene Objekte von einander zu unterscheiden. 
%
Man kann im Detail die Verfahren tiefer beschreiben, im Rahmen dieser Arbeit wird darauf allerdings verzichtet.
%
Die Genauigkeit (im technischen Sinne: Präzision und Wiederholbarkeit) der Messung muss sehr hoch sein. Das allein stellt viele Techniken vor eine großer Herausforderung. Hinzukommen weitere Anforderungen, die sich aus dem Ablauf einer Intervention ergeben. Ein System muss eine einfache Integrationsmöglichkeit in den Arbeitsablauf bieten.
