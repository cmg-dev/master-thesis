Mit der Entwicklung der minimal-invasiven Chirurgie, einer Operationsmethode bei der durch sehr kleine Einschnitte in den Körper mit besonders filigranen Operationsinstrumenten operiert wird, verändert sich die Art Operationen durchzuführen grundlegend. Eingriffe können schneller, schonender und effizienter durchgeführt werden. Möglich wird diese Entwicklung durch eine Vielzahl neuartiger technischer Systeme. Die Vorteile gegenüber herkömmlichen Operationstechniken begründen die weite Verbreitung und häufigen Einsatz der minimal-invasiven Techniken.\\
Mit fortschreitender Miniaturisierung der Instrumente geht die optische Kontrolle über das Operationsgebiet sowie Instrumentarium verloren. Diese Information ist unabdingbar für einen Erfolg der Operation und müssen dem Operierenden zu jeder Zeit zur Verfügung stehen. Um an diese Informationen zu gelangen ist es Stand der Technik, durch aufwändige bildgebende Verfahren intraoperativ, d.h. während der Operation, anzuwenden.\\
Beispielsweise werden bei kardiologischen Interventionen (z.B. Platzierung eines Stents durch die Arteria iliaca interna\footnote {innere Beckenarterie- Standardzugang für diese Art von Operationen} in den Coronargefäßen des Herzens) eine permanente Lagekontrolle der Katheter mittels Röntgentechnik durchgeführt. Oder es werden Bilder durch Magnetresonanztomografie oder durch andere bildgebende Verfahren erzeugt. Nicht nur das eine Gewinnung dieser Bilddaten schwierig (MRT) oder gar schädlich (Röntgen) ist, oft muss der Patient dafür samt Instrumentarium umgelagert werden. Das Umlagern bringt weitere Risiken mit sich und ist mit weiterem Aufwand verbunden.\\
Eine Lösung für diese Problem bringen sog. Trackingsysteme. Diese Systeme sind in der Lage eine Position, z.B. eines Instrumentes, zu ermitteln und stellen die benötigten Informationen für den Arzt zur Verfügung. Die verfügbaren Systeme basieren auf unterschiedlichen physikalischen Prinzipien und haben dadurch unterschiedliche Vor- und Nachteile.\\
Die Anwendung solcher Systeme erlaubt außerdem eine softwaregestützte Planung und assistierte Durchführung der Operation. Die Kombination dieser Techniken wird Navigation genannt. Die Möglichkeit der Planung und Kontrolle macht diese Systeme im Zuge der stets steigenden Ansprüche an das Qualitätsmanagement interessant. Die Anforderungen die vom Anwender im klinischen Alltag an die Systeme gestellt werden sind:
%
\begin{table} [H]
	\begin{center}
		\begin{tabular}{l}
		Gute Genauigkeit\\
		Hohe Verfügbarkeit\\
		Leichte Bedienbarkeit\\
		Einfache Einbindung Workflow\\
		Geringe Kosten\\
		Sicherheit\\
		\end{tabular}
	\end{center}
	\caption[Anforderungen Trackingsysteme]{Anforderungen an ein medizintechnisches Messsystem.}
	\label{tab:requirements_system}
\end{table}
%
Die Anforderungen an ein solches System sind somit sehr hoch. Sie müssen über eine entsprechende Technik verfügen und gleichzeitig muss der Umgang mit ihnen leicht sein. Zusätzlich dürfen die Systeme möglichst wenig kosten.\\
%
\subsubsection{Stand der Technik}
Es befinden sich Trackingsysteme unterschiedlicher Hersteller am Markt. Sie beruhen auf unterschiedlichsten Messprinzipien und unterliegen den daraus resultierenden Limitierungen. Die wichtigsten Technischen Unterschiede sind im Folgenden tabellarisch zusammengefasst:
%
\begin{table} [H]
	\begin{center}
		\begin{tabular}{rllll}
			\textbf{Arbeitsweise} & Optisch & Magnetisch & Ultraschall & Funk (UHF) \\
			\textbf{Genauigkeit} & gut & ausreichend & gut & sehr gut\footnote{Abhängig vom Messprinzip} \\
			\textbf{Frequenz} & mittel & hoch & gering & hoch \\
			\textbf{Volumen} & mittel & klein & mittel & groß \\
			\textbf{LOS} & Ja & Ja & Nein & Ja \\
			\textbf{IV\footnote{in vivo lat. im Lebendigen; med. im Patienten}} & Nein   & Nein & Nein & Ja \\
%			
		\end{tabular}
	\end{center}
	\caption[Übersicht Navigationsverfahren]{Grobe Übersicht und Einteilung verschiedener Navigationsverfahren anhand ihres physikalischen Messprinzips.}
	\label{tab:overview_tracking}
\end{table}
%
Die Tabelle~\ref{tab:overview_tracking} teilt die unterschiedlichen Systeme anhand ihres physikalischen Messprinzips ein. Herausgestellt werden vor Allem die wesentlichen Messparameter der betreffende Aspekte der Verfahren. Aus der Auflistung lassen sich Vor. und Nachteile ableiten.\\
Das größte Problem ist das Benötigen einer direkten Sicht auf die Objekte. Dem sog. LOS-Problem unterliegen fast alle Verfahren, die ein großes Messvolumen abdecken. Die auf Funk basierenden Verfahren haben das Problem nicht, unterliegen jedoch anderen Schwierigkeiten. Der größte Vorteil des auf Funk basierenden RFID-Verfahrens ist es verschiedene Objekte von einander zu unterscheiden, zu identifizieren.\\
Die Genauigkeit (im technischen Sinne: Präzision und Wiederholbarkeit) der Messung ist bei allen Verfahren mindestens ausreichend. Das allein stellt viele Techniken vor eine großer Herausforderung. Hinzukommen weitere Anforderungen, die sich aus dem Ablauf einer Intervention ergeben. Ein System muss eine einfache Integrationsmöglichkeit in den Arbeitsablauf bieten.\\
%

Im Folgenden wird auf die Besonderheiten und Merkmale des auf Funk basierenden RFID-Verfahrens eingegangen. Die anderen Verfahren werden, aufgrund der Unterschiedlichkeit der Systeme wird im Rahmen dieser Arbeit wird darauf verzichtet.

