Mit der Entwicklung der minimal-invasiven Chirurgie, einer Operationsmethode bei der durch sehr kleine Einschnitte in den Körper mit besonders filigranen Operationsinstrumenten operiert wird, verändert sich die Art Operationen durchzuführen grundlegend. Eingriffe können schneller, schonender und effizienter durchgeführt werden. Möglich wird diese Entwicklung durch eine Vielzahl neuartiger technischer Systeme. Die Vorteile gegenüber herkömmlichen Operationstechniken begründen die weite Verbreitung und häufigen Einsatz der minimal-invasiven Techniken.\\
Mit fortschreitender Miniaturisierung der Instrumente und der Verlagerung des Operationsgebietes in den Patienten geht die optische Kontrolle über das Operationsgebiet sowie Instrumentarium verloren. Diese ist unabdingbar für einen Erfolg der Operation und die Information (z.B. Lage der Instrumente) müssen dem Operierenden jeder Zeit zur Verfügung stehen. Eine Übersicht über das Operationsgebiet im Inneren des Körpers kann durch Kameratechniken (Endoskope) gewonnen werden, allerdings wird dafür viel Platz benötigt. Die Kenntnis über den Aufenthaltsort und die Lage des Instrumentariums ist, gerade bei schwierigen Operationen, unabdingbar. Um an diese Informationen zu gelangen ist es Stand der Technik, durch bildgebende Verfahren intraoperativ, d.h. während der Operation, Bildvolumina zu generieren und auszuwerten.\\
Beispielsweise werden bei kardiologischen Interventionen (z.B. Platzierung eines Stents durch die Arteria iliaca interna\footnote {innere Beckenarterie- Standardzugang für diese Art von Operationen} in den Coronargefäßen des Herzens) eine permanente Lagekontrolle der Katheter mittels Röntgentechnik durchgeführt. Dabei wird der Patient und das Operationsteam permanent ionisierender Bestrahlung ausgesetzt. Es können auch Bilder durch Magnetresonanztomografie oder durch andere bildgebende Verfahren erzeugt werden. Die Gewinnung dieser Bilddaten ist schwierig (MRT) oder nicht gut geeignet (Ultraschall) ist. Ein weiteres Erschwernis ist, dass der Patient dafür häufig mitsamt den Instrumenten umgelagert werden muss. Das Umlagern bringt weitere Risiken mit sich und ist mit weiterem Aufwand verbunden. Besonders könne die so gewonnenen Informationen nicht als genau angenommen weden.\\
Eine Lösung für diese Probleme bringen sog. Trackingsysteme. Diese sind in der Lage eine Position, z.B. eines Instrumentes, zu ermitteln und stellen die benötigten Informationen für den Arzt zur Verfügung. Dabei kommen bei den verfügbaren Systeme unterschiedliche physikalische Prinzipien zu Einsatz, die verschiedene Vor- und Nachteile bieten.\\
Die Anwendung solcher Systeme erlaubt außerdem eine softwaregestützte Planung und assistierte Durchführung der Operation. Die Kombination dieser Techniken wird Navigation genannt. Die Möglichkeit der Planung und Kontrolle macht diese Systeme im Zuge der stets steigenden Ansprüche an das Qualitätsmanagement interessant. Die Anforderungen die vom Anwender im klinischen Alltag an die Systeme gestellt werden sind:
%
\begin{table} [H]
	\begin{center}
		\begin{tabular}{l}
		Gute Genauigkeit\\
		Hohe Verfügbarkeit\\
		Leichte Bedienbarkeit\\
		Einfache Einbindung Workflow\\
		Geringe Kosten\\
		Sicherheit\\
		\end{tabular}
	\end{center}
	\caption[Anforderungen Trackingsysteme]{Anforderungen an ein medizintechnisches Messsystem.}
	\label{tab:requirements_system}
\end{table}
%
Die Anforderungen an ein solches System sind somit sehr hoch. Sie müssen über eine entsprechende Technik verfügen und gleichzeitig muss der Umgang mit ihnen leicht sein. Zusätzlich dürfen die Systeme möglichst wenig kosten.\\
%
\subsubsection{Stand der Technik}
Es befinden sich Trackingsysteme unterschiedlicher Hersteller am Markt. Sie beruhen auf unterschiedlichsten Messprinzipien und unterliegen den daraus resultierenden Limitierungen. Die wichtigsten Technischen Unterschiede sind im Folgenden tabellarisch zusammengefasst:
%
\begin{table} [H]
	\begin{center}
		\begin{tabular}{rllll}
			\textbf{Arbeitsweise} & Optisch & Magnetisch & Ultraschall & Funk (UHF) \\
			\textbf{Genauigkeit} & gut & ausreichend & gut & sehr gut\footnote{Abhängig vom Messprinzip} \\
			\textbf{Frequenz} & mittel & hoch & gering & hoch \\
			\textbf{Volumen} & mittel & klein & mittel & groß \\
			\textbf{LOS} & Ja & Ja & Nein & Ja \\
			\textbf{IV\footnote{in vivo lat. im Lebendigen; med. im Patienten}} & Nein   & Nein & Nein & Ja \\
%			
		\end{tabular}
	\end{center}
	\caption[Übersicht Navigationsverfahren]{Grobe Übersicht und Einteilung verschiedener Navigationsverfahren anhand ihres physikalischen Messprinzips.}
	\label{tab:overview_tracking}
\end{table}
%
Die Tabelle~\ref{tab:overview_tracking} teilt die unterschiedlichen Systeme anhand ihres physikalischen Messprinzips ein. Herausgestellt werden vor Allem die wesentlichen Messparameter der betreffende Aspekte der Verfahren. Aus der Auflistung lassen sich Vor. und Nachteile ableiten.\\
Das größte Problem ist das Benötigen einer direkten Sicht auf die Objekte, Line of Sight (LOS) genannt. Dem sog. LOS-Problem unterliegen fast alle Verfahren, die ein großes Messvolumen abdecken. Auf Funk basierende Methoden erfahren dieses Problem nicht, ihre Wechselwirkungsmechanismen erlauben die Durchdringung von Materie. Andere Wechselwirkungen machen diesen Verfahren zu schaffen und verursachen Schwierigkeiten. Der größte Vorteil des auf Funk basierenden RFID-Verfahrens ist es verschiedene Objekte von einander zu unterscheiden, bzw. sogar zu identifizieren. Die Genauigkeit (im technischen Sinne: hohe Präzision plus hohe Richtigkeit) der Messung ist bei allen Verfahren mindestens ausreichend. \\
Diese Anforderungen allein sind eine große Herausforderung an die Technik. Hinzukommen weitere Anforderungen, die sich z.B. aus dem Ablauf einer Intervention ergeben. Ein System muss eine einfache Integrationsmöglichkeit in diesen Arbeitsablauf bieten. Zusätzlich dürfen durch die Anschaffung und das Material keine großen Kosten entstehen.\\
%
