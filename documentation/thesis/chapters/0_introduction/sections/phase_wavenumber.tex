\begin{figure}[h!]
	\caption[Zusammenhang Wellenlänge - Wellenzahl]{Dargestellt ist der Zusammenhang zwischen der Wellenlänge $\lambda$ und der Wellenzahl $n$. Da die Phase alle $2\pi$ den gleichen Wert annimmt, wird mit dem Faktor $n$ ein vielfaches der Wellenlänge aufaddiert. Dadurch erhält man die Entfernung zu dem Tag.}
	\label{fig:wavenumber_wavelength}
	\begin{gnuplot} % aufrufen des Befehls und anschließende eingabe der Gnuplot Befehle
		set terminal epslatex color % WICHTIG: diese Zeile teilt Gnuplot 
		                            % mit das die Ausgabe in Latex umgeleitet werden soll
		set nokey % ab hier folgt üblicher Gnuplot Code
		set parametric
		set hidden3d
		set view 45,60
		set isosamples 200,15
		splot [-3*pi:3*pi][-pi:pi] cos(u)*cos(v)+3*cos(u)*(1.5+sin(u*5/3)/2),\
		sin(u)*cos(v)+3*sin(u)*(1.5+sin(u*5/3)/2), sin(v)+2*cos(u*5/3)
	\end{gnuplot}	
%	
	\ZeigerdiagrammText{110}{-80}{2}{2}
%	
\end{figure}


Aus der Abbildung~\ref{fig:wavenumber_wavelength} lässt sich folgender Zusammenhang ableiten.
%
\begin{equation}
\label{eq:Phase_Wavenumber}
	d(\Theta, n)=\lambda(\sfrac{\Theta}{2\pi}+n)
\end{equation}
%
\subsubsection{Berechnung idealer Phasenwerte}
\label{sec:PhaseCalculation}
%
Die Gleichung~\ref{eq:Phase_Wavenumber} kann dazu verwendet werden künstliche Messdaten zu generieren. Dazu muss zuerst die Wellenzahl aus der bekannten Entfernung und der Wellenlänge ermittelt werden.
$$
n_k=\frac{d_k}{\lambda}
$$
Nachdem die Wellenzahl für Antenne $k$ ermittelt wurde kann man das jeweilige $\Theta$ bestimmen.
$$
\Theta= 2\pi(\frac{d_k }{\lambda}-n)
$$