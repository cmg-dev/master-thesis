\begin{figure}[ht!]
	\begin{center}
		\caption[Ablauf Evolutionsstrategie]{ Der Ablauf des $(\lambda,\mu)$- Evolutionsalgorithmus ist in dieser Abbildung gezeigt. Dies ist die einfachste Variante der Algorithmen\footnote{Augenommen die $(1+1)$- Strategie}. Die wesentlichen Schritte gleichen sich in den Varianten. }
		\label{fig:es_flowchart}
		\vspace{0.5cm}
		\begin{tikzpicture}[auto]
		\scriptsize
			\tikzstyle{decision} = [diamond, draw=black, thick, fill=black!20, text width=5em, text badly centered, inner sep=1pt]
%			
			\tikzstyle{block} = [rectangle, draw=black, thick, fill=gray!20, text width=15em, text centered, rounded corners, minimum height=4em]
%	
			\tikzstyle{line} = [draw, thick, -latex',shorten >=1pt];
			\tikzstyle{commentline} = [draw, dashed, green!50,-latex',shorten >=1pt];
%	
			\tikzstyle{cloud} = [ dotted, draw=green!50, thick, ellipse,,fill=green!20, minimum height=2em, text width= 10em, text badly centered];
%	
			\matrix [column sep=5mm,row sep=7mm]
			{
				% row 1
				& \node [block] (start) { Start }; & \\
				% row 2
				&\node [block] (init) {Erstelle Startpopulation $X_p^1$ bestehend aus $\mu$-Individuen }; & 
				\node [cloud] (comment1) {Initialisierung, mit Zufallswerten}; \\
				% row 4
				& \node [block] (identify) {Erzeuge eine Menge von $\lambda$ Nachkommen $X_c^k$ aus der aktuellen Elterngeneration $X_p^k$ durch Rekombination \&\&, || Mutation}; & \\
				% row 5
				\node [block] (update) {Nächste Stufe der Evolution; k++}; &
				\node [block] (evaluate) {Durch Selektion die besten $\mu$ Nachkommen für die Generation $X_p^{k+1}$ auswählen}; & \\
				% row 6
				& \node [decision] (decide) {$\Delta \geq \Delta_{min}$}; & 
				\node [cloud] (criteria) {Abbruchkriterium; Muss geeignet gewählt werden, bspw. max. Anzahl der Generationen oder Erreichen des Optimums};\\
				% row 7
				& \node [block] (stop) {Ende}; & \\
			};
% Arrows
			\tikzstyle{every path}=[line]
			\path (init) -- (identify);
			\path (identify) -- (evaluate);
			\path (evaluate) -- (decide);
			\path (update) |- (identify);
			\path (decide) -| node [near start] {Ja} (update);
			\path (decide) -- node [midway] {Nein} (stop);
			\path (start) -- (init);
			
			\tikzstyle{every path}=[commentline]
			\path (criteria) -- (decide);
			\path (comment1) -- (init);
			
		\end{tikzpicture}
	\end{center}
\end{figure}