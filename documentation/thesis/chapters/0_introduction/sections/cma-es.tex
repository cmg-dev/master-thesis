Der Folgende Abschnitt behandelt das 
%
Dieses Verfahren stellt den "State of the Art"- der Evolutionären Berechnungsverfahren dar. Für spezialisierte Probleme gibt es bessere Lösungen, als allgemeiner Solver ist dieses Verfahren mehr als tauglich. Es wurde um die Jahrtausendwende Entwickelt und veröffentlicht. Es gibt verschiedene Abwandlungen des Algorithmus und sogar eine Lösung zur Multiobjekt-Optimierung (MO-CMA-ES) existiert.~\cite{HansenMOO:1} Das Verfahren wird aktuell stets Weiterentwickelt. [REFERENZEN]\\
%

Der Algorithmus steht in verschiedene Implementationen, in unterschiedlichen Programmiersprachen und Umgebungen zur Verfügung. Teilweise sind die Implementationen Propritär (z.B. Matlab), teilweise quelloffen. Die in dieser Arbeit Zur Anwendung kommende Variante ist die Shark-Library. Diese Bibiliothek ist eine in \cpp geschriebene, quelloffene Software, die am Institut für Neuroinformatik der Ruhr Universität Bochum entwickelt wird. Detailliert wird Shark im Rahmen des Hauptteils in Abschnitt~\ref{sec:Shark} vorgestellt.\\
%