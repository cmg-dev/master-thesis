%-----------------------------------------------------------------------
\section[Motivation]{Motivation}
Die Positionsbestimmung mittels RFID ist eine vielversprechende Technik. Die Bestimmung der Position (im Folgenden "Tracking" genannt) mittels RFID  bietet gegenüber vergleichbaren Methoden (z.B. Ultraschall, Optisch) verschiedene Vorteile. Das wesentlichste Unterscheidungsmerkmal ist, dass keine direkte Sichtlinie sog. LOS notwendig ist um ein Objekt zu lokalisieren. Der Grund dafür ist das zugrunde liegende Messprinzip. Es werden elektromagnetische Signale ausgewertet, die anderen Wechselwirkungen unterliegen und somit Materie durchdringen. Insbesondere im Vergleich mit optischen Verfahren ist RFID damit überlegen. Die Eigenschaft Materie zu durchdringen erlaubt es Tags im Patienten zu lokalisieren, entsprechende Untersuchungen über die Positionsgenauigkeit im Körper sind vielversprechend.[REFERENZEN]\\
Auf den Tags können zusätzliche Informationen hinterlegt werden, beispielsweise eine Identifikationsnummer oder Ähnliches. Dadurch wächst das Anwendungsspektrum weiter[REFERENZEN]. Das Auslesen von zusätzlichen Informationen ist mit keiner der anderen Technologien möglich.\\
Das von dem Messsystem der {Amedo GmbH} verwendete Verfahren basiert auf der Messung der Phasenlage der Antwort eines Tags. Die Phasenlage ist direkt proportional zu einer Entfernung, sie ist jedoch nicht Eindeutig (siehe \ref{sec:Measurement1})\\

Eine analytische Lösung des Problems ist schwierig und bisher nicht gelungen. Diese Ansätze scheiterten an der Komplexität des Problems\footnote{siehe \ref{sec:Komplexity1} und \ref{sec:Komplexity2}} oder benötigen sehr aufwändige Messreihen mit großer Anzahl an Messpunkten \cite{amedo1}. Das limitiert die Praxistauglichkeit der Verfahren.\\

In dieser Arbeit soll mittels Evolutionärer Verfahren die beschriebenen Probleme zu gelöst werden. Im Endergebnis soll dabei eine Abschätzung der Wellenzahl \ref{eq:wavenumber} möglich sein.
%
%-----------------------------------------------------------------------
\section{Voraussetzungen}
%
Dieses Kapitel stellt die Grundlagen dieser Arbeit vor. Zuerst werden mathematische und technische Grundlagen erläutert, anschließend wird das Messsystem der amedo GmbH vorgestellt.
%
%-----------------------------------------------------
\subsection[mathematisches]{Mathematische Voraussetzungen}
%
\subsection{Kondition}
%
Gegeben ist ein lineares Gleichungssystem der Form:
$$ \mathbf{A}\mathbf{x}-\mathbf{b} =\mathbf{0} $$
Eine numerische Lösung führt in der Regel zu einer von $\mathbf{0}$ verschiedenen Lösung (insbesondere bei überbestimmten Systemen), so das wir:
$$ \mathbf{A}\mathbf{\tilde{x}}-\mathbf{b} =\mathbf{r} $$
schreiben. Man nennt $\mathbf{r}$ den Residuumvektor. Es ist offensichtlich, dass ein kleines Residuum nicht hinreichend ist um von einem kleinen relativen Fehler auszugehen.\\
Weiter folgt aus $\mathbf{A}\mathbf{x}-\mathbf{b} =\mathbf{0}$ und $\mathbf{A}\mathbf{\tilde{x}}-\mathbf{b} =\mathbf{r}$,dass 
$$ \mathbf{A}\Delta\mathbf{x}=\mathbf{r}$$
und damit:
$ 
\lVert \mathbf{b} \rVert=\lVert \mathbf{Ax} \rVert \leq \lVert \mathbf{A} \rVert \lVert \mathbf{x} \rVert
$, 
$
\lVert \Delta\mathbf{x} \rVert=\lVert -\mathbf{A^{-1}r} \rVert \leq \lVert \mathbf{A^{-1}} \rVert \lVert \mathbf{r} \rVert
$
Wir können nun für den relativen Fehler schreiben:
$$
\frac{\lVert \Delta\mathbf{x} \rVert}{\lVert \mathbf{x} \rVert} \leq 
\frac{\lVert \mathbf{A^{-1}} \rVert \lVert \mathbf{r} \rVert}{\lVert \mathbf{b} \rVert / \lVert \mathbf{A} \rVert} =
\lVert \mathbf{A} \rVert \lVert \mathbf{A^{-1}} \rVert \frac{\lVert \mathbf{r} \rVert}{\lVert \mathbf{b} \rVert}
$$
Der Term $\lVert \mathbf{A} \rVert \lVert \mathbf{A^{-1}} \rVert := \text{cond}(\mathbf{A})$ heißt Konditionszahl. Auch der Begriff Konditionsmaß ist gebräuchlich und bezieht sich auf die gewählte Matrixnorm.
Es kann gezeigt werden, dass $\text{cond}(\mathbf{A}) \gg 1$  für eine schlechte Konditionierung der Matrix steht. Wird im Folgenden von einer speziellen Matrixnorm gesprochen schreiben wir $\text{cond}(\mathbf{A})$ zu 
$$ 
\text{cond}_k(\mathbf{A}) = \lVert \mathbf{A} \rVert_k \lVert \mathbf{A^{-1}} \rVert_k
$$ \\
Der Index $k$ wird entsprechend für die verwendete Norm ersetzt. Beispielsweise ergibt sich für die Konditionszahl der Spektralnorm\footnote{\url{http://de.wikipedia.org/w/index.php?title=Spektralnorm&oldid=118988565}}:
$$ 
\text{cond}_2(\mathbf{A}) = \lVert \mathbf{A} \rVert_2 \lVert \mathbf{A^{-1}} \rVert_2=
\sqrt{\frac{\mu_{max}}{\mu_{min}}}\\
$$
Die Symbole $\mu_{max}$ und $\mu_{min}$ stehen für die Eigenwerte des Systems.\\

Die Konditionszahl ermöglicht eine Analyse der Güte einer Lösung, die mittels Numerischer Verfahren ermittelt wurde. Nach \cite{hermann2006numerische} kann man folgende Aussage über die Konditionszahl treffen:
%
\begin{quote}
"Wird ein lineares Gleichungssystem $Ax=b$ mit $t$-stelliger dezimaler Gleitpunktarithmetik gelöst und beträgt die Konditionszahl $\text{cond}(A) \approx10^\alpha$, so sind auf Grund der im allgemeinen unvermeidbaren Fehler in den Eingabedaten $A$ und $b$ nur $t-\alpha-1$ Dezimalstellen der berechneten Lösung $\tilde{x}$ (bezogen auf die betragsgrößte Komponente) sicher."
\end{quote}

\subsection{SVD}
%Um die Konditionszahl zu bestimmen sind aufwändige Berechnungen\footnote{siehe Wochenbericht KW 22} der Eigenwerte der Matrix notwendig. Es wurde nach eine Möglichkeit gesucht diese effizient abzuschätzen oder zu berechnen. Vor Allem soll es auch möglich sein mit dem Verfahren eine nicht symmetrische, nicht quadratische.\\
%Eine Methode die diese Anforderungen erfüllt, ist die sog. Singulärwertzerlegung (im Folgenden SVD := engl. Singular Value Decomposition). 
Bei dem Verfahren der Singular Value Decompostion (oder auch Singulärwertzerlegung), kurz SVD, handelt es sich um eine Faktorisierung einer Matrix. Die Matrix wird dabei als Produkt von drei Matrizen dargestellt. Diese Matrizen enthalten die sog. Singulärwerte und können aus einer der Matrizen abgelesen werden. Die Eigenschaften des Systems sind, ähnlich den Eigenwerten, aus den Singulärwerten bestimmbar. Besonders an der SVD ist, die Existenz für jede Form von Matrix - einschließlich nicht quadratischer Matrizen.\\
Die SVD basiert auf folgender Theorie der linearen Algebra: Jede $M \times N$ Matrix $\mathbf{A}$ kann als Produkt einer $M \times N$ Spalten-orthogonalen Matrix $\mathbf{U}$, einer $N \times N$ Diagonalmatrix $\mathbf{\Sigma}$ mit Werten $\geq 0$ und einer dritten adjungierten $N \times N$-Matrix $\mathbf{V^*}$, so ergibt sich:
%
\begin{equation}
\mathbf{A}= \mathbf{U \Sigma V^*} = \mathbf{U \Sigma V}^T
\end{equation}
Ist $\mathbf{A}$ eine reelwertige Matrix gilt: $ \mathbf{V^*} = \mathbf{V}^T $. Die Matrix $\mathbf{ \Sigma }$ ist im Rahmen dieser Arbeit von besonderem Interesse, denn sie enthält die Singulärwerte $\sigma_r$. Ihre Gestalt ist wie folgt:
%
\begin{equation}
	\mathbf{\Sigma} = \left(\begin{array}{ccc|ccc}
	\sigma_1 &          &          &        & \vdots &        \\
	         & \ddots   &          & \cdots & 0      & \cdots \\
	         &          & \sigma_r &        & \vdots &        \\
	\hline
	         &  \vdots  &          &        & \vdots &        \\
	\cdots   &  0       & \cdots   & \cdots & 0      & \cdots \\
	         &  \vdots  &          &        & \vdots &        \\
	
	\end{array}\right)\nonumber
\end{equation}
%
\phantomeq{\mathbf{\Sigma} = }{ ,wobei~\sigma_1\geq\sigma_2\geq\cdots\geq\sigma_r> 0 \nonumber}
%
Da die $\sigma_r$ der Matrix mit den Eigenwerten in Verbindung stehen, kann aus dieser Matrix die Konditionszahl bestimmt werden. Sie ist durch folgendes Verhältnis gegeben: 
\begin{equation}
	\label{eq:cond_from_svd}
	cond(\mathbf{A})=\frac{max(\sigma_r)}{min(\sigma_r)}=\frac{max(\sigma_1)}{min(\sigma_r)}
\end{equation} 

Es gibt bereits viele Implementationen des Verfahrens, z.B. \cite{press2007numerical}. Diese Implementation wird durch den Erwerb der entsprechenden Lizenz im Rahmen dieser Arbeit verwendet. Weitere Informationen zum Verfahren sind z.B. in \cite[Kaptiel 4.6.3]{bronstejn2012taschenbuch} zu finden.
%
%-----------------------------------------------------
\subsection[technisches]{Technische Voraussetzungen}
In diesem Abschnitt werden die technischen Grundlagen für diese Arbeit vorgestellt und das Wichtigste erörtert. Es kann nicht im vollem Umfang auf die Details der Technik eingegangen werden ohne den Rahmen dieser Arbeit zu sprengen. Interessierte sei die referenzierte Literatur für eine weite Lektüre empfohlen.
%
\subsection{RFID}
\label{sec:Measurement1}
%
\begin{enumerate}
	\item Die Messung der Position erfolgt über die Auswertung der Phasenlage des empfangenen Signals in Bezug auf ein Referenzsignal. In der EU gibt es verschiedene, zulässige RFID-Frequenzen\footnote{text} (865,5?867,5 MHz) kann man die Wellenlänge mit: $ \lambda\simeq0,35 m $ angeben. Daraus folgt, dass alle 35 cm die gleiche Konfiguration der Phase vorliegt. In dieser Arbeit wird dieser Umstand Isophasen genannt. Die gewonnene Information aus der Phase ist somit redundant, d.h. es lässt sich durch die Kenntnis der Phase nicht unmittelbar auf die korrekte Postion schließen. Man kann das Problem umgehen in dem man auf die errechnete Position ein ganzzahliges Vielfaches der Wellenlänge addiert. Die sog. Wellenzahl (vgl.~\eqref{eq:Wavenumbers}).
	\item Das System der Amedo STS verwendet eine spezielle Antennenanordnung um die Position zu ermitteln. Dabei wird eine Antennenanzahl >4 eingesetzt. Für jede dieser Antennen muss eine eigene Wellenzahl bestimmt werden. Durch Auslöschung des Signals, Absorption etc. kann es dazu kommen, dass eine Antenne eine unbestimmte Zeit lang kein Signal vom Tag empfängt. Wenn die Antenne nach dieser Zeit erneut ein Signal empfängt ist die ihr zugehörige Wellenzahl unbekannt und muss neu bestimmt werden. 
	\item In realen Umgebungen treten zusätzlich noch Ruflektionen und ein sog. Multipath-Effekt auf. Dabei wird das Signal nicht auf dem Direkten Weg Antenne-Tag-Antenne empfangen sondern über einen unbekannten, längeren Weg. Dadurch kommt es zu einem Fehler in der Phase. Zusätzlich ist dieser Effekt individuell für jede Antenne.
\end{enumerate}
\lipsum[1-2]

%
%
%-----------------------------------------------------------------------
\section{Anforderungen an das Verfahren}
Die bisher vorgestellten Überlegungen erlauben Anforderungen an das verwendete Verfahren zu stellen:
\begin{enumerate}
\item Lösung muss schnell (ideal < 1 Sekunde) gefunden werden
\item Unabhängigkeit von Stütz- Kalibrierpunkten
\item Eindeutigkeit der Lösung
\item Eignung für ein großes Messvolumen (mehrere Kubikmeter)
%
\end{enumerate}
%
\section{Ziel}
%
%-----------------------------------------------------------------------
