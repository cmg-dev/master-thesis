%-----------------------------------------------------------------------
%
\section[Allgemein]{Allgemein}
Mit der Entwicklung der minimal-invasiven Chirurgie, einer Operationsmethode bei der durch sehr kleine Einschnitte in den Körper mit besonders filigranen Operationsinstrumenten operiert wird, verändert sich die Art Operationen durchzuführen grundlegend. Eingriffe können schneller, schonender und effizienter durchgeführt werden. Möglich wird diese Entwicklung durch eine Vielzahl neuartiger technischer Systeme. Die Vorteile gegenüber herkömmlichen Operationstechniken begründen die weite Verbreitung und den häufigen Einsatz der minimal-invasiven Techniken.\\
Mit fortschreitender Miniaturisierung der Instrumente und der Verlagerung des Operationsgebietes in den Patienten geht die optische Kontrolle über das Operationsgebiet sowie Instrumentarium verloren. Diese ist unabdingbar für einen Erfolg der Operation und die Information (z.B. Lage der Instrumente) müssen dem Operierenden jeder Zeit zur Verfügung stehen. Eine Übersicht über das Operationsgebiet im Inneren des Körpers kann durch Kameratechniken (Endoskope) gewonnen werden, allerdings wird dafür viel Platz benötigt. Die Kenntnis über den Aufenthaltsort und die Lage des Instrumentariums ist, gerade bei schwierigen Operationen, unabdingbar. Um an diese Informationen zu gelangen ist es Stand der Technik, bildgebende Verfahren intraoperativ, d.h. während der Operation, Bildvolumina zu generieren und auszuwerten.\\
Beispielsweise werden bei kardiologischen Interventionen (z.B. Platzierung eines Stents durch die Arteria iliaca interna\footnote {innere Beckenarterie- Standardzugang für diese Art von Operationen} in den Coronargefäßen des Herzens) permanente Lagekontrolle der Katheter mittels Röntgentechnik durchgeführt. Dabei wird der Patient und das Operationsteam permanent ionisierender Bestrahlung ausgesetzt. Es können auch Bilder durch Magnetresonanztomografie oder durch andere bildgebende Verfahren erzeugt werden. Die Gewinnung dieser Bilddaten ist schwierig (MRT) oder nicht gut geeignet (Ultraschall) ist. Ein weiteres Erschwernis ist, dass der Patient dafür häufig mitsamt den Instrumenten umgelagert werden muss. Das Umlagern bringt weitere Risiken mit sich und ist mit weiterem Aufwand verbunden. Besonders könne die so gewonnenen Informationen nicht als genau angenommen weden.\\
Eine Lösung für diese Probleme bringen sog. Trackingsysteme. Diese sind in der Lage die Position, z.B. eines Instrumentes, zu ermitteln und stellen die benötigten Informationen für den Arzt zur Verfügung. Dabei kommen bei den verfügbaren Systeme unterschiedliche physikalische Prinzipien zu Einsatz, die verschiedene Vor- und Nachteile bieten.\\
Die Anwendung solcher Systeme erlaubt außerdem eine softwaregestützte Planung und assistierte Durchführung der Operation. Die Kombination dieser Techniken wird Navigation genannt. Die Möglichkeit der Planung und Kontrolle macht diese Systeme im Zuge der stets steigenden Ansprüche an das Qualitätsmanagement interessant. Die Anforderungen die vom Anwender im klinischen Alltag an die Systeme gestellt werden sind:
%
\begin{table} [H]
	\begin{center}
		\begin{tabular}{l}
		Gute Genauigkeit\\
		Hohe Verfügbarkeit\\
		Leichte Bedienbarkeit\\
		Einfache Einbindung Workflow\\
		Geringe Kosten\\
		Sicherheit\\
		\end{tabular}
	\end{center}
	\caption[Anforderungen Trackingsysteme]{Anforderungen an ein medizintechnisches Messsystem. Nicht vollständig.}
	\label{tab:requirements_system}
\end{table}
%
Die Anforderungen an ein solches System sind somit sehr hoch. Sie müssen über eine entsprechende technische Leistungsfähigkeit verfügen (Genauigkeit, Verfügbarkeit etc.) und gleichzeitig muss der Umgang mit ihnen leicht sein und dürfen keine hohen Kosten für Anschaffung und Betrieb generieren.\\
%
\subsubsection{Stand der Technik}
Es befinden sich Trackingsysteme unterschiedlicher Hersteller am Markt. Sie beruhen auf unterschiedlichsten Messprinzipien und unterliegen den daraus resultierenden Limitierungen. Die wichtigsten Technischen Unterschiede sind im Folgenden tabellarisch zusammengefasst:
%
\begin{table} [H]
	\begin{center}
		\begin{tabular}{rllll}
			\textbf{Arbeitsweise} & Optisch & Magnetisch & Ultraschall & Funk (UHF) \\
			\textbf{Genauigkeit} & gut & ausreichend & gut & sehr gut\footnote{Abhängig vom Messprinzip} \\
			\textbf{Frequenz} & mittel & hoch & gering & hoch \\
			\textbf{Volumen} & mittel & klein & mittel & groß \\
			\textbf{LOS} & Ja & Ja & Nein & Ja \\
			\textbf{IV\footnote{in vivo lat. im Lebendigen; med. im Patienten}} & Nein   & Nein & Nein & Ja \\
%			
		\end{tabular}
	\end{center}
	\caption[Übersicht Navigationsverfahren]{Grobe Übersicht und Einteilung verschiedener Navigationsverfahren anhand ihres physikalischen Messprinzips.}
	\label{tab:overview_tracking}
\end{table}
%
Die Tabelle~\ref{tab:overview_tracking} teilt die unterschiedlichen Systeme anhand ihres physikalischen Messprinzips ein. Herausgestellt werden vor Allem die wesentlichen Messparameter der betreffende Aspekte der Verfahren. Aus der Auflistung lassen sich Vor. und Nachteile ableiten.\\
Das größte Problem ist das eine direkten Sicht auf die Objekte vorausgesetzt wird, Line of Sight (LOS) genannt. Dem sog. LOS-Problem unterliegen fast alle Verfahren, die ein großes Messvolumen abdecken. Auf Funk basierende Methoden erfahren dieses Problem nicht, ihre Wechselwirkungsmechanismen erlauben die Durchdringung von Materie. Andere Wechselwirkungen machen diesen Verfahren zu schaffen und verursachen Schwierigkeiten. Der größte Vorteil des auf Funk basierenden RFID-Verfahrens ist es verschiedene Objekte von einander zu unterscheiden, bzw. sogar zu identifizieren. Die Genauigkeit (im technischen Sinne: hohe Präzision plus hohe Richtigkeit) der Messung ist bei allen Verfahren mindestens ausreichend. \\
Diese Anforderungen allein sind eine große Herausforderung an die Technik. Hinzukommen weitere Anforderungen, die sich z.B. aus dem Ablauf einer Intervention ergeben. Ein System muss eine einfache Integrationsmöglichkeit in diesen Arbeitsablauf bieten. Zusätzlich dürfen durch die Anschaffung und das Material keine großen Kosten entstehen.\\
%

%
%-----------------------------------------------------------------------
%
\section[Motivation]{Motivation}
Die Positionsbestimmung (Tracking) mittels RFID (Radio-Frequency Identification) bietet gegenüber vergleichbaren Methoden (z.B. Ultraschall, Optisch) verschiedene Vorteile. Das wesentlichste Unterscheidungsmerkmal ist, dass keine direkte Sichtlinie sog. LOS (Line of Sight) notwendig ist um ein Objekt zu lokalisieren. Der Grund dafür ist das zugrunde liegende Messprinzip. Insbesondere im Vergleich mit optischen Verfahren ist RFID damit überlegen. Weiterhin erlauben die als Positionsgeber verwendeten Tags zusätzliche Informationen auf ihnen abzulegen, beispielsweise eine Identifikationsnummer und Weiteres. Dadurch wächst das Anwendungsspektrum weiter. Das Auslesen von zusätzlichen Informationen ist in keiner der anderen Technologien möglich.\\

Das von dem Messsystem der {Amedo GmbH} verwendete Verfahren basiert auf der Messung der Phasenlage der Antwort eines Tags. Die Phasenlage ist direkt proportional zu einer Entfernung. Dabei kommt es aufgrund der Physik im wesentlichen zu folgenden Problemen:
\begin{enumerate}
	\item Die Messung der Position erfolgt über die Auswertung der Phasenlage des empfangenen Signals in Bezug auf ein Referenzsignal. Da in der EU sind nur bestimmte Frequenzen für die Verwendung für RFID erlaubt (865,5–867,5 MHz) kann man die Wellenlänge mit: $ \lambda\simeq0,35 m $ angeben. Daraus folgt, dass alle 35 cm die gleiche Konfiguration der Phase vorliegt. In dieser Arbeit wird dieser Umstand Isophasen genannt. Die gewonnene Information aus der Phase ist somit redundant, d.h. es lässt sich durch die Kenntnis der Phase nicht unmittelbar auf die korrekte Postion schließen. Man kann das Problem umgehen in dem man auf die errechnete Position ein ganzzahliges Vielfaches der Wellenlänge addiert. Die sog. Wellenzahl (vgl.~\eqref{eq:Wavenumbers}).
	\item Das System der Amedo STS verwendet eine spezielle Antennenanordnung um die Position zu ermitteln. Dabei wird eine Antennenanzahl >4 eingesetzt. Für jede dieser Antennen muss eine eigene Wellenzahl bestimmt werden. Durch Auslöschung des Signals, Absorption etc. kann es dazu kommen, dass eine Antenne eine unbestimmte Zeit lang kein Signal vom Tag empfängt. Wenn die Antenne nach dieser Zeit erneut ein Signal empfängt ist die ihr zugehörige Wellenzahl unbekannt und muss neu bestimmt werden. 
	\item In realen Umgebungen treten zusätzlich noch Ruflektionen und ein sog. Multipath-Effekt auf. Dabei wird das Signal nicht auf dem Direkten Weg Antenne-Tag-Antenne empfangen sondern über einen unbekannten, längeren Weg. Dadurch kommt es zu einem Fehler in der Phase. Zusätzlich ist dieser Effekt individuell für jede Antenne.
\end{enumerate}

Eine analytische Lösung des Problems ist schwierig und bisher nicht gelungen. In dieser Arbeit soll mittels numerischer Methoden und Modellen die beschriebenen Probleme zu gelöst werden.

%
%Das Problem liegt in den unbekannten, komplex zu modellierenden Verhalten der elektromagnetischen Funkwellen in geschlossenen Räumen (insb. Auslöschung, Multipath, Reflektion). Diese führen zu einem Fehler der Phase und damit direkt zu einer Falschaussage der Position.\\ \\
%\textbf{Beschreibung der Wellenzahl[Referenz auf die Dipl. Arbeit von Bernd]}\\\\
%Ziel dieser Arbeit ist es ein System zu
%implementieren, das eine direkte Abschätzung (Ad-Hoc-Messung) der Wellenzahl erlaubt.
%Dafür werden Methoden der Numerik verwendet um die Uneindeutigkeit der Phasenlage zu
%eliminieren.
%\\ \\ \\
%Tags gibt es mit unterschiedlichen Funktionsweisen, in dieser Arbeit und in dem von der {Amedo STS} verwendeten System kommen passive Tags zum Einsatz. Diese versorgen sich aus den Funksignalen des Abfragegeräts mit der notwendigen Energie und modulieren ihre "Antwort" auf das Trägersignal auf.\\\\\\
%In der Positionsbestimmung wird im Zusammenhang von "Marker" gesprochen. In der in dieser Arbeit werden RFID-Transponder (sog. Tags) als Marker verwendet. D.h. Es wird die Position im Raum von einem Transponder ermittelt. \\
%

%
%-----------------------------------------------------------------------
%
\section[Anforderungen]{Anforderungen an die Lösung}
%
Aus den bisher vorgestellten Überlegungen können nun folgende Anforderungen abgeleitet werden:
%
\begin{enumerate}[itemsep=0mm]
	\item Lösung muss schnell (ideal < 1 Sekunde) gefunden werden
	\item Unabhängigkeit von Stütz- und Kalibrierpunkten
	\item Eindeutigkeit der Lösung
	\item Eignung für ein großes Messvolumen
	\item Nahtlose Integration in das bestehende Software Ökosystem
	\item Stand der Softwaretechnik entsprechend
%
\end{enumerate}
%
\begin{figure}[ht!]
         \centering
         % Spiderweb Diagram
%
% Author: Dominik Renzel
% Date; 2009-11-11
\documentclass{article}
\usepackage{tikz}
\usetikzlibrary{patterns}
\usetikzlibrary{shapes}

\begin{document}

\newcommand{\D}{6} % number of dimensions (config option)
\newcommand{\U}{5} % number of scale units (config option)

\newdimen\R % maximal diagram radius (config option)
\R=3.5cm 
\newdimen\L % radius to put dimension labels (config option)
\L=4cm

\newcommand{\A}{360/\D} % calculated angle between dimension axes  

\begin{figure}[htbp]
 \centering

\begin{tikzpicture}[scale=1]
\def\firstcircle{\draw  [color=gray,line width=1.5pt,opacity=0.5, pattern=north west lines,pattern color=gray] (D2-3) -- (D3-4) -- (D4-4) -- (D3-3) -- (D2-3)}
\def\secondcircle{(45:2cm) circle (1.5cm)}

  \path (0:0cm) coordinate (O); % define coordinate for origin

  % draw the spiderweb
  \foreach \X in {1,...,\D}{
    \draw (\X*\A:0) -- (\X*\A:\R);
  }

  \foreach \Y in {0,...,\U}{
    \foreach \X in {1,...,\D}{
      \path (\X*\A:\Y*\R/\U) coordinate (D\X-\Y);
      \fill (D\X-\Y) circle (1pt);
    }
    \draw [opacity=0.3] (0:\Y*\R/\U) \foreach \X in {1,...,\D}{
        -- (\X*\A:\Y*\R/\U)
    } -- cycle;
  }

  % define labels for each dimension axis (names config option)
%  \path (1*\A:\L) node (L1) {\tiny Security};
%  \path (2*\A:\L) node (L2) {\tiny Content Quality};
%  \path (3*\A:\L) node (L3) {\tiny Performance};
%  \path (4*\A:\L) node (L4) {\tiny Stability};
%  \path (5*\A:\L) node (L5) {\tiny Usability};
%  \path (6*\A:\L) node (L6) {\tiny Generality};
  \path (1*\A:\L) node (L1) { Unabhängigkeit};
  \path (2*\A:\L) node (L2) { Sicherheit};
  \path (3*\A:\L*1.2) node (L3) { Performance};
  \path (4*\A:\L) node (L4) { Großes Messvolumen};
  \path (5*\A:\L) node (L5) { Integration};
  \path (6*\A:\L*1.1) node (L6) { Aktualität};
%  \path (7*\A:\L) node (L7) {\tiny Popularity};

  % for each sample case draw a path around the web along concrete values
  % for the individual dimensions. Each node along the path is labeled
  % with an identifier using the following scheme:
  %
  %   D<d>-<v>, dimension <d> a number between 1 and \D (#dimensions) and
  %             value <v> a number between 0 and \U (#scale units)
  %
  % The paths will be drawn half-opaque, so that overlapping parts will be
  % rendered in a composite color.

  % Example Case 1 (red)
  %
  % D1 (Security): 0/7; D2 (Content Quality): 5/7; D3 (Performance): 0/7;
  % D4 (Stability): 6/7; D5 (Usability): 0/7; D6 (Generality): 5/7;
  % D7 (Popularity): 0/7
%  \draw [color=red,line width=1.5pt,opacity=0.5]
%    (D1-0) --
%    (D2-5) --
%    (D3-0) --
%    (D4-6) --
%    (D5-0) --
%    (D6-5) --cycle;
%    (D7-0) -- cycle;

  % Example Case 2 (green)
  %
  % D1 (Security): 2/7; D2 (Content Quality): 2/7; D3 (Performance): 5/7;
  % D4 (Stability): 1/7; D5 (Usability): 4/7; D6 (Generality): 1/7;
  % D7 (Popularity): 7/7
  \draw [color=green,line width=1.5pt,opacity=0.5, pattern=north west lines,pattern color=green]
    (D1-4) --
    (D2-3) --
    (D3-3) --
    (D4-4) --
    (D5-5) --
    (D6-3) -- cycle;
%    (D7-7) -- cycle;

  % Example Case 3 (blue)
  %
  % D1 (Security): 1/7; D2 (Content Quality): 7/7; D3 (Performance): 4/7;
  % D4 (Stability): 4/7; D5 (Usability): 3/7; D6 (Generality): 5/7;
  % D7 (Popularity): 2/7
  \draw [color=blue,line width=1.5pt,opacity=0.5]
    (D1-4) --
    (D2-5) --
    (D3-4) --
    (D4-4) --
    (D5-5) --
    (D6-3) -- cycle;
%    (D7-2) -- cycle;

% \begin{scope}
%      \clip \draw
%          (D3-3) --
%          (D4-4) --
%          (D5-5) ;
%      \fill[red] \draw 
%          (D3-3) --
%          (D4-5) --
%          (D5-5) ;
%    \end{scope}
\firstcircle;

\end{tikzpicture}
\caption[Gewichtete Ziele der Arbeit ]{ Gewichtete Zielvorgaben der Arbeit }
\label{fig:spiderweb}
\end{figure}

\end{document}
         \caption[Anforderungsspinne]{ Grafische Übersicht der Anforderungen an das System }
         \label{fig:Requirements}
\end{figure}
%
%-----------------------------------------------------------------------
%
\section{Ziel und Herangehensweise}
%
Das Ziel der Arbeit ist die Entwicklung eines Systems zur Abschätzung der Postion eines Tags. Das Auffinden der Lösung soll die oben abgeleiteten Anforderungen erfüllt. Die Ermittelung einer korrekten Lösung ist jedoch das Wichtigste. Das System wird im Kern die Lösung über ein numerisches Optimierungsverfahren finden, im speziellen kommt das sog. \textit{'Covarianz Matrix Adaption - Evolutionary Strategy'} (CMA-ES) zum Einsatz. Bei diesem Verfahren handelt es sich um ein stochastische, Ableitungsfreies Verfahren, dass für nicht lineare, nicht konvexe, kontinuierliche Probleme geeignet ist. Dazu wird zuerst ein Modell entworfen werden, dass sich für einen Einsatz in diesem Verfahren eignet. Das eingesetzte Lösungsverfahren stellt praktisch keine Anforderungen an ein solches Modell. Daher soll es mit möglichst wenig Annahmen/ Einschränkungen auskommen und dennoch ein relativ sicheres, reproduzierbares Ergebnis liefern. Weiterhin soll eine eine Integration der Lösung in das Software-Ökosystem der \amedogmbh erfolgen. Ferner sollen Methoden in weiteren Projekten zum Einsatz kommen, beider Umsetzung ist auf eine größtmögliche Wiederverwendbarkeit zu achten. Es werden verschiedene Implementationen des CMA-ES-Algorithmus recherchiert, verglichen und die geeignetste gewählt. Das System soll unmittelbar in den Produkten der \amedogmbh zum Einsatz kommen können, daher wird eine entsprechende Schnittstelle für andere Software implementiert werden. Im Rahmen dieser Arbeit wird eine Methode entwickelt werden, um die Position von frei im Raum angeordnete Antennen zu ermitteln und dem Messaufbau zu kalibrieren.
%
