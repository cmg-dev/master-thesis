%-----------------------------------------------------------------------
\section[Motivation]{Motivation}
Die Positionsbestimmung mittels RFID ist eine vielversprechende Technik. Die Bestimmung der Position (im Folgenden "Tracking" genannt) mittels RFID  bietet gegenüber vergleichbaren Methoden (z.B. Ultraschall, Optisch) verschiedene Vorteile. Das wesentlichste Unterscheidungsmerkmal ist, dass keine direkte Sichtlinie sog. LOS notwendig ist um ein Objekt zu lokalisieren. Der Grund dafür ist das zugrunde liegende Messprinzip. Es werden elektromagnetische Signale ausgewertet, die anderen Wechselwirkungen unterliegen und somit Materie durchdringen. Insbesondere im Vergleich mit optischen Verfahren ist RFID damit überlegen. Die Eigenschaft Materie zu durchdringen erlaubt es Tags im Patienten zu lokalisieren, entsprechende Untersuchungen über die Positionsgenauigkeit im Körper sind vielversprechend.[REFERENZEN]\\
Auf den Tags können zusätzliche Informationen hinterlegt werden, beispielsweise eine Identifikationsnummer oder Ähnliches. Dadurch wächst das Anwendungsspektrum weiter[REFERENZEN]. Das Auslesen von zusätzlichen Informationen ist mit keiner der anderen Technologien möglich.\\
Das von dem Messsystem der {Amedo GmbH} verwendete Verfahren basiert auf der Messung der Phasenlage der Antwort eines Tags. Die Phasenlage ist direkt proportional zu einer Entfernung, sie ist jedoch nicht Eindeutig (siehe \ref{sec:Measurement1})

Eine analytische Lösung des Problems ist schwierig und bisher nicht gelungen. Diese Ansätze scheiterten an der Komplexität des Problems\footnote{siehe \ref{sec:Komplexity1} und \ref{sec:Komplexity2}} oder benötigen sehr aufwändige Messreihen mit großer Anzahl an Messpunkten \cite{amedo1}. Das limitiert die Praxistauglichkeit der Verfahren.\\
In dieser Arbeit soll mittels numerischer Methoden und Modellen die beschriebenen Probleme zu gelöst werden.
%
%-----------------------------------------------------------------------
\section{Voraussetzungen}
%
Dieses Kapite stellt die Grundlagen dieser Arbeit vor. Zuerst werden mathematische und technische Grundlagen erläutert, anschließend wird das Messystem der amedo GmbH vorgestellt.
%
%-----------------------------------------------------
\subsection[mathematisches]{Mathematische Voraussetzungen}
%
\subsection{Kondition}
%
Gegeben ist ein lineares Gleichungssystem der Form:
$$ \mathbf{A}\mathbf{x}-\mathbf{b} =\mathbf{0} $$
Eine numerische Lösung führt in der Regel zu einer von $\mathbf{0}$ verschiedenen Lösung (insbesondere bei überbestimmten Systemen), so das wir:
$$ \mathbf{A}\mathbf{\tilde{x}}-\mathbf{b} =\mathbf{r} $$
schreiben. Man nennt $\mathbf{r}$ den Residuumvektor. Es ist offensichtlich, dass ein kleines Residuum nicht hinreichend ist um von einem kleinen relativen Fehler auszugehen.\\
Weiter folgt aus $\mathbf{A}\mathbf{x}-\mathbf{b} =\mathbf{0}$ und $\mathbf{A}\mathbf{\tilde{x}}-\mathbf{b} =\mathbf{r}$,dass 
$$ \mathbf{A}\Delta\mathbf{x}=\mathbf{r}$$
und damit:
$ 
\lVert \mathbf{b} \rVert=\lVert \mathbf{Ax} \rVert \leq \lVert \mathbf{A} \rVert \lVert \mathbf{x} \rVert
$, 
$
\lVert \Delta\mathbf{x} \rVert=\lVert -\mathbf{A^{-1}r} \rVert \leq \lVert \mathbf{A^{-1}} \rVert \lVert \mathbf{r} \rVert
$
Wir können nun für den relativen Fehler schreiben:
$$
\frac{\lVert \Delta\mathbf{x} \rVert}{\lVert \mathbf{x} \rVert} \leq 
\frac{\lVert \mathbf{A^{-1}} \rVert \lVert \mathbf{r} \rVert}{\lVert \mathbf{b} \rVert / \lVert \mathbf{A} \rVert} =
\lVert \mathbf{A} \rVert \lVert \mathbf{A^{-1}} \rVert \frac{\lVert \mathbf{r} \rVert}{\lVert \mathbf{b} \rVert}
$$
Der Term $\lVert \mathbf{A} \rVert \lVert \mathbf{A^{-1}} \rVert := \text{cond}(\mathbf{A})$ heißt Konditionszahl. Auch der Begriff Konditionsmaß ist gebräuchlich und bezieht sich auf die gewählte Matrixnorm.
Es kann gezeigt werden, dass $\text{cond}(\mathbf{A}) \gg 1$  für eine schlechte Konditionierung der Matrix steht. Wird im Folgenden von einer speziellen Matrixnorm gesprochen schreiben wir $\text{cond}(\mathbf{A})$ zu 
$$ 
\text{cond}_k(\mathbf{A}) = \lVert \mathbf{A} \rVert_k \lVert \mathbf{A^{-1}} \rVert_k
$$ \\
Der Index $k$ wird entsprechend für die verwendete Norm ersetzt. Beispielsweise ergibt sich für die Konditionszahl der Spektralnorm\footnote{\url{http://de.wikipedia.org/w/index.php?title=Spektralnorm&oldid=118988565}}:
$$ 
\text{cond}_2(\mathbf{A}) = \lVert \mathbf{A} \rVert_2 \lVert \mathbf{A^{-1}} \rVert_2=
\sqrt{\frac{\mu_{max}}{\mu_{min}}}\\
$$
Die Symbole $\mu_{max}$ und $\mu_{min}$ stehen für die Eigenwerte des Systems.\\

Die Konditionszahl ermöglicht eine Analyse der Güte einer Lösung, die mittels Numerischer Verfahren ermittelt wurde. Nach \cite{hermann2006numerische} kann man folgende Aussage über die Konditionszahl treffen:
%
\begin{quote}
"Wird ein lineares Gleichungssystem $Ax=b$ mit $t$-stelliger dezimaler Gleitpunktarithmetik gelöst und beträgt die Konditionszahl $\text{cond}(A) \approx10^\alpha$, so sind auf Grund der im allgemeinen unvermeidbaren Fehler in den Eingabedaten $A$ und $b$ nur $t-\alpha-1$ Dezimalstellen der berechneten Lösung $\tilde{x}$ (bezogen auf die betragsgrößte Komponente) sicher."
\end{quote}

\subsection{SVD}
%Um die Konditionszahl zu bestimmen sind aufwändige Berechnungen\footnote{siehe Wochenbericht KW 22} der Eigenwerte der Matrix notwendig. Es wurde nach eine Möglichkeit gesucht diese effizient abzuschätzen oder zu berechnen. Vor Allem soll es auch möglich sein mit dem Verfahren eine nicht symmetrische, nicht quadratische.\\
%Eine Methode die diese Anforderungen erfüllt, ist die sog. Singulärwertzerlegung (im Folgenden SVD := engl. Singular Value Decomposition). 
Bei dem Verfahren der Singular Value Decompostion (oder auch Singulärwertzerlegung), kurz SVD, handelt es sich um eine Faktorisierung einer Matrix. Die Matrix wird dabei als Produkt von drei Matrizen dargestellt. Diese Matrizen enthalten die sog. Singulärwerte und können aus einer der Matrizen abgelesen werden. Die Eigenschaften des Systems sind, ähnlich den Eigenwerten, aus den Singulärwerten bestimmbar. Besonders an der SVD ist, die Existenz für jede Form von Matrix - einschließlich nicht quadratischer Matrizen.\\
Die SVD basiert auf folgender Theorie der linearen Algebra: Jede $M \times N$ Matrix $\mathbf{A}$ kann als Produkt einer $M \times N$ Spalten-orthogonalen Matrix $\mathbf{U}$, einer $N \times N$ Diagonalmatrix $\mathbf{\Sigma}$ mit Werten $\geq 0$ und einer dritten adjungierten $N \times N$-Matrix $\mathbf{V^*}$, so ergibt sich:
%
\begin{equation}
\mathbf{A}= \mathbf{U \Sigma V^*} = \mathbf{U \Sigma V}^T
\end{equation}
Ist $\mathbf{A}$ eine reelwertige Matrix gilt: $ \mathbf{V^*} = \mathbf{V}^T $. Die Matrix $\mathbf{ \Sigma }$ ist im Rahmen dieser Arbeit von besonderem Interesse, denn sie enthält die Singulärwerte $\sigma_r$. Ihre Gestalt ist wie folgt:
%
\begin{equation}
	\mathbf{\Sigma} = \left(\begin{array}{ccc|ccc}
	\sigma_1 &          &          &        & \vdots &        \\
	         & \ddots   &          & \cdots & 0      & \cdots \\
	         &          & \sigma_r &        & \vdots &        \\
	\hline
	         &  \vdots  &          &        & \vdots &        \\
	\cdots   &  0       & \cdots   & \cdots & 0      & \cdots \\
	         &  \vdots  &          &        & \vdots &        \\
	
	\end{array}\right)\nonumber
\end{equation}
%
\phantomeq{\mathbf{\Sigma} = }{ ,wobei~\sigma_1\geq\sigma_2\geq\cdots\geq\sigma_r> 0 \nonumber}
%
Da die $\sigma_r$ der Matrix mit den Eigenwerten in Verbindung stehen, kann aus dieser Matrix die Konditionszahl bestimmt werden. Sie ist durch folgendes Verhältnis gegeben: 
\begin{equation}
	\label{eq:cond_from_svd}
	cond(\mathbf{A})=\frac{max(\sigma_r)}{min(\sigma_r)}=\frac{max(\sigma_1)}{min(\sigma_r)}
\end{equation} 

Es gibt bereits viele Implementationen des Verfahrens, z.B. \cite{press2007numerical}. Diese Implementation wird durch den Erwerb der entsprechenden Lizenz im Rahmen dieser Arbeit verwendet. Weitere Informationen zum Verfahren sind z.B. in \cite[Kaptiel 4.6.3]{bronstejn2012taschenbuch} zu finden.
%
%-----------------------------------------------------
\subsection[technisches]{Technische Voraussetzungen}
In diesem Abschnitt werden die technischen Grundlagen für diese Arbeit vorgestellt und das Wichtigste erörtert. Es kann nicht im vollem Umfang auf die Details der Technik eingegangen werden ohne den Rahmen dieser Arbeit zu sprengen. Interessierte sei die referenzierte Literatur für eine weite Lektüre empfohlen.
%
\subsection{RFID}
\label{sec:Measurement1}
%
\begin{enumerate}
	\item Die Messung der Position erfolgt über die Auswertung der Phasenlage des empfangenen Signals in Bezug auf ein Referenzsignal. In der EU gibt es verschiedene, zulässige RFID-Frequenzen\footnote{text} (865,5?867,5 MHz) kann man die Wellenlänge mit: $ \lambda\simeq0,35 m $ angeben. Daraus folgt, dass alle 35 cm die gleiche Konfiguration der Phase vorliegt. In dieser Arbeit wird dieser Umstand Isophasen genannt. Die gewonnene Information aus der Phase ist somit redundant, d.h. es lässt sich durch die Kenntnis der Phase nicht unmittelbar auf die korrekte Postion schließen. Man kann das Problem umgehen in dem man auf die errechnete Position ein ganzzahliges Vielfaches der Wellenlänge addiert. Die sog. Wellenzahl (vgl.~\eqref{eq:Wavenumbers}).
	\item Das System der Amedo STS verwendet eine spezielle Antennenanordnung um die Position zu ermitteln. Dabei wird eine Antennenanzahl >4 eingesetzt. Für jede dieser Antennen muss eine eigene Wellenzahl bestimmt werden. Durch Auslöschung des Signals, Absorption etc. kann es dazu kommen, dass eine Antenne eine unbestimmte Zeit lang kein Signal vom Tag empfängt. Wenn die Antenne nach dieser Zeit erneut ein Signal empfängt ist die ihr zugehörige Wellenzahl unbekannt und muss neu bestimmt werden. 
	\item In realen Umgebungen treten zusätzlich noch Ruflektionen und ein sog. Multipath-Effekt auf. Dabei wird das Signal nicht auf dem Direkten Weg Antenne-Tag-Antenne empfangen sondern über einen unbekannten, längeren Weg. Dadurch kommt es zu einem Fehler in der Phase. Zusätzlich ist dieser Effekt individuell für jede Antenne.
\end{enumerate}
\lipsum[1-2]

%
%
%-----------------------------------------------------------------------
\section{Anforderungen an das Verfahren}
Die bisher vorgestellten Überlegungen erlauben Anforderungen an das verwendete Verfahren zu stellen:
\begin{enumerate}
\item Lösung muss schnell (ideal < 1 Sekunde) gefunden werden
\item Unabhängigkeit von Stütz- Kalibrierpunkten
\item Eindeutigkeit der Lösung
\item Eignung für ein großes Messvolumen (mehrere Kubikmeter)
%
\end{enumerate}
%
%-----------------------------------------------------------------------
%- Section 1 ----------------------------------------------------------------
\label{seq:EvolutionaryStrategies}
Folgende Information entstammen im Wesentlichen aus \cite{kost2003optimierung},\cite{bronstejn2012taschenbuch}\ sowie \cite{Hansen:1} und sind auf den folgenden Seiten lediglich zusammengefasst und neu arrangiert um eine Einarbeitung in die Thematik zu ermöglichen.\\
%------------------------------------------------------------------
\subsection{Evolutionsstrategien - Grundlagen }
%
Nach dem Vorbild natürlicher Evolution entworfene stochastische Optimierungsverfahren werden Evolutionsstrategie bezeichnet. Sie verwenden die Prinzipien der Mutation, Rekombination und Selektion analog zu der nat. Evolution. Der Grundlegende Ablauf dieser Strategien zeigt die Abbildung~\ref{es_flowchart}\\
Wie in der Natur auch werden Nachkommen aus der Menge der verfügbaren Eltern gebildet. Dabei bezeichnet im Folgenden:
%
\begin{itemize}
	\item $\mu$ die Anzahl der Eltern (=> Größe der Population)
	\item $\lambda$\footnote{Anmerkung: Die Verwendung des Symbols $\lambda$ ist in diesem Kontext nicht eindeutig. Im Rahmen dieser Arbeit steht dieses Symbol auch für die Wellenlänge. In diesem Abschnitt wird jedoch weiterhin $\lambda$ verwendet um die gleiche Nomenklatur wie bei dieser Thematik üblich zu verwenden.} die Anzahl der Eltern die bei Rekombination neue Kinder erzeugt; Die Anzahl der erzeugten Nachkommen einer neuen Generation
	\item $\mathbf{x}_p$ Elternpunkt (Parent)
	\item $\mathbf{x}_c$ Nachkomme einer Generation (Child)
	\item $X_p^1$ Die Menge aller Eltern der ersten Generation $X_p = \{\mathbf{x}_{p_1}^1,..,\mathbf{x}_{p_\mu}^1\}$
	\item $X_p^k$ Die Menge aller Eltern der k-ten Generation $X_p = \{\mathbf{x}_{p_1}^k,..,\mathbf{x}_{p_\mu}^k\}$
\end{itemize}
%
Wir wollen nun in Abbildung~\ref{fig:es_flowchart} einen Blick auf den prinzipiellen Ablauf dieses Algorithmus werfen und anschließend auf die Details eingehen.
%
%------------------------------------------------------------------------------
%------------------------------------------------------------------------------
%------------------------------------------------------------------------------
\begin{figure}[h]
	\begin{center}
		\caption[Ablauf Evolutionsstrategie]{Der Prinzipielle Ablauf des $(\lambda,\mu)$-Evolutionsalgorithmus.}
		\label{fig:es_flowchart}
		\vspace{0.5cm}
		\begin{tikzpicture}[auto]
		\scriptsize
			\tikzstyle{decision} = [diamond, draw=black, thick, fill=black!20, text width=5em, text badly centered, inner sep=1pt]
%			
			\tikzstyle{block} = [rectangle, draw=black, thick, fill=gray!20, text width=15em, text centered, rounded corners, minimum height=4em]
%	
			\tikzstyle{line} = [draw, thick, -latex',shorten >=1pt];
			\tikzstyle{commentline} = [draw, dashed, green!50,-latex',shorten >=1pt];
%	
			\tikzstyle{cloud} = [ dotted, draw=green!50, thick, ellipse,,fill=green!20, minimum height=2em, text width= 10em, text badly centered];
%	
			\matrix [column sep=5mm,row sep=7mm]
			{
				% row 1
				& \node [block] (start) { Start }; & \\
				% row 2
				&\node [block] (init) {Erstelle Startpopulation $X_p^1$ bestehend aus $\mu$-Individuen }; & 
				\node [cloud] (comment1) {Initialisierung, mit Zufallswerten}; \\
				% row 4
				& \node [block] (identify) {Erzeuge eine Menge von $\lambda$ Nachkommen $X_c^k$ aus der aktuellen Elterngeneration $X_p^k$ durch Rekombination \&\&, || Mutation}; & \\
				% row 5
				\node [block] (update) {Nächste Stufe der Evolution; k++}; &
				\node [block] (evaluate) {Durch Selektion die besten $\mu$ Nachkommen für die Generation $X_p^{k+1}$ auswählen}; & \\
				% row 6
				& \node [decision] (decide) {$\Delta \geq \Delta_{min}$}; & 
				\node [cloud] (criteria) {Abbruchkriterium; Muss geeignet gewählt werden, bspw. max. Anzahl der Generationen oder Erreichen des Optimums};\\
				% row 7
				& \node [block] (stop) {Ende}; & \\
			};
% Arrows
			\tikzstyle{every path}=[line]
			\path (init) -- (identify);
			\path (identify) -- (evaluate);
			\path (evaluate) -- (decide);
			\path (update) |- (identify);
			\path (decide) -| node [near start] {Ja} (update);
			\path (decide) -- node [midway] {Nein} (stop);
			\path (start) -- (init);
			
			\tikzstyle{every path}=[commentline]
			\path (criteria) -- (decide);
			\path (comment1) -- (init);
			
		\end{tikzpicture}
	\end{center}
\end{figure}
%------------------------------------------------------------------
\subsubsection[Mutation]{Mutation}
Ein Nachkomme $\mathbf{x}_C$ wird aus seinem Elternteil $\mathbf{x}_P$ und einer zufälligen Variation $\mathbf{d}$ gebildet.
\begin{equation} \label{eq:Mutation_Child}
	\mathbf{x}_c = \mathbf{x}_P + \mathbf{d}
\end{equation}
Dabei ist $\mathbf{d}$ ein bei jeder Mutation neu zu bestimmender $(0,\sigma^2)-normalverteilte$ Zufallszahl $Z(0,\sigma^2)$:
\begin{equation}\label{eq:wavenumber_trilateration_model2}
\mathbf{d}=
\left(
	\begin{array}{c}
		d_1 \\
		\vdots\\
		d_n 
	\end{array}
\right)
=
\left(
	\begin{array}{c}
		Z(0,\sigma_1^2) \\
		\vdots\\
		Z(0,\sigma_n^2) 
	\end{array}
\right)
=
\left(
	\begin{array}{c}
		Z(0,1) \sigma_1 \\
		\vdots\\
		Z(0,1) \sigma_n 
	\end{array}
\right)
\end{equation}
%
Die Normalverteilung der Variation ist nützlich, da kleine Änderungen wahrscheinlicher sind als große. Die maximale Größe der Variation wird durch die Standardabweichung $\sigma_i$ bestimmt. Sie steuert somit die Schrittweite von Generation zu Generation.
%
%------------------------------------------------------------------
\subsubsection[Rekombination]{Rekombination}
Durch Rekombination zweier oder mehr Eltern aus der Menge aller $\mu$-Eltern $X_{\varrho} \subset X_E$. Die Wahl der Eltern sollte zufällig erfolgen um Inzuchtprobleme zu verhindern.\\
Zwei Arten der Rekombination sind denkbar:\\

Die \textit{intermediär Rekombination} erstellt einen Nachkommen durch das gewichtete Mittel von $\varrho$ Eltern.
%
\begin{equation}
\mathbf{x}_c = \Sigma^\varrho_{i=1} \alpha_i\mathbf{x}_{p_i},\\ \Sigma^\varrho_{i=1} \alpha_i = 1,\\ 2\leq\varrho\leq\mu
\end{equation} 
%
Bei der \textit{diskreten Rekombination} vom $\varrho$-Eltern wird die \textit{i}-te Komponente $x_{ic}$ eines Nachkommen $\mathbf{x}_c$ mit der \textit{i}-te Komponente eines zufällig gewählten Elternpunktes gleichgesetzt.
%
\begin{equation}
\mathbf{x}_{ic} = \mathbf{x}_{ip_j},\\ j\in\{1,...,\varrho\},\\i=1,...,n
\end{equation} 
%
%- Section .4 -----------------------------------------------------------------
\subsubsection[Selektion]{Selektion}
Die durch Rekombination und/oder Mutation erzeugten Nachkommen werden in dem Schritt Ausgewählt um einen Evolutionsfortschritt zu erreichen. Dies erfolgt anhand des Vergleichs mit dem Zielfunktionswert $f(\mathbf{x})$. Das beste Individuum oder die besten werden für die nachfolgende Generation ausgewählt. Dabei gibt es Strategien bei denen nur die Nachkommen an der Auswahl beteiligt sind und welche bei denen Eltern und Kinder teilnehmen.

%- Section .5 -----------------------------------------------------------------
\subsubsection{Evolutionsalgorithmus}
%
Der eigentliche Evolutionsalgorithmus ist in Abbildung~\ref{fig:es_flowchart} dargestellt. Er enthält im wesentlichen die in den vorherigen Abschnitten beschriebenen Schritte. Der prinzipielle Ablauf ist für alle Evolutionsalgorithmen gleich. Eine Unterscheidung der Verfahren kann durch verschiedene Parameter beschrieben werden. Wesentlich dabei sind die Populationsgröße $\mu$, die Anzahl an der Rekombination beteiligten Eltern $\varrho$, die gewählte Selektionsstrategie sowie die Anzahl der Nachkommen $\lambda$. Im Folgenden sind zuerst einige Beispiele für die Nomenklatur der Selektionsstrategie aufgeführt, die im Anschluss genauer beschrieben werden.\\
Für Strategien die nur auf Mutation für die Erzeugung von Nachkommen setzten sind folgende Nomenklaturen gebräuchlich:
\begin{itemize}
\item $(\mu+\lambda)$ Elternelemente werden in der Selektion berücksichtigt
\item $(\mu,\lambda)$ Ausschließlich Nachkommen nehmen an der Selektion teil
\end{itemize}
%
Die Strategien werden Plus- bzw. Komma-Strategie genannt. bei der Plus-Strategie wird zusätzlich noch ein gewichtungsfaktor eingeführt, der das "altern" der Elterngeneration darstellt. Dieser Mechanismus soll verhindern, dass die Eltern, nach einer gewissen Anzahl an Generationen, nicht mehr berücksichtigt werden.\\
Wird die Rekombination eingesetzt kann auch die Anzahl der beteiligten Elternelemente angegeben werden:
\begin{itemize}
\item $({\mu}/{\varrho}+\lambda)$ \& $({\mu}/{\varrho},\lambda)$ Angabe der Anzahl beteiligter Eltern bei der Rekombination.
\end{itemize}
%
Mithilfe der hier beschrieben Klassifikationen werden die Algorithmen im Folgenden stets angegeben.\\

In Abbildung~\ref{fig:es_flowchart} wird der Ablauf einer Optimierung mit evolutionären Verfahren dargestellt. Es wird die Komma-Strategie gezeigt, ein Struktogramm der Plus-, oder anderer Strategien ist nicht gezeigt. Die Unterschiede würden sich in dem Punkt Rekombination zeigen.
%
%------------------------------------------------------------------------------
%- Section .6 -----------------------------------------------------------------
\subsection{Strategien mit mehreren Populationen}
Es ist möglich die Strategien auf die Ebene von Populationen zu erweitern. Das bedeutet, man lässt ganze Populationen miteinander in Wettstreit treten und nur diejenige überleben, die die besten Ergebnisse liefern. Das mündet in einem zweistufigen Evolutionsprozess. Man kann die Notation um diesen Umstand erweitern und erhält so:
$$
[\mu_2/\varrho_2,^{+}\lambda_2(\mu_1/\varrho_1,^{+}\lambda_1)]
$$
Sprich aus $\mu_2$-Elternpopulationen werden durch Rekombination mit jeweils $\varrho_2$ Populationen, $\lambda_2$ Nachkommenpopulationen generiert. Innerhalb der Populationen erfolgt die Optimierung anhand einer $({\mu_1}/{\varrho_1}+\lambda_1)$ oder $({\mu_1}/{\varrho_1},\lambda_1)$-Strategie. Nun kann nach einer bestimmten Zahl von Generationen die besten Populationen für die nächste Generation ausgewählt werden. Auch hier stehen verschiedene Auswahlkriterien zur Verfügung. Man kann z.B. die Population anhand des Zielfunktionswert des besten Individuums wählen oder den Mittelwert über alle Individuen wählen.
%
%- Section .7 -----------------------------------------------------------------
\subsection{Optimierungsräume}
\lipsum[1]
%
\subsubsection{Kontinuierliche Optimierung}
%
\lipsum[1]
%
\subsubsection{Diskrete Optimierung}
%
\lipsum[1]
%
\subsubsection{Gemischte Optimierung}
%
\lipsum[1]
%
%- Section .8 -----------------------------------------------------------------

