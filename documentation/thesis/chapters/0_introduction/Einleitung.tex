%-----------------------------------------------------------------------
%
\section[Allgemein]{Allgemein}
Mit der Entwicklung der minimal-invasiven Chirurgie, einer Operationsmethode bei der durch sehr kleine Einschnitte in den Körper mit besonders filigranen Operationsinstrumenten operiert wird, verändert sich die Art Operationen durchzuführen grundlegend. Eingriffe können schneller, schonender und effizienter durchgeführt werden. Möglich wird diese Entwicklung durch eine Vielzahl neuartiger technischer Systeme. Die Vorteile gegenüber herkömmlichen Operationstechniken begründen die weite Verbreitung und häufigen Einsatz der minimal-invasiven Techniken.\\
Mit fortschreitender Miniaturisierung der Instrumente geht die optische Kontrolle über das Operationsgebiet sowie Instrumentarium verloren. Diese Information ist unabdingbar für einen Erfolg der Operation und müssen dem Operierenden zu jeder Zeit zur Verfügung stehen. Um an diese Informationen zu gelangen ist es Stand der Technik, durch aufwändige bildgebende Verfahren intraoperativ, d.h. während der Operation, anzuwenden.\\
Beispielsweise werden bei kardiologischen Interventionen (z.B. Platzierung eines Stents durch die Arteria iliaca interna\footnote {innere Beckenarterie- Standardzugang für diese Art von Operationen} in den Coronargefäßen des Herzens) eine permanente Lagekontrolle der Katheter mittels Röntgentechnik durchgeführt. Oder es werden Bilder durch Magnetresonanztomografie oder durch andere bildgebende Verfahren erzeugt. Nicht nur das eine Gewinnung dieser Bilddaten schwierig (MRT) oder gar schädlich (Röntgen) ist, oft muss der Patient dafür samt Instrumentarium umgelagert werden. Das Umlagern bringt weitere Risiken mit sich und ist mit weiterem Aufwand verbunden.\\
Eine Lösung für diese Problem bringen sog. Trackingsysteme. Diese Systeme sind in der Lage eine Position, z.B. eines Instrumentes, zu ermitteln und stellen die benötigten Informationen für den Arzt zur Verfügung. Die verfügbaren Systeme basieren auf unterschiedlichen physikalischen Prinzipien und haben dadurch unterschiedliche Vor- und Nachteile.\\
Die Anwendung solcher Systeme erlaubt außerdem eine softwaregestützte Planung und assistierte Durchführung der Operation. Die Kombination dieser Techniken wird Navigation genannt. Die Möglichkeit der Planung und Kontrolle macht diese Systeme im Zuge der stets steigenden Ansprüche an das Qualitätsmanagement interessant. Die Anforderungen die vom Anwender im klinischen Alltag an die Systeme gestellt werden sind:
%
\begin{table} [H]
	\begin{center}
		\begin{tabular}{l}
		Gute Genauigkeit\\
		Hohe Verfügbarkeit\\
		Leichte Bedienbarkeit\\
		Einfache Einbindung Workflow\\
		Geringe Kosten\\
		Sicherheit\\
		\end{tabular}
	\end{center}
	\caption[Anforderungen Trackingsysteme]{Anforderungen an ein medizintechnisches Messsystem.}
	\label{tab:requirements_system}
\end{table}
%
Die Anforderungen an ein solches System sind somit sehr hoch. Sie müssen über eine entsprechende Technik verfügen und gleichzeitig muss der Umgang mit ihnen leicht sein. Zusätzlich dürfen die Systeme möglichst wenig kosten.\\
%
\subsubsection{Stand der Technik}
Es befinden sich Trackingsysteme unterschiedlicher Hersteller am Markt. Sie beruhen auf unterschiedlichsten Messprinzipien und unterliegen den daraus resultierenden Limitierungen. Die wichtigsten Technischen Unterschiede sind im Folgenden tabellarisch zusammengefasst:
%
\begin{table} [H]
	\begin{center}
		\begin{tabular}{rllll}
			\textbf{Arbeitsweise} & Optisch & Magnetisch & Ultraschall & Funk (UHF) \\
			\textbf{Genauigkeit} & gut & ausreichend & gut & sehr gut\footnote{Abhängig vom Messprinzip} \\
			\textbf{Frequenz} & mittel & hoch & gering & hoch \\
			\textbf{Volumen} & mittel & klein & mittel & groß \\
			\textbf{LOS} & Ja & Ja & Nein & Ja \\
			\textbf{IV\footnote{in vivo lat. im Lebendigen; med. im Patienten}} & Nein   & Nein & Nein & Ja \\
%			
		\end{tabular}
	\end{center}
	\caption[Übersicht Navigationsverfahren]{Grobe Übersicht und Einteilung verschiedener Navigationsverfahren anhand ihres physikalischen Messprinzips.}
	\label{tab:overview_tracking}
\end{table}
%
Die Tabelle~\ref{tab:overview_tracking} teilt die unterschiedlichen Systeme anhand ihres physikalischen Messprinzips ein. Herausgestellt werden vor Allem die wesentlichen Messparameter der betreffende Aspekte der Verfahren. Aus der Auflistung lassen sich Vor. und Nachteile ableiten.\\
Das größte Problem ist das Benötigen einer direkten Sicht auf die Objekte. Dem sog. LOS-Problem unterliegen fast alle Verfahren, die ein großes Messvolumen abdecken. Die auf Funk basierenden Verfahren haben das Problem nicht, unterliegen jedoch anderen Schwierigkeiten. Der größte Vorteil des auf Funk basierenden RFID-Verfahrens ist es verschiedene Objekte von einander zu unterscheiden, zu identifizieren.\\
Die Genauigkeit (im technischen Sinne: Präzision und Wiederholbarkeit) der Messung ist bei allen Verfahren mindestens ausreichend. Das allein stellt viele Techniken vor eine großer Herausforderung. Hinzukommen weitere Anforderungen, die sich aus dem Ablauf einer Intervention ergeben. Ein System muss eine einfache Integrationsmöglichkeit in den Arbeitsablauf bieten.\\
%

Im Folgenden wird auf die Besonderheiten und Merkmale des auf Funk basierenden RFID-Verfahrens eingegangen. Die anderen Verfahren werden, aufgrund der Unterschiedlichkeit der Systeme wird im Rahmen dieser Arbeit wird darauf verzichtet.


%-----------------------------------------------------------------------
%
\section[Motivation]{Motivation}
Die Positionsbestimmung (Tracking) mittels RFID (Radio-Frequency Identification) bietet gegenüber vergleichbaren Methoden (z.B. Ultraschall, Optisch) verschiedene Vorteile. Das wesentlichste Unterscheidungsmerkmal ist, dass keine direkte Sichtlinie sog. LOS notwendig ist um ein Objekt zu lokalisieren. Der Grund dafür ist das zugrunde liegende Messprinzip. Insbesondere im Vergleich mit optischen Verfahren ist RFID damit überlegen. Weiterhin erlauben die als Positionsgeber verwendeten Tags zusätzliche Informationen auf ihnen abzulegen, beispielsweise eine Identifikationsnummer und Weiteres. Dadurch wächst das Anwendungsspektrum weiter. Das Auslesen von zusätzlichen Informationen ist in keiner der anderen Technologien möglich.\\

Das von dem Messsystem der {Amedo GmbH} verwendete Verfahren basiert auf der Messung der Phasenlage der Antwort eines Tags. Die Phasenlage ist direkt proportional zu einer Entfernung. Dabei kommt es aufgrund der Physik im wesentlichen zu folgenden Problemen:
\begin{enumerate}
	\item Die Messung der Position erfolgt über die Auswertung der Phasenlage des empfangenen Signals in Bezug auf ein Referenzsignal. Da in der EU sind nur bestimmte Frequenzen für die Verwendung für RFID erlaubt (865,5–867,5 MHz) kann man die Wellenlänge mit: $ \lambda\simeq0,35 m $ angeben. Daraus folgt, dass alle 35 cm die gleiche Konfiguration der Phase vorliegt. In dieser Arbeit wird dieser Umstand Isophasen genannt. Die gewonnene Information aus der Phase ist somit redundant, d.h. es lässt sich durch die Kenntnis der Phase nicht unmittelbar auf die korrekte Postion schließen. Man kann das Problem umgehen in dem man auf die errechnete Position ein ganzzahliges Vielfaches der Wellenlänge addiert. Die sog. Wellenzahl (vgl.~\eqref{eq:Wavenumbers}).
	\item Das System der Amedo STS verwendet eine spezielle Antennenanordnung um die Position zu ermitteln. Dabei wird eine Antennenanzahl >4 eingesetzt. Für jede dieser Antennen muss eine eigene Wellenzahl bestimmt werden. Durch Auslöschung des Signals, Absorption etc. kann es dazu kommen, dass eine Antenne eine unbestimmte Zeit lang kein Signal vom Tag empfängt. Wenn die Antenne nach dieser Zeit erneut ein Signal empfängt ist die ihr zugehörige Wellenzahl unbekannt und muss neu bestimmt werden. 
	\item In realen Umgebungen treten zusätzlich noch Ruflektionen und ein sog. Multipath-Effekt auf. Dabei wird das Signal nicht auf dem Direkten Weg Antenne-Tag-Antenne empfangen sondern über einen unbekannten, längeren Weg. Dadurch kommt es zu einem Fehler in der Phase. Zusätzlich ist dieser Effekt individuell für jede Antenne.
\end{enumerate}

Eine analytische Lösung des Problems ist schwierig und bisher nicht gelungen. In dieser Arbeit soll mittels numerischer Methoden und Modellen die beschriebenen Probleme zu gelöst werden.

%
%Das Problem liegt in den unbekannten, komplex zu modellierenden Verhalten der elektromagnetischen Funkwellen in geschlossenen Räumen (insb. Auslöschung, Multipath, Reflektion). Diese führen zu einem Fehler der Phase und damit direkt zu einer Falschaussage der Position.\\ \\
%\textbf{Beschreibung der Wellenzahl[Referenz auf die Dipl. Arbeit von Bernd]}\\\\
%Ziel dieser Arbeit ist es ein System zu
%implementieren, das eine direkte Abschätzung (Ad-Hoc-Messung) der Wellenzahl erlaubt.
%Dafür werden Methoden der Numerik verwendet um die Uneindeutigkeit der Phasenlage zu
%eliminieren.
%\\ \\ \\
%Tags gibt es mit unterschiedlichen Funktionsweisen, in dieser Arbeit und in dem von der {Amedo STS} verwendeten System kommen passive Tags zum Einsatz. Diese versorgen sich aus den Funksignalen des Abfragegeräts mit der notwendigen Energie und modulieren ihre "Antwort" auf das Trägersignal auf.\\\\\\
%In der Positionsbestimmung wird im Zusammenhang von "Marker" gesprochen. In der in dieser Arbeit werden RFID-Transponder (sog. Tags) als Marker verwendet. D.h. Es wird die Position im Raum von einem Transponder ermittelt. \\
%

%
%-----------------------------------------------------------------------
%
\section[Mathematische Voraussetzungen]{Mathematische Voraussetzungen}
%-----------------------------------------------------
Dieser Abschnitt behandelt die mathematischen Voraussetzungen für diese Arbeit.
%
\subsection{Kondition}
\label{sec:condition}
{
\small
Folgende Nomenklatur und Symbole gelten für diesen Abschnitt:
\begin{itemize}[itemsep=0mm]
	\item	fette Großbuchstaben stehen für Matrizen (bspw. $\mathbf{A}$)
	\item	fette Kleinbuchstaben stehen für Vektoren (bspw. $\mathbf{x}$)
	\item	$\mathbf{0} := \text{Nullvektor}$)
\end{itemize}
%
Gegeben ist ein lineares Gleichungssystem der Form:
$$ \mathbf{A}\mathbf{x}-\mathbf{b} =\mathbf{0} $$
Eine numerische Lösung für in der Regel zu einer von $\mathbf{0}$ verschiedenen Lösung so das wir:
$$ \mathbf{A}\mathbf{\tilde{x}}-\mathbf{b} =\mathbf{r} $$
schreiben. Man nennt $\mathbf{r}$ den Residuumvektor. Es ist offensichtlich, dass ein kleines Residuum nicht hinreichend ist um von einem kleinen relaitven Fehler auszugehen.\\
Aus $\mathbf{A}\mathbf{x}-\mathbf{b} =\mathbf{0}$ und $\mathbf{A}\mathbf{\tilde{x}}-\mathbf{b} =\mathbf{r}$ folgt $$ \mathbf{A}\Delta\mathbf{x}=\mathbf{r}$$
und damit:
$ 
\lVert \mathbf{b} \rVert=\lVert \mathbf{Ax} \rVert \leq \lVert \mathbf{A} \rVert \lVert \mathbf{x} \rVert
$, 
$
\lVert \Delta\mathbf{x} \rVert=\lVert -\mathbf{A^{-1}r} \rVert \leq \lVert \mathbf{A^{-1}} \rVert \lVert \mathbf{r} \rVert
$
Wir können nun für den relativen Fehler schreiben:
$$
\frac{\lVert \Delta\mathbf{x} \rVert}{\lVert \mathbf{x} \rVert} \leq 
\frac{\lVert \mathbf{A^{-1}} \rVert \lVert \mathbf{r} \rVert}{\lVert \mathbf{b} \rVert / \lVert \mathbf{A} \rVert} =
\lVert \mathbf{A} \rVert \lVert \mathbf{A^{-1}} \rVert \frac{\lVert \mathbf{r} \rVert}{\lVert \mathbf{b} \rVert}
$$
Der Term $\lVert \mathbf{A} \rVert \lVert \mathbf{A^{-1}} \rVert := \text{cond}(\mathbf{A})$ heißt Konditionszahl. Auch der Begriff Konditionsmaß ist gebräuchlich und bezieht sich auf die gewählte Marixnorm.
Es kann gezeigt werden, dass $\text{cond}(\mathbf{A}) \gg 1$  für eine schlechte Konditionierung der Matrix steht. Wird im Folgenden von einer speziellen Matrixnorm gesprochen schreiben wir $\text{cond}(\mathbf{A})$ zu 
$$ 
\text{cond}_k(\mathbf{A}) = \lVert \mathbf{A} \rVert_k \lVert \mathbf{A^{-1}} \rVert_k
$$ \\

\textbf{Konditionszahl der Spektralnorm}\\
%
Für die Spektralnorm einer Matrix berechnet sich die Konditionszahl zu:
$$ 
\text{cond}_2(\mathbf{A}) = \lVert \mathbf{A} \rVert_2 \lVert \mathbf{A^{-1}} \rVert_2=
\sqrt{\frac{\mu_{max}}{\mu_{min}}}
$$
Nach \{\} kann man folgende Aussage über die Konditionszahl treffen:\\
"Wird ein lineares Gleichungsysten $Ax=b$ mit $t$-stelliger dezimaler Gleitpunktarithmetik gelöst und beträgt die Konditionszahl $\text{cond}(A) \approx10^\alpha$, so sind auf Grund der im allgemeinen unvermeidbaren Fehler in den Eingabedaten $A$ und $b$ nur $t-\alpha-1$ Dezimalstellen der berechneten Lösung $\tilde{x}$ (bezogen auf die betragsgrößte Komponente) sicher."
%\begin{equation}\label{eq:final_trilateration_model}
%0=   \\m\ma\mat\math\mathb\mathbf
%\left(
%	\begin{array}{ccc}
%		x_k-x_0 & y_k-y_0 & z_k-z_0 
%	\end{array}
%\right)
%\left(
%   \begin{array}{c}
%	   x-x_0\\
%	   y-y_0\\
%	   z-z_0
%   \end{array}
%\right)
%-
%\left(
%	\begin{array}{c}
%		c_{k0}
%	\end{array}
%\right) 
%\end{equation}
%%
%Dabei ist:
%\begin{equation}\label{eq:c_k0}
%	c_{k0}=\frac{1}{2}[d_{k0}^2+r_{0}^2-r_k^2]
%\end{equation}
%%
%Ziel dieser Erweiterung ist es, einen Zusammenhang zwischen diesem Modell und der Wellenzahl zu erzeugen. Folgender Ansatz wird gewählt:
%	\begin{equation}\label{eq:r_0_theta} r(\Theta)=\frac{\lambda}{2}\left(\frac{\Theta}{2\pi}+n\right),\\\lambda=\frac{c}{f}, n:= \text{Wellenzahl}
%\end{equation}
%%
%%
%Weiterhin ist $\Theta$ die gemessene Phase, die das PRPS-System liefert und $n$ die gesuchte Wellenzahl.\\
%Durch einsetzen von \eqref{eq:r_0_theta} in \eqref{eq:c_k0}, erhalten wir:
%\begin{equation}\label{eq:c_k0_extended}
%	c_{k0}(\Theta_0, \Theta_k, n_0, n_k) =\frac{1}{2}\left[d_{k0}^2+\frac{\lambda^2}{4}\left(\frac{\Theta_0}{2\pi}+n_0\right)^2-\frac{\lambda^2}{4}\left(\frac{\Theta_k}{2\pi}+n_k\right)^2\right]
%\end{equation}
%%
%Wir stellen Gleichung~\eqref{eq:c_k0_extended} um:
%\begin{align}
%%	
%	c_{k0}(\Theta_0, \Theta_k, n_0, n_k) &= \frac{1}{2}\left\{d_{k0}^2+\frac{\lambda^2}{4}\left[\left(\frac{\Theta_0}{2\pi}\right)^2+2\frac{\Theta_0}{2\pi}n_0+n_0^2 \right.\right.\nonumber\\
%	&\phantom{=}\; 
%	\left.\left.-\left(\frac{\Theta_k}{2\pi}\right)^2-2\frac{\Theta_k}{2\pi}n_k-n_k^2\right]\right\}\\
%%    
%    &=\frac{1}{2}\left\{d_{k0}^2+\frac{\lambda^2}{4}\left[\left(\frac{\Theta_0}{2\pi}\right)^2-\left(\frac{\Theta_k}{2\pi}\right)^2 \right.\right.\nonumber\\
%    &\phantom{=}\;
%   	\left.\left.+2\frac{\Theta_0}{2\pi}n_0-2\frac{\Theta_k}{2\pi}n_k+n_0^2-n_k^2\right]\right\}\\
%%	
%	&=\frac{1}{2}d_{k0}^2+\frac{\lambda^2}{8}\left[\frac{1}{(2\pi)^2}\left(\Theta_0^2-\Theta_k^2\right) \right.\nonumber\\
%	&\phantom{=}\;
%	\left. +\frac{1}{\pi}\left(\Theta_0n_0-\Theta_kn_k\right)+\left(n_0^2-n_k^2\right)\right]\label{c_k0_rearragend}
%\end{align}
%%
%Führen wir nun:
%\phantomeq{c_{k0}(\Theta_0, \Theta_k, n_0, n_k)}{a_{0k} := \frac{1}{2}d_{k0}^2\nonumber}
%\phantomeq{c_{k0}(\Theta_0, \Theta_k, n_0, n_k)}{a_1 := \frac{\lambda^2}{8}\nonumber}
%\phantomeq{c_{k0}(\Theta_0, \Theta_k, n_0, n_k)}{a_2 := a_1\frac{1}{\pi}\nonumber}
%\phantomeq{c_{k0}(\Theta_0, \Theta_k, n_0, n_k)}{a_{3k0} := a_1\frac{1}{(2\pi)^2}(\Theta_0^2-\Theta_k^2)\nonumber}
%%
%in Gleichung~\eqref{c_k0_rearragend} ein, erhalten die finale Form der Gleichung:
%\begin{equation}
%c_{k0}(\Theta_0, \Theta_k, n_0, n_k) = a_{0k}+a_1(n_0^2-n_k^2)+a_2(\Theta_0n_0-\Theta_kn_k)-a_{3k0}\label{c_k0_final_form}   
%\end{equation}
%%
%Die Einführung der Konstanten macht zum Einen die Gleichung übersichtlicher. Zum Anderen können so, mit Blick auf eine spätere Softwareimplementation, Rechenschritte gespart werden. Das sollte sich positiv auf den späteren Berechnungsaufwand auswirken.\\
%%
%Im Weiteren erkennt man durch scharfes hinsehen das in Gleichung~\eqref{c_k0_final_form}, für $\Theta_k=\text{const.}$ \& $\Theta_0=\text{const.}$ gilt. Das resultiert aus der Tatsache, dass . Es ermöglicht uns zu schreiben:
%\begin{equation}
%c_{k0}(\Theta_0, \Theta_k, n_0, n_k) = c_{k0}(n_0, n_k)
%\end{equation}
%%
%Im engeren Sinne einer mathematischen Funktion sollten wir die Parameter alle als Argument aufnehmen. Diese Form soll darstellen, welche Größen von Interesse sind. Im späteren Gebrauch wird diese Gleichung in der Optimierung eingesetzt werden.
%Für unser Gleichungssystem aus\eqref{eq:final_trilateration_model} ergibt sich:
%\begin{equation}\label{eq:wavenumber_trilateration_model}
%0=
%\left(
%	\begin{array}{ccc}
%		x_k-x_0 & y_k-y_0 & z_k-z_0 
%	\end{array}
%\right)
%\left(
%   \begin{array}{c}
%	   x-x_0\\
%	   y-y_0\\
%	   z-z_0
%   \end{array}
%\right)
%-
%\left(
%	\begin{array}{c}
%		c_{k0}(n_0, n_k)
%	\end{array}
%\right)
%\end{equation}
%%
%Betrachten wir nun \eqref{eq:wavenumber_trilateration_model} und setzen $N'=4$, d.h. wir verwenden 4 Antennen. Wir beschreiben die Konfiguration wie folgt: Antenne 0 ist die Referenz-Antenne und Antenne 0-3 sind Messwertgeber für die Phaseninformation. 
%%
%\begin{equation}\label{eq:wavenumber_trilateration_model_explicit}
%0=
%\underbrace{\left(
%	\begin{array}{ccc}
%		x_1-x_0 & y_1-y_0 & z_1-z_0 \\
%		x_2-x_0 & y_2-y_0 & z_2-z_0 \\
%		x_3-x_0 & y_3-y_0 & z_3-z_0 
%	\end{array}
%\right)}_{\textbf{A}}
%\underbrace{\left(
%   \begin{array}{c}
%	   x-x_0\\
%	   y-y_0\\
%	   z-z_0
%   \end{array}
%\right)}_{\textbf{x}}
%-
%\underbrace{\left(
%	\begin{array}{c}
%		c_{10}(n_0, n_1) \\
%		c_{20}(n_0, n_2) \\
%		c_{30}(n_0, n_3)
%	\end{array}
%\right)}_{\textbf{b}}
%\end{equation}
%%
%\begin{equation}
%\mathbf{b}=
%\left(
%	\begin{array}{c}
%		a_{01}+a_1( n_0^2-n_1^2)+a_2(\Theta_0n_0-\Theta_1n_1)-a_{310} \\
%		a_{02}+a_1(n_0^2-n_2^2)+a_2(\Theta_0n_0-\Theta_2n_2)-a_{320} \\
%		a_{03}+a_1(n_0^2-n_3^2)+a_2(\Theta_0n_0-\Theta_3n_3)-a_{330}
%	\end{array}
%\right)
%\end{equation}
%%
%Das Ergebnis ist ein um $\Theta$ und $n$ erweitertes Gleichungssystem. Zusätzlich enthält  es mehrere geometrische Konstanten ($a_{0k}, k=\{1,..,N-1\}$), mehrere Phasen-Konstanten ($a_{3k0}, k=\{1,..,N-1\}$), sowie zwei allgemeine ($a_1$ und $a_2$). Allgemeiner formuliert ergibt sich:
%%
%\begin{multline}\label{eq:final_equation}
%0=
%\left(
%	\begin{array}{ccc}
%		x_k-x_0 & y_k-y_0 & z_k-z_0 
%	\end{array}
%\right)
%\left(
%   \begin{array}{c}
%	   x-x_0\\
%	   y-y_0\\
%	   z-z_0
%   \end{array}
%\right) \\
%-
%\left(
%	\begin{array}{c}
%		a_{0k}+a_1(n_0^2-n_k^2)+a_2(\Theta_0k_0-\Theta_kn_k)-a_{3k0}
%	\end{array}
%	\right)
%\end{multline}
%%
%Aus Gleichung~\eqref{eq:final_equation} ist durch eine geeignete Wahl von $N'=\{4,..,N\}$ sofort ersichtlich wie viele Veränderliche sich für eine gewählte Konstellation an Antennen ergeben. Für $k$ gilt in diesem Fall $k=\{1,..,N'-1\}$.\\
%%
%Beispielsweise ergibt sich für das Modell aus Gleichung~\eqref{eq:final_equation} mit $N'=4$, insgesamt 7 Variablen ($\mathbf{x},n_0,n_1,n_2,n_3$) . Analog würde sich für ein Modell mit allen 8 Antennen, 11 Variablen ($\mathbf{x},n_0,..,n_7$) ergeben.
}
%
\subsection{SVD}
\label{sec:svd}
%Um die Konditionszahl zu bestimmen sind aufwändige Berechnungen\footnote{siehe Wochenbericht KW 22} der Eigenwerte der Matrix notwendig. Es wurde nach eine Möglichkeit gesucht diese effizient abzuschätzen oder zu berechnen. Vor Allem soll es auch möglich sein mit dem Verfahren eine nicht symmetrische, nicht quadratische.\\
%Eine Methode die diese Anforderungen erfüllt, ist die sog. Singulärwertzerlegung (im Folgenden SVD := engl. Singular Value Decomposition). Die SVD basiert auf folgender Theorie der linearen Algebra: Jede $M \times N$ Matrix $\mathbf{A}$ kann als Produkt einer $M \times N$ Spalten-orthogonalen Matrix $\mathbf{U}$, einer $N \times N$ Diagonalmatrix $\mathbf{\Sigma}$ mit Werten $\geq 0$ und einer dritten adjungierten $N \times N$-Matrix $\mathbf{V^*}$, so ergibt sich:
Bei dem Verfahren der SVD (oder auch Singulärwertzerlegung) handelt es sich um eine Darstellung aus dem Produkt von drei Matrizen. Diese Matrizen enthalten die sog. Singulärwerte. Sie können aus einer der Matrizen abgelesen werden. Die Eigenschaften des Systems sind, ähnlich den Eigenwerten, aus den Singulärwerten bestimmbar. Besonders an der SVD ist, die Existenz für jede Form von Matrix - einschließlich nicht quadratischer Matrizen.
%
\begin{equation}
\mathbf{A}= \mathbf{U \Sigma V^*} = \mathbf{U \Sigma V}^T
\end{equation}
Ist $\mathbf{A}$ eine reelwertige Matrix gilt: $ \mathbf{V^*} = \mathbf{V}^T $. Die Matrix $\mathbf{ \Sigma }$ ist von besonderem Interesse, denn sie enthält die Singulärwerte $\sigma_r$ und hat folgende Gestalt.
%
\begin{equation}
	\mathbf{\Sigma} = \left(\begin{array}{ccc|ccc}
	\sigma_1 &          &          &        & \vdots &        \\
	         & \ddots   &          & \cdots & 0      & \cdots \\
	         &          & \sigma_r &        & \vdots &        \\
	\hline
	         &  \vdots  &          &        & \vdots &        \\
	\cdots   &  0       & \cdots   & \cdots & 0      & \cdots \\
	         &  \vdots  &          &        & \vdots &        \\
	
	\end{array}\right)\nonumber
\end{equation}
\begin{equation}
\sigma_1\geq\sigma_2\geq\cdots\geq\sigma_r> 0 \nonumber
\end{equation}
%
Da die $\sigma_r$ der Matrix mit den Eigenwerten in Verbindung stehen, kann aus dieser Matrix die Konditionszahl bestimmt werden. Sie ist durch folgendes Verhältnis gegeben: 
\begin{equation}
	\label{eq:cond_from_svd}
	cond(\mathbf{A})=\frac{max(\sigma_r)}{min(\sigma_r)} 
\end{equation} 

Es gibt bereits viele Implementationen des Verfahrens, z.B. \cite{press2007numerical}. Diese Implementation wird durch den Erwerb der entsprechenden Lizenz im Rahmen dieser Arbeit verwendet.\\
Weiter Informationen zum Verfahren sind in \cite[Kaptiel 4.6.3]{bronstejn2012taschenbuch} zu finden.
%
\section{Optimierung}
\label{sec:Optimization}
%
Dieser Abschnitt führt in die Optimierung ein und gibt einen Überblick über die Grundbegriffe, die im Rahmen dieser Arbeit verwendet werden. Tiefe Einsichten und die theoretischen Grundlagen sind z.B. in \cite{Bomze1,Spellucci1} zu finden.
%
\begin{figure}[ht!]
	\centering
	\caption[Übersicht numerische Verfahren]{ Übersicht über numerische Verfahren (unvollständig). Interessant für diese Arbeit ist der teil \textit{nichtlineare Optimierung}. Dieser wird an andere Stelle ausführlicher gezeigt. }
	\label{fig:overview_numericals}
	\vspace{2mm}
	\begin{tikzpicture}
	  \path[small mindmap,concept color=black,text=white]
	    node[concept] {\tiny\textsf{Numerische Mathematik}}
	    [clockwise from=0]
	    child[concept color=green!55!black] {
	      node[concept] {\tiny\textsf{Optimierung}}
	      [clockwise from=100]
%	      
	      child [concept color=green!55!black] {
	      		      node[concept] {\tiny\textsf{lineare-Opti-mierung}}
	      }
%	      
	      child { node[concept] {\tiny\textsf{nichtlinear-Opti-mierung}} }
	      child { node[concept] {\tiny\textsf{skalare-Opti-mierung}} }
	      child { node[concept] {\tiny\textsf{Vektor-opti-mierung}} }
	    }
	    child[concept color=blue] {
	      node[concept] {\tiny\textsf{Approximation}}
	      [clockwise from=-35]
%	      child { node[concept] {databases} }
%	      child { node[concept] {WWW} }	      
	    }	    
	    child[concept color=red] { node[concept] {\tiny\textsf{partiellen Differentialgleichungen}} }
	    child[concept color=orange] { node[concept] {\tiny\textsf{Integral\-gleichungen}} };
	\end{tikzpicture}
\end{figure}
%
\subsection[Objektfunktion]{Objektfunktion}
%
Für eine Optimierung wird eine Größe benötigt, die optimal werden soll. Dazu bedarf es einer Formulierung der Funktion die das Optimierungsziel enthält. Diese Funktion wird Objektfunktion\footnote{Wird in dieser Arbeit verwendet}, Fitnessfunktion oder Zielfunktion genannt Auch Güte- oder Qualitätsfunktion sind gebräuchlich. Es sind eine Reihe von Optimierungskriterien denkbar, z.B. können Gewicht, Größe, Fläche etc. optimiert werden. Es kann auch mehr als ein Ziel Bedingung sein, in diesem Fall spricht man von Mehrzieloptimierung\footnote{Keine weitere Erläuterung im Rahmen dieser Arbeit}.\\
Damit man eine Optimierung durchführen kann, muss die Objektfunktion freie Parameter (Variablen) enthalten. 
$$
y=f(x),\qquad x\in\mathbb{R}
$$
Diese Formulierung ist eine eindimensionale Objektfunktion. Dabei ist $x$ der freie Parameter den man variieren kann, um ein Optimum der Funktion $f(x)]$ zu finden.  In der Regel ist der Wert des erreichten Optimums nicht sehr interessant. Vielmehr ist man an dem auffinden der Optimalen Einstellung der freien Parameter interessiert.
%
\subsection[Arten von Optima]{Arten von Optima}
%
Man kann leicht verstehen, dass es unterschiedliche Optima gibt. Es kann das z.B. das geringste Gewicht oder der größte Gewinn Ziel der Optimierung sein. Dazu notieren wir:
$$
y=min\{f(\mathbf{x})\} = -max\{-f(\mathbf{x})\},\qquad \mathbf{x}\in\mathbb{R}^n
$$
Die Gleichung drückt aus, dass ich jedes Maximierungsproblem in ein äquivalentes Minimierungsproblem überführen lässt. Daraus kann eine allg. Formulierung des Optimierungsproblem angegeben werden.
\subsubsection{Allgemeine Formulierung des Optimierungsproblems}
%
Ableitung der allgemeinem Formulierung aus den bisherigen Ausführungen. Die Funktion:
$$
y=f(\mathbf{x}), \qquad \mathbf{x}\in\mathbb{R}^n
$$
sei zu optimieren. Dabei ist eine Minimierungsaufgabe gleich der Maximierungsaufgabe. Das bring die Formulierung:
%
\begin{equation}
\label{eq:optimazion1}
min\{~y(\mathbf{x})~|~\mathbf{x}\in\mathbf{X}~\}
\end{equation}
%
In der Formulierung ist eine Einschränkung enthalten. Wir fordern, dass $\mathbf{x}\in\mathbf{X}$ sein soll. Diese Einschränkung entstammt den Nebenbedingung des Problems. Herleitung:
$$
g_j(\mathbf{x})\leq 0, \qquad j=1,2,...,n
$$
Ergibt den zulässigen Bereich:
$$
\mathbf{X}=\{\mathbf{x}\in \mathbb{R}^n | g_j(\mathbf{x})\leq 0, \qquad j=1,2,...,n\}
$$
%
\subsubsection{Allgemeine von Minima}
%
Eine Objektfunktion kann zwei Arten von Minima enthalten:
\begin{itemize}[itemsep=1mm]
\item \textbf{Lokales Minimum} - Ein zulässiger Punkt, bei dem in der Nachbarschaft keine niedrigeren Funktionswerte zu finden sind
\item \textbf{Globales Minimum} - Ein zulässiger Punkt, der den geringsten Funktionswert des gesamten zulässigen Bereichs aufweist. Gleichzeitig ein lokales Minimum.
\end{itemize}
%
Wir haben ein mathematisches Modell der Optimierungsaufgabe erstellt. Wir können noch keine Aussage über die Komplexität des Problem treffen. Um den Schwierigkeitsgrad bestimmen zu können muss man die Zielfunktion analysieren. Eine Analyse der in dieser Arbeit verwendeten Objektfunktion wird im Hauptteil gezeigt.

Ein Optimierungsproblem mit Nebenbedingungen wird registiertes Optimierungsproblem genannt. Ohne spricht man von unregistrierten Optimierungsproblemen.
%
\section{Auffinden von Minima}
%
Die Kenntnis der Zielfunktion und des mathematischen Modells erlaubt die Anwendung eine Algorithmus, der das Minima bestimmt. Übergibt man die Zielfunktion und die Fragestellung an einen solchen Algorithmus berechnet dieser auf unterschiedlichen Wegen mögliche Optima. Man kann zwei Hauptklassen von Verfahren unterscheiden:
%
\begin{itemize}
\item Lineare Optimierungsverfahren
\item Nichtlineare Optimierungsverfahren
\end{itemize}
%
Auf die linearen Verfahren wird im Rahmen dieser Arbeit nicht eingegangen. Sie eignen sich nicht für Komplexe Fragestellungen. Eine Übersicht über die nichtlinearen Verfahren ist in der Abbildung~\ref{fig:overview_optimizations}
%
%- Section .7 -----------------------------------------------------------------
\subsection{Optimierungsräume}
%
In der Optimierung kann man verschiedene Optimierungsräume betrachten. Der häufigste und auf den alle Algorithmen Anwendbar sind ist der reellwertige oder kontinuierliche Raum. Daneben kommen Probleme vor, die auf einem ganzzahligen (diskreten) Raum ablaufen sowie gemischte Probleme. Im Folgenden wird die Modellformulierung für alle Räume besprochen.
%
\subsubsection{Kontinuierliche Optimierung}
%
Es wurde bereits das Modell für die kontinuierliche Optimierung besprochen siehe Gleichung~\ref{eq:optimazion1}. Wird im Folgenden von kontinuierlichen Problemen gesprochen, wird ein $_k$ an die Bezeichnungen angehängt.
%
\subsubsection{Diskrete Optimierung}
%
Darf ein Variablenvektor nur Werte einer diskreten Wertemenge annehmen, spricht man von einer diskreten Optimierung. Als Spezialfälle lassen sich eine ganzzahlige sowie eine binäre Optimierung ableiten, siehe unten. Die Definition von kontinuierlichen und diskreten Optimierungsproblem sind gleich:
\begin{equation}
	min\{~y(\mathbf{x})~|~\mathbf{x}\in\mathbf{X}_D~\}
\end{equation}
%
Dabei stammt der zulässige Bereich $\mathbf{X_D}$ aus der Menge diskreter Werte:
$$
\mathbf{X}_D=\{~\mathbf{x}\in \mathbb{R}^n~|~x_i\in \mathbf{D}_i,\qquad i=1,2,..,m;\qquad g_j(\mathbf{x})\leq 0, \qquad j=1,2,...,n~\}
$$
Für den diskreten Charakter sorgt die Auswahl von $m$ Wertemengen aus $\mathbb{R}$:
$$
\mathbf{D}_i=\{x_{i1},x_{i2}..x_{id_i}\},\qquad \mathbf{D}_i \subset \mathbb{R},\qquad i=1,2,..,m;
$$
\subsubsection{Spezialfall - Ganzzahlige Optimierung}
%
Der Spezialfall einer diskreten Optimiering ist die ganzzahlige
\begin{equation}
	min\{~y(\mathbf{x})~|~\mathbf{x}\in\mathbf{X}_Z~\}
\end{equation}
%
$$
\mathbf{X}_Z=\{~\mathbf{x}\in \mathbb{Z}^n~|~x_i\in \mathbf{Z}_i,\qquad i=1,2,..,m;\qquad g_j(\mathbf{x})\leq 0, \qquad j=1,2,...,n~\}
$$
Dadurch entstehen die $m$ Wertemengen:
$$
\mathbf{Z}_i=\{-d_1^-,..,-1,0,+1,..d_{i}^+\},\qquad \mathbf{Z}_i \subset \mathbb{Z},\qquad i=1,2,..,m;
$$
Ein weiterer Spezialfall ist die binäre Optimierung. Die Erweiterung ist für diesen Fall trivial und wird hier nicht gezeigt.
%
\subsubsection{Gemischte Optimierung}
%
Die auch als diskret-kontinuierliche Optimierung bekannte Methode formuliert das Modell für einen gemischten Variablenvektor. Mann kann in einem solchen Fall den Variablenvektor aus zwei Teilen zusammensetzen:
$$
\mathbf{x} = [\mathbf{x}_K~\mathbf{x}_D]^T
$$
Die Notation meint mit
\begin{itemize}
\item $\mathbf{x}_K$ den Vektor der $m_K$ kont. Variablen und mit
\item $\mathbf{x}_D$ den Vektor der $m_D$ disk. Varaiblen.
\end{itemize}
%
Analog dazu lässt sich der Suchraum in zwei Teile Zerlegen:
$$
\mathbf{x}_K\in\mathbf{X}_K \qquad und \qquad \mathbf{x}_D\in\mathbf{X}_D
$$
Nun lautet das disk.-kont. Optimierungsproblem wie folgt:
%
\begin{equation}
	min\{~y(\mathbf{x})~|~\mathbf{x}_K\in\mathbf{X}_K,\mathbf{x}_D\in\mathbf{X}_D~\}
\end{equation}
%
Diskret-kontinuierliche Probleme sind nur schlecht lösbar. In dieser Arbeit wird ein rein reellwertiges Problem behandelt.
%
%\section[Konvexität]{Formulierung}
%%
%\section[Konvexität]{Konvexität}
%%
%\section[Objektfunktion]{Konvexität}
%
\section{Evolutionäre Strategien}
\label{sec:es-common}
%- Section 1 ----------------------------------------------------------------
\label{seq:EvolutionaryStrategies}
Folgende Information entstammen im Wesentlichen aus \cite{kost2003optimierung},\cite{bronstejn2012taschenbuch}\ sowie \cite{Hansen:1} und sind auf den folgenden Seiten lediglich zusammengefasst und neu arrangiert um eine Einarbeitung in die Thematik zu ermöglichen.\\
%------------------------------------------------------------------
\subsection{Evolutionsstrategien - Grundlagen }
%
Nach dem Vorbild natürlicher Evolution entworfene stochastische Optimierungsverfahren werden Evolutionsstrategie bezeichnet. Sie verwenden die Prinzipien der Mutation, Rekombination und Selektion analog zu der nat. Evolution. Der Grundlegende Ablauf dieser Strategien zeigt die Abbildung~\ref{es_flowchart}\\
Wie in der Natur auch werden Nachkommen aus der Menge der verfügbaren Eltern gebildet. Dabei bezeichnet im Folgenden:
%
\begin{itemize}
	\item $\mu$ die Anzahl der Eltern (=> Größe der Population)
	\item $\lambda$\footnote{Anmerkung: Die Verwendung des Symbols $\lambda$ ist in diesem Kontext nicht eindeutig. Im Rahmen dieser Arbeit steht dieses Symbol auch für die Wellenlänge. In diesem Abschnitt wird jedoch weiterhin $\lambda$ verwendet um die gleiche Nomenklatur wie bei dieser Thematik üblich zu verwenden.} die Anzahl der Eltern die bei Rekombination neue Kinder erzeugt; Die Anzahl der erzeugten Nachkommen einer neuen Generation
	\item $\mathbf{x}_p$ Elternpunkt (Parent)
	\item $\mathbf{x}_c$ Nachkomme einer Generation (Child)
	\item $X_p^1$ Die Menge aller Eltern der ersten Generation $X_p = \{\mathbf{x}_{p_1}^1,..,\mathbf{x}_{p_\mu}^1\}$
	\item $X_p^k$ Die Menge aller Eltern der k-ten Generation $X_p = \{\mathbf{x}_{p_1}^k,..,\mathbf{x}_{p_\mu}^k\}$
\end{itemize}
%
Wir wollen nun in Abbildung~\ref{fig:es_flowchart} einen Blick auf den prinzipiellen Ablauf dieses Algorithmus werfen und anschließend auf die Details eingehen.
%
%------------------------------------------------------------------------------
%------------------------------------------------------------------------------
%------------------------------------------------------------------------------
\begin{figure}[ht!]
	\begin{center}
		\caption[Ablauf Evolutionsstrategie]{ Der Ablauf des $(\lambda,\mu)$- Evolutionsalgorithmus ist in dieser Abbildung gezeigt. Dies ist eine leicht zu verstehende Variante der Algorithmen. Die wesentlichen Schritte gleichen sich in den Varianten. }
		\label{fig:es_flowchart}
		\vspace{0.5cm}
		\begin{tikzpicture}[auto]
		\scriptsize
			\tikzstyle{decision} = [diamond, draw=black, thick, fill=black!20, text width=5em, text badly centered, inner sep=1pt]
%			
			\tikzstyle{block} = [rectangle, draw=black, thick, fill=gray!20, text width=15em, text centered, rounded corners, minimum height=4em]
%	
			\tikzstyle{line} = [draw, thick, -latex',shorten >=1pt];
			\tikzstyle{commentline} = [draw, dashed, green!50,-latex',shorten >=1pt];
%	
			\tikzstyle{cloud} = [ dotted, draw=green!50, thick, ellipse,,fill=green!20, minimum height=2em, text width= 10em, text badly centered];
%	
			\matrix [column sep=5mm,row sep=7mm]
			{
				% row 1
				& \node [block] (start) { Start }; & \\
				% row 2
				&\node [block] (init) {Erstelle Startpopulation $X_p^1$ bestehend aus $\mu$-Individuen }; & 
				\node [cloud] (comment1) {Initialisierung, mit Zufallswerten}; \\
				% row 4
				& \node [block] (identify) {Erzeuge eine Menge von $\lambda$ Nachkommen $X_c^k$ aus der aktuellen Elterngeneration $X_p^k$ durch Rekombination \&\&, || Mutation}; & \\
				% row 5
				\node [block] (update) {Nächste Stufe der Evolution; k++}; &
				\node [block] (evaluate) {Durch Selektion die besten $\mu$ Nachkommen für die Generation $X_p^{k+1}$ auswählen}; & \\
				% row 6
				& \node [decision] (decide) {$\Delta \geq \Delta_{min}$}; & 
				\node [cloud] (criteria) {Abbruchkriterium; Muss geeignet gewählt werden, bspw. max. Anzahl der Generationen oder Erreichen des Optimums};\\
				% row 7
				& \node [block] (stop) {Ende}; & \\
			};
% Arrows
			\tikzstyle{every path}=[line]
			\path (init) -- (identify);
			\path (identify) -- (evaluate);
			\path (evaluate) -- (decide);
			\path (update) |- (identify);
			\path (decide) -| node [near start] {Ja} (update);
			\path (decide) -- node [midway] {Nein} (stop);
			\path (start) -- (init);
			
			\tikzstyle{every path}=[commentline]
			\path (criteria) -- (decide);
			\path (comment1) -- (init);
			
		\end{tikzpicture}
	\end{center}
\end{figure}
%------------------------------------------------------------------
\subsubsection[Mutation]{Mutation}
Ein Nachkomme $\mathbf{x}_C$ wird aus seinem Elternteil $\mathbf{x}_P$ und einer zufälligen Variation $\mathbf{d}$ gebildet.
\begin{equation} \label{eq:Mutation_Child}
	\mathbf{x}_c = \mathbf{x}_P + \mathbf{d}
\end{equation}
Dabei ist $\mathbf{d}$ ein bei jeder Mutation neu zu bestimmender $(0,\sigma^2)-normalverteilte$ Zufallszahl $Z(0,\sigma^2)$:
\begin{equation}\label{eq:wavenumber_trilateration_model2}
\mathbf{d}=
\left(
	\begin{array}{c}
		d_1 \\
		\vdots\\
		d_n 
	\end{array}
\right)
=
\left(
	\begin{array}{c}
		Z(0,\sigma_1^2) \\
		\vdots\\
		Z(0,\sigma_n^2) 
	\end{array}
\right)
=
\left(
	\begin{array}{c}
		Z(0,1) \sigma_1 \\
		\vdots\\
		Z(0,1) \sigma_n 
	\end{array}
\right)
\end{equation}
%
Die Normalverteilung der Variation ist nützlich, da kleine Änderungen wahrscheinlicher sind als große. Die maximale Größe der Variation wird durch die Standardabweichung $\sigma_i$ bestimmt. Sie steuert somit die Schrittweite von Generation zu Generation.
%
%------------------------------------------------------------------
\subsubsection[Rekombination]{Rekombination}
Durch Rekombination zweier oder mehr Eltern aus der Menge aller $\mu$-Eltern $X_{\varrho} \subset X_E$. Die Wahl der Eltern sollte zufällig erfolgen um Inzuchtprobleme zu verhindern.\\
Zwei Arten der Rekombination sind denkbar:\\

Die \textit{intermediär Rekombination} erstellt einen Nachkommen durch das gewichtete Mittel von $\varrho$ Eltern.
%
\begin{equation}
\mathbf{x}_c = \Sigma^\varrho_{i=1} \alpha_i\mathbf{x}_{p_i},\\ \Sigma^\varrho_{i=1} \alpha_i = 1,\\ 2\leq\varrho\leq\mu
\end{equation} 
%
Bei der \textit{diskreten Rekombination} vom $\varrho$-Eltern wird die \textit{i}-te Komponente $x_{ic}$ eines Nachkommen $\mathbf{x}_c$ mit der \textit{i}-te Komponente eines zufällig gewählten Elternpunktes gleichgesetzt.
%
\begin{equation}
\mathbf{x}_{ic} = \mathbf{x}_{ip_j},\\ j\in\{1,...,\varrho\},\\i=1,...,n
\end{equation} 
%
%- Section .4 -----------------------------------------------------------------
\subsubsection[Selektion]{Selektion}
Die durch Rekombination und/oder Mutation erzeugten Nachkommen werden in dem Schritt Ausgewählt um einen Evolutionsfortschritt zu erreichen. Dies erfolgt anhand des Vergleichs mit dem Zielfunktionswert $f(\mathbf{x})$. Das beste Individuum oder die besten werden für die nachfolgende Generation ausgewählt. Dabei gibt es Strategien bei denen nur die Nachkommen an der Auswahl beteiligt sind und welche bei denen Eltern und Kinder teilnehmen.

%- Section .5 -----------------------------------------------------------------
\subsubsection{Evolutionsalgorithmus}
%
Der eigentliche Evolutionsalgorithmus ist in Abbildung~\ref{fig:es_flowchart} dargestellt. Er enthält im wesentlichen die in den vorherigen Abschnitten beschriebenen Schritte. Der prinzipielle Ablauf ist für alle Evolutionsalgorithmen gleich. Eine Unterscheidung der Verfahren kann durch verschiedene Parameter beschrieben werden. Wesentlich dabei sind die Populationsgröße $\mu$, die Anzahl an der Rekombination beteiligten Eltern $\varrho$, die gewählte Selektionsstrategie sowie die Anzahl der Nachkommen $\lambda$. Im Folgenden sind zuerst einige Beispiele für die Nomenklatur der Selektionsstrategie aufgeführt, die im Anschluss genauer beschrieben werden.\\
Für Strategien die nur auf Mutation für die Erzeugung von Nachkommen setzten sind folgende Nomenklaturen gebräuchlich:
\begin{itemize}
\item $(\mu+\lambda)$ Elternelemente werden in der Selektion berücksichtigt
\item $(\mu,\lambda)$ Ausschließlich Nachkommen nehmen an der Selektion teil
\end{itemize}
%
Die Strategien werden Plus- bzw. Komma-Strategie genannt. bei der Plus-Strategie wird zusätzlich noch ein gewichtungsfaktor eingeführt, der das "altern" der Elterngeneration darstellt. Dieser Mechanismus soll verhindern, dass die Eltern, nach einer gewissen Anzahl an Generationen, nicht mehr berücksichtigt werden.\\
Wird die Rekombination eingesetzt kann auch die Anzahl der beteiligten Elternelemente angegeben werden:
\begin{itemize}
\item $({\mu}/{\varrho}+\lambda)$ \& $({\mu}/{\varrho},\lambda)$ Angabe der Anzahl beteiligter Eltern bei der Rekombination.
\end{itemize}
%
Mithilfe der hier beschrieben Klassifikationen werden die Algorithmen im Folgenden stets angegeben.\\

In Abbildung~\ref{fig:es_flowchart} wird der Ablauf einer Optimierung mit evolutionären Verfahren dargestellt. Es wird die Komma-Strategie gezeigt, ein Struktogramm der Plus-, oder anderer Strategien ist nicht gezeigt. Die Unterschiede würden sich in dem Punkt Rekombination zeigen.
%
%------------------------------------------------------------------------------
%- Section .6 -----------------------------------------------------------------
\subsection{Strategien mit mehreren Populationen}
Es ist möglich die Strategien auf die Ebene von Populationen zu erweitern. Das bedeutet, man lässt ganze Populationen miteinander in Wettstreit treten und nur diejenige überleben, die die besten Ergebnisse liefern. Das mündet in einem zweistufigen Evolutionsprozess. Man kann die Notation um diesen Umstand erweitern und erhält so:
$$
[\mu_2/\varrho_2,^{+}\lambda_2(\mu_1/\varrho_1,^{+}\lambda_1)]
$$
Sprich aus $\mu_2$-Elternpopulationen werden durch Rekombination mit jeweils $\varrho_2$ Populationen, $\lambda_2$ Nachkommenpopulationen generiert. Innerhalb der Populationen erfolgt die Optimierung anhand einer $({\mu_1}/{\varrho_1}+\lambda_1)$ oder $({\mu_1}/{\varrho_1},\lambda_1)$-Strategie. Nun kann nach einer bestimmten Zahl von Generationen die besten Populationen für die nächste Generation ausgewählt werden. Auch hier stehen verschiedene Auswahlkriterien zur Verfügung. Man kann z.B. die Population anhand des Zielfunktionswert des besten Individuums wählen oder den Mittelwert über alle Individuen wählen.
%
%- Section .7 -----------------------------------------------------------------
\subsection{Optimierungsräume}
\lipsum[1]
%
\subsubsection{Kontinuierliche Optimierung}
%
\lipsum[1]
%
\subsubsection{Diskrete Optimierung}
%
\lipsum[1]
%
\subsubsection{Gemischte Optimierung}
%
\lipsum[1]
%
%- Section .8 -----------------------------------------------------------------


%
\section{Covariance Matrix Adaption - Anpassung der Kovarianzmatrix}
\label{sec:cma-es}
Der Folgende Abschnitt behandelt ein spezielles Verfahren der Evolutionären Optimierung. Das CMA-ES-Verfahren. Es stellt den "State of the Art"- der Evolutionären Berechnungsverfahren dar. Für spezialisierte Probleme gibt es bessere Lösungen, als allgemeiner Solver ist dieses Verfahren mehr als tauglich. Es wurde um die Jahrtausendwende entwickelt und veröffentlicht. Es gibt verschiedene Abwandlungen des Algorithmus und sogar eine Lösung zur Multiobjekt-Optimierung (MO-CMA-ES) existiert.~\cite{HansenMOO:1} Das Verfahren wird aktuell stets weiterentwickelt. [REFERENZEN]\\
%
%------------------------------------------------------------------
\begin{figure} [ht!]
\centering
         \caption[Konzept direkter Optimierung mittels CMA-ES]{Die Abbildung zeigt sechs Lösungsschritte des CMA-ES-Verfahrens. Eingangs wird eine Population mit einer Gauß-Verteilung generiert. Diese bewegt sich von Generation näher an das globale Optimum. Die gestrichelte Linie zeigt dabei eine Isolinie der Wahrscheinlichkeitsdichte. Wir befinden uns auf einem 2D-Problem, somit wird das vorankommen der Population über zwei $\sigma$'s gesteuert. Dadurch kommt die Ellipse zu Stande. Die Linie bedeutet nicht, dass nur in diesem Bereich Nachkommen erzeugt werden, siehe z.B. Generation 4. Je nach Beschaffenheit des Problems und nach Nähe zum Optimum werden die steuernden $\sigma$'s kleiner. }
         \label{fig:conecpt_cma.es}
         \vspace{0.5cm}
         \includegraphics[width=\textwidth]{img/CMA-ES_algorithm.png}
%      
\end{figure}

%
Durch die Anpassung der Kovarianz-Matrix wird eine enge Abschätzung der Konturlinien der Objektfunktion $f$.\\

%
Der Algorithmus steht in verschiedene Implementationen, Programmiersprachen und Umgebungen zur Verfügung. Teilweise sind die Implementationen Proprietär (z.B. Matlab), teilweise quelloffen. Die in dieser Arbeit Zur Anwendung kommende Variante ist die Shark-Library. Diese Bibiliothek ist eine in \cpp geschriebene, quelloffene Software, die am Institut für Neuroinformatik der Ruhr Universität Bochum entwickelt wird. Detailliert wird Shark im Rahmen des Hauptteils in Abschnitt~\ref{sec:Shark} vorgestellt.
%

%
%-----------------------------------------------------------------------
%
\section[Technische Voraussetzungen]{Technische Voraussetzungen}
%-----------------------------------------------------
In diesem Abschnitt werden die technischen Grundlagen für diese Arbeit vorgestellt und das Wichtigste erörtert. Es kann nicht im vollem Umfang auf die Details der Technik eingegangen werden ohne den Rahmen dieser Arbeit zu sprengen. Interessierte sei die referenzierte Literatur für eine weite Lektüre empfohlen.
%
\subsection{RFID}
%
Bei \textit{Radio-Frequency Identification} (RFID) handelt es sich um einen Funkstandard der die kontaktlose Identifikation bei gleichzeitiger Erfassung zusätzlicher Informationen ermöglicht. Zur Technik gehört ein Auslesegerät (Reader) und ein oder mehrere Transponder (Tags). Eine sehr grobe Übersicht über typische Bauformen von Tags und Reader ist in \ref{fig:RFID_TAGS_AND_READER} zu finden.  Heute verfügbare Transponder lasen sich auf nahezu jeder beliebigen Oberfläche anbringen lassen. Das ermöglicht ein großes Anwendungsspekrum, praktisch wird die Technik in jeder Umgebung eingesetzt in der es erforderlich oder nützlich ist, Dinge kontaktlos zu identifizieren. Eine gute Übersicht über Branchen und Anwendungsgebiete für RFID ist in \cite{RFIDJournal} zu finden. Im Rahmen dieser Arbeit wird kein umfassender Überblick über die Technik geboten, da die Bauformen und Spezifikationen sehr stark variieren. Eine gute Einführung und Übersicht zur Technik ist in \cite{finkenzeller2008rfid} zu finden. Dort werden auch detailliert die physikalischen Grundlagen von erläutert. Aufgrund des großen Anwendungsspektrums und der weiten Verbreitung ist die Technik in die Kritik geraten. Unter dem Dach des Vereins digitalcourage e.V. exisitiert die Kampange \textit{StopRFID}. Die Kampagne hat sich zum Thema gemacht über die Anwendungsmöglichkeiten und Gefahren von RFID aufzuklären \cite{stoprfid2013}. Die Seiten der Kampagne bieten eine sehr weitgehende Auflistung der Anwendungen für RFID. Ziel der Kampagne ist es die Gefahren in den gesellschaftlichen Fokus zu rücken und für den Umgang mit der allgegenwärtigen Technik zu sensibilisieren. Die Kampagne über sich selbst:
\begin{quote}
"Wir wollen RFID nicht komplett verhindern. Es geht uns nicht darum, die RFID-Entwicklung zum Erliegen zu bringen ... Im Gegenteil." \footnote{\url{http://www.foebud.org/rfid/was-kann-ich-tun/}}
\end{quote}
%
\begin{figure} [h]
\centering
         \caption[Beispiele für Transponder und Lesegeräte]{ Hier gezeigt sind Beispiele für Transponder und Lesegeräte. Das linke Bild zeigt drei typische Tags, nahezu jede Gestalt ist mittlerweile erhältlich. Die hier gezeigten Tags eignen sich für eine Anbringung an glatten Oberflächen. Es gibt zig weitere Bauformen, die unterschiedlichste Anwendungsspektren bedienen und sogar eine Implantation ermöglichen (nicht gezeigt). Im rechten Bild ist ein Handlesegerät gezeigt. Zum Mobilen Auslesen über mittlere bis kurze Distanzen. Auch bei den Readern gibt es unterschiedlichste Bauformen, die je nach Anwendungsfall ausgewählt werden. }
         \label{fig:RFID_TAGS_AND_READER}
         \vspace{0.5cm}%         
         \begin{subfigure}[h]{0.4\textwidth}
                 \centering
                 \includegraphics[width=\textwidth]{img/667px-RFID_Tags_gs.png}
                 \vspace{.1cm}
                 \caption{ RFID- Transponder }
                 \label{fig:TAGS}\textit{}
         \end{subfigure}
%         
\qquad
%
         \begin{subfigure}[h]{0.4\textwidth}
                 \centering
                 \includegraphics[width=\textwidth]{img/RFID-Reader_gs.png}
                 \vspace{.1cm}
                 \caption{ RFID- Handlesegerät }
                 \label{fig:READER}
         \end{subfigure}
\end{figure}
%
\label{sec:Measurement1}
%
\begin{enumerate}
	\item Die Messung der Position erfolgt über die Auswertung der Phasenlage des empfangenen Signals in Bezug auf ein Referenzsignal. In der EU gibt es verschiedene, zulässige RFID-Frequenzen\footnote{insert reference here} (865,5?867,5 MHz) kann man die Wellenlänge mit: $ \lambda\simeq0,35 m $ angeben. Daraus folgt, dass alle 35 cm die gleiche Konfiguration der Phase vorliegt. Im Rahmen dieser Arbeit wird dabei von \textit{Isophasen} gesprochen. Die gewonnene Information aus der Phase ist nicht eindeutig, d.h. es lässt sich durch die Kenntnis der Phase nicht unmittelbar auf die korrekte Postion schließen. Man kann das Problem umgehen in dem man auf die errechnete Position ein ganzzahliges Vielfaches der Wellenlänge addiert. Die sog. Wellenzahl (siehe~\eqref{eq:Wavenumbers}).
	\item Das System der Amedo STS verwendet eine spezielle Antennenanordnung um die Position zu ermitteln. Dabei wird eine Antennenanzahl >4 eingesetzt. Für jede dieser Antennen muss eine eigene Wellenzahl bestimmt werden. Durch Auslöschung des Signals, Absorption etc. kann es dazu kommen, dass eine Antenne eine unbestimmte Zeit lang kein Signal vom Tag empfängt. Wenn die Antenne nach dieser Zeit erneut ein Signal empfängt ist die ihr zugehörige Wellenzahl unbekannt und muss neu bestimmt werden. 
	\item In realen Umgebungen treten zusätzlich noch Ruflektionen und ein sog. Multipath-Effekt auf. Dabei wird das Signal nicht auf dem Direkten Weg Antenne-Tag-Antenne empfangen sondern über einen unbekannten, längeren Weg. Dadurch kommt es zu einem Fehler in der Phase. Zusätzlich ist dieser Effekt individuell für jede Antenne.
\end{enumerate}
%
\begin{figure}[h]
         \centering
         \includegraphics[width=0.5\textwidth]{img/00_placeholder.png}
         \caption[Messystem der Amedo GmbH]{Das Bild zeigt das PRPS-Messystem zu erkennen sind die wesentlichen elektronischen Komponenten, sowie weitere periphere Hardware }
         \label{fig:System}
\end{figure}
%
\begin{figure}[h]
         \centering
         \includegraphics[width=0.5\textwidth]{img/00_placeholder.png}
         \caption[Messaufbau der Amedo GmbH]{Abgebildet ist der Messaufbau mit unterschiedlichen Antennen}
         \label{fig:Setup}
\end{figure}
\subsection{Messystem der Amedo GmbH}
%
\lipsum[1-2]
%
%
%-----------------------------------------------------------------------
\section[Anforderungen]{Anforderungen an die Lösung}
%
Aus den bisher vorgestellten Überlegungen können nun folgende Anforderungen abgeleitet werden:
%
\begin{enumerate}
	\item Lösung muss schnell (ideal < 1 Sekunde) gefunden werden
	\item Unabhängigkeit von Stütz- und Kalibrierpunkten
	\item Eindeutigkeit der Lösung
	\item Eignung für ein großes Messvolumen
	\item Nahtlose Integration in das bestehende Software Ökosystem
	\item Stand der Softwaretechnik entsprechend
%
\end{enumerate}
%
\section{Ziel und Herangehensweise}
%
Das Ziel der Arbeit ist die Entwicklung eines Systems zur Ermittelung der Wellenzahl. Gleichzeitig sollen die oben abgeleiteten Anforderungen erfüllt werden. Das System wird im Kern die Lösung über numerische Verfahren finden, im speziellen kommt das sog CMA-ES zum Einsatz. Dazu wird zuerst ein Modell entworfen werden, dass sich für dieses Verfahren eignet. Das Modell soll mit möglichst wenig Annahmen/ Einschränkungen auskommen und dennoch ein relativ sicheres, reproduzierbares Ergebnis liefern. Das System soll unmittelbar in den Produkten der \amedogmbh zum Einsatz kommen können. Darüber hinaus soll im Rahmen dieser Arbeit eine Methode entwickelt werden, um die Position von frei im Raum angeordnete Antennen zu ermitteln.
%
%-----------------------------------------------------------------------
