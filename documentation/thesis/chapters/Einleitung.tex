\section[Motivation]{Motivation}

Die Positionsbestimmung mittels RFID ist eine vielversprechende Technik. Die Bestimmung der Position (im Folgenden "Tracking" genannt) mittels RFID  bietet gegenüber vergleichbaren Methoden (z.B. Ultraschall, Optisch) verschiedene Vorteile. Das wesentlichste Unterscheidungsmerkmal ist, dass keine direkte Sichtlinie sog. LOS notwendig ist um ein Objekt zu lokalisieren. Der Grund dafür ist das zugrunde liegende Messprinzip. Es werden elektromagnetische Signale ausgewertet, die anderen Wechselwirkungen unterliegen und somit Materie durchdringen. Insbesondere im Vergleich mit optischen Verfahren ist RFID damit überlegen. Die Eigenschaft Materie zu durchdringen erlaubt es Tags im Patienten zu lokalisieren, entsprechende Untersuchungen über die Positionsgenauigkeit im Körper sind vielversprechend.[REFERENZEN]\\
Auf den Tags können zusätzliche Informationen hinterlegt werden, beispielsweise eine Identifikationsnummer oder Ähnliches. Dadurch wächst das Anwendungsspektrum weiter[REFERENZEN]. Das Auslesen von zusätzlichen Informationen ist mit keiner der anderen Technologien möglich.\\
Das von dem Messsystem der {Amedo GmbH} verwendete Verfahren basiert auf der Messung der Phasenlage der Antwort eines Tags. Die Phasenlage ist direkt proportional zu einer Entfernung, sie ist jedoch nicht Eindeutig (siehe \ref{sec:Measurement1})

Eine analytische Lösung des Problems ist schwierig und bisher nicht gelungen. Diese Ansätze scheiterten an der Komplexität des Problems\footnote{siehe \ref{sec:Komplexity1} und \ref{sec:Komplexity2}} oder benötigen sehr aufwändige Messreihen mit großer Anzahl an Messpunkten \cite{amedo1}. Das limitiert die Praxistauglichkeit der Verfahren.\\
In dieser Arbeit soll mittels numerischer Methoden und Modellen die beschriebenen Probleme zu gelöst werden.

\section{Vorraussetzungen}
\lipsum[1-2]

\subsection{RFID}
\label{sec:Measurement1}
\begin{enumerate}
	\item Die Messung der Position erfolgt über die Auswertung der Phasenlage des empfangenen Signals in Bezug auf ein Referenzsignal. Da in der EU sind nur bestimmte Frequenzen für die Verwendung für RFID erlaubt (865,5–867,5 MHz) kann man die Wellenlänge mit: $ \lambda\simeq0,35 m $ angeben. Daraus folgt, dass alle 35 cm die gleiche Konfiguration der Phase vorliegt. In dieser Arbeit wird dieser Umstand Isophasen genannt. Die gewonnene Information aus der Phase ist somit redundant, d.h. es lässt sich durch die Kenntnis der Phase nicht unmittelbar auf die korrekte Postion schließen. Man kann das Problem umgehen in dem man auf die errechnete Position ein ganzzahliges Vielfaches der Wellenlänge addiert. Die sog. Wellenzahl (vgl.~\eqref{eq:Wavenumbers}).
	\item Das System der Amedo STS verwendet eine spezielle Antennenanordnung um die Position zu ermitteln. Dabei wird eine Antennenanzahl >4 eingesetzt. Für jede dieser Antennen muss eine eigene Wellenzahl bestimmt werden. Durch Auslöschung des Signals, Absorption etc. kann es dazu kommen, dass eine Antenne eine unbestimmte Zeit lang kein Signal vom Tag empfängt. Wenn die Antenne nach dieser Zeit erneut ein Signal empfängt ist die ihr zugehörige Wellenzahl unbekannt und muss neu bestimmt werden. 
	\item In realen Umgebungen treten zusätzlich noch Ruflektionen und ein sog. Multipath-Effekt auf. Dabei wird das Signal nicht auf dem Direkten Weg Antenne-Tag-Antenne empfangen sondern über einen unbekannten, längeren Weg. Dadurch kommt es zu einem Fehler in der Phase. Zusätzlich ist dieser Effekt individuell für jede Antenne.
\end{enumerate}
\lipsum[1-2]

\subsection{Anforderungen an das Verfahren}
Aus den vorherigen Überlegungen leiten sich folgende Anforderungen ab
\begin{enumerate}
\item Lösung muss schnell (ideal < 1 Sekunde) gefunden werden
\item Unabhängigkeit von Stütz- Kalibrierpunkten
\item Eindeutigkeit der Lösung
\item Eignung für ein großes Messvolumen (mehrere Kubikmeter)

\end{enumerate}

\subsection{Evolutionäre Verfahren}
\lipsum[1-2]

\subsubsection{Evolutionäre Verfahren, warum?}
\lipsum[1]

\subsubsection{CMA-ES}
\lipsum[1]