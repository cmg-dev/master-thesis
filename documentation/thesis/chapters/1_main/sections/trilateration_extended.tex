{
\subsubsection{Zusammenhang mit der Wellenzahl}
Wie gezeigt wurde ergibt sich für den Fall der Trilateration und der Annahme, dass vier Antennen Messwerte liefern, die Gleichung:
\begin{equation}\label{eq:final_trilateration_model}
\mathbf{0}=
\left(
	\begin{array}{ccc}
		x_k-x_0 & y_k-y_0 & z_k-z_0 
	\end{array}
\right)
\left(
   \begin{array}{c}
	   x-x_0\\
	   y-y_0\\
	   z-z_0
   \end{array}
\right)
-
\left(
	\begin{array}{c}
		c_{kj}
	\end{array}
\right) 
\end{equation}
%
Wir stellen fest, dass dieses Modell rein geometrisch ist. Es erlaubt bereits einen Einsatz im Rahmen der Kalibrierung (siehe~\ref{sec:calibration}). Es wird im Folgenden eine Erweiterung dieses Modells gezeigt. Ziel ist es, einen Zusammenhang zwischen diesem Modell, der gemessenen Phase und der Wellenzahl zu erzeugen. Folgender Ansatz wird gewählt:
%
\begin{equation}
	\label{eq:r_0_varrho} r(\varrho,n)=\frac{\lambda}{2}\left(\frac{\varrho}{2\pi}+n\right),\\\lambda=\frac{c}{f}, n:= \text{Wellenzahl}
\end{equation}
%
In dem Modell steht $\varrho_k$ für die gemessene Phase vom Messsystem und $n_k$ ist die gesuchte Wellenzahl. Der Index $k$ deutet eine Existenz der beiden Parameter für jede Antenne an. Durch einsetzen von \eqref{eq:r_0_varrho} in \eqref{eq:c_kj}, erhalten wir:
%
\begin{equation}\label{eq:c_k0_extended}
	c_{kj}(\varrho_0, \varrho_k, n_0, n_k) =\frac{1}{2}\left[d_{kj}^2+\frac{\lambda^2}{4}\left(\frac{\varrho_j}{2\pi}+n_0\right)^2-\frac{\lambda^2}{4}\left(\frac{\varrho_k}{2\pi}+n_k\right)^2\right]
\end{equation}
%
Wir stellen Gleichung~\eqref{eq:c_k0_extended} um:
\begin{align}
%	
	c_{kj}(\varrho_0, \varrho_k, n_0, n_k) &= \frac{1}{2}\left\{d_{kj}^2+\frac{\lambda^2}{4}\left[\left(\frac{\varrho_j}{2\pi}\right)^2+2\frac{\varrho_j}{2\pi}n_0+n_0^2 \right.\right.\nonumber\\
	&\phantom{=}\; 
	\left.\left.-\left(\frac{\varrho_k}{2\pi}\right)^2-2\frac{\varrho_k}{2\pi}n_k-n_k^2\right]\right\}\\
%    
    &=\frac{1}{2}\left\{d_{kj}^2+\frac{\lambda^2}{4}\left[\left(\frac{\varrho_j}{2\pi}\right)^2-\left(\frac{\varrho_k}{2\pi}\right)^2 \right.\right.\nonumber\\
    &\phantom{=}\;
   	\left.\left.+2\frac{\varrho_j}{2\pi}n_0-2\frac{\varrho_k}{2\pi}n_k+n_0^2-n_k^2\right]\right\}\\
%	
	&=\frac{1}{2}d_{kj}^2+\frac{\lambda^2}{8}\left[\frac{1}{(2\pi)^2}\left(\varrho_0^2-\varrho_k^2\right) \right.\nonumber\\
	&\phantom{=}\;
	\left. +\frac{1}{\pi}\left(\varrho_0n_0-\varrho_kn_k\right)+\left(n_0^2-n_k^2\right)\right]\label{c_k0_rearragend}
\end{align}
%
Führen wir nun:
\phantomeq{c_{kj}(\varrho_0, \varrho_k, n_0, n_k)}{a_{0k} := \frac{1}{2}d_{kj}^2\nonumber}
\phantomeq{c_{kj}(\varrho_0, \varrho_k, n_0, n_k)}{a_1 := \frac{\lambda^2}{8}\nonumber}
\phantomeq{c_{kj}(\varrho_0, \varrho_k, n_0, n_k)}{a_2 := a_1\frac{1}{\pi}\nonumber}
\phantomeq{c_{kj}(\varrho_0, \varrho_k, n_0, n_k)}{a_{3kj} := a_1\frac{1}{(2\pi)^2}(\varrho_j^2-\varrho_k^2)\nonumber}
%
in Gleichung~\eqref{c_k0_rearragend} ein, erhalten die finale Form der Gleichung:
\begin{equation}
c_{kj}(\varrho_0, \varrho_k, n_0, n_k) = a_{0k}+a_1(n_0^2-n_k^2)+a_2(\varrho_0n_0-\varrho_kn_k)-a_{3kj}\label{c_k0_final_form}   
\end{equation}
%
Die Einführung der Konstanten macht zum Einen die Gleichung übersichtlicher. Zum Anderen können so in der spätere Softwareimplementation, Rechenschritte gespart werden. Was sich günstig auf den Rechenaufwand auswirkt. Im Weiteren erkennt man, dass in Gleichung~\eqref{c_k0_final_form}, für $\varrho_k=\text{const.}$ \& $\varrho_0=\text{const.}$ gilt. Der Grund dafür liegt darin, dass $\varrho$ zwar die Messwerte beschreibt, diese jedoch nur in dem Modell eingeführt werden. Im Sinne der später durchgeführten Optimierung sind diese Parameter keine Variablen. Es ermöglicht uns zu schreiben:
\begin{equation}
c_{kj}(\varrho_0, \varrho_k, n_0, n_k) = c_{kj}(n_0, n_k)
\end{equation}
%
Im engeren Sinne einer mathematischen Funktion sollten wir die Parameter alle als Argument aufnehmen. Diese Form soll darstellen, welche Größen von Interesse sind. Im späteren Gebrauch wird diese Gleichung in der Optimierung eingesetzt werden.\\
%
Für unser Gleichungssystem aus\eqref{eq:final_trilateration_model} ergibt sich:
%
\begin{equation}\label{eq:wavenumber_trilateration_model}
\mathbf{0}=
\left(
	\begin{array}{ccc}
		x_k-x_0 & y_k-y_0 & z_k-z_0 
	\end{array}
\right)
\left(
   \begin{array}{c}
	   x-x_0\\
	   y-y_0\\
	   z-z_0
   \end{array}
\right)
-
\left(
	\begin{array}{c}
		c_{kj}(n_0, n_k)
	\end{array}
\right)
\end{equation}
%
\subsubsection{Konkretes Beispiel}
%
Für ein konkretes Beispiel Betrachten wir nun \eqref{eq:wavenumber_trilateration_model}. Dabei wählen wir $|N'|=4$ (d.h. wir verwenden 4 Antennen) und setzen $j=0$. Diese exemplarische Konfiguration kann wie folgt beschrieben werden: Antenne 0 ist die Referenz-Antenne und Antennen 1, 2 und 3 sind Messwertgeber für die Phaseninformation. Im praktischen Gebrauch werden die Konfigurationen anders zusammengestellt. Strategien für die Zusammenstellung werden später beschrieben.\\
Für die gewählte Konfiguration ergibt sich explizit:
%
\begin{equation}\label{eq:wavenumber_trilateration_model_explicit}
\mathbf{0}=
\underbrace{\left(
	\begin{array}{ccc}
		x_1-x_0 & y_1-y_0 & z_1-z_0 \\
		x_2-x_0 & y_2-y_0 & z_2-z_0 \\
		x_3-x_0 & y_3-y_0 & z_3-z_0 
	\end{array}
\right)}_{\textbf{A}}
\underbrace{\left(
   \begin{array}{c}
	   x-x_0\\
	   y-y_0\\
	   z-z_0
   \end{array}
\right)}_{\textbf{x}}
-
\underbrace{\left(
	\begin{array}{c}
		c_{10}(n_0, n_1) \\
		c_{20}(n_0, n_2) \\
		c_{30}(n_0, n_3)
	\end{array}
\right)}_{\textbf{b}}
\end{equation}
%
Wir wollen den Vektor $\mathbf{b}$ nun explizit betrachten:
%
\begin{equation}
\mathbf{b}=
\left(
	\begin{array}{c}
		a_{01}+a_1( n_0^2-n_1^2)+a_2(\varrho_0n_0-\varrho_1n_1)-a_{310} \\
		a_{02}+a_1(n_0^2-n_2^2)+a_2(\varrho_0n_0-\varrho_2n_2)-a_{320} \\
		a_{03}+a_1(n_0^2-n_3^2)+a_2(\varrho_0n_0-\varrho_3n_3)-a_{330}
	\end{array}
\right)
\end{equation}
%
Das Ergebnis ist ein um $\varrho$ und $n$ erweitertes Gleichungssystem. Zusätzlich enthält  es mehrere geometrische Konstanten ($a_{0k}$), mehrere Phasen-Konstanten ($a_{3k0}$), sowie zwei Systemparameter abhängige Konstanten ($a_1$ und $a_2$). Allgemeiner formuliert ergibt sich:
%
\begin{multline}\label{eq:final_equation}
0=
\left(
	\begin{array}{ccc}
		x_k-x_0 & y_k-y_0 & z_k-z_0 
	\end{array}
\right)
\left(
   \begin{array}{c}
	   x-x_0\\
	   y-y_0\\
	   z-z_0
   \end{array}
\right) \\
-
\left(
	\begin{array}{c}
		a_{0k}+a_1(n_0^2-n_k^2)+a_2(\varrho_0k_0-\varrho_kn_k)-a_{3kj}
	\end{array}
	\right)
\end{multline}
%
\subsubsection{Hinzufügen von Antennen - Der allgemeine Fall}
Aus den oben beschriebenen Beispiel, Gleichung~\eqref{eq:final_equation}, und die dort getroffene Wahl von $|N'|=4$ ergibt sich wie viele Veränderliche sich für eine gewählte Konstellation an Antennen ergeben. Leiten wir daraus nun einen allgemeinen Fall ab. Für $k$ gilt in diesem Fall $k=\{1,..,N'-1\}$, wir wählen die Referenzantenne $j=0$ und die Menge an Verwendeten Antennen gleich der Anzahl der Verfügbaren $N'=N$. Es ist leicht ersichtlich, dass sich die Anzahl der verwendeten Antenne unmittelbar auf die Zahl der Variablen auswirkt. Es ergibt sich für das Modell mit vier Antennen insgesamt 7 Variablen ($\mathbf{x},n_0,n_1,n_2,n_3$), wobei sich für ein Modell mit allen 8 Antennen, 11 Variablen ($\mathbf{x},n_0,..,n_7$) ergeben. Andere Konfigurationen verhalten sich analog dazu.
%
\subsubsection{Relevanz dieses Modells}
Dieses Modell hat unmittelbare Relevanz für die Praxis. Es trägt dem Umstand Rechnung, dass zu einem Messzeitpunkt ein Teil der Antennen keine Messwerte liefern könnte. Das Modell erlaubt daher, dass die Anzahl und die Auswahl der Antennen variieren kann. Damit ist das Modell uneingeschränkt tauglich für den Einsatz in dem PRPS-Messsystem.\\
}