\begin{figure}[h]
	\begin{center}
		\caption[Modelllösung bestimmen]{Die Grafik zeigt den Ablauf im Modell um eine Lösung zu finden. Es werden so lange Evolutionsschritte durchgeführt bis das Konvergenzkriterium erreicht wird.}
		\label{fig:find_model_solution}
		\vspace{0.5cm}
		\begin{tikzpicture}[auto]
		\scriptsize
			\tikzstyle{decision} = [diamond, draw=black, thick, fill=black!20, text width=5em, text badly centered, inner sep=1pt]
%			
			\tikzstyle{block} = [rectangle, draw=black, thick, fill=gray!20, text width=15em, text centered, rounded corners, minimum height=4em]
%	
			\tikzstyle{line} = [draw, thick, -latex',shorten >=1pt];
			\tikzstyle{commentline} = [draw, dashed, green!50,-latex',shorten >=1pt];
%	
			\tikzstyle{cloud} = [ dotted, draw=green!40, thick, ellipse,,fill=green!10, minimum height=1em, text width= 15em, text badly centered];
%	
			\matrix [column sep=5mm,row sep=7mm]
			{
%			 row 1
				& \node [block] (start) { Start }; & \\
%	 row 2
				&\node [block] (create1) {Erstelle '\textit{model}'-Instanz}; & \\
%	 row 3
				&\node [block] (set1) {Setze Anzahl der Objektvariablen}; & 
					\node [cloud] (comment1) { Teile dem Modell die Anzahl der Veränderlichen mit, der Solver verwendet später diese Initialisierung }; \\
				
				&\node [block] (set2) {Set Modell-Parameter}; & 
					\node [cloud] (comment2) { Die Modelle erhalten unterschiedliche Parameter, z.B. die ausgewählten Matrizen }; \\
%	 row 4
				& \node [block] (create2) {Erstelle Instanz von \textit{CMA-ES-Solver}}; & 
					\node [cloud] (comment3) {Hier wird das Modell aufgerufen }; \\
				& \node [block] (init) {Parametrisiere den Solver}; & 
					\node [cloud] (comment4) { Einstellen von $\lambda$ und $\mu$ sowie Lösungsstrategie}; \\
%	 row 5
				& \node [block] (step) {cma.\textit{step}()}; & \\
				
				& \node [decision] (decide) {Optimum reached?}; & 
					\node [cloud] (comment5) {Konvergenzkriterium erreicht? (z.B. Ziel-Fitness, max. Evaluationen erreicht etc.)}; \\
%	 row 6
				& \node [block] (done) {done\\ \textbf{return final-solution}}; & \\
			};
% Arrows
			\tikzstyle{every path}=[line]
			\path (start) -- (create1);
			\path (create1) -- (set1);
			\path (set1) -- (set2);
			\path (set2) -- (create2);
			\path (create2) -- (init);
			\path (init) -- (step);
			\path (step) -- (decide);
%		
			\path (decide) -- node [near start] {[Ja]} (done);
			\path (decide) -|  node [above] {[Nein] - i++;} +(-2,0) -- node [] {} +(-3.5,0) |-  (step);
		
%			
			\tikzstyle{every path}=[commentline]
			
			\path (comment1) -- (set1);
			\path (comment2) -- (set2);
			\path (comment3) -- (create2);
			\path (comment4) -- (init);
			\path (comment5) -- (decide);
%			
		\end{tikzpicture}
	\end{center}
\end{figure}
%\begin{figure}[h]
%	\begin{center}
%		\caption[Modellablauf mit CMA-ES]{Ablauf im Modell; Der hier dargestellte Ablauf ist exemplarisch für alle in dieser Arbeit entstandenen Modelle; Die Modelle Variieren nur minimal }
%		\label{fig:struct_model}
%		\vspace{0.5cm}
%		\begin{tikzpicture}[auto]
%		\scriptsize
%			\tikzstyle{decision} = [diamond, draw=black, thick, fill=black!20, text width=5em, text badly centered, inner sep=1pt]
%%			
%			\tikzstyle{block} = [rectangle, draw=black, thick, fill=gray!20, text width=15em, text centered, rounded corners, minimum height=4em]
%%	
%			\tikzstyle{line} = [draw, thick, -latex',shorten >=1pt];
%			\tikzstyle{commentline} = [draw, dashed, green!50,-latex',shorten >=1pt];
%%	
%			\tikzstyle{cloud} = [ dotted, draw=green!40, thick, ellipse,,fill=green!10, minimum height=2em, text width= 12em, text badly centered];
%%	
%			\matrix [column sep=5mm,row sep=7mm]
%			{
%				% row 1
%%				& \node [block] (start) { Start }; & \\
%				% row 2
%%				&\node [block] (create1) {Erstelle '\textit{model}'-Instanz}; & \\
%				% row 3
%%				&\node [block] (set1) {Setze Anzahl der Objektvariablen}; & 
%%					\node [cloud] (comment2) { Teile dem Modell die Anzahl der Veränderlichen mit, der Solver verwendet später diese Initialisierung }; \\
%				%
%%				&\node [block] (set2) {Set Modell-Parameter}; & 
%%					\node [cloud] (comment2) { Die Modelle erhalten unterschiedliche Parameter, z.B. die ausgewählten Matrizen }; \\
%				% row 4
%%				& \node [block] (create2) {Erstelle Instanz von \textit{CMA-ES-Solver}}; & 
%%					\node [cloud] (comment1) {Hier wird das Modell aufgerufen }; \\
%%				& \node [block] (init) {Parametrisiere den Solver}; & 
%%					\node [cloud] (comment3) { Auswahl von $\lambda$ und $\mu$ sowie Lösungsstrategie}; \\
%				% row 5
%%				& \node [block] (step) {cma.\textit{step}()}; & \\
%				%
%%				& \node [decision] (decide) {Optimum reached?}; & 
%%					\node [cloud] (comment4) {Konvergenzkriterium erreicht? (z.B. Ziel-Fitness, max. Evaluationen erreicht etc.)}; \\
%				% row 6
%%				& \node [block] (done) {done}; & \\
%
%			};
%			
%% Arrows
%%			\tikzstyle{every path}=[line]
%%			\path (start) -- (create1);
%%			\path (create1) -- (set1);
%%			\path (set1) -- (set2);
%%			\path (set2) -- (create2);
%%			\path (create2) -- (init);
%%			\path (init) -- (step);
%%			\path (step) -- (decide);
%%%		
%%			\path (decide) -- node [near start] {[Ja]} (done);
%%			\path (decide) -|  node [above] {[Nein] - i++;} +(-2,0) -- node [] {} +(-3.5,0) |-  (call);
%%			
%%			\path (write) -- (stop);
%%%			
%%			\tikzstyle{every path}=[commentline]
%%			
%%			\path (comment1) -- (call);
%%			\path (comment2) -- (decide);
%%			\path (comment3) -- (load);
%%			\path (comment4) -- (create1);
%%			\path (comment5) -- (create);
%%			
%		\end{tikzpicture}
%	\end{center}
%\end{figure}