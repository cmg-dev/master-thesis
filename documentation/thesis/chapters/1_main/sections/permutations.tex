Die sich ergebenen Matrizen sind immer auf eine Referenzantenne bezogen. Es ergeben sich für eine Referenzantenne folgende Anzahl an Matrizen:
% 
\begin{equation}
\frac{7!}{3!(7-3)!}=35
\end{equation}
%
Insgesamt ergeben sich so $8\times 35 = 280$ mögliche Anordnungen. Das erstellte Tool soll in ein entsprechendes C++-Programm portiert werden. Diese Berechnungen werden im späteren System auf jeden Fall anfallen.