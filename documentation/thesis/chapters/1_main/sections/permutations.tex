%
\begin{figure}[hr!]
	\begin{center}
		\caption[Ablauf libPermuatate]{ lorem }
		\label{fig:libpermutate_flowchart}
		\vspace{0.5cm}
		\begin{tikzpicture}[scale=.3]
		\scriptsize
			\tikzstyle{decision} = [diamond, draw=black, thick, fill=black!20, text width=5em, text badly centered, inner sep=1pt]
%			
			\tikzstyle{block} = [rectangle, draw=black, thick, fill=gray!20, text width=15em, text centered, rounded corners, minimum height=4em]
%	
			\tikzstyle{line} = [draw, thick, -latex',shorten >=1pt];
			\tikzstyle{commentline} = [draw, dashed, green!50,-latex',shorten >=1pt];
%	
			\tikzstyle{cloud} = [ dotted, draw=green!50, thick, ellipse,,fill=green!20, minimum height=2em, text width= 10em, text badly centered];
%	
			\matrix [column sep=5mm,row sep=7mm]
			{
				% row 1
				& \node [block] (start) { Start }; & \\
				% row 2
				&\node [block] (read) { Lese Koordinaten aus der Datei }; &\\
				% row 4
				& \node [block] (calcPermutations) {Berechne Permutationen}; & \\
				%				
				& \node [block] (calcMatrixes) {Berechne Matrizen}; & \\	
				% row 5
				& \node [block] (store) {Speichere die Konfigurationen in entsprechender Struktur}; & \\
				%				
				& \node [block] (write) {Schreibe Konfigurationen in CSV-Datei}; & \\
				% row 6
				& \node [block] (stop) {Ende}; & \\
			};
% Arrows
			\tikzstyle{every path}=[line]
			\path (start) -- (read);
			\path (read) -- (calcPermutations);
			\path (calcPermutations) -- (calcMatrixes);
			\path (calcMatrixes) -- (store);
			\path (store) -- (write);
			\path (write) -- (stop);
			
		\end{tikzpicture}
	\end{center}
\end{figure}
%
Um die Beurteilung der Konditionszahl der Matrizen effizient durchführen zu können, wurde im Rahmen dieser Arbeit ein Programm geschrieben. Es erstellt automatisch, auf Basis der durch die Kalibrierung bestimmten Koordinaten, alle möglichen Permutationen von Antennen. Die sich ergebenden Matrizen sind immer auf eine Referenzantenne bezogen. Es ergeben sich, bezogen auf eine Referenzantenne, folgende Anzahl an Matrizen:
% 
\begin{equation}
	\frac{7!}{3!(7-3)!}=35
% 
\end{equation}
%Wdh, steht oben schon: Die sich ergebenen Matrizen sind immer auf eine Referenzantenne bezogen. Es ergeben sich für eine Referenzantenne folgende Anzahl an Matrizen:
%
Für einen Aufbau mit acht Antennen ergeben sich so $8\times 35 = 280$ mögliche Anordnungen. Die Implementation der Berechnungen der Permutationen findet sich in dem Modul \textit{'libPermutate'}. Das Modul generiert bei der Instanziierung automatisch alle möglichen Kombinationen von Antennen und speichert diese in einer geeigneten Struktur für den späteren Gebrauch. Das Ablaufdiagramm ist in Abbildung~\ref{fig:libpermutate_flowchart} zu finden.\\
%
%\begin{equation}
%\frac{7!}{3!(7-3)!}=35
%\end{equation}
%%
%Insgesamt ergeben sich so $8\times 35 = 280$ mögliche Anordnungen. Das erstellte Tool soll in ein entsprechendes C++-Programm portiert werden. Diese Berechnungen werden im späteren System auf jeden Fall anfallen.



