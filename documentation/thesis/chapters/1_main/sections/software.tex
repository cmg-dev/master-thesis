%
\lstset{
	basicstyle=\scriptsize,
	language=C++,
	breaklines=true,
	%		frameround=fttt,
	frame=tbrl,
	breakatwhitespace=false
	breaklines=true,  
	xleftmargin=1cm,
	tabsize=2,
	showstringspaces=false}
%---------------------------------------------------------------------
%
\subsection{Shark}
\label{sec:Shark}
\lipsum[1]
%
%---------------------------------------------------------------------
\subsection{Implementation}
In diesem Abschnitt wird auf interessante Details der Softwareimplementation eingegangen. Es ist nicht möglich im vollen Umfang die Implementation der Softare aufgrund ihres Umfangs zu besprechen. Lediglich sollen hier die Implementationen verschiedener Evolutionsalgorithmen unter Verwendung von Shark gezeigt werden. Es wird auch beispielhaft eine Implementation einer Objektfunktion beschrieben, wie sie typischerweise in Shark implementiert wird.
%
%\begin{figure}[ht!]
	\begin{center}
		\caption[Integration PRPS Software]{Übersicht über die Integration des PRPS Evolution Moduls in die bestehende PRPS-Software}
		\label{fig:Integration-prpsevolutions}
		\vspace{1cm}
		\begin{tikzpicture}[auto]
		\scriptsize
			\tikzstyle{decision} = [diamond, draw=black, thick, fill=black!20, text width=5em, text badly centered, inner sep=1pt]
%			
			\tikzstyle{block} = [rectangle, draw=black, thick, fill=gray!20, text width=15em, text centered, rounded corners, minimum height=4em]
%	
			\tikzstyle{line} = [draw, thick, -latex',shorten >=1pt];
			\tikzstyle{commentline} = [draw, dashed, green!50,-latex',shorten >=1pt];
%	
			\tikzstyle{cloud} = [ dotted, draw=green!50, thick, ellipse,,fill=green!20, minimum height=2em, text width= 10em, text badly centered];
%	
			\matrix [column sep=5mm,row sep=7mm]
			{
				% row 1
				& \node [block] (start) { Start }; & \\
				% row 2
				& \node [block] (setup) {Aufstellen des Kalibrierstücks}; & 
					\node [cloud] (comment1) {Gezeigt in Abbildung \ref{fig:calib_piece}}; \\
				% row 4
				& \node [block] (measure) {Vermessen der Entfernungen zu den Antennen}; & 
					\node [cloud] (comment2) {z.B. mit Laser-Entfernungsmesser, gezeigt in Abbildung \ref{fig:laser_meter}}; \\
				% row 5
				&\node [block] (writefile) {Eintragen der Vermessenen Werte in Maschinenlesbare Datei}; &\\
				% row 6
				\node (temp){}; &\node [block] (startsw) {Starte die Kalibiersoftware}; &\\
				% row 7
				&\node [block] (viewresults) {Speichern der berechneten Werte}; &\\
				% row 8				
				& \node [decision] (decide) {$\Delta \geq \Delta_{max}$}; & 
					\node [cloud] (criteria) {Ergebnisse haben eine geringe Abweichung};\\
				% row 9
				& \node [block] (stop) {Ende}; & \\
			};
			
%
%			Draw the arrows
%
			\path (decide) -| node [near start] {Nein} (temp);
%			\tikzstyle{every path}=[line]
%			\path (start) -- (setup);
%			\path (setup) -- (measure);
%			\path (measure) -- (writefile);
%			\path (writefile) -- (startsw);
%			\path (startsw) -- (viewresults);
%			\path (viewresults) -- (decide);
%			\path (decide)	-| +(-3,0)  |- (measure);
%			\path (decide) -- node [midway] {Ja} (stop);
			
%			
%			draw the comments 
%
			\tikzstyle{every path}=[commentline]
%			\path (criteria) -- (decide);
%			\path (comment1) -- (setup);
%			\path (comment2) -- (measure);
				
		\end{tikzpicture}
	\end{center}
\end{figure}
%\newpage
%---------------------------------------------------------------------
\subsection{Objektfunktion in Shark}
%
Als Beispiel für die Implementation einer Objektfunktion wird im Folgenden das Modell der evolutionären Kalibrierung besprochen. Dieses Modell, bzw. Objektfunktion, hat eine überschaubare Komplexität und wurde bereits im Abschnitt~\ref{sec:calibration} zur Veranschaulichung verwendet. Daher eignet es sich gut um die Implementation zu zeigen. Im Rahmen dieser Arbeit sind eine Vielzahl von Modellen entstanden, die nicht in aller Ausführlichkeit diskutiert werden können.\\
%
%---------------------------------------------------------------------
Das Listing~\ref{lst:EvolutionaryCalibration.h} zeigt die Headerdatei für die Implementation einer Objektfunktion, sofort ist zu erkennen, das sie von der abstakten Klasse \textit{SingleObjectiveFunction} abgeleitet ist. Diese ist einer von Shark bereitgestellte Klasse. Sie beschreibt die Funktionen die eine Objektfunktion implementieren muss damit sie von Optimizern verwendet werden kann. Ein von dieser Klasse abgeleitetes Modell erlaubt die Verwendung in verschiedenen Algorithmen. Die Funktion:
%
\begin{lstlisting}
double eval(const SearchPointType &p) const;
\end{lstlisting}
%
wird in von den Optimizern aufgerufen und implementiert die eigentliche Funktion des Modells.
%
\begin{lstlisting}[label=EvolutionaryCalibration_2]
void proposeStartingPoint(SearchPointType &x) const;
\end{lstlisting}
%
Diese Methode wird von einem Solver beim Start aufgerufen und liefert passende Startwerte für das Modell zurück. Der Rückgabewert der Funktion ist ein normalverteilter Vektor der Dimension \textit{m\_numberOfVariables} in einem Intervall von $[-10,10]$ zurück. Der Wert für \textit{m\_numberOfVariables} wird bei der Instanziierung des Modells, in dem Konstruktor, gesetzt. Es wird auch die Eigenschaft \textit{CAN\_PROPOSE\_STARTING\_POINT} gesetzt, sie wird von den unterschiedlichen Optimizern verwendet um zu erkennen, welche Features vom Modell unterstützt werden.
%
\begin{lstlisting}[label=EvolutionaryCalibration_3]
EvolutionaryCalibration( ) {
	m_numberOfVariables = Solve::ProblemDimensions::Calibration;
	m_features |= CAN_PROPOSE_STARTING_POINT;
}
\end{lstlisting}	
%
%---------------------------------------------------------------------
%
\lstinputlisting[caption=Quellcode Schnipsel für die Deklaration einer Objektfunktion \vspace{2mm},
				firstline=18, lastline=108]{src/EvolutionaryCalibration.h}
\label{lst:EvolutionaryCalibration.h}
%
Das Listig~\ref{lst:EvolutionaryCalibration.cpp} gezeigten cpp-Datei ist die Implementation der oben beschriebenen Modellfunktionen. Aus diesen Beiden Dateien ist ersichtlich, dass die Deklaration des Modells sehr überschaubar ist. Praktische jedes Modell hat diese übersichtliche Struktur. 
%
\begin{lstlisting}[label=EvolutionaryCalibration_4]
inline double EvolutionaryCalibration::mkII( const NRmatrix<Doub> &A, const double* x, const NRvector<Doub> &b ) const
\end{lstlisting}
%
Diese Funktion ist die eigentliche Berechnung des Gleichungssystems der Form $\mathbf{A}\mathbf{x}=\textbf{b}$. Die Lösung wird an die Aufrufende Funktion übergeben. Als Eingabe wird der Variablenvektor \textit{const double*x} von Shark sowie die geom. Matrix und der Distanz Vektor erwartet.
%
Der komplette Quellcode findet sich im Anhang~\ref{app:EvolutionaryCalibration1} und \ref{app:EvolutionaryCalibration2} gelistet.
%
%---------------------------------------------------------------------
%
\lstinputlisting[caption=Quellcode Schnipsel für die Implementation einer Objektfunktion für verschiedene Optimzier im Shark\vspace{2mm},
				 firstline=7, lastline=44]{src/EvolutionaryCalibration.cpp}
				 \label{lst:EvolutionaryCalibration.cpp}
%
\label{lst:shark_model}
%
%
\subsection{Paralleler Ablauf}
\label{lst:parallel_computing}
%
Die \textit{ProcessMkII}-Klasse ist Threadfähig. Eine Implementierung wurde im Rahmen der Arbeit durchgeführt. Durch das parallele Ablaufen verschiedener Optimierungen kann die Performance wesentlicher verbessert werden und so die Multi-Kern-Architektur aktueller Rechner ausgenutzt werden. Einen großen Beitrag zur schnellen Umsetzung lieferte der neue Standard \cpp11. Dieser Implementiert ein High-Level Konstrukt für Threads und Tasks, sodass es einfacher ist, Abläufe zu Synchronisieren und Ergebnisse abzufragen. Zusätzlich ist die Implementation sehr leicht, wie im folgenden Listing gezeigt:
%
\lstinputlisting[caption=Erstellen von mehreren Threads bei gleichzeitiger Übergabe der Parameter\vspace{2mm},
				 firstline=1, lastline=19]{src/async.cpp}
				 \label{lst:Parallel_example1.cpp}
%
\vspace{2mm}
Das Abfragen der Ergebnisse ist genauso einfach:
\vspace{2mm}
%
\lstinputlisting[caption=Abfragen der Asynchronen Ergebnisse mit Hilfe des auto-Datentyp. Ausgabe des Ergebnisses in einer Datei.\vspace{2mm},
				 firstline=23, lastline=36]{src/async.cpp}
				 \label{lst:Parallel_example2.cpp}
%
\vspace{2mm}
%
\subsection{Schnittstelle für Dateneingabe}
%
Im Folgenden wird die implementierte Schnittstelle besprochen mit der die Daten unter den Programmteilen ausgetauscht werden können. Die Schnittstelle umfasst im Wesentliche zwei Teile:
\begin{enumerate}
	\item Eingabe für die gemessenen Phasenwerte
	\item Ausgabe für die ermittelten Wellenzahlen
\end{enumerate}
%
Hinzukommt eine pseudo-Schnittstelle über die die Systemparameter eingelesen werden können. Die aktuelle Implementation ließt lediglich eine CSV-Eingabedatei, mit den Systemparametern in eine entsprechende Struktur ein. Diese Parameter stehen Anschließend dem PRPS-Evolution System zur Verfügung.
%
Die Kommunikation zwischen dem PRPS-Dienst und dem PRPS-Evolution ist einfach. Der Dienst teilt die gemessenen Phasendaten mit, die vom PRPS-Evolution für die Berechnung der Wellenzahlen verwendet werden. Das bedeutet das der Ablauf des Auffindens der Wellenzahl vom PRPS-Dienst als 'Black-Box' angesehen wird.\\
%
%\subsubsection{Schnittstelle für Dateneingabe}
%
\subsection{Ablaufdiagramme}
%
Vieles der Funktionalität der Software ist zu umfangreich für dieses Dokument. Es kann in diesem Rahmen keine vollständige Besprechung des Quellcodes durchgeführt werden. Wichtige Funktionalitäten und Abläufe werden in Ablaufplänen zusammenfassend dargestellt.
%
\begin{figure}[h]
	\begin{center}
		\caption[Ablauf Programmstart]{Prozessschritte nach Start des Programms}
		\label{fig:start_programm}
		\vspace{0.5cm}
		\begin{tikzpicture}[auto]
		\scriptsize
			\tikzstyle{decision} = [diamond, draw=black, thick, fill=black!20, text width=5em, text badly centered, inner sep=1pt]
%			
			\tikzstyle{block} = [rectangle, draw=black, thick, fill=gray!20, text width=15em, text centered, rounded corners, minimum height=4em]
%	
			\tikzstyle{line} = [draw, thick, -latex',shorten >=1pt];
			\tikzstyle{commentline} = [draw, dashed, green!50,-latex',shorten >=1pt];
%	
			\tikzstyle{cloud} = [ dotted, draw=green!50, thick, ellipse,,fill=green!20, minimum height=2em, text width= 10em, text badly centered];
%	
			\matrix [column sep=5mm,row sep=7mm]
			{
				% row 1
				& \node [block] (start) { Start }; & \\
				% row 2
				&\node [block] (load) {Lade System}; &  \\
				% row 3
				&\node [block] (perm) {Generiere Permutationen}; & \\
				% row 4
				& \node [block] (preprocess) {Vorberechnung der Matrizen}; & \\
				% row 5
				& \node [block] (stop) {Ende}; & \\
			};
% Arrows
			\tikzstyle{every path}=[line]
			\path (start) -- (load);
			\path (load) -- (perm);
			\path (perm) -- (preprocess);
			\path (preprocess) -- (stop);
%		
%			\tikzstyle{every path}=[commentline]
%			\path (criteria) -- (decide);
%			\path (comment1) -- (init);
%			
			
		\end{tikzpicture}
	\end{center}
\end{figure}
%
\begin{figure}[h]
	\begin{center}
		\caption[Finde Modelllösung]{Schritte die zur Lösungsfindung abgearbeitet werden. Dies ist ein Teil des Hauptprogramms, der die erstellten Informationen vom Programmstart in ein Modell zum Auffinden einer Lösung übergibt. Es werden eine beliebige Anzahl an Lösungen generiert.}
		\label{fig:find_model_solution}
		\vspace{0.5cm}
		\begin{tikzpicture}[auto]
		\scriptsize
			\tikzstyle{decision} = [diamond, draw=black, thick, fill=black!20, text width=5em, text badly centered, inner sep=1pt]
%			
			\tikzstyle{block} = [rectangle, draw=black, thick, fill=gray!20, text width=15em, text centered, rounded corners, minimum height=4em]
%	
			\tikzstyle{line} = [draw, thick, -latex',shorten >=1pt];
			\tikzstyle{commentline} = [draw, dashed, green!50,-latex',shorten >=1pt];
%	
			\tikzstyle{cloud} = [ dotted, draw=green!40, thick, ellipse,,fill=green!10, minimum height=2em, text width= 12em, text badly centered];
%	
			\matrix [column sep=5mm,row sep=7mm]
			{
				% row 1
				& \node [block] (start) { Start }; & \\
				% row 2
				&\node [block] (load) {Lade Phasendaten vom Interface}; &  
					\node [cloud] (comment3) {Verschiedene Quellen sind mögliche (z.B. CSV-Datei, Stream etc.) }; \\
				% row 3
				&\node [block] (create1) {Erstelle '\textit{preprocess}'-Instanz}; & 
					\node [cloud] (comment4) {Berechne die möglichen Matrizen; Treffe Auswahl an 'guten' Matizen für die Lösung }; \\
				%
				&\node [block] (create) {Erstelle '\textit{process}'-Instanz}; & 
					\node [cloud] (comment5) {Übergebe benötigte Informationen) }; \\
				% row 4
				& \node [block] (call) {Lösung für Modell finden:\\ process.\textit{model}()}; & 
					\node [cloud] (comment1) {Finde Lösung für '\textit{model}' - mehrere Modelle sind in '\textit{process}' implementiert}; \\
				% row 5
				& \node [decision] (decide) {$i==N$}; & 
					\node [cloud] (comment2) {Anzahl der Lösungen $N$ erreicht?}; \\
				% row 6
				& \node [block] (write) { Rufe '\textit{post-processing}' auf }; & \\
				% row 7
				& \node [block] (stop) {Ende}; & \\
			};
% Arrows
			\tikzstyle{every path}=[line]
			\path (start) -- (load);
			\path (load) -- (create1);
			\path (create1) -- (create);
			\path (create) -- (call);
			\path (call) -- (decide);
%		
			\path (decide) -- node [near start] {[Ja]} (write);
			\path (decide) -|  node [above] {[Nein] - i++;} +(-2,0) -- node [] {} +(-3.5,0) |-  (call);
			
			\path (write) -- (stop);
%			
			\tikzstyle{every path}=[commentline]
			
			\path (comment1) -- (call);
			\path (comment2) -- (decide);
			\path (comment3) -- (load);
			\path (comment4) -- (create1);
			\path (comment5) -- (create);
%			
		\end{tikzpicture}
	\end{center}
\end{figure}
%
\begin{figure}[h]
	\begin{center}
		\caption[Modelllösung bestimmen]{Die Grafik zeigt den Ablauf im Modell um eine Lösung zu finden. Es werden so lange Evolutionsschritte durchgeführt bis das Konvergenzkriterium erreicht wird.}
		\label{fig:find_model_solution}
		\vspace{0.5cm}
		\begin{tikzpicture}[auto]
		\scriptsize
			\tikzstyle{decision} = [diamond, draw=black, thick, fill=black!20, text width=5em, text badly centered, inner sep=1pt]
%			
			\tikzstyle{block} = [rectangle, draw=black, thick, fill=gray!20, text width=15em, text centered, rounded corners, minimum height=4em]
%	
			\tikzstyle{line} = [draw, thick, -latex',shorten >=1pt];
			\tikzstyle{commentline} = [draw, dashed, gray!50,-latex',shorten >=1pt];
%	
			\tikzstyle{cloud} = [ dotted, draw=gray!50, thick, ellipse,,fill=gray!5, minimum height=2em, text width= 10em, text badly centered];
%			\tikzstyle{cloud} = [ dotted, draw=green!40, thick, ellipse,,fill=green!10, minimum height=1em, text width= 15em, text badly centered];
%	
			\matrix [column sep=5mm,row sep=7mm]
			{
%			 row 1
				& \node [block] (start) { Start }; & \\
%	 row 2
				&\node [block] (create1) {Erstelle '\textit{model}'-Instanz}; & \\
%	 row 3
				&\node [block] (set1) {Setze Anzahl der Objektvariablen}; & 
					\node [cloud] (comment1) { Teile dem Modell die Anzahl der Veränderlichen mit, der Solver verwendet später diese Initialisierung }; \\
				
				&\node [block] (set2) {Set Modell-Parameter}; & 
					\node [cloud] (comment2) { Die Modelle erhalten unterschiedliche Parameter, z.B. die ausgewählten Matrizen }; \\
%	 row 4
				& \node [block] (create2) {Erstelle Instanz von \textit{CMA-ES-Solver}}; & 
					\node [cloud] (comment3) {Hier wird das Modell aufgerufen }; \\
				& \node [block] (init) {Parametrisiere den Solver}; & 
					\node [cloud] (comment4) { Einstellen von $\lambda$ und $\mu$ sowie Lösungsstrategie}; \\
%	 row 5
				& \node [block] (step) {cma.\textit{step}()}; & \\
				
				& \node [decision] (decide) {Optimum reached?}; & 
					\node [cloud] (comment5) {Konvergenzkriterium erreicht? (z.B. Ziel-Fitness, max. Evaluationen erreicht etc.)}; \\
%	 row 6
				& \node [block] (done) {done\\ \textbf{return final-solution}}; & \\
			};
% Arrows
			\tikzstyle{every path}=[line]
			\path (start) -- (create1);
			\path (create1) -- (set1);
			\path (set1) -- (set2);
			\path (set2) -- (create2);
			\path (create2) -- (init);
			\path (init) -- (step);
			\path (step) -- (decide);
%		
			\path (decide) -- node [near start] {[Ja]} (done);
			\path (decide) -|  node [above] {[Nein] - i++;} +(-2,0) -- node [] {} +(-3.5,0) |-  (step);
		
%			
			\tikzstyle{every path}=[commentline]
			
			\path (comment1) -- (set1);
			\path (comment2) -- (set2);
			\path (comment3) -- (create2);
			\path (comment4) -- (init);
			\path (comment5) -- (decide);
%			
		\end{tikzpicture}
	\end{center}
\end{figure}
%\begin{figure}[h]
%	\begin{center}
%		\caption[Modellablauf mit CMA-ES]{Ablauf im Modell; Der hier dargestellte Ablauf ist exemplarisch für alle in dieser Arbeit entstandenen Modelle; Die Modelle Variieren nur minimal }
%		\label{fig:struct_model}
%		\vspace{0.5cm}
%		\begin{tikzpicture}[auto]
%		\scriptsize
%			\tikzstyle{decision} = [diamond, draw=black, thick, fill=black!20, text width=5em, text badly centered, inner sep=1pt]
%%			
%			\tikzstyle{block} = [rectangle, draw=black, thick, fill=gray!20, text width=15em, text centered, rounded corners, minimum height=4em]
%%	
%			\tikzstyle{line} = [draw, thick, -latex',shorten >=1pt];
%			\tikzstyle{commentline} = [draw, dashed, green!50,-latex',shorten >=1pt];
%%	
%			\tikzstyle{cloud} = [ dotted, draw=green!40, thick, ellipse,,fill=green!10, minimum height=2em, text width= 12em, text badly centered];
%%	
%			\matrix [column sep=5mm,row sep=7mm]
%			{
%				% row 1
%%				& \node [block] (start) { Start }; & \\
%				% row 2
%%				&\node [block] (create1) {Erstelle '\textit{model}'-Instanz}; & \\
%				% row 3
%%				&\node [block] (set1) {Setze Anzahl der Objektvariablen}; & 
%%					\node [cloud] (comment2) { Teile dem Modell die Anzahl der Veränderlichen mit, der Solver verwendet später diese Initialisierung }; \\
%				%
%%				&\node [block] (set2) {Set Modell-Parameter}; & 
%%					\node [cloud] (comment2) { Die Modelle erhalten unterschiedliche Parameter, z.B. die ausgewählten Matrizen }; \\
%				% row 4
%%				& \node [block] (create2) {Erstelle Instanz von \textit{CMA-ES-Solver}}; & 
%%					\node [cloud] (comment1) {Hier wird das Modell aufgerufen }; \\
%%				& \node [block] (init) {Parametrisiere den Solver}; & 
%%					\node [cloud] (comment3) { Auswahl von $\lambda$ und $\mu$ sowie Lösungsstrategie}; \\
%				% row 5
%%				& \node [block] (step) {cma.\textit{step}()}; & \\
%				%
%%				& \node [decision] (decide) {Optimum reached?}; & 
%%					\node [cloud] (comment4) {Konvergenzkriterium erreicht? (z.B. Ziel-Fitness, max. Evaluationen erreicht etc.)}; \\
%				% row 6
%%				& \node [block] (done) {done}; & \\
%
%			};
%			
%% Arrows
%%			\tikzstyle{every path}=[line]
%%			\path (start) -- (create1);
%%			\path (create1) -- (set1);
%%			\path (set1) -- (set2);
%%			\path (set2) -- (create2);
%%			\path (create2) -- (init);
%%			\path (init) -- (step);
%%			\path (step) -- (decide);
%%%		
%%			\path (decide) -- node [near start] {[Ja]} (done);
%%			\path (decide) -|  node [above] {[Nein] - i++;} +(-2,0) -- node [] {} +(-3.5,0) |-  (call);
%%			
%%			\path (write) -- (stop);
%%%			
%%			\tikzstyle{every path}=[commentline]
%%			
%%			\path (comment1) -- (call);
%%			\path (comment2) -- (decide);
%%			\path (comment3) -- (load);
%%			\path (comment4) -- (create1);
%%			\path (comment5) -- (create);
%%			
%		\end{tikzpicture}
%	\end{center}
%\end{figure}
%

%\lipsum[1-5]