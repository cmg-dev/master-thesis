\begin{figure}[h]
	\begin{center}
		\caption[Ablauf Programmstart]{Prozessschritte nach Start des Programms. Mit dem Punkt "Ende" ist nicht das Ende des Programms gemeint, sondern das Ende des Programmstarts. Nach diesen Schritten stehen alle statisch generierbaren verfügbaren Daten für das Programm zur Verfügung. }
		\label{fig:start_programm}
		\vspace{0.5cm}
		\begin{tikzpicture}[auto]
		\scriptsize
			\tikzstyle{decision} = [diamond, draw=black, thick, fill=black!20, text width=5em, text badly centered, inner sep=1pt]
%			
			\tikzstyle{block} = [rectangle, draw=black, thick, fill=gray!20, text width=15em, text centered, rounded corners, minimum height=4em]
%	
			\tikzstyle{line} = [draw, thick, -latex',shorten >=1pt];
			\tikzstyle{commentline} = [draw, dashed, green!50,-latex',shorten >=1pt];
%	
			\tikzstyle{cloud} = [ dotted, draw=green!50, thick, ellipse,,fill=green!20, minimum height=2em, text width= 10em, text badly centered];
%	
			\matrix [column sep=5mm,row sep=7mm]
			{
				% row 1
				& \node [block] (start) { Start }; & \\
				% row 2
				&\node [block] (load) {Lade System}; &  \\
				% row 3
				&\node [block] (perm) {Generiere Permutationen}; & \\
				% row 4
				& \node [block] (preprocess) {Vorberechnung der Matrizen}; & \\
				% row 5
				& \node [block] (stop) {Ende}; & \\
			};
% Arrows
			\tikzstyle{every path}=[line]
			\path (start) -- (load);
			\path (load) -- (perm);
			\path (perm) -- (preprocess);
			\path (preprocess) -- (stop);
%		
%			\tikzstyle{every path}=[commentline]
%			\path (criteria) -- (decide);
%			\path (comment1) -- (init);
%			
			
		\end{tikzpicture}
	\end{center}
\end{figure}