\begin{figure}[h]
	\begin{center}
		\caption[Finde Modelllösung]{Schritte die zur Lösungsfindung abgearbeitet werden. Dies ist ein Teil des Hauptprogramms, der die erstellten Informationen vom Programmstart in ein Modell zum Auffinden einer Lösung übergibt. Es werden eine beliebige Anzahl an Lösungen generiert.}
		\label{fig:find_model_solution}
		\vspace{0.5cm}
		\begin{tikzpicture}[auto]
		\scriptsize
			\tikzstyle{decision} = [diamond, draw=black, thick, fill=black!20, text width=5em, text badly centered, inner sep=1pt]
%			
			\tikzstyle{block} = [rectangle, draw=black, thick, fill=gray!20, text width=15em, text centered, rounded corners, minimum height=4em]
%	
			\tikzstyle{line} = [draw, thick, -latex',shorten >=1pt];
			\tikzstyle{commentline} = [draw, dashed, green!50,-latex',shorten >=1pt];
%	
			\tikzstyle{cloud} = [ dotted, draw=green!40, thick, ellipse,,fill=green!10, minimum height=2em, text width= 12em, text badly centered];
%	
			\matrix [column sep=5mm,row sep=7mm]
			{
				% row 1
				& \node [block] (start) { Start }; & \\
				% row 2
				&\node [block] (load) {Lade Phasendaten vom Interface}; &  
					\node [cloud] (comment3) {Verschiedene Quellen sind mögliche (z.B. CSV-Datei, Stream etc.) }; \\
				% row 3
				&\node [block] (create1) {Erstelle '\textit{preprocess}'-Instanz}; & 
					\node [cloud] (comment4) {Berechne die möglichen Matrizen; Treffe Auswahl an 'guten' Matizen für die Lösung }; \\
				%
				&\node [block] (create) {Erstelle '\textit{process}'-Instanz}; & 
					\node [cloud] (comment5) {Übergebe benötigte Informationen) }; \\
				% row 4
				& \node [block] (call) {Lösung für Modell finden:\\ process.\textit{model}()}; & 
					\node [cloud] (comment1) {Finde Lösung für '\textit{model}' - mehrere Modelle sind in '\textit{process}' implementiert}; \\
				% row 5
				& \node [decision] (decide) {$i==N$}; & 
					\node [cloud] (comment2) {Anzahl der Lösungen $N$ erreicht?}; \\
				% row 6
				& \node [block] (write) { Rufe '\textit{post-processing}' auf }; & \\
				% row 7
				& \node [block] (stop) {Ende}; & \\
			};
% Arrows
			\tikzstyle{every path}=[line]
			\path (start) -- (load);
			\path (load) -- (create1);
			\path (create1) -- (create);
			\path (create) -- (call);
			\path (call) -- (decide);
%		
			\path (decide) -- node [near start] {[Ja]} (write);
			\path (decide) -|  node [above] {[Nein] - i++;} +(-2,0) -- node [] {} +(-3.5,0) |-  (call);
			
			\path (write) -- (stop);
%			
			\tikzstyle{every path}=[commentline]
			
			\path (comment1) -- (call);
			\path (comment2) -- (decide);
			\path (comment3) -- (load);
			\path (comment4) -- (create1);
			\path (comment5) -- (create);
%			
		\end{tikzpicture}
	\end{center}
\end{figure}