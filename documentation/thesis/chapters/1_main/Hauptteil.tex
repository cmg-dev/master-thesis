\section{Vorüberlegung zur Komplexität}
\label{sec:Komplexity1}
%
\begin{floatingfigure}[hr!]{6cm}
 \centering
         \includegraphics[width=7cm]{img/Plate0_A1.png}
         \caption[Profil einer Phasenmessung]{Normiertes Höhenprofil einer Phasenmessung aus der Sicht von Antenne 1 }
         \label{fig:Plate0_A1_}
\end{floatingfigure}
%
In diesem Abschnitt wird eine Übersicht über die Komplexität des Problems gegeben. 
Abbildung~\ref{fig:Plate0_A1_} zeigt die Visualisierung einer typischen Kalibriermessung. Der verwendete Aufbau ist in Abbildung~\ref{fig:Spider1}. gezeigt. Er besteht aus vier Antennen, die in einer Ebene angeordnet sind. Es wurde eine reproduzierbare Aufstellung verwendet (Abbildung~\ref{fig:Spider_setup1}) und eine Fläche von $1\times1$ Meter vermessen. Alle $10$ cm wurde eine Messung gespeichert. In der Abbildung kann man deutlich das Verhalten der Phasendaten sehen. Um diesen Verlauf deutlicher zu zeigen, wurden die Phasenwerte normiert und als Oberfläche in den Plot gelegt. Am Boden gezeigt ist der Kontur-Plot der Werte. Zwischen den Werten wurde interpoliert um die Nulldurchgänge deutlicher zu zeigen. Die Übersicht aus der Sicht aller Antennen ist in Abbildung~\ref{fig:Real_Measurements} gezeigt.\\
%

In der Abbildung~\ref{fig:Complexity1} werden die Daten ohne Interpolation dargestellt. Es wurden die Höhenlinien eingezeichnet. Die Anordnung der Plots soll ein Gefühl dafür vermitteln, wie die Messwerte eines Tags sich an unterschiedlichen Postionen und aus Sicht verschiedener Antennen verhalten.\\
%

Die Darstellung echter Messwerte lässt Rückschlüsse auf die Komplexität des Problems zu. Es ist leicht nachzuvollziehen, dass das Zusammenspiel der Messwerte eine sehr komplexe Szene ergibt. Hier dargestellt ist bereits das Verhalten bei der Verwendung von vier Antennen. Der aktuelle Messaufbau erlaubt sogar acht Antennen. Das ergibt insgesamt eine komplexe Szenerie.\\
%
\begin{figure}[ht!]
        \centering
        \begin{subfigure}[b]{0.4\textwidth}
            \centering
            \includegraphics[width=\textwidth]{img/Plate0_A1.png}
            \caption[lorem]{Antenne 1}
            \label{fig:Plate0_A1}
        \end{subfigure}%
\\
        \begin{subfigure}[b]{0.4\textwidth}
            \centering
            \includegraphics[width=\textwidth]{img/Plate0_A2.png}
          	\caption[Loren ipsum]{Antenne 2}
         	\label{fig:Plate0_A2}
        \end{subfigure}
\qquad\qquad
        \begin{subfigure}[b]{0.4\textwidth}
			\centering
			\includegraphics[width=\textwidth]{img/Plate0_A4.png}
			\caption[Loren ipsum]{Antenne 4}
			\label{fig:Plate0_A3}
        \end{subfigure}
\\
        \begin{subfigure}[b]{0.4\textwidth}
			\centering
			\includegraphics[width=\textwidth]{img/Plate0_A3.png}
			\caption[Loren ipsum]{Antenne 3}
			\label{fig:Plate0_A4}
        \end{subfigure}
        \caption[Reale Messwerte visualisiert]{Blick auf die Messwerte der  Kalibrierplatte aus der "Sicht" der Antennen. Dabei zeigt sich deutlich der Wellencharakter der Messung, dieser ist zu erwarten. Die Messungen wurden mit einer Frequenz von $865,7$ MHz unter Laborbedingungen aufgenommen. }\label{fig:Real_Measurements}
\end{figure}
%
\begin{figure}[ht!]
         \centering
         \includegraphics[width=0.6\textwidth]{img/complexitiy1.pdf}
%         \includegraphics[width=0.7\textwidth]{img/00_placeholder.png}
         \caption[Normierte Messwerte von Kalibriermessung]{Diese Grafik zeigt die Visualisierung von realen Phasen-Messwerten. Die Daten wurden durch Vermessung einer $1\times1$-Kalibrierplatte mit reproduzierbarer Aufstellung\footnote{In dieser Arbeit nicht gezeigt} gewonnen. Die Daten wurden normiert. In jeder Dimension wurden $10\times10$ Werte aufgenommen. Die Darstellung der Phasenwerte erfolgt als Heatmap, es soll qualitativ der Verlauf der Phasenwerte gezeigt werden. Zur Orientierung sind in jedem Plot Höhenlinien eingezeichnet. Pro Plot werden die Daten einer Antenne dargestellt. Die Antenne von der die Daten stammen ist angegeben.}
         \label{fig:Complexity1}
%
\end{figure}


%
\section{Entwicklung des Modells}
\label{sec:model_developement}
Im folgenden Abschnitt wird das Modell für die Lösung des Zusammenhangs entwickelt. Zur Veranschaulichung des Sachverhalts dient die Abbildung~\ref{fig:TrilaterationScene}. Dort skizziert ist der Messaufbau mit einem Tag. Die Szene ist in 2D dargestellt die Ableitung des Modells erfolgt direkt für drei Raumkoordinaten.
%
\begin{figure}
	\begin{center}
		\caption[Antennen-Szene mit einem Tag]{2D-Übersicht auf die Szene mit drei Antennen, einem Tag und einer Landmarke. Die Position von $\{A_1,A_3,A_3\}$, sowie der Landmarke, zum Koordinatenursprung sind bekannt. Die Vektoren $r_1,r_2,r_3$ sind die gemessene Entfernung zu einer Antenne. Die Landmarke wird im späteren Verlauf eine Antenne sein, die ihrerseits ein gemessene Entfernung $r_0$ produziert. Der Schnittpunkt aller Kreise ist die Lösung der gemessenen Entfernung und der geom. Anordnung, die sich für die Position des Tags ergibt.} 
		\label{fig:TrilaterationScene}
		\begin{tikzpicture}[
    scale=1,
    axis/.style={thick, ->, >=stealth'},
%    vector/.style={thick, ->,-latex, >=stealth'},
%    antenna/.style={thick},
     important line/.style={thick},
     antenna/.style={thick, cyan!70},
%    dashed line/.style={dashed, thin},
%    pile/.style={thick, ->, >=stealth', shorten <=2pt, shorten
%    >=2pt},
%    every node/.style={color=black},
%    main node/.style={circle,fill=blue!20,draw},
%    help lines/.style={gray,very thin}
    ]
    % axis
    \draw[axis] (-.1,0)  -- (1,0) node(xline)[right] {$x$};
    \draw[axis] (0,-.1) -- (0,1) node(yline)[above] {$y$};

	\draw[gray, very thin, dotted] (0,0) grid (15,6);

	\coordinate (A1_start) at (4,3);
	\coordinate (A1_end) at (4,4);
	\coordinate (A2_start) at (7,5);
	\coordinate (A2_end) at (8,5);
	\coordinate (A3_start) at (8,1);
	\coordinate (A3_end) at (8,2);

	\coordinate (A1_end_) at ($(A1_start)!1!-10:(A1_end)$);
	\coordinate (A2_end_) at ($(A2_start)!1!-10:(A2_end)$);
	\coordinate (A3_end_) at ($(A3_start)!1!-35:(A3_end)$);
	
	\coordinate (Tag_0) at (6,2);
	\coordinate (REF_0) at (12,5);
	\coordinate (Int1) at ($(A1_start)!.5!(A1_end_)$);
	\coordinate (Int2) at ($(A2_start)!.5!(A2_end_)$);
	\coordinate (Int3) at ($(A3_start)!.5!(A3_end_)$);
	
	\begin{scope}
		\node [draw,orange!50,dashed] at (Int1) [circle through={(Tag_0)}] {};
		\node [draw,orange!50,dashed] at (Int2) [circle through={(Tag_0)}] {};
		\node [draw,orange!50,dashed] at (Int3) [circle through={(Tag_0)}] {};
	\end{scope}
	
	\draw[antenna] (A1_start) node[font=\scriptsize,black,below] {$A_1$} -- ($(A1_start)!1!-10:(A1_end)$);
	\draw[antenna] (A2_start) node[font=\scriptsize,black,above] {$A_2$}-- ($(A2_start)!1!-10:(A2_end)$);
	\draw[antenna] (A3_start) node[font=\scriptsize,black,below] {$A_3$}-- ($(A3_start)!1!-35:(A3_end)$);
	
	\node [green!60!black!90, right,font=\scriptsize ] at (REF_0) {$\text{Landmarke}@(x_0,y_0,z_0)$};

	\draw[latex-latex] (Tag_0) -- node[sloped,above,midway] {$r_1$}(Int1);
	\draw[latex-latex] (Tag_0) -- node[sloped,above,midway] {$r_2$}(Int2);
	\draw[latex-latex] (Tag_0) -- node[sloped,above,midway] {$r_2$}(Int3);
	\draw[-latex,dashed,green!60!black!90] (REF_0) -- node[sloped,above,midway] {$r_0$}(Tag_0);
	
	\draw[ -latex,violet!60,font=\scriptsize,dotted] (REF_0) -- node[sloped,above,midway] {$d_{10}$}(Int1);
	\draw[ -latex,violet!60,font=\scriptsize,dotted] (REF_0) -- node[sloped,above,midway] {$d_{20}$}(Int2);
	\draw[ -latex,violet!60,font=\scriptsize,dotted] (REF_0) -- node[sloped,above,midway] {$d_{30}$}(Int3);
		
	\fill[red!70] (Tag_0) circle [radius=2pt];
	\node[font=\scriptsize,black,below] at (Tag_0) {$Tag$} ;
	\fill[green!60!black!90] (REF_0) circle [radius=2pt];
	
\end{tikzpicture}



%		
	\end{center}
\end{figure}
%
Folgende Nomenklatur und Symbole gelten für diesen Abschnitt:
\begin{itemize}[itemsep=0mm]
	\item	$r_{k}$ := Abstand vom Tag zur Antenne
	\item	$d_{kJ}$ := Abstand zur Landmarke
	\item	$N_0:=$ Menge der verfügbaren Antennen $N=\{1,..,8\}$
	\item	$N:=$ Menge der Antennen für die Optimierung verfügbar sind\footnote{d.h. ein Messergebnis liefern}($N \subseteq N_0$)
	\item	$N':=$ Menge der Antennen für die Optimierung ($N' \subseteq N$)
%	; Dabei ist $|N'| \geq 3$
%	\item	Es gilt $|N'| \geq |N| \geq |N_0|$   
	\item	$j$ ist der Index der Referenzantenne, es gilt $j = \{1,2,..,8\}$
	\item	$k$ ist der Index der Antennen einer Messung, es gilt $k = 1,2,..,|N'|-1$
\end{itemize}
%
Wir starten mit der Überlegung über den geometrischen Zusammenhang zwischen der Antennenposition von Antenne $k$ zu der Position des Tags $r_k$:
\begin{align}
	\label{eq:base_vactor}
	r_{k}^2 &= (x-x_k)^2+(y-y_k)^2+(z-z_k)^2
\end{align}
%
Diese Gleichung stellt die Euklidische Vektornorm dar und entspricht der Strecke Antenne-Tag. Für die Ermittelung einer Postion (mit drei Raumkoordinaten) sind drei Antennen Notwendig. Daraus ergibt sich:
%
\begin{itemize}[itemsep=0mm]
\item 3 Gleichunge n
\item 3 Unbekannte
\item Quadratisches Gleichungssystem
\end{itemize}
%
Das Gleichungssystem sieht wie folgt aus:
%
\begin{align}
	r_{1}^2 &= (x-x_1)^2+(y-y_1)^2+(z-z_1)^2 \nonumber\\
	r_{2}^2 &= (x-x_2)^2+(y-y_2)^2+(z-z_2)^2 \nonumber\\
	r_{3}^2 &= (x-x_3)^2+(y-y_3)^2+(z-z_3)^2 \nonumber
%	
\end{align}
%
Es ist trivial und wird in verschiedenen Beispielen gezeigt\footnote{z.B. \url{http://en.wikipedia.org/w/index.php?title=Trilateration&oldid=553215995}}, dass man die Koordinaten aus dem quadratischen Gleichungssystem unmittelbar berechnen kann. Es muss jedoch ein quadratisches Gleichungssystem gelöst werden, was zu den bekannten Problematiken führt, insbesondere der Ausschluss mehrdeutiger Ergebnisse. Der Messaufbau der \amedogmbh erlaubt die Verwendung von mehr als 3 Messwertgebern. Diese zusätzliche Informationen lassen sich für eine Linearisierung des Gleichungssystems verwenden. Dieser Ansatz wird für ein Modell im Rahmen dieser Arbeit verwendet und wird im Folgenden beschrieben.\\
%
Von den Antennen sind die Raumkoordinaten ($x,y,z-Koordinaten$) bekannt, bzw. wurden durch Kalibrierung \ref{sec:calibration} in einem vorherigen Schritt bestimmt. Wir können zusätzlich zu notieren:
%
\begin{equation}\label{eq:d_k0}
	d_{kj}^2= (x_k-x_0)^2+(y_k-y_0)^2+(z_k-z_0)^2
\end{equation}
%
Linearisierung des Modells. Dazu wird Gleichung~\ref{eq:base_vactor} in mehreren Schritten umgebaut. Zuerst wird eine neutrale Erweiterung durchgeführt und die Terme geschickt zusammengefasst. Das führt zu:
%
\begin{align}
	r_{k}^2 &= (x-x_k)^2+(y-y_k)^2+(z-z_k)^2 \nonumber \\
	&=(x-x_k+x_0-x_0)^2+(y-y_k+y_0-y_0)^2+(z-z_k+z_0-z_0)^2 \nonumber \\
	&=((x-x_0)-(x_k-x_0))^2+((y-y_0)-(y_k-y_0))^2+((z-z_0)-(z_k-z_0))^2 \nonumber \\ 
	%2 bin. Form
	&=(x-x_0)^2-2(x-x_0)(x_k-x_0)+(x_k-x_0)^2\underbrace{+\dots{}+\dots{}}_\text{y-\& z-Terme analog}
	\label{eq:tri_temp1}
%
\end{align}
%
Um Platz zu sparen sind die y- und z-Terme nicht explizit notiert. Sie ergeben sich durch einfaches Ersetzen der Indizes und werden im Finalen Modell eingefügt. Durch Umstellen von \eqref{eq:tri_temp1} erhalten wir:
\begin{align}
(x-x_0)(x_k-x_0)+\dots{}+\dots{}&=-\frac{1}{2}[r_k^2-(x_k-x_0)^2 -(x-x_0)^2 +\dots{} +\dots{}]\nonumber\\
(x-x_0)(x_k-x_0)+\dots{}+\dots{}&=\phantom{-}\frac{1}{2}[(x_k-x_0)^2 +(x-x_0)^2 +\dots{}+\dots{}-r_k^2]\nonumber
%
\end{align}
%
\begin{multline}\label{eq:rk_final}
(x-x_0)(x_k-x_0)+(y-y_0)(y_k-y_0)+(z-z_0)(z_k-z_0)= \\\frac{1}{2}[(x_k-x_0)^2+(x-x_0)^2-(y_k-y_0)^2+(y-y_0)^2
\\-(z_k-z_0)^2 +(z-z_0)^2-r_k^2]
\end{multline}
%
Vergleich von \eqref{eq:rk_final} mit \eqref{eq:d_k0} bringt: 
%
\begin{multline}
(x-x_0)(x_k-x_0)+(y-y_0)(y_k-y_0)+(z-z_0)(z_k-z_0)= \\\frac{1}{2}[\underbrace{(x_k-x_0)^2+(z_k-z_0)^2+(y_k-y_0)^2}_\text{\boldmath{$d_{kj}^2$}}
\\+\underbrace{(x-x_0)^2+(y-y_0)^2 +(z-z_0)^2}_\text{\boldmath{$r_j^2$}}-r_k^2]
\end{multline}
%
\begin{equation}
(x-x_0)(x_k-x_0)+(y-y_0)(y_k-y_0)+(z-z_0)(z_k-z_0)=\frac{1}{2}[d_{kj}^2+r_{j}^2-r_k^2]\label{eq:rk_final_simplyfied}
\end{equation}
mit 
\begin{equation}\label{eq:c_kj}
\mathbf{c_{kj}}=\frac{1}{2}[d_{kj}^2+r_{j}^2-r_k^2]
\end{equation}
können wir das lineare Gleichungssystem abschließend schreiben:
%
\begin{equation}
%\label{eq:final_trilateration_model}
\mathbf{0}=
\left(
	\begin{array}{ccc}
		x_1-x_j & y_1-y_j & z_1-z_j \\
		x_2-x_j & y_2-y_j & z_2-z_j \\
		x_3-x_j & y_3-y_j & z_3-z_j
	\end{array}
\right)
\left(
   \begin{array}{c}
	   x-x_j\\
	   y-y_j\\
	   z-z_j
   \end{array}
\right)
-
\left(
	\begin{array}{c}
		c_{1j}\\
		c_{2j}\\
		c_{3j}
	\end{array}
\right)
\end{equation}
%
Das Gleichungssystem entspricht ist linear und hat die allg. Form: $\mathbf{0} = \mathbf{Ax}+\mathbf{b}$ es lässt sich mit bekannten Methoden lösen.




{
\small
Folgende Nomenklatur und Symbole gelten für diesen Abschnitt:
\begin{itemize}[itemsep=0mm]
	\item	$N:=$Anzahl der Antennen $N=\{1,..,8\}$
	\item	$k$ ist der Index der Antennen, es gilt $k = \{1,2,..,N-1\}$
	\item	$r_{k}$ := Abstand vom Tag zur Antenne
	\item	$d_{k0}$ := Abstand zur Landmarke
	\item	fette Großbuchstaben stehen für Matrizen (bspw. $\mathbf{A}$)
	\item	fette Kleinbuchstaben stehen für Vektoren (bspw. $\mathbf{x}$)
	
\end{itemize}
%
Wie gezeigt werden konnte\footnote{Wochenbericht KW 20, Anhang B} ergibt sich für den Fall der Trilateration und der Annahme, dass vier Antennen Messwerte liefern, die Gleichung:
\begin{equation}\label{eq:final_trilateration_model}
0=
\left(
	\begin{array}{ccc}
		x_k-x_0 & y_k-y_0 & z_k-z_0 
	\end{array}
\right)
\left(
   \begin{array}{c}
	   x-x_0\\
	   y-y_0\\
	   z-z_0
   \end{array}
\right)
-
\left(
	\begin{array}{c}
		c_{k0}
	\end{array}
\right) 
\end{equation}
%
Dabei ist:
\begin{equation}\label{eq:c_k0}
	c_{k0}=\frac{1}{2}[d_{k0}^2+r_{0}^2-r_k^2]
\end{equation}
%
Ziel dieser Erweiterung ist es, einen Zusammenhang zwischen diesem Modell und der Wellenzahl zu erzeugen. Folgender Ansatz wird gewählt:
	\begin{equation}\label{eq:r_0_theta} r(\Theta)=\frac{\lambda}{2}\left(\frac{\Theta}{2\pi}+n\right),\\\lambda=\frac{c}{f}, n:= \text{Wellenzahl}
\end{equation}
%
%
\begin{shaded} 
Weiterhin ist $\Theta$ die gemessene Phase, die das PRPS-System liefert und $n$ die gesuchte Wellenzahl.\\
Durch einsetzen von \eqref{eq:r_0_theta} in \eqref{eq:c_k0}, erhalten wir:
\begin{equation}\label{eq:c_k0_extended}
	c_{k0}(\Theta_0, \Theta_k, n_0, n_k) =\frac{1}{2}\left[d_{k0}^2+\frac{\lambda^2}{4}\left(\frac{\Theta_0}{2\pi}+n_0\right)^2-\frac{\lambda^2}{4}\left(\frac{\Theta_k}{2\pi}+n_k\right)^2\right]
\end{equation}
%
Wir stellen Gleichung~\eqref{eq:c_k0_extended} um:
\begin{align}
%	
	c_{k0}(\Theta_0, \Theta_k, n_0, n_k) &= \frac{1}{2}\left\{d_{k0}^2+\frac{\lambda^2}{4}\left[\left(\frac{\Theta_0}{2\pi}\right)^2+2\frac{\Theta_0}{2\pi}n_0+n_0^2 \right.\right.\nonumber\\
	&\phantom{=}\; 
	\left.\left.-\left(\frac{\Theta_k}{2\pi}\right)^2-2\frac{\Theta_k}{2\pi}n_k-n_k^2\right]\right\}\\
%    
    &=\frac{1}{2}\left\{d_{k0}^2+\frac{\lambda^2}{4}\left[\left(\frac{\Theta_0}{2\pi}\right)^2-\left(\frac{\Theta_k}{2\pi}\right)^2 \right.\right.\nonumber\\
    &\phantom{=}\;
   	\left.\left.+2\frac{\Theta_0}{2\pi}n_0-2\frac{\Theta_k}{2\pi}n_k+n_0^2-n_k^2\right]\right\}\\
%	
	&=\frac{1}{2}d_{k0}^2+\frac{\lambda^2}{8}\left[\frac{1}{(2\pi)^2}\left(\Theta_0^2-\Theta_k^2\right) \right.\nonumber\\
	&\phantom{=}\;
	\left. +\frac{1}{\pi}\left(\Theta_0n_0-\Theta_kn_k\right)+\left(n_0^2-n_k^2\right)\right]\label{c_k0_rearragend}
\end{align}
%
Führen wir nun:
\phantomeq{c_{k0}(\Theta_0, \Theta_k, n_0, n_k)}{a_{0k} := \frac{1}{2}d_{k0}^2\nonumber}
\phantomeq{c_{k0}(\Theta_0, \Theta_k, n_0, n_k)}{a_1 := \frac{\lambda^2}{8}\nonumber}
\phantomeq{c_{k0}(\Theta_0, \Theta_k, n_0, n_k)}{a_2 := a_1\frac{1}{\pi}\nonumber}
\phantomeq{c_{k0}(\Theta_0, \Theta_k, n_0, n_k)}{a_{3k0} := a_1\frac{1}{(2\pi)^2}(\Theta_0^2-\Theta_k^2)\nonumber}
%
in Gleichung~\eqref{c_k0_rearragend} ein, erhalten die finale Form der Gleichung:
\begin{equation}
c_{k0}(\Theta_0, \Theta_k, n_0, n_k) = a_{0k}+a_1(n_0^2-n_k^2)+a_2(\Theta_0n_0-\Theta_kn_k)-a_{3k0}\label{c_k0_final_form}   
\end{equation}
%
Die Einführung der Konstanten macht zum Einen die Gleichung übersichtlicher. Zum Anderen können so, mit Blick auf eine spätere Softwareimplementation, Rechenschritte gespart werden. Das sollte sich positiv auf den späteren Berechnungsaufwand auswirken.\\
%
Im Weiteren erkennt man durch scharfes hinsehen das in Gleichung~\eqref{c_k0_final_form}, für $\Theta_k=\text{const.}$ \& $\Theta_0=\text{const.}$ gilt. Das resultiert aus der Tatsache, dass . Es ermöglicht uns zu schreiben:
\begin{equation}
c_{k0}(\Theta_0, \Theta_k, n_0, n_k) = c_{k0}(n_0, n_k)
\end{equation}
%
Im engeren Sinne einer mathematischen Funktion sollten wir die Parameter alle als Argument aufnehmen. Diese Form soll darstellen, welche Größen von Interesse sind. Im späteren Gebrauch wird diese Gleichung in der Optimierung eingesetzt werden.
Für unser Gleichungssystem aus\eqref{eq:final_trilateration_model} ergibt sich:
\begin{equation}\label{eq:wavenumber_trilateration_model}
0=
\left(
	\begin{array}{ccc}
		x_k-x_0 & y_k-y_0 & z_k-z_0 
	\end{array}
\right)
\left(
   \begin{array}{c}
	   x-x_0\\
	   y-y_0\\
	   z-z_0
   \end{array}
\right)
-
\left(
	\begin{array}{c}
		c_{k0}(n_0, n_k)
	\end{array}
\right)
\end{equation}
%
Betrachten wir nun \eqref{eq:wavenumber_trilateration_model} und setzen $N'=4$, d.h. wir verwenden 4 Antennen. Wir beschreiben die Konfiguration wie folgt: Antenne 0 ist die Referenz-Antenne und Antenne 0-3 sind Messwertgeber für die Phaseninformation. 
%
\begin{equation}\label{eq:wavenumber_trilateration_model_explicit}
0=
\underbrace{\left(
	\begin{array}{ccc}
		x_1-x_0 & y_1-y_0 & z_1-z_0 \\
		x_2-x_0 & y_2-y_0 & z_2-z_0 \\
		x_3-x_0 & y_3-y_0 & z_3-z_0 
	\end{array}
\right)}_{\textbf{A}}
\underbrace{\left(
   \begin{array}{c}
	   x-x_0\\
	   y-y_0\\
	   z-z_0
   \end{array}
\right)}_{\textbf{x}}
-
\underbrace{\left(
	\begin{array}{c}
		c_{10}(n_0, n_1) \\
		c_{20}(n_0, n_2) \\
		c_{30}(n_0, n_3)
	\end{array}
\right)}_{\textbf{b}}
\end{equation}
%
\begin{equation}
\mathbf{b}=
\left(
	\begin{array}{c}
		a_{01}+a_1( n_0^2-n_1^2)+a_2(\Theta_0n_0-\Theta_1n_1)-a_{310} \\
		a_{02}+a_1(n_0^2-n_2^2)+a_2(\Theta_0n_0-\Theta_2n_2)-a_{320} \\
		a_{03}+a_1(n_0^2-n_3^2)+a_2(\Theta_0n_0-\Theta_3n_3)-a_{330}
	\end{array}
\right)
\end{equation}

\end{shaded}
%
Das Ergebnis ist ein um $\Theta$ und $n$ erweitertes Gleichungssystem. Zusätzlich enthält  es mehrere geometrische Konstanten ($a_{0k}, k=\{1,..,N-1\}$), mehrere Phasen-Konstanten ($a_{3k0}, k=\{1,..,N-1\}$), sowie zwei allgemeine ($a_1$ und $a_2$). Allgemeiner formuliert ergibt sich:
%
\begin{multline}\label{eq:final_equation}
0=
\left(
	\begin{array}{ccc}
		x_k-x_0 & y_k-y_0 & z_k-z_0 
	\end{array}
\right)
\left(
   \begin{array}{c}
	   x-x_0\\
	   y-y_0\\
	   z-z_0
   \end{array}
\right) \\
-
\left(
	\begin{array}{c}
		a_{0k}+a_1(n_0^2-n_k^2)+a_2(\Theta_0k_0-\Theta_kn_k)-a_{3k0}
	\end{array}
	\right)
\end{multline}
%
Aus Gleichung~\eqref{eq:final_equation} ist durch eine geeignete Wahl von $N'=\{4,..,N\}$ sofort ersichtlich wie viele Veränderliche sich für eine gewählte Konstellation an Antennen ergeben. Für $k$ gilt in diesem Fall $k=\{1,..,N'-1\}$.\\
%
Beispielsweise ergibt sich für das Modell aus Gleichung~\eqref{eq:final_equation} mit $N'=4$, insgesamt 7 Variablen ($\mathbf{x},n_0,n_1,n_2,n_3$) . Analog würde sich für ein Modell mit allen 8 Antennen, 11 Variablen ($\mathbf{x},n_0,..,n_7$) ergeben.
}
{
Abschließend soll das das bisher verwendete Modell umgeschrieben werden, damit die Allgemeingültigkeit darin enthalten ist.
\begin{align}
%
%\mathbf{0}&=\mathbf{A}\mathbf{x}-\mathbf{b}\\
%
\mathbf{A}&=
\left(
	\begin{array}{cccccc}
		x_k-x_0 & y_k-y_0 & z_k-z_0 & \sum_{i=1,j=0}^{k}(-a_1\delta_{ij}) &  -a_2\Theta_0 & \sum_{i=1,j=0}^{k}(a_2\Theta_k\delta_{ij})
	\end{array}
\right)\nonumber\\
%
\mathbf{x}&=
\left(
   \begin{array}{c}
	   x-x_0\\
	   y-y_0\\
	   z-z_0\\
	   n_0^2-n_k^2\\
	   n_0\\
	   n_k
   \end{array}
\right)\nonumber\\
%
\mathbf{b}&=
	\begin{array}{c}
		a_{0k}-a_{3kj} 
	\end{array}
	= c_{kj}'\nonumber
\end{align}
%
Dabei steht $\delta_{ij}$ für den bekannten Kronecker-Operator und bedeutet:
\begin{equation*}
\delta_{ij} = \begin{cases}1 ~\text{für}~ i=j\\ 0 ~\text{für}~ i\neq j\end{cases}
\end{equation*}
%
Im Expliziten sehen die Matrix $\mathbf{A}$ und der Vektor $\mathbf{b}$, für denn Fall $N'=3$ und $k=\{1,2,3\}$, wie folgt aus:
%
\begin{multline}
\mathbf{A}=\\
\left(
	\begin{array}{cccccccccc}
		x_1-x_0 & y_1-y_0 & z_1-z_0 & -a_1 & 0 & 0 & -a_2\Theta_0 & a_2\Theta_3 & 0 & 0 \\
		x_2-x_0 & y_2-y_0 & z_2-z_0 & 0 & -a_1 & 0 & -a_2\Theta_0& 0 & a_2\Theta_3 & 0 \\
		x_3-x_0 & y_3-y_0 & z_3-z_0 & 0 & 0 & -a_1 & -a_2\Theta_0& 0 & 0 & a_2\Theta_3
	\end{array}
\right) \nonumber
\end{multline}
%
\begin{equation}
\mathbf{x}=
\left(
	\begin{array}{c}
		x-x_0	\\
		y-y_0	\\
		z-z_0	\\
		n_0^2-n_1^2	\\
		(\dots)	\\
		n_0^2-n_3^2	\\
		n_0 \\
		n_1	\\
		(\dots)	\\
		n_3	
	\end{array}
\right)\nonumber
\end{equation}
%
\subsubsection{Bemerkungen - Finales Modell}
%
Das Ergebnis ist eine $3\times10$ Matrix und ein $1\times10$ Vaktor. Es ist möglich diesem Modell eine beliebige Anzahl an Antennen hinzuzufügen. Fügt man eine Antenne zur Berechnung hinzufügen würde sich die Matrix $\mathbf{A}$ um zwei Spalten und eine Zeile erweitern, der Vektor $\mathbf{x}$ analog um 2 Zeilen.

}

%

\section{Erweiterte Betrachtung der Kondition}
Die vorgestellte erweiterte Form des Modells erleichtert Implementation und Verifikation, da große Teile vorberechnet und in geeigneten Strukturen abgelegt werden können. Diese statischen Teile des Models sind in Gleichung~\ref{eq:block_matrix_form} ersichtlich. Es sind nun auch die gemessenen Phasenwerte Teil des Modells, genauer: der Matrix $\mathbf{A}$. Im Folgenden werden die Auswirkungen auf die Kondition der Matrix betrachtet, wenn man diese Phasendaten hinzurechnet. Weiterhin wird Untersucht inwieweit die Zerlegung in Blockmatrizen und die Untersuchung der Kondition dieser eine Abschätzung der vollständigen Konditionszahl im Allgemeinen darstellt. 
%
\begin{equation}
\label{eq:block_matrix_form}
\mathbf{A}=\bigg( \mathbf{Z}\quad \mathbf{P}\quad \mathbf{V}\bigg)
\end{equation}
%
Dabei ist:
\begin{equation}
\mathbf{Z} \in \mathbb{R}^{3x3} \quad \mathbf{P} \in \mathbb{R}^{3x3} \quad \mathbf{V}\in \mathbb{R}^{4x3}
\end{equation}
%
Die Matrizen $\mathbf{Z}$ und $\mathbf{P}$ sind statisch. Hingegen enthält die Matrix $\mathbf{V}$ die gemessenen Phasenwerte $\Theta_k$ der Antennen für diese Konfiguration. \\
%
Die Abbildung~\ref{fig:CondNumberAnalyze} zeigt die bereits angestellte Untersuchung zu dieser Überlegung. Abbildung~\ref{fig:AnalyzeOf3x3} stellt die Konditionszahl der rein geometrischen $3\times3$-Matrix dar. In der Abbildung~\ref{fig:AnalyzeOf10x3} sehen wir die Kondition der erweiterten Matrix. Neben der geometrischen sind auch die beiden anderen Blockmatrizen in diese Konditionsbetrachtung eingeflossen. Als zusätzliche Angabe wird ist sind die Skalierungsfaktoren angegeben. Legt man beide Grafiken übereinander erkennt man:
\begin{enumerate}
\item Geometrisch gut konditionierte Konfigurationen (linke Grafik), bleiben im erweiterten Modell (rechte Grafik) weiterhin gut konditioniert.
\item Die Konditionszahl der \textit{schlechteste} ist wesentlich kleiner (ca. Faktor $10$) als im rein geometrischen Modell
\end{enumerate}
%
\begin{figure}[h!]
         \centering
	     \caption[Ergebnisse der Konditionsanalyse alle Permutationen]{Analyse der Konditionszahlen aller möglichen Matrizen für den Messaufbau; Die Konditionszahl ist für jede mögliche Permutation an Messantennen für eine Referenzantenne angegeben}\label{fig:CondNumberAnalyze}
         \begin{subfigure}[t]{0.45\textwidth}
                 \centering
                 \includegraphics[width=\textwidth]{img/fenceModell3x3.png}
                 \caption{Konditionszahl der rein geometrischen $3\times3$ Matrix normiert auf den größten vorkommenden Wert ($=2149,16$). Auf den Achsen finden sich der Index der Referenzantenne sowie die Nummer der Permutation. Die z-Achse enthält die normierte Kondition}
                 \label{fig:AnalyzeOf3x3}
         \end{subfigure}
\qquad        
         \begin{subfigure}[t]{0.45\textwidth}
                 \centering
                 \includegraphics[width=\textwidth]{img/fenceModell9x3.png}
                 \caption{Konditionszahl der $10\times3$ Matrix normiert auf den größten vorkommenden Wert ($=257,13$); In dieser Konfiguration sind die Konstanten ($a_1$ \& $a_2$) sowie die variablen, gemessenen Phasen $\Theta_k$ enthalten}
                 \label{fig:AnalyzeOf10x3}
         \end{subfigure}
%
\end{figure}
%
Aus der Grafik lässt sich entnehmen, dass es für jede Referenzantenne aus der Geometrie alleine gute Konfigurationen existieren. Aus diesen Erkenntnissen kann in späteren Aufbauten, die Position der Antennen optimiert werden. Diese Verfahren wird in Abschnitt~\ref{sec:Calibration_Optimaztion} weiter beschrieben. Die Grafik lässt erkennen, dass eine Konfiguration die ohne Phasendaten eine gut Kondition aufwies, eine ähnliche Kondition behält wenn diese Daten in der Modell einfließen.
%
%- Section 2.3 --------------------------------------------------------------
\subsection{Weitere Anwendung der Konditionszahl}
Weitere Anwendungen, die sich aus der Konditionszahl der Matrix ableiten, sind denkbar. Für die FPGA-Software ist, parallel zu diesem Projekt, eine intelligente Umschaltung der Antennen in der Planung. Die Kondition der geometrische Matrix verändert sich nach dem Kalibrieren nicht mehr. Dadurch und durch die oben beschriebenen Überlegungen kann statisch eine Abschätzung für die Konditionszahl, von zwei der drei Blockmatrizen, im Vorfeld erstellt werden. Die Konditionszahl dient zum Steuern der Umschaltung. Ordnet man die möglichen Konfiguration anhand ihrer Konditionszahl (niedrigste zuerst) in einer statischen Liste an so kann im FPGA eine einfache, schlaue Umschaltung implementiert werden. Diese würde immer dafür sorgen, dass Messdaten von einer Konfiguration bevorzugt werden, die eine niedrige Konditionszahl hat und somit relativ sicher zu einer guten Lösung führen. Diese Überlegungen werden im Rahmen dieser Arbeit nicht näher beschrieben.\\
Eine Weitere Anwendung ergibt sich für die Kalibrierung. Der Aufbau der Antennen kann unter Berücksichtigung der Kondition optimiert werden. Ziel der Optimierung wäre es durch eine geeignete Positionierung der Antennen, die Anzahl der Antennenpermutationen mit kleiner Konditionszahl zu maximieren.
%
\section{Realisierung der Kalibrierung}
\label{sec:calibration}
In diesem Abschnitt wir die Implementierung der Kalibrierung des Messaufbaus und die Ergebnisse beschrieben. Es werden zwei unterschiedliche Berechnungsverfahren vorgestellt. Zuerst die Berechnung über das SVD-Verfahren, danach durch das CMA-ES-Verfahren. Es ist sinnvoll zu erwarten, dass beide Ergebnisse die gleichen Koordinaten liefern.
\subsection{Implementation}
%
Der Ablauf der Kalibrierung ist in Abbildung~\ref{fig:calibration_flowchart} in Form eine Ablaufdiagramms dargestellt. Beschrieben werden die wesentlichen Schritte. Es sind sowohl Interaktion mit der Person enthalten, die die Kalibrierung durchführt, als auch die Schritte, die von den beteiligten Softwarekomponenten ausgeführt werden enthalten. Es wurden im Rahmen der Arbeit zwei unterschiedliche Wege implementiert, um ein Ergebnis für die Kalibrierung zu berechnen. Diese Wege werden im Folgenden vorgestellt und die Ergebnisse miteinander verglichen. Die Präsentation der Resultate wird vor allem dazu verwendet werden, die gewählte Form der Diagramme zu erläutern. Diese werden in den Ergebnissen des komplexeren Modells ebenfalls verwendet.
%
\subsubsection{SVD}
%
Das unter \ref{sec:svd} vorgestellte Verfahren der Singular-Value-Decomposition kann dazu verwendet werden eine Lösung eines linearen Gleichungssystems zu berechnen. Das Modell, das zur Kalibrierung verwendet wird, ist ein Gleichungssystem der Form $\mathbf{A}\mathbf{x}=\mathbf{b}$ und hat drei Gleichungen mit drei Unbekannten. Daher kann sofort eine Lösung mit dem Verfahren hergeleitet werden. Das Ergebnis eines Messaufbaus mit 3 Antennen ist in Tabelle~\ref{tab:FinalCoords} und in Abbildung~\ref{fig:3dplot_coordinates} gezeigt. Die Implementation des Algorithmus stammt aus \cite{press2007numerical} und wurde für diese Arbeit angeschafft.
%
\subsubsection{CMA-ES}
Da in dieser Arbeit der CMA-ES-Algorithmus eingesetzt wird und damit ohnehin eine Implementation vorgenommen wird, kann die Kalibrierung dazu verwendet werden die Umsetzung des Algorithmus zu verifizieren. Dazu vergleichen wir die Lösung der SVD-Methode mit der des CMA-ES.\\
Das über den evolutionären Algorithmus gefundene Ergebnis gleicht dem des SVD-Verfahrens (siehe Tabelle~\ref{tab:FinalCoords}). Der SVD-Algorithmus ist um ein vielfaches effizienter\footnote{d.h. weniger Rechenzeit ist erforderlich} beim Lösen des Gleichungssystems. Die Gründe, warum an dieser Stelle die Berechnung mit dem evolutionären Verfahren durchgeführt und hier dargestellt wird sind folgende:
%
\begin{enumerate}
 \item Die Komplexität ist gering, daher kann der Ablauf des evolutionären Verfahrens besser dargestellt und verstanden werden
 \item Der Vergleich der beiden Ergebnisse ermöglicht die Verifizierung der Implementation beider Verfahren.
\end{enumerate}
%
Dem ersten Punkt kommt im Rahmen dieser Arbeit eine besondere Stellung zu. Es ist einfacher anhand dieses übersichtlichen Problems (mit nur drei Unbekannten) den Ablauf des Algorithmus sowie die Visualisierung der Ergebnisse zu veranschaulichen. Die verwendete Darstellung gleicht der, die später bei der Präsentation und Beurteilung der komplexeren Modell verwendet wird.
%
%-------------------------------------------------------------------------
\begin{figure}[H]
	\begin{center}
		\caption[Ablauf der Kalibierung]{Ablauf der Kalibierung}
		\label{fig:calibration_flowchart}
		\vspace{0.5cm}
		\begin{tikzpicture}[auto]
		\scriptsize
			\tikzstyle{decision} = [diamond, draw=black, thick, fill=black!20, text width=5em, text badly centered, inner sep=1pt]
%			
			\tikzstyle{block} = [rectangle, draw=black, thick, fill=gray!20, text width=15em, text centered, rounded corners, minimum height=4em]
%	
			\tikzstyle{line} = [draw, thick, -latex',shorten >=1pt];
			\tikzstyle{commentline} = [draw, dashed, green!50,-latex',shorten >=1pt];
%	
			\tikzstyle{cloud} = [ dotted, draw=green!50, thick, ellipse,,fill=green!20, minimum height=2em, text width= 10em, text badly centered];
%	
			\matrix [column sep=5mm,row sep=7mm]
			{
				% row 1
				& \node [block] (start) { Start }; & \\
				% row 2
				& \node [block] (setup) {Aufstellen des Kalibrierstücks}; & 
					\node [cloud] (comment1) {Gezeigt in Abbildung \ref{fig:calib_piece}}; \\
				% row 4
				& \node [block] (measure) {Vermessen der Entfernungen zu den Antennen}; & 
					\node [cloud] (comment2) {z.B. mit Laser-Entfernungsmesser, gezeigt in Abbildung \ref{fig:laser_meter}}; \\
				% row 5
				&\node [block] (writefile) {Eintragen der Vermessenen Werte in Mashinenlesbare Datei}; &\\
				% row 6
				\node (temp){}; &\node [block] (startsw) {Starte die Kalibiersoftware}; &\\
				% row 7
				&\node [block] (viewresults) {Speichern der berechneten Werte}; &\\
				% row 8				
				& \node [decision] (decide) {$\Delta \geq \Delta_{max}$}; & 
					\node [cloud] (criteria) {Ergebnisse haben eine geringe Abweichung};\\
				% row 9
				& \node [block] (stop) {Ende}; & \\
			};
			
%
%			Draw the arrows
%
			\path (decide) -| node [near start] {Nein} (temp);
			\tikzstyle{every path}=[line]
			\path (start) -- (setup);
			\path (setup) -- (measure);
			\path (measure) -- (writefile);
			\path (writefile) -- (startsw);
			\path (startsw) -- (viewresults);
			\path (viewresults) -- (decide);
			\path (decide)	-| +(-3,0)  |- (measure);
			\path (decide) -- node [midway] {Ja} (stop);
			
%			
%			draw the comments 
%
			\tikzstyle{every path}=[commentline]
			\path (criteria) -- (decide);
			\path (comment1) -- (setup);
			\path (comment2) -- (measure);
				
		\end{tikzpicture}
	\end{center}
\end{figure}
%
%-------------------------------------------------------------------------
\begin{figure}[ht!]
	\begin{center}
		\caption[Ablauf der libCalibration]{Ablauf der libCalibration}
		\label{fig:calibration_flowchart_}
		\vspace{1cm}
		\begin{tikzpicture}[auto]
		\scriptsize
			\tikzstyle{decision} = [diamond, draw=black, thick, fill=black!20, text width=5em, text badly centered, inner sep=1pt]
%			
			\tikzstyle{block} = [rectangle, draw=black, thick, fill=gray!20, text width=15em, text centered, rounded corners, minimum height=4em]
%	
			\tikzstyle{line} = [draw, thick, -latex',shorten >=1pt];
			\tikzstyle{commentline} = [draw, dashed, green!50,-latex',shorten >=1pt];
%	
			\tikzstyle{cloud} = [ dotted, draw=green!50, thick, ellipse,,fill=green!20, minimum height=2em, text width= 10em, text badly centered];
%	
			\matrix [column sep=5mm,row sep=7mm]
			{
				% row 1
				& \node [block] (start) { Start }; & \\
				% row 2
				& \node [block] (read) {Lese gemessene Werte aus CSV-Datei}; &  \\
				% row 4
				& \node [block] (read2) {Lade Geometrie des Kalibrierstücks}; & 
				 \node [cloud] (comment1) {Diagonalmatrix};\\
				% row 5
				&\node [block] (calc1) { Berechne die Matrizen für jede Antenne $\mathbf{A_k}\qquad,1 \leq k \leq |N|$ }; &  \\
				% row 6
				&\node [block] (calc2) { Berechne den Distanzvektor $\mathbf{b}$ }; &  \\			 
				% row 7
				&\node [block] (run) {Berechne die Positionen}; &
 				 \node [cloud] (comment2) {mittels SVD};\\
				% row 8
				&\node [block] (write) {Schreibe Werte}; &\\
				% row 9				
				& \node [block] (stop) {Ende}; & \\
			};
			
%
%			Draw the arrows
%
			\tikzstyle{every path}=[line]
			\path (start) -- (read);
			\path (read) -- (read2);
			\path (read2) -- (calc1);
			\path (calc1) -- (calc2);
			\path (calc2) -- (run);
			\path (run) -- (write);
			\path (write) -- (stop);
%			
%			draw the comments 
%
			\tikzstyle{every path}=[commentline]
			\path (comment1) -- (read2);
			\path (comment2) -- (run);
				
		\end{tikzpicture}
	\end{center}
\end{figure}
%\newpage
%
 
%
%\subsection{Konkurierende Modelle}
% SuWi - Zeug 
%\lipsum[1-1]
%
\section{Betrachtung der Komplexität}
\label{sec:Komplexity2}
Im Folgenden wird eine Betrachtung der Komplexität des in Abschnitt~\ref{sec:model_developement} entwickelten Modells präsentiert. Diese Betrachtung ist wichtig für die Parametrisierung des Optimierungsverfahrens. Es wird eine Visualisierung des Fitness-Raums gezeigt und mit Benchmark-Funktionen verglichen.

\section{Software}
\label{sec:sw}
\lipsum[1-3]

\section{Hardware}
\label{sec:hw}
\lipsum[1-3]

