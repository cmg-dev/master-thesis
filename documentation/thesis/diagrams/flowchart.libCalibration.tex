\begin{figure}[ht!]
	\begin{center}
		\caption[Ablauf der libCalibration]{Ablauf der libCalibration}
		\label{fig:calibration_flowchart_}
		\vspace{1cm}
		\begin{tikzpicture}[auto]
		\scriptsize
			\tikzstyle{decision} = [diamond, draw=black, thick, fill=black!20, text width=5em, text badly centered, inner sep=1pt]
%			
			\tikzstyle{block} = [rectangle, draw=black, thick, fill=gray!20, text width=15em, text centered, rounded corners, minimum height=4em]
%	
			\tikzstyle{line} = [draw, thick, -latex',shorten >=1pt];
			\tikzstyle{commentline} = [draw, dashed, gray!50,-latex',shorten >=1pt];
%	
			\tikzstyle{cloud} = [ dotted, draw=gray!50, thick, ellipse,,fill=gray!5, minimum height=2em, text width= 10em, text badly centered];
%			\tikzstyle{cloud} = [ dotted, draw=green!50, thick, ellipse,,fill=green!20, minimum height=2em, text width= 10em, text badly centered];
%	
			\matrix [column sep=5mm,row sep=7mm]
			{
				% row 1
				& \node [block] (start) { Start }; & \\
				% row 2
				& \node [block] (read) {Lese gemessene Werte aus CSV-Datei}; &  \\
				% row 4
				& \node [block] (read2) {Lade Geometrie des Kalibrierstücks}; & 
				 \node [cloud] (comment1) {Diagonalmatrix};\\
				% row 5
				&\node [block] (calc1) { Berechne die Matrizen für jede Antenne $\mathbf{A_k}\qquad,1 \leq k \leq |N|$ }; &  \\
				% row 6
				&\node [block] (calc2) { Berechne den Distanzvektor $\mathbf{b}$ }; &  \\			 
				% row 7
				&\node [block] (run) {Berechne die Positionen}; &
 				 \node [cloud] (comment2) {mittels SVD};\\
				% row 8
				&\node [block] (write) {Schreibe Werte}; &\\
				% row 9				
				& \node [block] (stop) {Ende}; & \\
			};
			
%
%			Draw the arrows
%
			\tikzstyle{every path}=[line]
			\path (start) -- (read);
			\path (read) -- (read2);
			\path (read2) -- (calc1);
			\path (calc1) -- (calc2);
			\path (calc2) -- (run);
			\path (run) -- (write);
			\path (write) -- (stop);
%			
%			draw the comments 
%
			\tikzstyle{every path}=[commentline]
			\path (comment1) -- (read2);
			\path (comment2) -- (run);
				
		\end{tikzpicture}
	\end{center}
\end{figure}
%\newpage