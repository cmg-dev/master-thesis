%
%
%
\begin{appendix}

%----------------------------------------------------------------------------
%----------------------------------------------------------------------------
%\newpage
%
%\begin{center}
%	\huge{Anhänge}
%\end{center}

\normalsize
%
%----------------------------------------------------------------------------
%----------------------------------------------------------------------------
\chapter{Abbildungen}
\section{Messaufbauten}
\begin{figure}[h!]
 \centering
         \includegraphics[width=\textwidth]{img/Calibration_Plate1.png}
         \caption[PRPS-Kalibiersystem]{Reproduzierbare Aufstellung für die Messungen. Im Bild zu erkennen ist ein Taghalter mit drei Tags und eine Grundplatte, die eine Fläche von $1\times1$ Meter abdeckt. Dabei beträgt die Distanz zwischen den einzelnen Positionen $10$~cm, sodass sich insgesamt $100$ Messpunkte anfahren lassen. Die Postionen sind aufsteigend nach Richtung der Ebenen benannt. Der \textit{Ursprung} befindet sich in der Linken unteren Ecke (vom Bild nicht erfasst). Der Aufsteller mit den Tags befindet sich demnach an Postion $(3,6)$.}
         \label{fig:Spider1}
\end{figure}
\newpage
%
%----------------------------------------------------------------------------
%----------------------------------------------------------------------------
\begin{figure}[h!]
 \centering
         \includegraphics[width=\textwidth]{img/4AntennaSetup.png}
         \caption[Übersicht Kalibrieraufbau]{Messaufbau mit \textit{Blick} auf die reproduzierbare Aufstellung. Antenne~1 ist die oben gelegene Antenne. Im Uhrzeigersinn ordnen sich die restlichen Antennen an. Zur Orientierung: Die reproduzierbare Aufstellung befindet sich in Richtung des Betrachters, von dem Aufbau aus gesehen.}
         \label{fig:Spider_setup1}
\end{figure}
\newpage
%
%----------------------------------------------------------------------------
%----------------------------------------------------------------------------
\chapter{Quellcodeauszüge}
\section{ObjectiveFunktion - Evolutionary Calibration Header}

\lstset{
%	caption= Modelldeklaration für Shark,
	basicstyle=\scriptsize,
	language=C++,
	numbers=left,
	breaklines=true,
	%		frameround=fttt,
	frame=tbrl,
	breakatwhitespace=false
	breaklines=true,  
	xleftmargin=1cm,
	tabsize=2,
	showstringspaces=false}

\lstinputlisting[title=Vollständiger Quellcode der EvolutionaryCalibration.h]{src/EvolutionaryCalibration.h}
\label{app:EvolutionaryCalibration1}
\newpage

\section{ObjectiveFunktion - Evolutionary Calibration Source}
\lstinputlisting[title=Vollständiger Quellcode der EvolutionaryCalibration.cpp]{src/EvolutionaryCalibration.cpp}
\label{app:EvolutionaryCalibration2}
\newpage
%
%----------------------------------------------------------------------------
%----------------------------------------------------------------------------
\chapter{Gnuplot Skripte}
\section{Boxplot}
\tiny
\lstinputlisting[
		title=Gnuplot Boxplot-Skript,
		caption=Gnuplot Boxplot-Skript,
		language=Gnuplot,
		numbers=left,
%		frameround=fttt,
		frame=trbl,
		breakatwhitespace=false,         % sets if automatic breaks should only happen at whitespace
   	    breaklines=true,  
		xleftmargin=1cm,
		showstringspaces=false]{../../dev/src/c-cpp/AntConfApp/build/Debug/test/output/mkII/plot/kondensierte_boxen.gp}
\label{append_Script_Box-plot}
\newpage
%
%----------------------------------------------------------------------------
%----------------------------------------------------------------------------
\section{Lineplot}
%\begin{footnote}
\tiny 
%
% listings print source code

% define colors for source code list
\definecolor{colKeys}{rgb}{0,0,1}
\definecolor{colIdentifier}{rgb}{0,0,0}
\definecolor{colComments}{rgb}{0,1,0.3}
\definecolor{colString}{rgb}{0,0.5,0}

\definecolor{dkgreen}{rgb}{0,0.6,0}
\definecolor{gray}{rgb}{0.5,0.5,0.5}

\lstset{language=Matlab,
   keywords={break,case,catch,continue,else,elseif,end,for,function,
   global,if,otherwise,persistent,return,switch,try,while,ones,zeros},
   float=hbp,
   basicstyle=\ttfamily\small,
   identifierstyle=\color{colIdentifier},
   keywordstyle=\color{blue},
   commentstyle=\color{dkgreen},
   stringstyle=\color{red},
   columns=flexible,
   tabsize=2,
   frame=single,
   numbers=left,
   extendedchars=true,
   showspaces=false,
   numberstyle=\tiny\color{gray},
   stepnumber=1,
   numbersep=10pt,
   showspaces=false,
   showstringspaces=false,
   breakautoindent=true}

\lstinputlisting[
		title=Gnuplot Lineplot-Skript,
		caption=Gnuplot Lineplot-Skript,
		language=Gnuplot,
		numbers=left,
%		frameround=fttt,
		frame=trbl,
		breakatwhitespace=false,         % sets if automatic breaks should only happen at whitespace
   	    breaklines=true,  
		xleftmargin=1cm,
		showstringspaces=false]{../../dev/src/c-cpp/AntConfApp/build/Debug/test/output/mkII/plot/kondensierte_linien.gp}
\label{append_Script_Line-plot}
\newpage
%----------------------------------------------------------------------------
%----------------------------------------------------------------------------
\section{Scatterplot}
%\begin{footnote}
%
% listings print source code

% define colors for source code list
\definecolor{colKeys}{rgb}{0,0,1}
\definecolor{colIdentifier}{rgb}{0,0,0}
\definecolor{colComments}{rgb}{0,1,0.3}
\definecolor{colString}{rgb}{0,0.5,0}

\definecolor{dkgreen}{rgb}{0,0.6,0}
\definecolor{gray}{rgb}{0.5,0.5,0.5}

\lstset{language=Matlab,
   keywords={break,case,catch,continue,else,elseif,end,for,function,
   global,if,otherwise,persistent,return,switch,try,while,ones,zeros},
   float=hbp,
   basicstyle=\ttfamily\small,
   identifierstyle=\color{colIdentifier},
   keywordstyle=\color{blue},
   commentstyle=\color{dkgreen},
   stringstyle=\color{red},
   columns=flexible,
   tabsize=2,
   frame=single,
   numbers=left,
   extendedchars=true,
   showspaces=false,
   numberstyle=\tiny\color{gray},
   stepnumber=1,
   numbersep=10pt,
   showspaces=false,
   showstringspaces=false,
   breakautoindent=true}

\tiny
\lstinputlisting[
		title=Gnuplot Scatterplot-Skript,
		caption=Gnuplot Scatterplot-Skript,
		language=Gnuplot,
		numbers=left,
%		frameround=fttt,
		frame=trbl,
		breakatwhitespace=false,         % sets if automatic breaks should only happen at whitespace
   	    breaklines=true,  
		xleftmargin=1cm,
		showstringspaces=false]{../../dev/src/c-cpp/AntConfApp/build/Debug/test/output/mkII/plot/scatter.gp}
\label{append_Script_Scatter-plot}
\newpage
%
%----------------------------------------------------------------------------
%----------------------------------------------------------------------------
\chapter{Fitness Plots}
\label{app:fitness:plots1}
\newpage
%\begin{\landscape}
\begin{figure}[!ht]
	\centering
	\begin{subfigure}[t]{0.3\textwidth}
	     \centering
	     \includegraphics[width=\textwidth]{img/fitness/xy/a0.png}
	%             \caption{Statistisch verteilte Endwerte für die Koordinaten der Kalibrierung.}
	%             \label{fig:abortedFinal_Calibration_Ant0_ES-boxes}
	\end{subfigure}
	\begin{subfigure}[t]{0.3\textwidth}
		\centering
	     \includegraphics[width=\textwidth]{img/fitness/xz/a0.png}
	%			\caption{x-z-Ebene, vergrößert}
	%			\label{fig:abortedFinal_Calibration_Ant0_ES-boxes}
	\end{subfigure}
	\begin{subfigure}[t]{0.3\textwidth}
			\centering
	   \includegraphics[width=\textwidth]{img/fitness/yz/a0.png}
	%			\caption{x-z-Ebene, vergrößert}
	%			\label{fig:abortedFinal_Calibration_Ant0_ES-boxes}
	\end{subfigure}
\\
	\centering
	\begin{subfigure}[t]{0.3\textwidth}
	     \centering
	     \includegraphics[width=\textwidth]{img/fitness/xy/a1.png}
	%             \caption{Statistisch verteilte Endwerte für die Koordinaten der Kalibrierung.}
	%             \label{fig:abortedFinal_Calibration_Ant0_ES-boxes}
	\end{subfigure}
	\begin{subfigure}[t]{0.3\textwidth}
		\centering
	     \includegraphics[width=\textwidth]{img/fitness/xz/a1.png}
	%			\caption{x-z-Ebene, vergrößert}
	%			\label{fig:abortedFinal_Calibration_Ant0_ES-boxes}
	\end{subfigure}
	\begin{subfigure}[t]{0.3\textwidth}
			\centering
	   \includegraphics[width=\textwidth]{img/fitness/yz/a1.png}
	%			\caption{x-z-Ebene, vergrößert}
	%			\label{fig:abortedFinal_Calibration_Ant0_ES-boxes}
	\end{subfigure}
\\
	\centering
	\begin{subfigure}[t]{0.3\textwidth}
	     \centering
	     \includegraphics[width=\textwidth]{img/fitness/xy/a2.png}
	%             \caption{Statistisch verteilte Endwerte für die Koordinaten der Kalibrierung.}
	%             \label{fig:abortedFinal_Calibration_Ant0_ES-boxes}
	\end{subfigure}
	\begin{subfigure}[t]{0.3\textwidth}
		\centering
	     \includegraphics[width=\textwidth]{img/fitness/xz/a2.png}
	%			\caption{x-z-Ebene, vergrößert}
	%			\label{fig:abortedFinal_Calibration_Ant0_ES-boxes}
	\end{subfigure}
	\begin{subfigure}[t]{0.3\textwidth}
			\centering
	   \includegraphics[width=\textwidth]{img/fitness/yz/a2.png}
	%			\caption{x-z-Ebene, vergrößert}
	%			\label{fig:abortedFinal_Calibration_Ant0_ES-boxes}
	\end{subfigure}
\\
	\centering
	\begin{subfigure}[t]{0.3\textwidth}
	     \centering
	     \includegraphics[width=\textwidth]{img/fitness/xy/a3.png}
	%             \caption{Statistisch verteilte Endwerte für die Koordinaten der Kalibrierung.}
	%             \label{fig:abortedFinal_Calibration_Ant0_ES-boxes}
	\end{subfigure}
	\begin{subfigure}[t]{0.3\textwidth}
		\centering
	     \includegraphics[width=\textwidth]{img/fitness/xz/a3.png}
	%			\caption{x-z-Ebene, vergrößert}
	%			\label{fig:abortedFinal_Calibration_Ant0_ES-boxes}
	\end{subfigure}
	\begin{subfigure}[t]{0.3\textwidth}
			\centering
	   \includegraphics[width=\textwidth]{img/fitness/yz/a3.png}
	%			\caption{x-z-Ebene, vergrößert}
	%			\label{fig:abortedFinal_Calibration_Ant0_ES-boxes}
	\end{subfigure}
\end{figure}
%
\newpage
\begin{figure}[!ht]
	\centering
	\begin{subfigure}[t]{0.3\textwidth}
	     \centering
	     \includegraphics[width=\textwidth]{img/fitness/xy/a4.png}
	%             \caption{Statistisch verteilte Endwerte für die Koordinaten der Kalibrierung.}
	%             \label{fig:abortedFinal_Calibration_Ant0_ES-boxes}
	\end{subfigure}
	\begin{subfigure}[t]{0.3\textwidth}
		\centering
	     \includegraphics[width=\textwidth]{img/fitness/xz/a4.png}
	%			\caption{x-z-Ebene, vergrößert}
	%			\label{fig:abortedFinal_Calibration_Ant0_ES-boxes}
	\end{subfigure}
	\begin{subfigure}[t]{0.3\textwidth}
			\centering
	   \includegraphics[width=\textwidth]{img/fitness/yz/a4.png}
	%			\caption{x-z-Ebene, vergrößert}
	%			\label{fig:abortedFinal_Calibration_Ant0_ES-boxes}
	\end{subfigure}
\\
	\centering
	\begin{subfigure}[t]{0.3\textwidth}
	     \centering
	     \includegraphics[width=\textwidth]{img/fitness/xy/a5.png}
	%             \caption{Statistisch verteilte Endwerte für die Koordinaten der Kalibrierung.}
	%             \label{fig:abortedFinal_Calibration_Ant0_ES-boxes}
	\end{subfigure}
	\begin{subfigure}[t]{0.3\textwidth}
		\centering
	     \includegraphics[width=\textwidth]{img/fitness/xz/a5.png}
	%			\caption{x-z-Ebene, vergrößert}
	%			\label{fig:abortedFinal_Calibration_Ant0_ES-boxes}
	\end{subfigure}
	\begin{subfigure}[t]{0.3\textwidth}
			\centering
	   \includegraphics[width=\textwidth]{img/fitness/yz/a5.png}
	%			\caption{x-z-Ebene, vergrößert}
	%			\label{fig:abortedFinal_Calibration_Ant0_ES-boxes}
	\end{subfigure}
\\
	\centering
	\begin{subfigure}[t]{0.3\textwidth}
	     \centering
	     \includegraphics[width=\textwidth]{img/fitness/xy/a6.png}
	%             \caption{Statistisch verteilte Endwerte für die Koordinaten der Kalibrierung.}
	%             \label{fig:abortedFinal_Calibration_Ant0_ES-boxes}
	\end{subfigure}
	\begin{subfigure}[t]{0.3\textwidth}
		\centering
	     \includegraphics[width=\textwidth]{img/fitness/xz/a6.png}
	%			\caption{x-z-Ebene, vergrößert}
	%			\label{fig:abortedFinal_Calibration_Ant0_ES-boxes}
	\end{subfigure}
	\begin{subfigure}[t]{0.3\textwidth}
			\centering
	   \includegraphics[width=\textwidth]{img/fitness/yz/a6.png}
	%			\caption{x-z-Ebene, vergrößert}
	%			\label{fig:abortedFinal_Calibration_Ant0_ES-boxes}
	\end{subfigure}
\\
	\centering
	\begin{subfigure}[t]{0.3\textwidth}
	     \centering
	     \includegraphics[width=\textwidth]{img/fitness/xy/a7.png}
	%             \caption{Statistisch verteilte Endwerte für die Koordinaten der Kalibrierung.}
	%             \label{fig:abortedFinal_Calibration_Ant0_ES-boxes}
	\end{subfigure}
	\begin{subfigure}[t]{0.3\textwidth}
		\centering
	     \includegraphics[width=\textwidth]{img/fitness/xz/a7.png}
	%			\caption{x-z-Ebene, vergrößert}
	%			\label{fig:abortedFinal_Calibration_Ant0_ES-boxes}
	\end{subfigure}
	\begin{subfigure}[t]{0.3\textwidth}
			\centering
	   \includegraphics[width=\textwidth]{img/fitness/yz/a7.png}
	%			\caption{x-z-Ebene, vergrößert}
	%			\label{fig:abortedFinal_Calibration_Ant0_ES-boxes}
	\end{subfigure}      
\end{figure}
\newpage
%
%
%----------------------------------------------------------------------------
%----------------------------------------------------------------------------
%\section{Filter Entwurf - Ergebnisse}
%\begin{landscape}
%\label{FirFilterResult}
%\begin{figure} [h]
         \centering
         \caption{ lorem ipsum }
         \label{fig:1}
         \centering
         \includegraphics[width=.8\textwidth]{common/img/AmpGefiltert_small.png}

\end{figure}
%---------------------------------------------------------------------------------------
\vspace{.5cm}
%---------------------------------------------------------------------------------------
\begin{figure} [h]
         \centering
         \caption{ Spektrum des Messsignals, vor und nach der Filterung  }
         \label{fig:2}
	     \centering
	     \includegraphics[width=.6\textwidth]{common/img/SpektrumAmp.PNG} \\
\vspace{.2cm}
Die Grafik zeigt das Spektrum des Messsignals der Amplitude. Im linken Bild ist das ungefilterte Signal und im Rechten das gefilterte.
%
\end{figure}
%---------------------------------------------------------------------------------------
\vspace{.5cm}
%---------------------------------------------------------------------------------------
\begin{figure} [h]
         \centering
         \caption{ Frequenzgänge der entworfenen Filter. Beide ähneln sich in den Parametern, verfügen jedoch über etwas unterschiedliche Eckfrequenzen. Als Entwurfsmethode wurde die sog. "Least-squares"-Methode verwendet. Diese Methode liefert gute Ergebnisse im Hinblick auf aöglichst kleine Sidelobes und eine geringe Anzahl an Taps. }
         \label{fig:3}
%         
         \begin{subfigure}[t]{0.5\textwidth}
                 \centering
                 \includegraphics[width=\textwidth]{common/img/filter.png}
                 \vspace{.1cm}
                 \caption{Erstes Filter mit den Parametern wpass~=~0.1 und wstop~=~0.15. Das Ergebnis ist ein schmalbandigeres Filter. }
                 \label{fig:Filter1_A}\textit{}
         \end{subfigure}
%         
\qquad
         \begin{subfigure}[t]{0.5\textwidth}
                 \centering
                 \includegraphics[width=\textwidth]{common/img/filter2.png}
                 \vspace{.1cm}
                 \caption{ Zweites Filter mit den Parametern wpass~=~0.1 und wstop~=~0.2. Der Durchlas bereich ist etwas breit, dafür sind die Sidelobes besser gedämpft }
                 \label{fig:Filter2_B}
         \end{subfigure}
%
\end{figure}
%---------------------------------------------------------------------------------------
%\end{landscape}
%
%----------------------------------------------------------------------------
%----------------------------------------------------------------------------
%\newpage
%\begin{landscape}
%	\section{Projektlaufplan KW 31}
%	\label{sec:projectplan}
%	\scalebox{.75}{
%		\begin{ganttchart}[vgrid={draw=none,*1{gray, dashed}},
				hgrid=true,
				today=24,
				title height=1,
				y unit title=0.6cm,
				y unit chart=0.8cm,
				group right shift=0,
				group top shift=.3,
				group height=.3,
				milestone width=.8,
				group peaks={}{}{.2},
				incomplete/.style={fill=black!15}, %
				bar/.style={fill=white}, %
				today label={Heute},
				today rule/.style={dashed, thick}]{44}


\gantttitle{\textbf{2013}}{44} \\
\gantttitlelist{16,...,37}{2} \\
%-------------------------------------------------------------
\ganttgroup{Projekt Evaluation}{3}{14} \\
\ganttbar[progress=100, progress label font=\small\color{black!75},
	progress label anchor/.style={right=4pt}]{Installation der Umgebungen}{3}{6} \\
	
\ganttbar[progress=100, progress label font=\small\color{black!75},
	progress label anchor/.style={right=4pt},
	bar label font=\normalsize\color{black},
	name=rech]{Recherche}{3}{7} \\
	
\ganttmilestone[name=ms1]{Vorstellung der Ergebnisse}{7} \\
	
\ganttbar[progress=90, progress label font=\small\color{black!75},
	progress label anchor/.style={right=4pt},
	bar label font=\normalsize\color{black},
	name=pflichten]
	{Pflichtenheft}{5}{8} \\
	
\ganttmilestone[name=ms2]{Pflichtenheft fertig}{8} \\

\ganttbar[progress=100, progress label font=\small\color{black!75},
	progress label anchor/.style={right=4pt},
	bar label font=\normalsize\color{black},
	name=bNumVerf]
	{Einarbeitung num. Verfahren}{5}{16} \\

\ganttbar[progress=95, progress label font=\small\color{black!75},
	progress label anchor/.style={right=34pt},
	bar label font=\normalsize\color{black},
	name=bCMAES]
	{speziell CMA-ES}{7}{10} \\

\ganttmilestone[name=ms3]{Beurteilung num. Verfahren}{16} \\

\ganttlinkedbar[progress=100, progress label font=\small\color{black!75},
	progress label anchor/.style={right=34pt},
	bar label font=\normalsize\color{black}]
	{Shark Einarbeitung}{17}{18} \\

\ganttlinkedmilestone[name=ms7]{Abschluss Evaluation}{18} \\
	
%-------------------------------------------------------------
\ganttgroup{Erstellung Prototyp}{15}{26} \\
\ganttgroup{(optional)}{15}{18} \\
\ganttbar[progress=25, progress label font=\small\color{black!75},
	progress label anchor/.style={right=4pt},
	bar label font=\normalsize\color{black}]
	{(Entwurf digi. Filter)}{15}{15} \\

\ganttlinkedbar[progress=10, progress label font=\small\color{black!75},
	progress label anchor/.style={right=4pt},
	bar label font=\normalsize\color{black},
	name=bImpFPGA]
	{(Implementation FPGA)}{16}{18} \\

\ganttmilestone[name=ms4]{(Verifikation dig. Filter)}{18} \\
	
\ganttbar[progress=90, progress label font=\small\color{black!75},
	progress label anchor/.style={right=4pt},
	bar label font=\normalsize\color{black},
	name=bImplAlgo]
	{Implementation Algorithmus}{15}{26} \\

\ganttlinkedmilestone[name=ms5]{Implementation Done}{26} \\

%-------------------------------------------------------------
\ganttgroup{Verifikation}{27}{34} \\
\ganttbar[progress=10, progress label font=\small\color{black!75},
	progress label anchor/.style={right=4pt},
	bar label font=\normalsize\color{black},
	name=bVerf]
	{Durchf\"uhrung Verifikation}{27}{34} \\

\ganttlinkedmilestone[name=ms6]{Verifikation Done}{34} \\

%-------------------------------------------------------------
\ganttgroup{Projektdokumentation}{35}{42} \\

\ganttbar[progress=0, progress label font=\small\color{black!75},
	progress label anchor/.style={right=4pt},
	bar label font=\normalsize\color{black},
	name=thesis]
	{Thesis schreiben}{35}{42} \\
	
\ganttmilestone[name=msthesis,milestone label font=\color{red}, 
	milestone/.style={fill=red}]{Abgabe}{42}

%\ganttlink{ms7}{bImplAlgo}
\ganttlink{bImpFPGA}{ms4}
\ganttlink{bNumVerf}{ms3}
\ganttlink{bCMAES}{ms3}
\ganttlink{rech}{ms1}
\ganttlink{pflichten}{ms2}
\ganttlink{thesis}{msthesis}

	\end{ganttchart}
%		}
%\end{landscape}
%
%----------------------------------------------------------------------------

\end{appendix}
