% Interaction diagram, LaTeX user level and TeX system software level
% Author: Agostino De Marco
% Based on diagram from Marco Miani and Pascal Seppecher.
\documentclass{article}
\usepackage{tikz}
\usepackage{amsmath,amssymb,amsfonts,amstext}
\usepackage{floatflt}
\usetikzlibrary[arrows,snakes,backgrounds,shapes]
\usetikzlibrary{through}
\usetikzlibrary{calc}
\usepackage{caption}
\usepackage{subcaption}
\usepackage{wrapfig}
\usepackage{makeidx}
\usepackage{transparent}
\usepackage[utf8x]{inputenc}
\usetikzlibrary{positioning} 
\usetikzlibrary{mindmap,trees}
%%%<
\usepackage{verbatim}
\usepackage[active,tightpage]{preview}
\PreviewEnvironment{tikzpicture}
\setlength\PreviewBorder{5pt}%
%%%>
\usetikzlibrary{positioning}

\newcommand{\ypos}{3}
\newcommand{\xpos}{-.7}
\newcommand{\zpos}{0}

\newcommand{\yslant}{-.3}
\newcommand{\xslant}{.20}

\newcommand{\xoffset}[1]{\xpos*#1+3}
\newcommand{\yoffset}[1]{\ypos*#1}
\newcommand{\zoffset}[1]{\zpos+#1}

\begin{document}

\begin{tikzpicture}[auto]
	\scriptsize
		\tikzstyle{decision} = [diamond, draw=black, thick, fill=gray!5, text width=5em, text badly centered, inner sep=1pt]
%			
		\tikzstyle{block} = [rectangle, draw=black, thick, fill=gray!15, text width=15em, text centered, rounded corners, minimum height=5em]
%	
		\tikzstyle{blocknew} = [rectangle, draw=green!50!black!80, dashed, fill=green!5, text width=10em, text centered, rounded corners, minimum height=4em]
%
		\tikzstyle{input} = [circle, draw=black, thick, fill=gray!5, text width=5em, text centered, rounded corners, minimum height=4em]
%
		\tikzstyle{output} = [circle, draw=black, thick, fill=gray!5, text width=5em, text centered, rounded corners, minimum height=4em]
%
		\tikzstyle{line} = [draw, thick, -latex',shorten >=0.1pt];
		\tikzstyle{commentline} = [draw, dashed, green!50!black!80,-latex',shorten >=1pt];
%	
		\tikzstyle{cloud} = [ dotted, draw=gray!50, thick, ellipse,fill=gray!5, minimum height=2em, text width= 10em, text badly centered];
%	
		\matrix [column sep=7mm,row sep=5mm]
		{
			% row 1
			\node [input] (one) { Messwerte }; & \node [block] (two) { PRPS-Software }; &  \node [output] (three) { Kunde };\\
			 &  & \\
			% row 2
			&\node [blocknew] (four) { PRPS Evolution }; & \\
			% row 4
%																						 v--- durch Rekombination und/ oder Mutation
%			& \node [block] (identify) {Erzeuge $\lambda$ Nachkommen $X_c^k$ aus $X_p^k$ }; & \\
			% row 5
%			\node [block] (update) {Nächste Stufe der Evolution; k++}; &
%			\node [block] (evaluate) {Selektion von $\mu$ Nachkommen für die Generation $X_p^{k+1}$ }; & \\
			% row 6
%			& \node [decision] (decide) {$\Delta \geq \Delta_{min}$}; & \\
			% row 7
%			& \node [block] (stop) {Ende}; & \\
		};
		\draw[gray, thin, dashed] (-2.7,-3) -- (-2.7,3);
		\draw[gray, thin, dashed] (2.7,-3) -- (2.7,3);
		
%		edge[pil,bend left=45] (market.east)
%		\draw[green, thin, dashed,-latex', bend left=45 ,shorten >=1pt] 
		\tikzstyle{every path}=[commentline];
		\path[every node/.style={font=\sffamily\tiny}]
			(two) edge[bend right] node [left] {$\mathbf{\Theta_k}$} (four)
			(four) edge[bend right] node [right] {$\mathbf{n_k}$} (two);
%		\edge[thick,<->, bend left=45] (one);
		
% Arrows
		\tikzstyle{every path}=[line]
		\path (one) -- (two);
		\path (two) -- (three);
%		\path (evaluate) -- (decide);
%		\path (update) |- (identify);
%		\path (decide) -| node [near start] {Ja} (update);
%		\path (decide) -- node [midway] {Nein} (stop);
%		\path (start) -- (init);
		
%		\tikzstyle{every path}=[commentline]
%		\path (criteria) -- (decide);
%		\path (comment1) -- (init);
		
\end{tikzpicture}
\end{document}
%
%Approximation
%Die Theorie des numerischen Lösens linearer und nichtlinearer Gleichungen
%Die Theorie des numerischen Lösens von partiellen Differentialgleichungen
%Die Theorie des numerischen Lösens von Integralgleichungen
%Experimentelle Mathematik
%
%6.1.1 Genetische Algorithmen (GA)
%6.1.2 Evolutionsstrategien (ES)
%6.1.3 Genetische Programmierung (GP)
%6.1.4 Evolutionäre Programmierung (EP)
