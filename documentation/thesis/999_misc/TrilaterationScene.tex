% Interaction diagram, LaTeX user level and TeX system software level
% Author: Agostino De Marco
% Based on diagram from Marco Miani and Pascal Seppecher.
\documentclass{article}
\usepackage{tikz}
\usepackage{amsmath,amssymb,amsfonts,amstext}
\usepackage{floatflt}
\usetikzlibrary[arrows,snakes,backgrounds,shapes]
\usetikzlibrary{through}
\usetikzlibrary{calc}
\usepackage{caption}
\usepackage{subcaption}
\usepackage{wrapfig}
\usepackage{makeidx}
\usepackage{transparent}
\usepackage[utf8x]{inputenc}
\usetikzlibrary{mindmap,trees}
%%%<
\usepackage{verbatim}
\usepackage[active,tightpage]{preview}
\PreviewEnvironment{tikzpicture}
\setlength\PreviewBorder{5pt}%
%%%>
\usetikzlibrary{positioning}

\newcommand{\ypos}{3}
\newcommand{\xpos}{-.7}
\newcommand{\zpos}{0}

\newcommand{\yslant}{-.3}
\newcommand{\xslant}{.20}

\newcommand{\xoffset}[1]{\xpos*#1+3}
\newcommand{\yoffset}[1]{\ypos*#1}
\newcommand{\zoffset}[1]{\zpos+#1}


\begin{document}

\begin{tikzpicture}[
    scale=1,
    axis/.style={thick, ->, >=stealth'},
%    vector/.style={thick, ->,-latex, >=stealth'},
%    antenna/.style={thick},
     important line/.style={thick},
     antenna/.style={thick, black!70},
%    dashed line/.style={dashed, thin},
%    pile/.style={thick, ->, >=stealth', shorten <=2pt, shorten
%    >=2pt},
%    every node/.style={color=black},
%    main node/.style={circle,fill=blue!20,draw},
%    help lines/.style={gray,very thin}
    ]
    % axis
    \draw[axis] (-.1,0)  -- (1,0) node(xline)[right] {$x$};
    \draw[axis] (0,-.1) -- (0,1) node(yline)[above] {$y$};

%	\draw[gray, very thin, dotted] (-2,-2) grid (13,10);

	\coordinate (A1_start) at (1,3);
	\coordinate (A1_end) at (1,4);
	\coordinate (A2_start) at (4,6);
	\coordinate (A2_end) at (5,6);
	\coordinate (A3_start) at (8,1);
	\coordinate (A3_end) at (8,2);

	\coordinate (A1_end_) at ($(A1_start)!1!-10:(A1_end)$);
	\coordinate (A2_end_) at ($(A2_start)!1!-10:(A2_end)$);
	\coordinate (A3_end_) at ($(A3_start)!1!-35:(A3_end)$);
	
	\coordinate (Tag_0) at (4,2.5);
	\coordinate (REF_0) at (12,5);
	\coordinate (Int1) at ($(A1_start)!.5!(A1_end_)$);
	\coordinate (Int2) at ($(A2_start)!.5!(A2_end_)$);
	\coordinate (Int3) at ($(A3_start)!.5!(A3_end_)$);
	
%	\begin{scope}
%		\node [draw,thick,orange!50,dashed] at (Int1) [circle through={(Tag_0)}] {};
%		\node [draw,thick,orange!50,dashed] at (Int2) [circle through={(Tag_0)}] {};
%		\node [draw,thick,orange!50,dashed] at (Int3) [circle through={(Tag_0)}] {};
%	\end{scope}
	\begin{scope}
		\node [draw,very thick,gray!50,dashed] at (Int1) [circle through={(Tag_0)}] {};
		\node [draw,very thick,gray!50,dashed] at (Int2) [circle through={(Tag_0)}] {};
		\node [draw,very thick,gray!50,dashed] at (Int3) [circle through={(Tag_0)}] {};
	\end{scope}
		
	\draw[antenna] (A1_start) node[font=\scriptsize,black,below] {$A_1$} -- ($(A1_start)!1!-10:(A1_end)$);
	\draw[antenna] (A2_start) node[font=\scriptsize,black,above] {$A_2$}-- ($(A2_start)!1!-10:(A2_end)$);
	\draw[antenna] (A3_start) node[font=\scriptsize,black,below] {$A_3$}-- ($(A3_start)!1!-35:(A3_end)$);
	
%	\node [green!60!black!90, right,font=\scriptsize ] at (REF_0) {$\text{Landmarke}@(x_0,y_0,z_0)$};
	\node [black!90, right,font=\scriptsize ] at (REF_0) {$\text{Landmarke}@(x_0,y_0,z_0)$};
	
	\draw[latex-latex] (Tag_0) -- node[sloped,above,midway] {$r_1$}(Int1);
	\draw[latex-latex] (Tag_0) -- node[sloped,above,midway] {$r_2$}(Int2);
	\draw[latex-latex] (Tag_0) -- node[sloped,above,midway] {$r_3$}(Int3);
%	\draw[-latex,dashed,thick,green!60!black!90] (REF_0) -- node[sloped,above,midway] {$r_0$}(Tag_0);
	\draw[-latex,dashed,thick,black!90] (REF_0) -- node[sloped,above,midway] {$r_0$}(Tag_0);

	\draw[ -latex,gray!60,font=\tiny,dashed] (REF_0) -- node[sloped,above,midway] {$d_{10}$}(Int1);
	\draw[ -latex,gray!60,font=\tiny,dashed] (REF_0) -- node[sloped,above,midway] {$d_{20}$}(Int2);
	\draw[ -latex,gray!60,font=\tiny,dashed] (REF_0) -- node[sloped,above,midway] {$d_{30}$}(Int3);
		
	\fill[red!70] (Tag_0) circle [radius=2pt];
	\node[font=\scriptsize,black,below] at (Tag_0) {$Tag$} ;
	\fill[gray!60!black!90] (REF_0) circle [radius=2pt];
	
\end{tikzpicture}
\end{document}
%
%Approximation
%Die Theorie des numerischen Lösens linearer und nichtlinearer Gleichungen
%Die Theorie des numerischen Lösens von partiellen Differentialgleichungen
%Die Theorie des numerischen Lösens von Integralgleichungen
%Experimentelle Mathematik
%
%6.1.1 Genetische Algorithmen (GA)
%6.1.2 Evolutionsstrategien (ES)
%6.1.3 Genetische Programmierung (GP)
%6.1.4 Evolutionäre Programmierung (EP)
