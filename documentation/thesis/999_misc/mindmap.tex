% Interaction diagram, LaTeX user level and TeX system software level
% Author: Agostino De Marco
% Based on diagram from Marco Miani and Pascal Seppecher.
\documentclass{article}
\usepackage{tikz}
\usepackage{amsmath,amssymb,amsfonts,amstext}
\usepackage{floatflt}
\usetikzlibrary[arrows,snakes,backgrounds,shapes]
\usetikzlibrary{through}
\usetikzlibrary{calc}
\usepackage{caption}
\usepackage{subcaption}
\usepackage{wrapfig}
\usepackage{makeidx}
\usepackage{transparent}
\usepackage[utf8x]{inputenc}
\usetikzlibrary{mindmap,trees}
%%%<
\usepackage{verbatim}
\usepackage[active,tightpage]{preview}
\PreviewEnvironment{tikzpicture}
\setlength\PreviewBorder{5pt}%
%%%>
\usetikzlibrary{positioning}

\newcommand{\ypos}{3}
\newcommand{\xpos}{-.7}
\newcommand{\zpos}{0}

\newcommand{\yslant}{-.3}
\newcommand{\xslant}{.20}

\newcommand{\xoffset}[1]{\xpos*#1+3}
\newcommand{\yoffset}[1]{\ypos*#1}
\newcommand{\zoffset}[1]{\zpos+#1}


\begin{document}

\def\pgfsnakecirclestartradius{1cm}
\def\pgfsnakecircleendradius{.2cm}
\def\pgfsnakesegmentamplitude{1mm}
\def\pgfsnakesegmentangle{10}
	\begin{tikzpicture}
		  \path[small mindmap,concept color=black,text=white]
		    node[concept] {\tiny\textsf{Numerische Mathematik}}
		    [clockwise from=0]
		    child[concept color=green!55!black] {
		      node[concept] {\tiny\textsf{Optimierung}}
		      [clockwise from=100]
	%	      
		      child [concept color=green!55!black] {
		      		      node[concept] {\tiny\textsf{lineare-Opti-mierung}}
		      }
	%	      
		      child { node[concept] {\tiny\textsf{nichtlinear-Opti-mierung}} }
		      child { node[concept] {\tiny\textsf{skalare-Opti-mierung}} }
		      child { node[concept] {\tiny\textsf{Vektor-opti-mierung}} }
		    }
		    child[concept color=blue] {
		      node[concept] {\tiny\textsf{Approximation}}
		      [clockwise from=-35]
	%	      child { node[concept] {databases} }
	%	      child { node[concept] {WWW} }	      
		    }	    
		    child[concept color=red] { node[concept] {\tiny\textsf{partiellen Differentialgleichungen}} }
		    child[concept color=orange] { node[concept] {\tiny\textsf{Integral\-gleichungen}} };
		\end{tikzpicture}

	\begin{tikzpicture}
	 \path[small mindmap,concept color=green!60!black,text=white]
	    node[concept] {\tiny\textsf{ Nichtlineare\\Optimierung}}
		     	[clockwise from=0]
	%	      
				child [concept color=green!35!black]{ node[concept] {\tiny\textsf{Ableitungs-freie}}
				[clockwise from=110]
	%	   			   		       
	     child[concept color=violet!80!black]{ node[concept] {\tiny\textsf{Enum-erations-verfahren}} }
	     child [concept color=violet!80!black]{ node[concept] {\tiny\textsf{Monte-Carlo}} }
	     child [concept color=violet!100!black] {
	 			    node[concept] {\tiny\textsf{Evo\-lutionäre Algo\-rithmen}}
	 			    [clockwise from=45]
	      			%		      child[concept color=green!50!black] {
	      							child[grow=55] { node[concept] {\tiny\textsf{Genetische Algorithmen (GA)}} }
	      						 	child[grow=15] { node[concept] {\tiny\textsf{Evolutions\-strategien (ES)}} }
	      						 	child[grow=-25] { node[concept] {\tiny\textsf{Genetische Programmierung (GP)}} }
	      						 	child[grow=-70] { node[concept] {\tiny\textsf{Evolutionäre Programmierung (EP)}} }
	      			%			 }
	 			   }
	     child [concept color=violet!80!black]{ node[concept] {\tiny\textsf{Diskrete Suche}} }
	    }
	    child [concept color=green!35!black]{ node[concept]  (test) {\tiny\textsf{mit part. Ableitungen}}
	%	      		      [clockwise from=-20]
	    						child[concept color=orange!90!black, grow=-70] { node[concept] {\tiny\textsf{ Gradienten-verfahren}} }
							child[concept color=orange!90!black,grow=-110] { node[concept] {\tiny\textsf{ NEWTON-Verfahren}} }
							child[concept color=orange!90!black,grow=-150] { node[concept] {\tiny\textsf{ Quasi-NEWTON-Verfahren}} }
	%    	   			   
			     };
	%			\node [annotation] at (test.west)
	%			{\tiny Keine \\weitere \\Betrachtung\\ hier};
		\end{tikzpicture}
\end{document}
%
%Approximation
%Die Theorie des numerischen Lösens linearer und nichtlinearer Gleichungen
%Die Theorie des numerischen Lösens von partiellen Differentialgleichungen
%Die Theorie des numerischen Lösens von Integralgleichungen
%Experimentelle Mathematik
%
%6.1.1 Genetische Algorithmen (GA)
%6.1.2 Evolutionsstrategien (ES)
%6.1.3 Genetische Programmierung (GP)
%6.1.4 Evolutionäre Programmierung (EP)
