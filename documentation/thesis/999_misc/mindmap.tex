% Interaction diagram, LaTeX user level and TeX system software level
% Author: Agostino De Marco
% Based on diagram from Marco Miani and Pascal Seppecher.
\documentclass{article}
\usepackage{tikz}
\usepackage{amsmath,amssymb,amsfonts,amstext}
\usepackage{floatflt}
\usetikzlibrary[arrows,snakes,backgrounds,shapes]
\usetikzlibrary{through}
\usetikzlibrary{calc}
\usepackage{caption}
\usepackage{subcaption}
\usepackage{wrapfig}
\usepackage{makeidx}
\usepackage{transparent}

\usetikzlibrary{mindmap,trees}
%%%<
\usepackage{verbatim}
\usepackage[active,tightpage]{preview}
\PreviewEnvironment{tikzpicture}
\setlength\PreviewBorder{5pt}%
%%%>
\usetikzlibrary{positioning}

\newcommand{\ypos}{3}
\newcommand{\xpos}{-.7}
\newcommand{\zpos}{0}

\newcommand{\yslant}{-.3}
\newcommand{\xslant}{.20}

\newcommand{\xoffset}[1]{\xpos*#1+3}
\newcommand{\yoffset}[1]{\ypos*#1}
\newcommand{\zoffset}[1]{\zpos+#1}


\begin{document}

\def\pgfsnakecirclestartradius{1cm}
\def\pgfsnakecircleendradius{.2cm}
\def\pgfsnakesegmentamplitude{1mm}
\def\pgfsnakesegmentangle{10}
	\begin{tikzpicture}
	  \path[small mindmap,concept color=black,text=white]
	    node[concept] {Numerische Mathematik}
	    [clockwise from=0]
	    child[concept color=green!55!black] {
	      node[concept] {Optimierung}
	      [clockwise from=100]
%	      
	      child [concept color=green!55!black] {
	      		      node[concept] {lineare-Optimierung}
	      }
%	      
	      child { node[concept] {nichtlinear-Optimierung} }
	      child { node[concept] {skalare-Optimierung} }
	      child { node[concept] {Vektor-optimierung} }
	    }
	    child[concept color=blue] {
	      node[concept] {Approximation}
	      [clockwise from=-35]
%	      child { node[concept] {databases} }
%	      child { node[concept] {WWW} }	      
	    }	    
	    child[concept color=red] { node[concept] {partiellen Differentialgleichungen} }
	    child[concept color=orange] { node[concept] {Integral\-gleichungen} };
	\end{tikzpicture}

	\begin{tikzpicture}
		  \path[small mindmap,concept color=green!60!black,text=white]
		      node[concept] {Nichtlineare\\Optimierung}
		     	[clockwise from=0]
	%	      
				child [concept color=green!35!black]{ node[concept] {Ableitungs-freie}
				[clockwise from=110]
%	   			   		       
		       child[concept color=violet!80!black]{ node[concept] {\tiny Enum-erations-verfahren} }
		       child [concept color=violet!80!black]{ node[concept] {\tiny Monte-Carlo} }
		       child [concept color=violet!100!black] {
 	   			    node[concept] {Evo\-lutionäre Algo\-rithmen}
 	   			    [clockwise from=45]
       			%		      child[concept color=green!50!black] {
       							child[grow=55] { node[concept] {\tiny Genetische Algorithmen (GA)} }
       						 	child[grow=15] { node[concept] {\tiny Evolutions\-strategien (ES)} }
       						 	child[grow=-25] { node[concept] {\tiny Genetische Programmierung (GP)} }
       						 	child[grow=-70] { node[concept] {\tiny Evolutionäre Programmierung (EP)} }
       			%			 }
 	   			   }
		       child [concept color=violet!80!black]{ node[concept] {\tiny Diskrete Suche} }
		      }
		      child [concept color=green!35!black]{ node[concept]  (test) {mit part. Ableitungen}
%	      		      [clockwise from=-20]
     						child[concept color=blue!45!black, grow=-90] { node[concept] {\tiny Gradienten-verfahren} }
							child[concept color=blue!45!black,grow=-120] { node[concept] {\tiny NEWTON-Verfahren} }
							child[concept color=blue!45!black,grow=-150] { node[concept] {\tiny Quasi-NEWTON-Verfahren} }
%    	   			   
    		     };
				\node [annotation] at (test.west)
				{\tiny Keine \\weitere \\Betrachtung\\ hier};
		\end{tikzpicture}
\end{document}
%
%Approximation
%Die Theorie des numerischen Lösens linearer und nichtlinearer Gleichungen
%Die Theorie des numerischen Lösens von partiellen Differentialgleichungen
%Die Theorie des numerischen Lösens von Integralgleichungen
%Experimentelle Mathematik
%
%6.1.1 Genetische Algorithmen (GA)
%6.1.2 Evolutionsstrategien (ES)
%6.1.3 Genetische Programmierung (GP)
%6.1.4 Evolutionäre Programmierung (EP)
