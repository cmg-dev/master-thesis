%\documentclass[a4paper,12pt,fleqn,twoside]{scrbook}
\documentclass[a4paper,12pt, twoside, openright]{scrbook}

\usepackage[T1]{fontenc}
\usepackage{ucs}
\usepackage[utf8x]{inputenc}
\usepackage{ngerman}
\usepackage[ngerman]{babel}
\usepackage{lastpage}
\usepackage[pdftex]{color,graphicx}
\usepackage{listings}
\usepackage{pdflscape}
\usepackage{longtable}

%\usepackage{showframe}
\usepackage[lmargin=142pt,rmargin=95pt,tmargin=117pt,bmargin=113pt]{geometry}
%\usepackage[inner=2cm,outer=2cm,top=1cm,bottom=1.5cm,includeheadfoot]{geometry}

\usepackage{fancyhdr}
\usepackage{url}

\usepackage{booktabs}
\usepackage{blindtext} 
\usepackage{lipsum}

\usepackage{framed} 
\usepackage{xcolor} 
\colorlet{shadecolor}{black} 
\usepackage{latexsym}
\usepackage{float}

\usepackage{bbm}
\usepackage{enumitem}

%\usepackage{draftwatermark}
%\SetWatermarkText{Entwurf}
%\SetWatermarkScale{4}
%\SetWatermarkLightness{0.9}
\usepackage[section]{placeins}
\usepackage{pgfgantt}
\usepackage{amsmath,amssymb,amsfonts,amstext}
\usepackage{floatflt}
\usepackage{tikz}
\usetikzlibrary[arrows,snakes,backgrounds,shapes]
\usetikzlibrary{through}
\usetikzlibrary{calc}
\usepackage{caption}
\usepackage{subcaption}
\usepackage{wrapfig}
\usepackage{makeidx}

% highlighting
\usepackage{xcolor,soul}

%---- PageLayout
\pagestyle{fancy}

%\setlength{\textheight}{225mm}
%\setlength{\headsep}{20mm}
\usepackage{eso-pic}
%\usepackage[numbers,round]{natbib}

%----------------------------------------------------------------------------
% HEADER --------------------------------------------------------------------
%----------------------------------------------------------------------------
\fancyhead[R]{
%  \includegraphics[width=100pt,keepaspectratio]{img/amedo2012.png}
}

\fancyhead[C]{ }

\fancyhead[LO]{
  \begin{tabular}[b]{l}
  Christoph Gnip\\
  Projekt: PRPS-Evolution
  \end{tabular}
}

\fancyhead[RE]{
%  \chaptername
}


%Linie oben
\renewcommand{\headrulewidth}{0.5pt}
%----------------------------------------------------------------------------

%----------------------------------------------------------------------------
%----------------------------------------------------------------------------
%----------------------------------------------------------------------------
%\fancyfoot[L]{Stand: \today}
\fancyfoot[C]{ Entwurf }
\fancyfoot[R]{\thepage{} von \pageref{LastPage}}

% Linie unten
\renewcommand{\footrulewidth}{0.5pt}
%----------------------------------------------------------------------------

% Import Macros  ------------------------------------------------------------
%--------------------------------------------------------------
%--------------------------------------------------------------
%--------------------------------------------------------------
\newcommand\confidentialoverlay{
  % Taken from the TikZ documentation.
  % NB: This requires \usepackage{tikz}!
  \begin{tikzpicture}[remember picture,overlay]
    \node [rotate=60,scale=10,text opacity=0.1]
      at (current page.center) {Vertraulich};
  \end{tikzpicture} 
 
} 
%--------------------------------------------------------------
%--------------------------------------------------------------
%--------------------------------------------------------------
\newcommand{\myvec}[1]{\hat{\mathbf{#1}}}% Vector notation

%--------------------------------------------------------------
%- This can be used for aligning equations --------------------
%--------------------------------------------------------------
\newcommand{\phantomeq}[2]{
\begin{equation}
	\phantom{#1}
	#2
\end{equation}
}% Vector notation

%--------------------------------------------------------------
%- seraches for input in the "extern" folder ------------------
%--------------------------------------------------------------
\newcommand{\externInput}[1]{\input{extern/#1}}

%--------------------------------------------------------------
%- seraches for input in the "intern" folder ------------------
%--------------------------------------------------------------
\newcommand{\internInput}[1]{\input{intern/#1}}

%--------------------------------------------------------------
%- seraches for input in the "common" folder ------------------
%--------------------------------------------------------------
\newcommand{\commonInput}[1]{\input{common/#1}}

\newcommand{\cpp}{%
  \mbox{\emph{\textrm{C\hspace{-1.5pt}\raisebox{1.75pt}{\scriptsize +}%
  \hspace{-2pt}\raisebox{.75pt}{\scriptsize +}}}}%
}

\newcommand{\amedogmbh}{%
  amedo GmbH
}

%\renewenvironment{itemize}[1]{\begin{compactitem}#1}{\end{compactitem}}
%\renewenvironment{enumerate}[1]{\begin{compactenum}#1}{\end{compactenum}}
%\renewenvironment{description}[0]{\begin{compactdesc}}{\end{compactdesc}}


%----------------------------------------------------------------------------
% Start the Document --------------------------------------------------------
%----------------------------------------------------------------------------
\begin{document}
\frontmatter 

%\setlength{\headheight}{36pt}

%----------------------------------------------------------------------------
% Titlepage -----------------------------------------------------------------
%----------------------------------------------------------------------------

%- the Title page --------------------------------------------------------
\begin{titlepage}
       \begin{center}
           %\vspace*{2.5cm}
			{
			\Huge 
			\textbf{
			  Entwicklung eines Systems zur
			  Entfernungsabschätzung für Phasen basiertes UHF RFID Tracking durch Verwendung
			  evolutionärer Berechnungsverfahren
			}\par
			}
			\vspace{3cm}
			{
			\Large Masterthesis eingereicht zur Erfüllung\\
			der Anforderungen zum Erwerb des akademischen Grades\\
			Master of Science der Medizintechnik\\
			}
\vspace{2cm}

\large{Erstellt von}\\

\Large{\textbf{Christoph Gnip}}


%\vspace{4cm}
\vfill

{\normalsize Fachbereicht Elektrotechnik und angewandte Naturwissenschaften
           \\Westfälische Hochschule\\[2ex]September 2013}


       \end{center}
   \end{titlepage}

\newpage
\pagenumbering{roman}
\thispagestyle{empty}
\hspace{1cm}

\newpage
%\vspace{1cm}
\normalsize


\begin{longtable}{p{4cm}p{95mm}}
{} & {}\\
\Large{Master's Thesis} & {}\\
{} & {}\\
{} & {}\\
Titel: & Entwicklung eines Systems zur
Entfernungsabschätzung für Phasen basiertes UHF RFID Tracking durch Verwendung
evolutionärer Berechnungsverfahren\\
Title: & Development of a Distance Estimation System for
Phase-Based UHF RFID Tracking by Utilizing Methodes of Evolutionary Computation\\
{} & {}\\
{} & {}\\
University:                 & Westphalian University of Applied Sciences \\
{} & Department Electrical Engineering and Applied Sciences \\
{}                          & Neidenburger Str. 43 \\
{}                          & 45897 Gelsenkirchen \\
{}                          & Germany \\
{}                          & {}\\
{}                          & {}\\
In Cooperation with:        & Amedo Smart Tracking Solutions GmbH\\
{}                          & Universitätstraße 142 \\
{}                          & Bochum \\
{}                          & {}\\
Author:                     & Christoph Gnip \\
{}                          & Luggendelle 28 \\
{}                          & 48954 Gelsenkirchen \\
{}                          & Germany \\
{}                          & {}\\
Matrikelnumber:				& 200720362 \\
{}                          & {}\\
Supervisor:                 & Prof. Dr. Frank Bärmann \\
Co-supervisor:              & Dipl.-Ing. Volker Trösken\\
\end{longtable}

\newpage
\pagenumbering{roman} 

%----------------------------------------------------------------------------
% Tables --------------------------------------------------------------------
%----------------------------------------------------------------------------
\tableofcontents 
%
\listoffigures
%
\listoftables
%
\lstlistoflistings
%
\newpage
%
\subsection*{Verwendete Abkürzungen}
%
\begin{table} [H]
	\begin{center}
		\begin{tabular}{p{25mm}p{95mm}}
		      	\textbf{ES} & Evolutionäre Strategie (\textit{Evolutionary Strategy})\\
		      	\textbf{CMA-ES}  & Covariance Matrix Adaption - Evolutionary Strategy\\
		      	\textbf{\cpp11} & Programmiersprache C++ in der Version 11\\
		      	\textbf{MRT}	& Magnetresonanztomografie\\
		      	\textbf{RFID} & Radio-Frequency Identification\\
		      	\textbf{LOS} & Line of Sight\\
		      	\textbf{CSV} & Comma seperated Values\\
		      	\textbf{Rö} & Röntgen\\
		      	\textbf{CT} & Computertomografie\\
		      	\textbf{EM} & Elektromagnetismus\\
		      	\textbf{TAG} & Transponder/ Receiver für Funk Kommunikation\\
		      	\textbf{Tracking} & Positionsbestimmung\\
		      	\textbf{PRPS} & Passiv RFID Positioning System - Produkt der \amedogmbh \\
		      	\textbf{TOF} & Time Of Flight\\
		      	\textbf{PD} & Phasendifferenz\\
		      	\textbf{RSSI} & Indikator für die empfangene Signalstärke (Received Signal Strength Indication)\\
%		      	
		\end{tabular}
	\end{center}
\end{table}


%
\newpage
\subsection*{Verwendete Symbole}
%
\begin{table} [H]
	\begin{center}
		\begin{tabular}{p{25mm}p{95mm}}
			$\mathbf{A}$ & Matrizen werden mit fetten Großbuchstaben notiert\\
			$\mathbf{b}$ & Vektoren werden mit fetten Kleinbuchstaben notiert\\
			$\mathbf{0}$ & \text{Nullvektor}\\
			$k$ & ist der Index der Antennen im Aufbau verwendeten Antennen\\
			$r_{k}$ & Abstand vom Tag zur indizierten Antenne\\
			$d_{k0}$ & Abstand zur Landmarke (Index $0$) zur indizierten Antenne\\
			$\mu $ & \text{Eigenwert}; Es wird von dem gebräuchlicheren Symbol $\lambda$ abgewichen, um Mehrdeutigkeiten im Rahmen der Arbeit zu vermeiden\\
			$(\mu,\lambda)$ & "Komma"- Evolutionsstrategie \\
			$(\mu+\lambda)$ & "Plus"- Evolutionsstrategie \\
			$\varrho$ & Phase \\
%			
		\end{tabular}
	\end{center}
\end{table}


\mainmatter 
%----------------------------------------------------------------------------
% Chapters ------------------------------------------------------------------
%----------------------------------------------------------------------------
% 1. ------------------------------------------------------------------------
\chapter{Einleitung}
Dieses Kapitel führt in die Arbeit ein. Zuerst wird die Motivation erläutert und der aktuelle Stand der Technik vorgestellt, im Anschluss werden in zwei Teilen die technischen und mathematischen Voraussetzungen beschrieben, zuletzt werden daraus die Anforderungen an die Lösung abgeleitet. Die Voraussetzungen werden in der für das Verständnis dieser Arbeit angebrachten Tiefe beschrieben. Allgemeine Zusammenhänge und Techniken, denen einen großer Stellenwert in dieser Arbeit zukommt, werden zusammengefasst präsentiert. Für detaillierte Beschreibungen wird auf entsprechende Fachliteratur verwiesen. Abschließen wird das Ziel dieser Arbeit aufgestellt.
%
%-----------------------------------------------------------------------
%
\section[Allgemein]{Allgemein}
Mit der Entwicklung der minimal-invasiven Chirurgie, einer Operationsmethode bei der durch sehr kleine Einschnitte in den Körper mit besonders filigranen Operationsinstrumenten operiert wird, verändert sich die Art Operationen durchzuführen grundlegend. Eingriffe können schneller, schonender und effizienter durchgeführt werden. Möglich wird diese Entwicklung durch eine Vielzahl neuartiger technischer Systeme. Die Vorteile gegenüber herkömmlichen Operationstechniken begründen die weite Verbreitung und häufigen Einsatz der minimal-invasiven Techniken.\\
Mit fortschreitender Miniaturisierung der Instrumente geht die optische Kontrolle über das Operationsgebiet sowie Instrumentarium verloren. Diese Information ist unabdingbar für einen Erfolg der Operation und müssen dem Operierenden zu jeder Zeit zur Verfügung stehen. Um an diese Informationen zu gelangen ist es Stand der Technik, durch aufwändige bildgebende Verfahren intraoperativ, d.h. während der Operation, anzuwenden.\\
Beispielsweise werden bei kardiologischen Interventionen (z.B. Platzierung eines Stents durch die Arteria iliaca interna\footnote {innere Beckenarterie- Standardzugang für diese Art von Operationen} in den Coronargefäßen des Herzens) eine permanente Lagekontrolle der Katheter mittels Röntgentechnik durchgeführt. Oder es werden Bilder durch Magnetresonanztomografie oder durch andere bildgebende Verfahren erzeugt. Nicht nur das eine Gewinnung dieser Bilddaten schwierig (MRT) oder gar schädlich (Röntgen) ist, oft muss der Patient dafür samt Instrumentarium umgelagert werden. Das Umlagern bringt weitere Risiken mit sich und ist mit weiterem Aufwand verbunden.\\
Eine Lösung für diese Problem bringen sog. Trackingsysteme. Diese Systeme sind in der Lage eine Position, z.B. eines Instrumentes, zu ermitteln und stellen die benötigten Informationen für den Arzt zur Verfügung. Die verfügbaren Systeme basieren auf unterschiedlichen physikalischen Prinzipien und haben dadurch unterschiedliche Vor- und Nachteile.\\
Die Anwendung solcher Systeme erlaubt außerdem eine softwaregestützte Planung und assistierte Durchführung der Operation. Die Kombination dieser Techniken wird Navigation genannt. Die Möglichkeit der Planung und Kontrolle macht diese Systeme im Zuge der stets steigenden Ansprüche an das Qualitätsmanagement interessant. Die Anforderungen die vom Anwender im klinischen Alltag an die Systeme gestellt werden sind:
%
\begin{table} [H]
	\begin{center}
		\begin{tabular}{l}
		Gute Genauigkeit\\
		Hohe Verfügbarkeit\\
		Leichte Bedienbarkeit\\
		Einfache Einbindung Workflow\\
		Geringe Kosten\\
		Sicherheit\\
		\end{tabular}
	\end{center}
	\caption[Anforderungen Trackingsysteme]{Anforderungen an ein medizintechnisches Messsystem.}
	\label{tab:requirements_system}
\end{table}
%
Die Anforderungen an ein solches System sind somit sehr hoch. Sie müssen über eine entsprechende Technik verfügen und gleichzeitig muss der Umgang mit ihnen leicht sein. Zusätzlich dürfen die Systeme möglichst wenig kosten.\\
%
\subsubsection{Stand der Technik}
Es befinden sich Trackingsysteme unterschiedlicher Hersteller am Markt. Sie beruhen auf unterschiedlichsten Messprinzipien und unterliegen den daraus resultierenden Limitierungen. Die wichtigsten Technischen Unterschiede sind im Folgenden tabellarisch zusammengefasst:
%
\begin{table} [H]
	\begin{center}
		\begin{tabular}{rllll}
			\textbf{Arbeitsweise} & Optisch & Magnetisch & Ultraschall & Funk (UHF) \\
			\textbf{Genauigkeit} & gut & ausreichend & gut & sehr gut\footnote{Abhängig vom Messprinzip} \\
			\textbf{Frequenz} & mittel & hoch & gering & hoch \\
			\textbf{Volumen} & mittel & klein & mittel & groß \\
			\textbf{LOS} & Ja & Ja & Nein & Ja \\
			\textbf{IV\footnote{in vivo lat. im Lebendigen; med. im Patienten}} & Nein   & Nein & Nein & Ja \\
%			
		\end{tabular}
	\end{center}
	\caption[Übersicht Navigationsverfahren]{Grobe Übersicht und Einteilung verschiedener Navigationsverfahren anhand ihres physikalischen Messprinzips.}
	\label{tab:overview_tracking}
\end{table}
%
Die Tabelle~\ref{tab:overview_tracking} teilt die unterschiedlichen Systeme anhand ihres physikalischen Messprinzips ein. Herausgestellt werden vor Allem die wesentlichen Messparameter der betreffende Aspekte der Verfahren. Aus der Auflistung lassen sich Vor. und Nachteile ableiten.\\
Das größte Problem ist das Benötigen einer direkten Sicht auf die Objekte. Dem sog. LOS-Problem unterliegen fast alle Verfahren, die ein großes Messvolumen abdecken. Die auf Funk basierenden Verfahren haben das Problem nicht, unterliegen jedoch anderen Schwierigkeiten. Der größte Vorteil des auf Funk basierenden RFID-Verfahrens ist es verschiedene Objekte von einander zu unterscheiden, zu identifizieren.\\
Die Genauigkeit (im technischen Sinne: Präzision und Wiederholbarkeit) der Messung ist bei allen Verfahren mindestens ausreichend. Das allein stellt viele Techniken vor eine großer Herausforderung. Hinzukommen weitere Anforderungen, die sich aus dem Ablauf einer Intervention ergeben. Ein System muss eine einfache Integrationsmöglichkeit in den Arbeitsablauf bieten.\\
%

Im Folgenden wird auf die Besonderheiten und Merkmale des auf Funk basierenden RFID-Verfahrens eingegangen. Die anderen Verfahren werden, aufgrund der Unterschiedlichkeit der Systeme wird im Rahmen dieser Arbeit wird darauf verzichtet.


%
%-----------------------------------------------------------------------
%
\section[Motivation]{Motivation}
Die Positionsbestimmung (Tracking) mittels RFID (Radio-Frequency Identification) bietet gegenüber vergleichbaren Methoden (z.B. Ultraschall, Optisch) verschiedene Vorteile. Das wesentlichste Unterscheidungsmerkmal ist, dass keine direkte Sichtlinie sog. LOS notwendig ist um ein Objekt zu lokalisieren. Der Grund dafür ist das zugrunde liegende Messprinzip. Insbesondere im Vergleich mit optischen Verfahren ist RFID damit überlegen. Weiterhin erlauben die als Positionsgeber verwendeten Tags zusätzliche Informationen auf ihnen abzulegen, beispielsweise eine Identifikationsnummer und Weiteres. Dadurch wächst das Anwendungsspektrum weiter. Das Auslesen von zusätzlichen Informationen ist in keiner der anderen Technologien möglich.\\

Das von dem Messsystem der {Amedo GmbH} verwendete Verfahren basiert auf der Messung der Phasenlage der Antwort eines Tags. Die Phasenlage ist direkt proportional zu einer Entfernung. Dabei kommt es aufgrund der Physik im wesentlichen zu folgenden Problemen:
\begin{enumerate}
	\item Die Messung der Position erfolgt über die Auswertung der Phasenlage des empfangenen Signals in Bezug auf ein Referenzsignal. Da in der EU sind nur bestimmte Frequenzen für die Verwendung für RFID erlaubt (865,5–867,5 MHz) kann man die Wellenlänge mit: $ \lambda\simeq0,35 m $ angeben. Daraus folgt, dass alle 35 cm die gleiche Konfiguration der Phase vorliegt. In dieser Arbeit wird dieser Umstand Isophasen genannt. Die gewonnene Information aus der Phase ist somit redundant, d.h. es lässt sich durch die Kenntnis der Phase nicht unmittelbar auf die korrekte Postion schließen. Man kann das Problem umgehen in dem man auf die errechnete Position ein ganzzahliges Vielfaches der Wellenlänge addiert. Die sog. Wellenzahl (vgl.~\eqref{eq:Wavenumbers}).
	\item Das System der Amedo STS verwendet eine spezielle Antennenanordnung um die Position zu ermitteln. Dabei wird eine Antennenanzahl >4 eingesetzt. Für jede dieser Antennen muss eine eigene Wellenzahl bestimmt werden. Durch Auslöschung des Signals, Absorption etc. kann es dazu kommen, dass eine Antenne eine unbestimmte Zeit lang kein Signal vom Tag empfängt. Wenn die Antenne nach dieser Zeit erneut ein Signal empfängt ist die ihr zugehörige Wellenzahl unbekannt und muss neu bestimmt werden. 
	\item In realen Umgebungen treten zusätzlich noch Ruflektionen und ein sog. Multipath-Effekt auf. Dabei wird das Signal nicht auf dem Direkten Weg Antenne-Tag-Antenne empfangen sondern über einen unbekannten, längeren Weg. Dadurch kommt es zu einem Fehler in der Phase. Zusätzlich ist dieser Effekt individuell für jede Antenne.
\end{enumerate}

Eine analytische Lösung des Problems ist schwierig und bisher nicht gelungen. In dieser Arbeit soll mittels numerischer Methoden und Modellen die beschriebenen Probleme zu gelöst werden.

%
%Das Problem liegt in den unbekannten, komplex zu modellierenden Verhalten der elektromagnetischen Funkwellen in geschlossenen Räumen (insb. Auslöschung, Multipath, Reflektion). Diese führen zu einem Fehler der Phase und damit direkt zu einer Falschaussage der Position.\\ \\
%\textbf{Beschreibung der Wellenzahl[Referenz auf die Dipl. Arbeit von Bernd]}\\\\
%Ziel dieser Arbeit ist es ein System zu
%implementieren, das eine direkte Abschätzung (Ad-Hoc-Messung) der Wellenzahl erlaubt.
%Dafür werden Methoden der Numerik verwendet um die Uneindeutigkeit der Phasenlage zu
%eliminieren.
%\\ \\ \\
%Tags gibt es mit unterschiedlichen Funktionsweisen, in dieser Arbeit und in dem von der {Amedo STS} verwendeten System kommen passive Tags zum Einsatz. Diese versorgen sich aus den Funksignalen des Abfragegeräts mit der notwendigen Energie und modulieren ihre "Antwort" auf das Trägersignal auf.\\\\\\
%In der Positionsbestimmung wird im Zusammenhang von "Marker" gesprochen. In der in dieser Arbeit werden RFID-Transponder (sog. Tags) als Marker verwendet. D.h. Es wird die Position im Raum von einem Transponder ermittelt. \\
%

%
%-----------------------------------------------------------------------
%
\section[Anforderungen]{Anforderungen an die Lösung}
%
Aus den bisher vorgestellten Überlegungen können nun folgende Anforderungen abgeleitet werden:
%
\begin{enumerate}[itemsep=0mm]
	\item Lösung muss schnell (ideal < 1 Sekunde) gefunden werden
	\item Unabhängigkeit von Stütz- und Kalibrierpunkten
	\item Eindeutigkeit der Lösung
	\item Eignung für ein großes Messvolumen
	\item Nahtlose Integration in das bestehende Software Ökosystem
	\item Stand der Softwaretechnik entsprechend
%
\end{enumerate}
%
\begin{figure}[ht!]
         \centering
         \input{diagrams/spider_requirements.tex}
         \caption[Anforderungsspinne]{ Grafische Übersicht der Anforderungen an das System }
         \label{fig:Requirements}
\end{figure}
%
%-----------------------------------------------------------------------
%
\section{Ziel und Herangehensweise}
%
Das Ziel der Arbeit ist die Entwicklung eines Systems zur Abschätzung der Postion eines Tags. Das Auffinden der Lösung soll die oben abgeleiteten Anforderungen erfüllt. Die Ermittelung einer korrekten Lösung ist jedoch das Wichtigste. Das System wird im Kern die Lösung über ein numerisches Optimierungsverfahren finden, im speziellen kommt das sog. \textit{'Covarianz Matrix Adaption - Evolutionary Strategy'} (CMA-ES) zum Einsatz. Bei diesem Verfahren handelt es sich um ein stochastische, Ableitungsfreies Verfahren, dass für nicht lineare, nicht konvexe, kontinuierliche Probleme geeignet ist. Dazu wird zuerst ein Modell entworfen werden, dass sich für einen Einsatz in diesem Verfahren eignet. Das eingesetzte Lösungsverfahren stellt praktisch keine Anforderungen an ein solches Modell. Daher soll es mit möglichst wenig Annahmen/ Einschränkungen auskommen und dennoch ein relativ sicheres, reproduzierbares Ergebnis liefern. Weiterhin soll eine eine Integration der Lösung in das Software-Ökosystem der \amedogmbh erfolgen. Ferner sollen Methoden in weiteren Projekten zum Einsatz kommen, beider Umsetzung ist auf eine größtmögliche Wiederverwendbarkeit zu achten. Es werden verschiedene Implementationen des CMA-ES-Algorithmus recherchiert, verglichen und die geeignetste gewählt. Das System soll unmittelbar in den Produkten der \amedogmbh zum Einsatz kommen können, daher wird eine entsprechende Schnittstelle für andere Software implementiert werden. Im Rahmen dieser Arbeit wird eine Methode entwickelt werden, um die Position von frei im Raum angeordnete Antennen zu ermitteln und dem Messaufbau zu kalibrieren.
%

%
% 2. ------------------------------------------------------------------------
\chapter{Hauptteil}
Im Folgenden werden ausführlich die Lösungen zur beschriebenen Problemstellung präsentiert. Es werden die Modelle vorgestellt die zum Auffinden der Lösung verwendet wurden, Im Anschluss wird die weiterhin wird die Implementation der ES und die Schnittstellen zum PRPS beschrieben.
%
\subsection{Vorüberlegung zur Komplexität}
\label{sec:Komplexity1}
%
\begin{floatingfigure}[hr!]{6cm}
 \centering
         \includegraphics[width=7cm]{img/Plate0_A1.png}
         \caption[Profil einer Phasenmessung]{Normiertes Höhenprofil einer Phasenmessung aus der Sicht von Antenne 1 }
         \label{fig:Plate0_A1_}
\end{floatingfigure}
%
In diesem Abschnitt wird eine Übersicht über die Komplexität des Problems gegeben. 
Abbildung~\ref{fig:Plate0_A1_} zeigt die Visualisierung einer typischen Kalibriermessung. Der verwendete Aufbau ist in Abbildung~\ref{fig:Spider1}. gezeigt. Er besteht aus vier Antennen, die in einer Ebene angeordnet sind. Es wurde eine reproduzierbare Aufstellung verwendet (Abbildung~\ref{fig:Spider_setup1}) und eine Fläche von $1\times1$ Meter vermessen. Alle $10$ cm wurde eine Messung gespeichert. In der Abbildung kann man deutlich das Verhalten der Phasendaten sehen. Um diesen Verlauf deutlicher zu zeigen, wurden die Phasenwerte normiert und als Oberfläche in den Plot gelegt. Am Boden gezeigt ist der Kontur-Plot der Werte. Zwischen den Werten wurde interpoliert um die Nulldurchgänge deutlicher zu zeigen. Die Übersicht aus der Sicht aller Antennen ist in Abbildung~\ref{fig:Real_Measurements} gezeigt.\\
%

In der Abbildung~\ref{fig:Complexity1} werden die Daten ohne Interpolation dargestellt. Es wurden die Höhenlinien eingezeichnet. Die Anordnung der Plots soll ein Gefühl dafür vermitteln, wie die Messwerte eines Tags sich an unterschiedlichen Postionen und aus Sicht verschiedener Antennen verhalten.\\
%

Die Darstellung echter Messwerte lässt Rückschlüsse auf die Komplexität des Problems zu. Es ist leicht nachzuvollziehen, dass das Zusammenspiel der Messwerte eine sehr komplexe Szene ergibt. Hier dargestellt ist bereits das Verhalten bei der Verwendung von vier Antennen. Der aktuelle Messaufbau erlaubt sogar acht Antennen. Das ergibt insgesamt eine komplexe Szenerie.\\
%
\begin{figure}[ht!]
        \centering
        \begin{subfigure}[b]{0.4\textwidth}
            \centering
            \includegraphics[width=\textwidth]{img/Plate0_A1.png}
            \caption[lorem]{Antenne 1}
            \label{fig:Plate0_A1}
        \end{subfigure}%
\\
        \begin{subfigure}[b]{0.4\textwidth}
            \centering
            \includegraphics[width=\textwidth]{img/Plate0_A2.png}
          	\caption[Loren ipsum]{Antenne 2}
         	\label{fig:Plate0_A2}
        \end{subfigure}
\qquad\qquad
        \begin{subfigure}[b]{0.4\textwidth}
			\centering
			\includegraphics[width=\textwidth]{img/Plate0_A4.png}
			\caption[Loren ipsum]{Antenne 4}
			\label{fig:Plate0_A3}
        \end{subfigure}
\\
        \begin{subfigure}[b]{0.4\textwidth}
			\centering
			\includegraphics[width=\textwidth]{img/Plate0_A3.png}
			\caption[Loren ipsum]{Antenne 3}
			\label{fig:Plate0_A4}
        \end{subfigure}
        \caption[Reale Messwerte visualisiert]{Blick auf die Messwerte der  Kalibrierplatte aus der "Sicht" der Antennen. Dabei zeigt sich deutlich der Wellencharakter der Messung, dieser ist zu erwarten. Die Messungen wurden mit einer Frequenz von $865,7$ MHz unter Laborbedingungen aufgenommen. }\label{fig:Real_Measurements}
\end{figure}
%
\begin{figure}[ht!]
         \centering
         \includegraphics[width=0.6\textwidth]{img/complexitiy1.pdf}
%         \includegraphics[width=0.7\textwidth]{img/00_placeholder.png}
         \caption[Normierte Messwerte von Kalibriermessung]{Diese Grafik zeigt die Visualisierung von realen Phasen-Messwerten. Die Daten wurden durch Vermessung einer $1\times1$-Kalibrierplatte mit reproduzierbarer Aufstellung\footnote{In dieser Arbeit nicht gezeigt} gewonnen. Die Daten wurden normiert. In jeder Dimension wurden $10\times10$ Werte aufgenommen. Die Darstellung der Phasenwerte erfolgt als Heatmap, es soll qualitativ der Verlauf der Phasenwerte gezeigt werden. Zur Orientierung sind in jedem Plot Höhenlinien eingezeichnet. Pro Plot werden die Daten einer Antenne dargestellt. Die Antenne von der die Daten stammen ist angegeben.}
         \label{fig:Complexity1}
%
\end{figure}


%
\subsection{Entwicklung des Modells}
\label{sec:model_developement}
Im folgenden Abschnitt wird das Modell für die Lösung des Zusammenhangs entwickelt. Zur Veranschaulichung des Sachverhalts dient die Abbildung~\ref{fig:TrilaterationScene}. Dort skizziert ist der Messaufbau mit einem Tag. Die Szene ist in 2D dargestellt die Ableitung des Modells erfolgt direkt für drei Raumkoordinaten.
%
\begin{figure}
	\begin{center}
		\caption[Antennen-Szene mit einem Tag]{2D-Übersicht auf die Szene mit drei Antennen, einem Tag und einer Landmarke. Die Position von $\{A_1,A_3,A_3\}$, sowie der Landmarke, zum Koordinatenursprung sind bekannt. Die Vektoren $r_1,r_2,r_3$ sind die gemessene Entfernung zu einer Antenne. Die Landmarke wird im späteren Verlauf eine Antenne sein, die ihrerseits ein gemessene Entfernung $r_0$ produziert. Der Schnittpunkt aller Kreise ist die Lösung der gemessenen Entfernung und der geom. Anordnung, die sich für die Position des Tags ergibt.} 
		\label{fig:TrilaterationScene}
		\begin{tikzpicture}[
    scale=1,
    axis/.style={thick, ->, >=stealth'},
%    vector/.style={thick, ->,-latex, >=stealth'},
%    antenna/.style={thick},
     important line/.style={thick},
     antenna/.style={thick, cyan!70},
%    dashed line/.style={dashed, thin},
%    pile/.style={thick, ->, >=stealth', shorten <=2pt, shorten
%    >=2pt},
%    every node/.style={color=black},
%    main node/.style={circle,fill=blue!20,draw},
%    help lines/.style={gray,very thin}
    ]
    % axis
    \draw[axis] (-.1,0)  -- (1,0) node(xline)[right] {$x$};
    \draw[axis] (0,-.1) -- (0,1) node(yline)[above] {$y$};

	\draw[gray, very thin, dotted] (0,0) grid (15,6);

	\coordinate (A1_start) at (4,3);
	\coordinate (A1_end) at (4,4);
	\coordinate (A2_start) at (7,5);
	\coordinate (A2_end) at (8,5);
	\coordinate (A3_start) at (8,1);
	\coordinate (A3_end) at (8,2);

	\coordinate (A1_end_) at ($(A1_start)!1!-10:(A1_end)$);
	\coordinate (A2_end_) at ($(A2_start)!1!-10:(A2_end)$);
	\coordinate (A3_end_) at ($(A3_start)!1!-35:(A3_end)$);
	
	\coordinate (Tag_0) at (6,2);
	\coordinate (REF_0) at (12,5);
	\coordinate (Int1) at ($(A1_start)!.5!(A1_end_)$);
	\coordinate (Int2) at ($(A2_start)!.5!(A2_end_)$);
	\coordinate (Int3) at ($(A3_start)!.5!(A3_end_)$);
	
	\begin{scope}
		\node [draw,orange!50,dashed] at (Int1) [circle through={(Tag_0)}] {};
		\node [draw,orange!50,dashed] at (Int2) [circle through={(Tag_0)}] {};
		\node [draw,orange!50,dashed] at (Int3) [circle through={(Tag_0)}] {};
	\end{scope}
	
	\draw[antenna] (A1_start) node[font=\scriptsize,black,below] {$A_1$} -- ($(A1_start)!1!-10:(A1_end)$);
	\draw[antenna] (A2_start) node[font=\scriptsize,black,above] {$A_2$}-- ($(A2_start)!1!-10:(A2_end)$);
	\draw[antenna] (A3_start) node[font=\scriptsize,black,below] {$A_3$}-- ($(A3_start)!1!-35:(A3_end)$);
	
	\node [green!60!black!90, right,font=\scriptsize ] at (REF_0) {$\text{Landmarke}@(x_0,y_0,z_0)$};

	\draw[latex-latex] (Tag_0) -- node[sloped,above,midway] {$r_1$}(Int1);
	\draw[latex-latex] (Tag_0) -- node[sloped,above,midway] {$r_2$}(Int2);
	\draw[latex-latex] (Tag_0) -- node[sloped,above,midway] {$r_2$}(Int3);
	\draw[-latex,dashed,green!60!black!90] (REF_0) -- node[sloped,above,midway] {$r_0$}(Tag_0);
	
	\draw[ -latex,violet!60,font=\scriptsize,dotted] (REF_0) -- node[sloped,above,midway] {$d_{10}$}(Int1);
	\draw[ -latex,violet!60,font=\scriptsize,dotted] (REF_0) -- node[sloped,above,midway] {$d_{20}$}(Int2);
	\draw[ -latex,violet!60,font=\scriptsize,dotted] (REF_0) -- node[sloped,above,midway] {$d_{30}$}(Int3);
		
	\fill[red!70] (Tag_0) circle [radius=2pt];
	\node[font=\scriptsize,black,below] at (Tag_0) {$Tag$} ;
	\fill[green!60!black!90] (REF_0) circle [radius=2pt];
	
\end{tikzpicture}



%		
	\end{center}
\end{figure}
%
Folgende Nomenklatur und Symbole gelten für diesen Abschnitt:
\begin{itemize}[itemsep=0mm]
	\item	$r_{k}$ := Abstand vom Tag zur Antenne
	\item	$d_{kJ}$ := Abstand zur Landmarke
	\item	$N_0:=$ Menge der verfügbaren Antennen $N=\{1,..,8\}$
	\item	$N:=$ Menge der Antennen für die Optimierung verfügbar sind\footnote{d.h. ein Messergebnis liefern}($N \subseteq N_0$)
	\item	$N':=$ Menge der Antennen für die Optimierung ($N' \subseteq N$)
%	; Dabei ist $|N'| \geq 3$
%	\item	Es gilt $|N'| \geq |N| \geq |N_0|$   
	\item	$j$ ist der Index der Referenzantenne, es gilt $j = \{1,2,..,8\}$
	\item	$k$ ist der Index der Antennen einer Messung, es gilt $k = 1,2,..,|N'|-1$
\end{itemize}
%
Wir starten mit der Überlegung über den geometrischen Zusammenhang zwischen der Antennenposition von Antenne $k$ zu der Position des Tags $r_k$:
\begin{align}
	\label{eq:base_vactor}
	r_{k}^2 &= (x-x_k)^2+(y-y_k)^2+(z-z_k)^2
\end{align}
%
Diese Gleichung stellt die Euklidische Vektornorm dar und entspricht der Strecke Antenne-Tag. Für die Ermittelung einer Postion (mit drei Raumkoordinaten) sind drei Antennen Notwendig. Daraus ergibt sich:
%
\begin{itemize}[itemsep=0mm]
\item 3 Gleichunge n
\item 3 Unbekannte
\item Quadratisches Gleichungssystem
\end{itemize}
%
Das Gleichungssystem sieht wie folgt aus:
%
\begin{align}
	r_{1}^2 &= (x-x_1)^2+(y-y_1)^2+(z-z_1)^2 \nonumber\\
	r_{2}^2 &= (x-x_2)^2+(y-y_2)^2+(z-z_2)^2 \nonumber\\
	r_{3}^2 &= (x-x_3)^2+(y-y_3)^2+(z-z_3)^2 \nonumber
%	
\end{align}
%
Es ist trivial und wird in verschiedenen Beispielen gezeigt\footnote{z.B. \url{http://en.wikipedia.org/w/index.php?title=Trilateration&oldid=553215995}}, dass man die Koordinaten aus dem quadratischen Gleichungssystem unmittelbar berechnen kann. Es muss jedoch ein quadratisches Gleichungssystem gelöst werden, was zu den bekannten Problematiken führt, insbesondere der Ausschluss mehrdeutiger Ergebnisse. Der Messaufbau der \amedogmbh erlaubt die Verwendung von mehr als 3 Messwertgebern. Diese zusätzliche Informationen lassen sich für eine Linearisierung des Gleichungssystems verwenden. Dieser Ansatz wird für ein Modell im Rahmen dieser Arbeit verwendet und wird im Folgenden beschrieben.\\
%
Von den Antennen sind die Raumkoordinaten ($x,y,z-Koordinaten$) bekannt, bzw. wurden durch Kalibrierung \ref{sec:calibration} in einem vorherigen Schritt bestimmt. Wir können zusätzlich zu notieren:
%
\begin{equation}\label{eq:d_k0}
	d_{kj}^2= (x_k-x_0)^2+(y_k-y_0)^2+(z_k-z_0)^2
\end{equation}
%
Linearisierung des Modells. Dazu wird Gleichung~\ref{eq:base_vactor} in mehreren Schritten umgebaut. Zuerst wird eine neutrale Erweiterung durchgeführt und die Terme geschickt zusammengefasst. Das führt zu:
%
\begin{align}
	r_{k}^2 &= (x-x_k)^2+(y-y_k)^2+(z-z_k)^2 \nonumber \\
	&=(x-x_k+x_0-x_0)^2+(y-y_k+y_0-y_0)^2+(z-z_k+z_0-z_0)^2 \nonumber \\
	&=((x-x_0)-(x_k-x_0))^2+((y-y_0)-(y_k-y_0))^2+((z-z_0)-(z_k-z_0))^2 \nonumber \\ 
	%2 bin. Form
	&=(x-x_0)^2-2(x-x_0)(x_k-x_0)+(x_k-x_0)^2\underbrace{+\dots{}+\dots{}}_\text{y-\& z-Terme analog}
	\label{eq:tri_temp1}
%
\end{align}
%
Um Platz zu sparen sind die y- und z-Terme nicht explizit notiert. Sie ergeben sich durch einfaches Ersetzen der Indizes und werden im Finalen Modell eingefügt. Durch Umstellen von \eqref{eq:tri_temp1} erhalten wir:
\begin{align}
(x-x_0)(x_k-x_0)+\dots{}+\dots{}&=-\frac{1}{2}[r_k^2-(x_k-x_0)^2 -(x-x_0)^2 +\dots{} +\dots{}]\nonumber\\
(x-x_0)(x_k-x_0)+\dots{}+\dots{}&=\phantom{-}\frac{1}{2}[(x_k-x_0)^2 +(x-x_0)^2 +\dots{}+\dots{}-r_k^2]\nonumber
%
\end{align}
%
\begin{multline}\label{eq:rk_final}
(x-x_0)(x_k-x_0)+(y-y_0)(y_k-y_0)+(z-z_0)(z_k-z_0)= \\\frac{1}{2}[(x_k-x_0)^2+(x-x_0)^2-(y_k-y_0)^2+(y-y_0)^2
\\-(z_k-z_0)^2 +(z-z_0)^2-r_k^2]
\end{multline}
%
Vergleich von \eqref{eq:rk_final} mit \eqref{eq:d_k0} bringt: 
%
\begin{multline}
(x-x_0)(x_k-x_0)+(y-y_0)(y_k-y_0)+(z-z_0)(z_k-z_0)= \\\frac{1}{2}[\underbrace{(x_k-x_0)^2+(z_k-z_0)^2+(y_k-y_0)^2}_\text{\boldmath{$d_{kj}^2$}}
\\+\underbrace{(x-x_0)^2+(y-y_0)^2 +(z-z_0)^2}_\text{\boldmath{$r_j^2$}}-r_k^2]
\end{multline}
%
\begin{equation}
(x-x_0)(x_k-x_0)+(y-y_0)(y_k-y_0)+(z-z_0)(z_k-z_0)=\frac{1}{2}[d_{kj}^2+r_{j}^2-r_k^2]\label{eq:rk_final_simplyfied}
\end{equation}
mit 
\begin{equation}\label{eq:c_kj}
\mathbf{c_{kj}}=\frac{1}{2}[d_{kj}^2+r_{j}^2-r_k^2]
\end{equation}
können wir das lineare Gleichungssystem abschließend schreiben:
%
\begin{equation}
%\label{eq:final_trilateration_model}
\mathbf{0}=
\left(
	\begin{array}{ccc}
		x_1-x_j & y_1-y_j & z_1-z_j \\
		x_2-x_j & y_2-y_j & z_2-z_j \\
		x_3-x_j & y_3-y_j & z_3-z_j
	\end{array}
\right)
\left(
   \begin{array}{c}
	   x-x_j\\
	   y-y_j\\
	   z-z_j
   \end{array}
\right)
-
\left(
	\begin{array}{c}
		c_{1j}\\
		c_{2j}\\
		c_{3j}
	\end{array}
\right)
\end{equation}
%
Das Gleichungssystem entspricht ist linear und hat die allg. Form: $\mathbf{0} = \mathbf{Ax}+\mathbf{b}$ es lässt sich mit bekannten Methoden lösen.




{
\small
Folgende Nomenklatur und Symbole gelten für diesen Abschnitt:
\begin{itemize}[itemsep=0mm]
	\item	$N:=$Anzahl der Antennen $N=\{1,..,8\}$
	\item	$k$ ist der Index der Antennen, es gilt $k = \{1,2,..,N-1\}$
	\item	$r_{k}$ := Abstand vom Tag zur Antenne
	\item	$d_{k0}$ := Abstand zur Landmarke
	\item	fette Großbuchstaben stehen für Matrizen (bspw. $\mathbf{A}$)
	\item	fette Kleinbuchstaben stehen für Vektoren (bspw. $\mathbf{x}$)
	
\end{itemize}
%
Wie gezeigt werden konnte\footnote{Wochenbericht KW 20, Anhang B} ergibt sich für den Fall der Trilateration und der Annahme, dass vier Antennen Messwerte liefern, die Gleichung:
\begin{equation}\label{eq:final_trilateration_model}
0=
\left(
	\begin{array}{ccc}
		x_k-x_0 & y_k-y_0 & z_k-z_0 
	\end{array}
\right)
\left(
   \begin{array}{c}
	   x-x_0\\
	   y-y_0\\
	   z-z_0
   \end{array}
\right)
-
\left(
	\begin{array}{c}
		c_{k0}
	\end{array}
\right) 
\end{equation}
%
Dabei ist:
\begin{equation}\label{eq:c_k0}
	c_{k0}=\frac{1}{2}[d_{k0}^2+r_{0}^2-r_k^2]
\end{equation}
%
Ziel dieser Erweiterung ist es, einen Zusammenhang zwischen diesem Modell und der Wellenzahl zu erzeugen. Folgender Ansatz wird gewählt:
	\begin{equation}\label{eq:r_0_theta} r(\Theta)=\frac{\lambda}{2}\left(\frac{\Theta}{2\pi}+n\right),\\\lambda=\frac{c}{f}, n:= \text{Wellenzahl}
\end{equation}
%
%
\begin{shaded} 
Weiterhin ist $\Theta$ die gemessene Phase, die das PRPS-System liefert und $n$ die gesuchte Wellenzahl.\\
Durch einsetzen von \eqref{eq:r_0_theta} in \eqref{eq:c_k0}, erhalten wir:
\begin{equation}\label{eq:c_k0_extended}
	c_{k0}(\Theta_0, \Theta_k, n_0, n_k) =\frac{1}{2}\left[d_{k0}^2+\frac{\lambda^2}{4}\left(\frac{\Theta_0}{2\pi}+n_0\right)^2-\frac{\lambda^2}{4}\left(\frac{\Theta_k}{2\pi}+n_k\right)^2\right]
\end{equation}
%
Wir stellen Gleichung~\eqref{eq:c_k0_extended} um:
\begin{align}
%	
	c_{k0}(\Theta_0, \Theta_k, n_0, n_k) &= \frac{1}{2}\left\{d_{k0}^2+\frac{\lambda^2}{4}\left[\left(\frac{\Theta_0}{2\pi}\right)^2+2\frac{\Theta_0}{2\pi}n_0+n_0^2 \right.\right.\nonumber\\
	&\phantom{=}\; 
	\left.\left.-\left(\frac{\Theta_k}{2\pi}\right)^2-2\frac{\Theta_k}{2\pi}n_k-n_k^2\right]\right\}\\
%    
    &=\frac{1}{2}\left\{d_{k0}^2+\frac{\lambda^2}{4}\left[\left(\frac{\Theta_0}{2\pi}\right)^2-\left(\frac{\Theta_k}{2\pi}\right)^2 \right.\right.\nonumber\\
    &\phantom{=}\;
   	\left.\left.+2\frac{\Theta_0}{2\pi}n_0-2\frac{\Theta_k}{2\pi}n_k+n_0^2-n_k^2\right]\right\}\\
%	
	&=\frac{1}{2}d_{k0}^2+\frac{\lambda^2}{8}\left[\frac{1}{(2\pi)^2}\left(\Theta_0^2-\Theta_k^2\right) \right.\nonumber\\
	&\phantom{=}\;
	\left. +\frac{1}{\pi}\left(\Theta_0n_0-\Theta_kn_k\right)+\left(n_0^2-n_k^2\right)\right]\label{c_k0_rearragend}
\end{align}
%
Führen wir nun:
\phantomeq{c_{k0}(\Theta_0, \Theta_k, n_0, n_k)}{a_{0k} := \frac{1}{2}d_{k0}^2\nonumber}
\phantomeq{c_{k0}(\Theta_0, \Theta_k, n_0, n_k)}{a_1 := \frac{\lambda^2}{8}\nonumber}
\phantomeq{c_{k0}(\Theta_0, \Theta_k, n_0, n_k)}{a_2 := a_1\frac{1}{\pi}\nonumber}
\phantomeq{c_{k0}(\Theta_0, \Theta_k, n_0, n_k)}{a_{3k0} := a_1\frac{1}{(2\pi)^2}(\Theta_0^2-\Theta_k^2)\nonumber}
%
in Gleichung~\eqref{c_k0_rearragend} ein, erhalten die finale Form der Gleichung:
\begin{equation}
c_{k0}(\Theta_0, \Theta_k, n_0, n_k) = a_{0k}+a_1(n_0^2-n_k^2)+a_2(\Theta_0n_0-\Theta_kn_k)-a_{3k0}\label{c_k0_final_form}   
\end{equation}
%
Die Einführung der Konstanten macht zum Einen die Gleichung übersichtlicher. Zum Anderen können so, mit Blick auf eine spätere Softwareimplementation, Rechenschritte gespart werden. Das sollte sich positiv auf den späteren Berechnungsaufwand auswirken.\\
%
Im Weiteren erkennt man durch scharfes hinsehen das in Gleichung~\eqref{c_k0_final_form}, für $\Theta_k=\text{const.}$ \& $\Theta_0=\text{const.}$ gilt. Das resultiert aus der Tatsache, dass . Es ermöglicht uns zu schreiben:
\begin{equation}
c_{k0}(\Theta_0, \Theta_k, n_0, n_k) = c_{k0}(n_0, n_k)
\end{equation}
%
Im engeren Sinne einer mathematischen Funktion sollten wir die Parameter alle als Argument aufnehmen. Diese Form soll darstellen, welche Größen von Interesse sind. Im späteren Gebrauch wird diese Gleichung in der Optimierung eingesetzt werden.
Für unser Gleichungssystem aus\eqref{eq:final_trilateration_model} ergibt sich:
\begin{equation}\label{eq:wavenumber_trilateration_model}
0=
\left(
	\begin{array}{ccc}
		x_k-x_0 & y_k-y_0 & z_k-z_0 
	\end{array}
\right)
\left(
   \begin{array}{c}
	   x-x_0\\
	   y-y_0\\
	   z-z_0
   \end{array}
\right)
-
\left(
	\begin{array}{c}
		c_{k0}(n_0, n_k)
	\end{array}
\right)
\end{equation}
%
Betrachten wir nun \eqref{eq:wavenumber_trilateration_model} und setzen $N'=4$, d.h. wir verwenden 4 Antennen. Wir beschreiben die Konfiguration wie folgt: Antenne 0 ist die Referenz-Antenne und Antenne 0-3 sind Messwertgeber für die Phaseninformation. 
%
\begin{equation}\label{eq:wavenumber_trilateration_model_explicit}
0=
\underbrace{\left(
	\begin{array}{ccc}
		x_1-x_0 & y_1-y_0 & z_1-z_0 \\
		x_2-x_0 & y_2-y_0 & z_2-z_0 \\
		x_3-x_0 & y_3-y_0 & z_3-z_0 
	\end{array}
\right)}_{\textbf{A}}
\underbrace{\left(
   \begin{array}{c}
	   x-x_0\\
	   y-y_0\\
	   z-z_0
   \end{array}
\right)}_{\textbf{x}}
-
\underbrace{\left(
	\begin{array}{c}
		c_{10}(n_0, n_1) \\
		c_{20}(n_0, n_2) \\
		c_{30}(n_0, n_3)
	\end{array}
\right)}_{\textbf{b}}
\end{equation}
%
\begin{equation}
\mathbf{b}=
\left(
	\begin{array}{c}
		a_{01}+a_1( n_0^2-n_1^2)+a_2(\Theta_0n_0-\Theta_1n_1)-a_{310} \\
		a_{02}+a_1(n_0^2-n_2^2)+a_2(\Theta_0n_0-\Theta_2n_2)-a_{320} \\
		a_{03}+a_1(n_0^2-n_3^2)+a_2(\Theta_0n_0-\Theta_3n_3)-a_{330}
	\end{array}
\right)
\end{equation}

\end{shaded}
%
Das Ergebnis ist ein um $\Theta$ und $n$ erweitertes Gleichungssystem. Zusätzlich enthält  es mehrere geometrische Konstanten ($a_{0k}, k=\{1,..,N-1\}$), mehrere Phasen-Konstanten ($a_{3k0}, k=\{1,..,N-1\}$), sowie zwei allgemeine ($a_1$ und $a_2$). Allgemeiner formuliert ergibt sich:
%
\begin{multline}\label{eq:final_equation}
0=
\left(
	\begin{array}{ccc}
		x_k-x_0 & y_k-y_0 & z_k-z_0 
	\end{array}
\right)
\left(
   \begin{array}{c}
	   x-x_0\\
	   y-y_0\\
	   z-z_0
   \end{array}
\right) \\
-
\left(
	\begin{array}{c}
		a_{0k}+a_1(n_0^2-n_k^2)+a_2(\Theta_0k_0-\Theta_kn_k)-a_{3k0}
	\end{array}
	\right)
\end{multline}
%
Aus Gleichung~\eqref{eq:final_equation} ist durch eine geeignete Wahl von $N'=\{4,..,N\}$ sofort ersichtlich wie viele Veränderliche sich für eine gewählte Konstellation an Antennen ergeben. Für $k$ gilt in diesem Fall $k=\{1,..,N'-1\}$.\\
%
Beispielsweise ergibt sich für das Modell aus Gleichung~\eqref{eq:final_equation} mit $N'=4$, insgesamt 7 Variablen ($\mathbf{x},n_0,n_1,n_2,n_3$) . Analog würde sich für ein Modell mit allen 8 Antennen, 11 Variablen ($\mathbf{x},n_0,..,n_7$) ergeben.
}
{
Abschließend soll das das bisher verwendete Modell umgeschrieben werden, damit die Allgemeingültigkeit darin enthalten ist.
\begin{align}
%
%\mathbf{0}&=\mathbf{A}\mathbf{x}-\mathbf{b}\\
%
\mathbf{A}&=
\left(
	\begin{array}{cccccc}
		x_k-x_0 & y_k-y_0 & z_k-z_0 & \sum_{i=1,j=0}^{k}(-a_1\delta_{ij}) &  -a_2\Theta_0 & \sum_{i=1,j=0}^{k}(a_2\Theta_k\delta_{ij})
	\end{array}
\right)\nonumber\\
%
\mathbf{x}&=
\left(
   \begin{array}{c}
	   x-x_0\\
	   y-y_0\\
	   z-z_0\\
	   n_0^2-n_k^2\\
	   n_0\\
	   n_k
   \end{array}
\right)\nonumber\\
%
\mathbf{b}&=
	\begin{array}{c}
		a_{0k}-a_{3kj} 
	\end{array}
	= c_{kj}'\nonumber
\end{align}
%
Dabei steht $\delta_{ij}$ für den bekannten Kronecker-Operator und bedeutet:
\begin{equation*}
\delta_{ij} = \begin{cases}1 ~\text{für}~ i=j\\ 0 ~\text{für}~ i\neq j\end{cases}
\end{equation*}
%
Im Expliziten sehen die Matrix $\mathbf{A}$ und der Vektor $\mathbf{b}$, für denn Fall $N'=3$ und $k=\{1,2,3\}$, wie folgt aus:
%
\begin{multline}
\mathbf{A}=\\
\left(
	\begin{array}{cccccccccc}
		x_1-x_0 & y_1-y_0 & z_1-z_0 & -a_1 & 0 & 0 & -a_2\Theta_0 & a_2\Theta_3 & 0 & 0 \\
		x_2-x_0 & y_2-y_0 & z_2-z_0 & 0 & -a_1 & 0 & -a_2\Theta_0& 0 & a_2\Theta_3 & 0 \\
		x_3-x_0 & y_3-y_0 & z_3-z_0 & 0 & 0 & -a_1 & -a_2\Theta_0& 0 & 0 & a_2\Theta_3
	\end{array}
\right) \nonumber
\end{multline}
%
\begin{equation}
\mathbf{x}=
\left(
	\begin{array}{c}
		x-x_0	\\
		y-y_0	\\
		z-z_0	\\
		n_0^2-n_1^2	\\
		(\dots)	\\
		n_0^2-n_3^2	\\
		n_0 \\
		n_1	\\
		(\dots)	\\
		n_3	
	\end{array}
\right)\nonumber
\end{equation}
%
\subsubsection{Bemerkungen - Finales Modell}
%
Das Ergebnis ist eine $3\times10$ Matrix und ein $1\times10$ Vaktor. Es ist möglich diesem Modell eine beliebige Anzahl an Antennen hinzuzufügen. Fügt man eine Antenne zur Berechnung hinzufügen würde sich die Matrix $\mathbf{A}$ um zwei Spalten und eine Zeile erweitern, der Vektor $\mathbf{x}$ analog um 2 Zeilen.

}
%
\subsection{Erweiterte Betrachtung der Kondition}
Die vorgestellte erweiterte Form des Modells erleichtert Implementation und Verifikation, da große Teile vorberechnet und in geeigneten Strukturen abgelegt werden können. Diese statischen Teile des Models sind in Gleichung~\ref{eq:block_matrix_form} ersichtlich. Es sind nun auch die gemessenen Phasenwerte Teil des Modells, genauer: der Matrix $\mathbf{A}$. Im Folgenden werden die Auswirkungen auf die Kondition der Matrix betrachtet, wenn man diese Phasendaten hinzurechnet. Weiterhin wird Untersucht inwieweit die Zerlegung in Blockmatrizen und die Untersuchung der Kondition dieser eine Abschätzung der vollständigen Konditionszahl im Allgemeinen darstellt. 
%
\begin{equation}
\label{eq:block_matrix_form}
\mathbf{A}=\bigg( \mathbf{Z}\quad \mathbf{P}\quad \mathbf{V}\bigg)
\end{equation}
%
Dabei ist:
\begin{equation}
\mathbf{Z} \in \mathbb{R}^{3x3} \quad \mathbf{P} \in \mathbb{R}^{3x3} \quad \mathbf{V}\in \mathbb{R}^{4x3}
\end{equation}
%
Die Matrizen $\mathbf{Z}$ und $\mathbf{P}$ sind statisch. Hingegen enthält die Matrix $\mathbf{V}$ die gemessenen Phasenwerte $\Theta_k$ der Antennen für diese Konfiguration. \\
%
Die Abbildung~\ref{fig:CondNumberAnalyze} zeigt die bereits angestellte Untersuchung zu dieser Überlegung. Abbildung~\ref{fig:AnalyzeOf3x3} stellt die Konditionszahl der rein geometrischen $3\times3$-Matrix dar. In der Abbildung~\ref{fig:AnalyzeOf10x3} sehen wir die Kondition der erweiterten Matrix. Neben der geometrischen sind auch die beiden anderen Blockmatrizen in diese Konditionsbetrachtung eingeflossen. Als zusätzliche Angabe wird ist sind die Skalierungsfaktoren angegeben. Legt man beide Grafiken übereinander erkennt man:
\begin{enumerate}
\item Geometrisch gut konditionierte Konfigurationen (linke Grafik), bleiben im erweiterten Modell (rechte Grafik) weiterhin gut konditioniert.
\item Die Konditionszahl der \textit{schlechteste} ist wesentlich kleiner (ca. Faktor $10$) als im rein geometrischen Modell
\end{enumerate}
%
\begin{figure}[h!]
         \centering
	     \caption[Ergebnisse der Konditionsanalyse alle Permutationen]{Analyse der Konditionszahlen aller möglichen Matrizen für den Messaufbau; Die Konditionszahl ist für jede mögliche Permutation an Messantennen für eine Referenzantenne angegeben}\label{fig:CondNumberAnalyze}
         \begin{subfigure}[t]{0.45\textwidth}
                 \centering
                 \includegraphics[width=\textwidth]{img/fenceModell3x3.png}
                 \caption{Konditionszahl der rein geometrischen $3\times3$ Matrix normiert auf den größten vorkommenden Wert ($=2149,16$). Auf den Achsen finden sich der Index der Referenzantenne sowie die Nummer der Permutation. Die z-Achse enthält die normierte Kondition}
                 \label{fig:AnalyzeOf3x3}
         \end{subfigure}
\qquad        
         \begin{subfigure}[t]{0.45\textwidth}
                 \centering
                 \includegraphics[width=\textwidth]{img/fenceModell9x3.png}
                 \caption{Konditionszahl der $10\times3$ Matrix normiert auf den größten vorkommenden Wert ($=257,13$); In dieser Konfiguration sind die Konstanten ($a_1$ \& $a_2$) sowie die variablen, gemessenen Phasen $\Theta_k$ enthalten}
                 \label{fig:AnalyzeOf10x3}
         \end{subfigure}
%
\end{figure}
%
Aus der Grafik lässt sich entnehmen, dass es für jede Referenzantenne aus der Geometrie alleine gute Konfigurationen existieren. Aus diesen Erkenntnissen kann in späteren Aufbauten, die Position der Antennen optimiert werden. Diese Verfahren wird in Abschnitt~\ref{sec:Calibration_Optimaztion} weiter beschrieben. Die Grafik lässt erkennen, dass eine Konfiguration die ohne Phasendaten eine gut Kondition aufwies, eine ähnliche Kondition behält wenn diese Daten in der Modell einfließen.
%
%- Section 2.3 --------------------------------------------------------------
\subsection{Weitere Anwendung der Konditionszahl}
Weitere Anwendungen, die sich aus der Konditionszahl der Matrix ableiten, sind denkbar. Für die FPGA-Software ist, parallel zu diesem Projekt, eine intelligente Umschaltung der Antennen in der Planung. Die Kondition der geometrische Matrix verändert sich nach dem Kalibrieren nicht mehr. Dadurch und durch die oben beschriebenen Überlegungen kann statisch eine Abschätzung für die Konditionszahl, von zwei der drei Blockmatrizen, im Vorfeld erstellt werden. Die Konditionszahl dient zum Steuern der Umschaltung. Ordnet man die möglichen Konfiguration anhand ihrer Konditionszahl (niedrigste zuerst) in einer statischen Liste an so kann im FPGA eine einfache, schlaue Umschaltung implementiert werden. Diese würde immer dafür sorgen, dass Messdaten von einer Konfiguration bevorzugt werden, die eine niedrige Konditionszahl hat und somit relativ sicher zu einer guten Lösung führen. Diese Überlegungen werden im Rahmen dieser Arbeit nicht näher beschrieben.\\
Eine Weitere Anwendung ergibt sich für die Kalibrierung. Der Aufbau der Antennen kann unter Berücksichtigung der Kondition optimiert werden. Ziel der Optimierung wäre es durch eine geeignete Positionierung der Antennen, die Anzahl der Antennenpermutationen mit kleiner Konditionszahl zu maximieren.
%
%\subsection{Konkurierende Modelle}
% SuWi - Zeug 
%\lipsum[1-1]
%
\subsection{Betrachtung der Komplexität}
\label{sec:Komplexity2}
Im Folgenden wird eine Betrachtung der Komplexität des in Abschnitt~\ref{sec:model_developement} entwickelten Modells präsentiert. Diese Betrachtung ist wichtig für die Parametrisierung des Optimierungsverfahrens. Es wird eine Visualisierung des Fitness-Raums gezeigt und mit Benchmark-Funktionen verglichen.

\section{Software}
\label{sec:sw}
\lipsum[1-3]

\section{Hardware}
\label{sec:hw}
\lipsum[1-3]


%
% 3. ------------------------------------------------------------------------
\chapter{Ergebnisse und Erkenntnisse}
In diesem Kapitel wird die Implementation verifiziert, dafür wird die analytische Lösung für die Kalibrierung der Antennenposition der Lösung des CMA-ES-Verfahrens verglichen. Im Anschluss werden die Ergebnisse das Auffinden der Wellenzahl präsentiert.
%
%\section{Ergebnisse}
%
\section{Ergebnisse der Kalibrierung}
\label{sec:calibrationResults}
%
Es werden die Ergebnisse der Kalibrierung vorgestellt. Für eine der vermessenen Antennenkonfigurationen sind in der folgenden Tabelle die Koordinaten der Antennen gezeigt. Die Visualisierung der Konfiguration zeigt die Abbildung~\ref{fig:3dplot_coordinates}.
%
\begin{table} [ht]
	\begin{center}
		\begin{tabular}{cccccccc}
		      \textbf{Antenne} & \textbf{$x$} & \textbf{$y$} & \textbf{$z$} & \textbf{$d_{meas}$} & \textbf{$d_{result}$}& \textbf{$\varepsilon_{abs}$} & \textbf{$\varepsilon_{rel}$} \\
		      1 & 0.48		& -1.01	& 0.60 & 1,259 & 1,274& 0,015 & 1,14\% \\
		      2 & -0.77 	& -1.04 	& 1.34 & 1,894 & 1,872 & -0,022 & 1,19\% \\
		      3 & 1.52  	& -1.05 	& 1.37 & 2,334 & 2,307 & -0,027 & 1,15\% \\
		      4 & -0.92 	& -0.19 	& 1.32 & 1,661 & 1,628 & -0,033 & 2,01\% \\
		      5 & 1.92 		&  0.03 	& 1.39 & 2,399 & 2,375 & -0,024 & 1,01\% \\
		      6 & -0.55 	&  1.09 	& 1.43 & 1,851 & 1,887 & 0,036 & 1,93\% \\
		      7 & 1.06 		&  1.07 	& 1.35 & 2,055 & 2,031 & -0,024 & 1,19\% \\
		      8 & 0.45 		&  1.35 	& 0.67 & 1,574 & 1,578 & 0,004 & 0,26\% \\					
%
		\end{tabular}
		\caption[Finale Antennen Koordinaten]{Tabelle der finalen Antennenkoordinaten [m], berechnet mit dem in dieser Arbeit entwickelten Modell und dem SVD-Verfahren. Die Ergebnisse wurden auf zwei Nachkommastellen gerundet und sind identisch für beide Methoden. Die Spalte $d_{result}$ enthält die von der Berechnung gefundenen Distanz vom Referenzpunkt zur Antenne. Die Spalte $d_{meas}$ zeigt die gemessenen Werte. Die $\epsilon$-Spalten zeigen die Abweichung.}
		\label{tab:FinalCoords}
	\end{center}
\end{table}
%
Eine Berechnung mit dem evolutionären Verfahren dauerte ca. $170$~ms mit dem SVD-verfahren wurde eine Lösung in $\le 1$~ms gefunden. Für die in der Praxis eingesetzte Software wird es eine Implementation der Kalibrierung mit dem SVD-Verfahren geben. Das mit dieser Variante berechnete Ergebnis wird bei Bedarf mit einer Lösung des evolutionären Verfahrens verglichen. Das ermöglicht eine Build-In Verifikation der Kalibrierung.\\

\subsection{Visualisierung der Ergebnisse}
Für die in den folgenden Abbildungen präsentierten Ergebnisse wurden insgesamt $100$ Durchläufe des Algorithmus erstellt. Die Ergebnisse wurden mit einem vom Algorithmus selbst erstellten $\mu$ und $\lambda$ gefunden. In Abbildung~\ref{fig:Final_Calibration_Ant0_ES-boxes} wird eine statistische Auswertung der Ergebnisse gezeigt. In jedem Plot werden die Endwerte der Lösungen in einem sog. Boxplot gezeigt. Dabei wird die Verteilung mit Hilfe von Boxen dargestellt. Die Fähnchen der Boxen stellen die maximal- bzw. minimal-Werte dar. Die Größe der Boxen enthält das obere und untere Quartil der Daten, der horizontale Strich in der Box zeigt den Mittelwert der Daten. Ausreißer\footnote{Hier Werte die mindestens den $2$-Fachen Wert des oberen- bzw. unteren-Quartils aufweisen} in den Daten werden durch Punkte abseits der Box dargestellt.\\
%

Die Abbildung~\ref{fig:Final_Calibration_Ant0_ES-boxes} zeigt den Verlauf der drei Objektvariablen ($x,y,z$-Koordinaten) sowie die Entwicklung der Fitness und des mittleren Sigmas. Als Darstellungsart wird der Linienplot verwendet und die Verläufe einzelner Lösungen überlagern sich in diesem Plot. Das Abbruchkriterium war eine Fitness von $\leq 10^{-25}$.% Ist Fitness nicht etwas gutes und sollte maximiert werden?
Für Darstellungszwecke wurde die $x$-Achse nach $500$ Werten beschränkt, daher erreicht der Fitness-Plot diesen Wert in der Abbildung nicht. Der Verlauf ist typisch für den verwendeten Algorithmus. Deutlich zu erkennen ist eine Verbesserung des Ergebnisses mit steigender Zahl der Generationen. Der Verlauf der Variablen ist immer für den erfolgreichsten Nachkommen einer Generation dargestellt.\\
%

Abbildung~\ref{fig:Final_Calibration_Ant0_ES-Scatter} ist ein Scatter-Plot. Die Objektvariablen werden hier gegeneinander aufgetragen. Auf der Diagonalen befinden sich stets die Variable gegen sich selbst aufgetragen, daher zeigt sich dort immer eine Linie, bei streuenden Ergebnissen, bzw. ein einzelner Punkt, sollten die Ergebnisse nicht streuen. Der Plot ist praktisch um die Implementation des Algorithmus zu verifizieren. Er lässt Rückschlüsse auf Abhängigkeiten und Einflüsse der Objektvariablen zu. So können die Ergebnisse mit den Erwartungen an die Verläufe verglichen werden.\\
%
%---------------------------------------------------------------
%
\begin{figure}[!ht]
  \begin{center}
   \caption[Box-Plot der Endergebnisse der Kalibierung]{Boxplot des Kalibierergebnis aus $100$ Durchläufen. Im oberen Plot sind die x,y,z-Koordinaten gezeigt, diese landen in allen Durchläufen auf dem selben Ergebnis. Was nicht verwundert, das Problem ist eines der Art \glqq{}drei Gleichungen und drei Unbekannte\grqq. Die Streuung der Lösung zeigt sich in der Breite der Linien. Die unteren vier Plots zeigen die Anzahl der Evaluationen der Fitness-Funktion, den finalen Funktionswert, das Sigma für die Variablen und die Entfernung zum Referenzpunkt (v.l.n.r.). Das Ergebnis sind die Koordinaten für Antenne 1}
    \label{fig:Final_Calibration_Ant0_ES-boxes}
    \includegraphics[width=0.9\textwidth]{img/calibration/calibration_ant0-boxes.png}
  \end{center}
 
%
\end{figure}
%
%---------------------------------------------------------------
%
\begin{figure}[!ht]
  \begin{center}
    \caption[Linien-Plot der Endergebnisse der Kalibierung]{Zu erkennen ist, dass nach ca. 300 Evaluationen der Zielfunktion keine großen Änderungen der Variablen zu erkennen sind. Bis zum erreichen des Abbruchkriteriums (Function Value $\leq10^{-25}$) werden noch ca 400 Evaluationen benötigt, vgl. korrespondierender Boxplot.}
    \label{fig:Final_Calibration_Ant0_ES-Lines}  
    \includegraphics[width=0.9\textwidth]{img/calibration/calibration_ant0-lines.png}
  \end{center}
%  
\end{figure}
%---------------------------------------------------------------
%
\begin{figure}[!ht]
  \begin{center}
  
    \caption[Kalibierung Scatter-Plot]{Scatter-Plot der Ergebnisse der evolutionären Kalibrierung. Die Endergebnisse streuen in keiner Dimension, das wird aus dieser Darstellung deutlich.}
    \label{fig:Final_Calibration_Ant0_ES-Scatter}  
    \includegraphics[width=0.9\textwidth]{img/calibration/calibration_ant0-scatter.png}
  \end{center}
%  
\end{figure}
%---------------------------------------------------------------
%
\begin{figure}[!ht]
     \centering
     \begin{subfigure}[t]{0.45\textwidth}
             \centering
             \includegraphics[width=\textwidth]{img/calibration/aborted_calibration_ant0-boxes.png}
             \caption{Statistisch verteilte Endwerte für die Koordinaten der Kalibrierung.}
             \label{fig:abortedFinal_Calibration_Ant0_ES-boxes}
     \end{subfigure}
%
\qquad         
%
     \begin{subfigure}[t]{0.45\textwidth}
             \centering
             \includegraphics[width=\textwidth]{img/calibration/aborted_calibration_ant0-lines.png}
             \caption{Linienplot der bei $140$ Evaluationen abgebrochenen Verläufe. Gut zu sehen ist der Verlauf der Objektvariablen, die sich von Generation zu Generation dem realen Wert nähern.}
             \label{fig:abortedFinal_Calibration_Ant0_ES-Lines}
     \end{subfigure}
%
\\
%
     \begin{subfigure}[t]{0.4\textwidth}
             \centering
             \includegraphics[width=\textwidth]{img/calibration/aborted_calibration_ant0-scatter.png}
             \caption{Statistische Streuung um einen Mittelwert. So in etwa kann man die Lösungen der Komplexen Probleme erwarten.}
             \label{fig:abortedFinal_Calibration_Ant0_ES-Scatter}
     \end{subfigure}
%
     \caption[Statistisch verteilte Ergebnisse der Kalibrierung mittels ES]{Analog zu den Abbildungen~\ref        {fig:abortedFinal_Calibration_Ant0_ES-Lines}, \ref{fig:abortedFinal_Calibration_Ant0_ES-boxes} und \ref{fig:abortedFinal_Calibration_Ant0_ES-Scatter} zeigen die Plots die gleichen Darstellungen. Hier gezeigt wird, wie sich eine statistische Verteilung in den Plots manifestieren würde. Um das zu demonstrieren wurde das Abbruchkriterium auf lediglich $150$ Evaluationen der Zielfunktion eingestellt. Zu diesem Zeitpunkt können die Objektvariablen bereits einen passablen Wert erreicht haben oder noch abweichende Werte aufweisen (vgl. \ref{fig:Final_Calibration_Ant0_ES-Lines}).}
     \label{fig::abortedFinal_Calibration_Ant0_ES}
\end{figure}
%
\begin{figure}[ht!]
         \centering
         \includegraphics[width=0.7\textwidth]{img/calibration/calibration_results.png}
         \caption[Visualisierung des Kalibrierendergebnis]{Visualisierung des Kalibrierendergebnis. Abgebildet sind die gefundenen Antennenkoordinaten (Punkte) in drei Raumansichten. Die zusätzliche, dreidimensionale  Ansicht dient der Übersicht. Das Ergebnis und die reale Anordnung decken sich sehr gut. Siehe Tabelle~\ref{tab:FinalCoords}}
         \label{fig:3dplot_coordinates}
%
\end{figure}

Das Modell aus \ref{sec:model_developement} wurde in verschiedenen Experimenten untersucht. Dabei wurden die Parameter der Optimierung variiert und die Auswirkungen untersucht.

Es wird analysiert, wie gut die Lösbarkeit des unregistrierten Problems ist. Dazu werden zuerst für jede Referenzantenne immer eine mögliche Konfiguration gewählt und das Ergebnis aus $M$-Durchläufen untersucht. Im Anschluss %MENE: trinken wir Bier! %CG: DAS MACHEN WIR!

Zunächst werden die Ergebnisse der Experimente für idealen Messwerte vorgestellt, dannach wird dieselbe Darstellung für reale Messdaten vorgenommen. Die Darstellung der Ergebnisse ist eine kondensierte Form der Präsentation. Sie Zeigen 
%
%-----------------------------------------------------------------------------
%
\section{Ergebnisse des Trilaterationsmodells}
%
\label{sec:Results1}
%
Dieser Abschnitt enthält die Ergebnisse der evolutionären Optimierung für das entwickelte Modell der Trilateration. Es werden Experimente definiert und ihre Resultate als Falschfarbenbild (Heatmap) dargestellt. Es werden die Ergebnisse von künstlichen Eingabedaten und realen Messdaten verglichen. Im Anschluss wird der Verlauf der Optimierung auf Besonderheiten hin untersucht. Abschließend wir die Performance der evolutionären Lösung visualisiert.
%
\subsection{Experimente}
%
Die Tabellen~\ref{tab:experiments} und~\ref{tab:experiments2} geben an welche Parameter wie variiert wurden. Diese Experimente wurde für die Präsentation der Ergebnisse in dieser Arbeit entworfen. Alle wurden mit der gleichen Objektfunktion durchgeführt. Diese trägt den Namen '\textit{WholeTomatoMkII}'\footnote{Implementation des in dieser Arbeit entwickelten Trilaterationsmodells}. Der eingesetzte Algorithmus ist das CMA-ES. Andere evolutionäre Strategien werden nicht in separaten Experimenten untersucht und präsentiert. Die spätere Darstellung kondensieren die Parameter und Erkenntnisse die in dieser Arbeit gewonnen wurden. Ziel ist es insgesamt die Lösung zu quantifizieren. Wie im Rahmen der Komplexitätsuntersuchung beschrieben ist die Fragestellung die hier bearbeitet wird recht komplex. Es werden daher die Parameter der Evolutionsstrategie variiert. Dabei wird untersucht ob und wie sich die Güte der Lösung durch diese Parameter verbessern lässt. In den Abbildungen~\ref{fig:results1} und~\ref{fig:results2} sind die Ergebnisse der Experimente als Heatmap dargestellt.

%
\begin{table} [ht!]	
	\caption[Experimente - Ideale Messdaten]{Aufstellung der Experimente die in diesem Abschnitt vorgestellt werden. Die Gruppengröße wird bei jedem Experiment Variiert. Jedes Unterexperiment erhält seinen eigenen Namen. Daher steigt der Name der Experimente bei jedem Eintrag entsprechend der Anzahl der untersuchten Gruppen an. In dieser Untersuchung wurden auch Parameter des Evolutionären Algorithmus variiert. Es werde $\mu$ und $\lambda$ variiert. Der Einfluss der Populations- und Gruppengröße wird damit gleichzeitig untersucht. Es ergeben sich $100$ einzelne Experimente.}
	\label{tab:experiments}
	\begin{center}
		\begin{tabular}{ccccc}
			\textbf{Name} 	& \textbf{Trials $M$} 	& \textbf{Gruppengröße $L$} & \textbf{$\mathbf{\mu}+\mathbf{\lambda}$}\\
			E2000			& 50 				&    1-10		&  (30+100) \\
			E2010			& 50 				&    1-10		&  (40+150) \\
			E2020			& 50 				&    1-10		&  (50+200) \\
			E2030			& 50 				&    1-10		&  (60+250) \\
			E2040			& 50 				&    1-10		&  (70+300) \\			                        
			E2050			& 50 				&    1-10		&  (80+350) \\			                        
			E2060			& 50 				&    1-10		&  (90+400) \\			                        
			E2070			& 50 				&    1-10		&  (100+450) \\			                        
			E2080			& 50 				&    1-10		&  (110+500) \\			                        
			E2090			& 50 				&    1-10		&  (120+550) \\			                        
%			
		\end{tabular}
	\end{center}
\end{table}
%

\begin{table} [ht!]	
	\caption[Experimente - Reale Messdaten]{Für bestmögliche Vergleichbarkeit wurden diese Experimente mit den gleichen Parametern durchgeführt wie die für ideale Daten. In diesem Experiment werden die realen Messwerte des PRPS in der Positionsberechnung verwendet. }
		\label{tab:experiments2}
	\begin{center}
		\begin{tabular}{ccccc}
			\textbf{Name} 	& \textbf{Trials $M$} 	& \textbf{Gruppengröße $L$} & \textbf{$\mathbf{\mu}+\mathbf{\lambda}$}\\
			E3000			& 50 				&    1-10		&  (30+100) \\
			E3010			& 50 				&    1-10		&  (40+150) \\
			E3020			& 50 				&    1-10		&  (50+200) \\
			E3030			& 50 				&    1-10		&  (60+250) \\
			E3040			& 50 				&    1-10		&  (70+300) \\			                        
			E3050			& 50 				&    1-10		&  (80+350) \\			                        
			E3060			& 50 				&    1-10		&  (90+400) \\			                        
			E3070			& 50 				&    1-10		&  (100+450) \\			                        
			E3080			& 50 				&    1-10		&  (110+500) \\			                        
			E3090			& 50 				&    1-10		&  (120+550) \\			                        
%			
		\end{tabular}
	\end{center}
\end{table}
%
%-----------------------------------------------------------------------------
%
\subsection{Ergebnisse ideale Messwerte}
%
\begin{figure}[h!]
	\centering
	\caption[Ergebnis-Heatmap - Ideale Messwerte]{ Ergebnisse verschiedener Experimente mit idealen Messwerten. Die Abbildung zeigt farbkodiert den Betrag der Abweichung vom wahren Wert der Entfernung (Ausgemessen). Zu beachten gilt, dass die Farbskala eine unterschiedliche Skalierung hat und somit jede Antenne separat betrachtet werden muss. Die Parameter die in diesen Experimenten variiert wurden sind die Populationsgröße ($\mu$ und $\lambda$) und die Gruppengröße $L$. Mit der $x$-Achse steigt die Populationsgröße an. Die $y$-Achse trägt die steigende Gruppengrößen auf.}
	\label{fig:results1}
	\includegraphics[width=0.65\textwidth]{img/result.png}
\end{figure}
% 
Um die idealen Messwerte zu generieren wurde ein eigenes Modul entwickelt. Es verwendet die in Kapitel~\ref{sec:PhaseCalculation} beschrieben Formeln. Es wurden eine Reihe von Punkten definiert, von denen die idealen Phasenwerte ermittelt wurden. Um im Rahmen dieser Arbeit in der Präsentation der Ergebnisse konsistent zu bleiben, wird als Punkt der Ursprung der Kalibrierung genutzt. Dieser muss von dem Algorithmus wiedergefunden werden. Seine Koordinaten wurden bereits in Kapitel~\ref{sec:calibration} besprochen. Die in den Abbildungen gezeigte Darstellung stellt die Abweichung von dem korrekten Wert dar.
%
$$
\varepsilon=~|~d_{true}-d_{found}~|
$$
%
Die Dimension von $\varepsilon$ ist dementsprechend [m]. Für die Visualisierung wurden auf den bestimmten $M$-Lösungen der Median-Wert (mittlere) genommen. 
%
\subsection{Ergebnisse ideale Messwerte}
%
Die idealen Messewerte zeigen insgesamt ein gutes Ergebnis (Abbildung~\ref{fig:results1}). Im Mittel werden die besten Ergebnisse, bei mittlerer Gruppengröße und großer Populationsgröße gefunden. Es zeigt sich, dass eine steigende Populationsgröße das Ergebnis in der Regel verbessert. Bei der Gruppengröße kann ab einer bestimmten Größe keine weitere Verbesserung festgestellt werden. Eine Besonderheit zeigt sich bei den Antennen $3$ und $5$. Diese zeigen ein umgekehrtes Verhalten, bei niedriger Gruppengröße und kleiner Populationsgröße zeigen sie die geringste Abweichungen. Woran das liegt konnte nicht weiter untersucht werden. Die meisten Lösungen zeigen eine Abweichung von unter einer Wellenlänge, die wir mit $\lambda\simeq35$~cm angeben konnten. Das erlaubt prinzipiell eine korrekte Berechnung der Wellenzahl.\\

In Abbildung~\ref{fig:results4} wurden die Ergebnisse anders eingefärbt, es werden Abweichungen über $35$~cm konsequent rot eingefärbt. Damit zeigt sich deutlicher, dass eine recht große Population und eine geeignete Gruppengröße gewählt werden muss um vernünftige Resultate zu erhalten.
%
%-----------------------------------------------------------------------------
%
\subsection{Ergebnisse reale Messwerte}
%
\begin{figure}[h!]
	\centering
	\caption[Ergebnis-Heatmap - Reale Messwerte]{ Ergebnisse mit realen Messwerten. Die Antennen $2$ und $6$ lieferten bei dieser Messung keine Messdaten. Daher fehlen diese Plots in der Grafik. Das Antennen keine Daten liefern ist der Regelfall den man in der Praxis begegnet. Das Ergebnis ist in etwa identisch mit dem der idealen Messwerte. Auch hier finden sich gute Ergebnisse bei mittlerer Gruppen- und großer Populationsgröße.}
	\label{fig:results2}
	\includegraphics[width=0.65\textwidth]{img/resultRealData.png}
\end{figure}
%
Die Ergebnisse (Abbildung~\ref{fig:results2}) für reale Messwerte zeigen ein ähnliches Bild wie die idealen Messwerte. Eine steigende Gruppen- und Populationsgröße verbessert das Ergebnis in der Regel. Es fehlen die Plots der Antenne $2$ und $6$. Das liegt daran, dass diese beiden Antennen keine Messdaten lieferten. Es wurde der selbe Punkt ausgewählt, wie bei Vorstellung der ideal Messergebnisse. Damit sind die Ergebnisse unmittelbar vergleichbar. Genau wie bei den idealen Messdaten wurde der Ursprung der Kalibrierung als Messpunkt genommen. Das Ergebnis der realen Messwerte mutet sogar sicherer an, als das der idealen Phasenwerte. Der Grund dafür kann nicht exakt angegeben werden. Man kann sich dazu überlegen, da jeder Messwert ein Messrauschen enthält. Dieses Rauschen wird Teil des Modell und variiert dadurch unmittelbar die Fitnesslandschaft. Das Aussehen der Fitnesslandschaft wurde in Kapitel~\ref{sec:Komplexity2} untersucht. Weitere Untersuchungen und größere Messreihen müssen dieses Verhalten bestätigen.\\

Auch die neu kolorierte Darstellung der Ergebnisse in Abbildung~\ref{fig:results5} zeigt, wie die der idealen Messwerte, eine Verbesserung der Positionsbestimmung für steigende Populations- und Gruppengrößen.
%
%-----------------------------------------------------------------------------
%
\subsection{Evolutionsverlauf - real vs ideal}
%
Abschließend ist in Abbildung~\ref{fig:results3} der Verlauf der letzten beiden Experimente $2099$ und $3099$ gezeigt. Dort werden in $3$ Plots die Lösungen charakterisiert und die statistischen Eigenschaften sowie der Verlauf der Optimierung quantifiziert. Zu erkennen ist, dass sie die beiden Ergebnisse grundsätzlich ähneln. Unterschiede zeigen sich nur in Einzelheiten. So zeigt sich z.B., dass die Streuung der Parameter bei idealen Messwerten geringer ist, jedoch mehr Evaluationen notwendig waren. Allgemein zeigt sich der erwartete Verlauf einer evolutionären Optimierung.
%
%-----------------------------------------------------------------------------
%
\subsection{Performance}
%
Abschließend soll die Performance der Experimente gezeigt werden. Dabei wird die gleiche farbkodierte Darstellung verwendet, die auch bei den anderen Ergebnissen verwendet wird. Die Ausführungszeiten der Experimente sind in der Abbildung~\ref{fig:results6} und Abbildung~\ref{fig:results7} als Falschfarbenbild dargestellt. Die maximale Ausführungszeit betrug $\sim1200$~ms. Durch eine vernünftige Wahl von Populations- und Gruppengröße würde ein Ergebnis zwischen $400-600$~ms in Anspruch nehmen. Die Performance lässt sich durch Multithreading weiter verbessern.
%

%\textsc{First Letter}
%
%or
%
%{\scshape First Letter}

\begin{figure}[!ht]
	\centering
	\caption[Limitierte Ergebnisse - Ideale Messwerte]{ Alternative Visualisierung der zuvor gezeigten Ergebnisse, \textit{idealer} Messdaten. Zur besseren Übersicht wurde die Farbskala auf eine Wellenlänge reduziert. Abweichungen $\ge35$~cm werden rot eingefärbt. Gut $50\%$ der Werte führen nicht zu einen zufriedenstellenden Ergebnis. }
	\label{fig:results4}
	\vspace{3mm}
	\includegraphics[width=0.65\textwidth]{img/limitedIdeal.png}
	

\end{figure}
%
\begin{figure}[!ht]
	\centering
	\caption[Limitierte Ergebnisse - Reale Messwerte]{Alternative Visualisierung der zuvor gezeigten Ergebnisse, \textit{realer} Messdaten.  Abweichungen $\ge35$~cm werden rot eingefärbt. Gut $50\%$ der Werte führen nicht zu einen zufriedenstellenden Ergebnis. Das Ergebnis gleicht dem der idealen Messdaten. }
	\label{fig:results5}
	\vspace{3mm}
	\includegraphics[width=0.65\textwidth]{img/limitedReal.png}
\end{figure}
%
\begin{landscape}
\begin{figure}[!ht]
	\caption[Evolutionsverlauf der Ergebnisse]{ Diese Grafik zeigt den Verlauf der Evolution. Es werden die Beiden letzten Experimente $2099$ (opben, ideale Messdaten) und $3099$ (unten, reale Messdaten) gezeigt. Diese Plots dienen nur der Einschätzung über den generellen Verlauf der Evolution. Es ist nicht sinnvoll sie für alle Experimente hier darzustellen. Anhand des Boxplots (Mitte) erkennend man, dass die Resultate für ideale Messwerte nicht so stark streuen, die Lösung der realen Messdaten ist den der idealen mind. Ebenbürtig. Es zeigt sich sogar, dass weniger Evaluationen der Zielfunktion bei den realen Werten nötig waren.}
	\label{fig:results3}
	\vspace{3mm}
	\centering
	\begin{subfigure}[t]{0.45\textheight}
	     \centering
	     \includegraphics[width=\textwidth]{img/evo/lines2089.png}
	             \caption{Linienplot zur Analyse der Variablen und Optimierungsverlaufs.}
	%             \label{fig:abortedFinal_Calibration_Ant0_ES-boxes}
	\end{subfigure}
	\qquad
	\begin{subfigure}[t]{0.45\textheight}
		\centering
	     \includegraphics[width=\textwidth]{img/evo/boxes2089.png}
	     	    \caption{Boxplot zur Veranschaulichung der Lösungsstatistik }
	%			\label{fig:abortedFinal_Calibration_Ant0_ES-boxes}
	\end{subfigure}
	\qquad
	\begin{subfigure}[t]{0.45\textheight}
			\centering
	   \includegraphics[width=\textwidth]{img/evo/Scatter2089.png}
	   	       \caption{Streuplot zur Untersuchung von Abhängigkeiten der Parameter}
	%			\label{fig:abortedFinal_Calibration_Ant0_ES-boxes}
	\end{subfigure}
	\vspace{5mm}
\\
	\centering
	\begin{subfigure}[t]{0.45\textheight}
	     \centering
	     \includegraphics[width=\textwidth]{img/evo/lines4089.png}
	             \caption{Linienplot zur Analyse der Variablen und Optimierungsverlaufs.}
	%             \label{fig:abortedFinal_Calibration_Ant0_ES-boxes}
	\end{subfigure}
	\qquad
	\begin{subfigure}[t]{0.45\textheight}
		\centering
	     \includegraphics[width=\textwidth]{img/evo/boxes4089.png}
	     	    \caption{Boxplot zur Veranschaulichung der Lösungsstatistik }
	%			\label{fig:abortedFinal_Calibration_Ant0_ES-boxes}
	\end{subfigure}
	\qquad
	\begin{subfigure}[t]{0.45\textheight}
			\centering
	   \includegraphics[width=\textwidth]{img/evo/Scatter4089.png}
	   	       \caption{Streuplot zur Untersuchung von Abhängigkeiten der Parameter}
	%			\label{fig:abortedFinal_Calibration_Ant0_ES-boxes}
	\end{subfigure}

\end{figure}
\newpage
\end{landscape}
%
\begin{figure}[!ht]
	\centering
	\caption[Performance Ergebnisse - Ideale Messwerte]{ Visualisierung der Performance. Die Resultate wurde in einer virtuellen Umgebung generiert. Aufgrund limitierter Ressourcen (nur ein Prozessor verfügbar). Konnte die Multithreading-Fähigkeit nicht genutzt werden. Die Ausführungszeiten für vernünftige Konfigurationen liegen bei $\sim800$~ms ohne Threading. Gut zu erkennen ist, dass mit steigendem Aufwand die Ausführungszeit, wie erwartet steigt. }
	\label{fig:results6}
	\vspace{3mm}
	\includegraphics[width=0.65\textwidth]{img/resultstiming.png}
\end{figure}
%
%
\begin{figure}[!ht]
	\centering
	\caption[Performance Ergebnisse -- Reale Messwerte]{ Keine Überraschungen hier. Mit zunehmender Komplexität zeigt sich eine Zunahme der Ausführungszeit. Gleiches Ergebnis wie bei den idealen Messdaten. }
	\label{fig:results7}
	\vspace{3mm}
	\includegraphics[width=0.65\textwidth]{img/resultstimingreal.png}
\end{figure}
%
%
% 4. ------------------------------------------------------------------------
\chapter{Diskussion}
\lipsum[1-5]
%
Die in dieser Arbeit bearbeitete Aufgabestellung ist sehr komplex. Die erreichten Ergebnisse erfüllen die Anforderungen nicht ganz. Der Ansatz das Problem über evolutionäre Verfahren zu lösen funktioniert zuverlässig. Auch auf limitierten Ressourcen (wenig Rechenleistung) konnten brauchbare Ergebnisse in einer akzeptablen Zeit gefunden werden. Die Performance kann mit den hier beschriebenen Methoden (Threading) weiter gesteigert werden. Das erhöht die Sicherheit des Messergebnisses. In Abbildung~\ref{fig:Requirements_reached} wird die Anforderungsspinne (Abbildung~\ref{fig:Requirements}) mit den erreichten Resultaten in Deckung gebracht.\\
%
\begin{figure}[ht]
         \centering
         \caption[Anforderungsspinne]{ Grafische Übersicht der Ergebnisse gegen die gestellten Anforderungen (blau) an diese Arbeit. Der grüne Bereich stellt die erreichten Ziele dar. Wie bereits Diskutiert, werden die Anforderungen an die Genauigkeit nicht gut erfüllt. Auch die Performance ist nicht zufriedenstellend erreicht, die Gründe wurden bereits angegeben. In den anderen Bereichen zeigt sich eine gute Deckung mit den Anforderungen. }
         \vspace{2mm}
         \label{fig:Requirements_reached}
         \input{diagrams/spider_requirements_reached.tex}
         \vspace{5mm}
\end{figure}
%

Wie in Kapitel~\ref{sec:Results1} beschrieben liefern sowohl reale Messdaten als auch künstliche, ideale Eingabewerte korrekte Ergebnisse, wenn man die Evolutionsparameter passend wählt. Brauchbare Konfigurationen wurden präsentiert. Es ist nicht nachgewiesen, dass man daraus eine Allgemeingültigkeit für die Zukunft ableiten kann. Es wurde in dieser Arbeit nur ein Messaufbau vermessen. Die Lösung konnte ohne wesentliche Einschränkungen und Registrierung durchgeführt werden. Der in dem verwendeten Aufbau abgedeckte Messbereich umfasse in etwa $30 m^3$. Damit ist erwiesen, dass sich die Methode für ein großes Messvolumen eignet\\
%

Die dabei verwendete Kalibrierroutine wurde ausführlich vorgestellt (Kapitel~\ref{sec:calibrationResults}). Das Softwaremodul für die Kalibrierung komm bereits in anderes Software der \amedogmbh zum Einsatz. Die Integration in die bestehende Softwarestrukturen war einfach. \\
%

Die Software wurde mit \cpp11 entwickelt und ist somit auf dem Stand der Technik. Sie lässt sich einfach auf eine andere Plattform portieren, die das Buildtool \textit{CMake} unterstützt. \\

%
Über die Ziele dieser Arbeit hinaus konnten Vorschläge gemacht werden, die Problemstellung einer automatischen Antennenumschaltung zu lösen. 
%
% 5. ------------------------------------------------------------------------
\chapter{Schluss}
\lipsum[1-2]
%
\section{Verbesserungen}
%
Wie in den Ergebnissen gezeigt, können unter gewissen Umständen (schlechte Konditionierung, geringe Anzahl an Antennen) unzureichende Ergebnisse erzielt werden. Zur Zeit wird dieser Umstand durch eine häufigere Anzahl an Lösungsversuchen, sowie Verwendung verschiedener Antennenkombinationen kompensiert. Dadurch erhöht sich jedoch die Ausführungszeit, bzw. die Zeit bis ein Ergebnis Vorliegt auf zum Teil mehrere Sekunden. Das liegt außerhalb der Anforderungen. \\
%
Im Rahmen dieser Arbeit konnte eine automatische Kalibrierung nicht mehr entwickelt und erprobt werden. Die vorgestellten Ergebnisse sind prinzipiell dazu geeignet auf dieser Basis ein solches System zu entwickeln.\\


Die Ergebnisse dieser Arbeit weisen noch relativ große Abweichungen in der berechneten z-Koordinate auf. Warum diese auftreten konnte nicht mehr abschließend geklärt werden.

%
\section{Ausblick}
\label{sec:Calibration_Optimaztion}
%
Das Modell, das in dieser Arbeit entwickelt wurde, ermöglicht eine Anwendung Abseits der Positionsberechnung. Es erlaubt ein anderes Problem zu Lösung, dass durch die Freie Anordnung der Antennen entsteht. Wenn mehrere Tags im Raum identifiziert werden, muss die Steuerung der Antennenumschaltung (diese bestimmt von welcher Antenne gerade gelesen wird - es können nicht gleichzeitig alle Antennen gelesen werden) zur Zeit alle Antennen in einem Round-Robin-Verfahren nach einer gewissen Zeit umschalten. Nach einer Umschaltung kann es dazu kommen, dass keine der Antennen einen der zuvor identifizierten Tags "sieht". Da die Umschaltung zur Zeit zufällig zur nächsten Antenne spring, werden auch Antennen genommen, die keine günstige Positionsberechnung erlauben.\\
Die Bestimmung der Kondition für jede Konfiguration aus vier Antennen kann dazu verwendet werden eine gewisse Intelligenz beizutragen. Dazu würden aus dem gewählten Antennenaufbau die Antennenkombinationen ihrer Konditionszahl nach Aufsteigend\footnote{zur Erinnerung, gute Kondition = kleine Konditionszahl}] sortiert und diese Antennenkombinationen von der Antennenumschaltung bevorzugt. Dadurch werden stets gut konditionierte Kombinationen gewählt.
%
%----------------------------------------------------------------------------
% Appendix ------------------------------------------------------------------
%----------------------------------------------------------------------------

\begin{frame}[noframenumbering]
  \frametitle{Appendix I} 
%	bla bla bla bla bla bla bla bla \\ bla bla bla bla  bla bla bla bla 
%	\begin{right} 
	\centering
	\includegraphics[page=2, width=.6\textwidth]{../img/mindmap.pdf}
\end{frame}
%------------------------------------------------------
\begin{frame}[noframenumbering]
  	\frametitle{Appendix II}
%  
	Wir notieren:
%	
	\begin{align}
		r_1^2&= (x_1-x )^2 + (y_1-y )^2 + (z_1-z )^2\\
		r_2^2&= (x_2-x )^2 + (y_2-y )^2 + (z_2-z )^2\\
		r_3^2&= (x_3-x )^2 + (y_3-y )^2 + (z_3-z )^2\\
		\nonumber\\
		r_0^2&= (x_0-x )^2 + (y_0-y )^2 + (z_0-z )^2\\
		\nonumber\\
		r_{0k}^2&= (x_0-x_k )^2 + (y_0-y_k )^2 + (z_0-z_k )^2 \\
		\nonumber\\
		r(\Theta_k,n_k)&=\frac{\lambda}{2}\left(\Theta_k+n_k\right)
%		
	\end{align}
%
\end{frame}
%------------------------------------------------------
\begin{frame}[noframenumbering]
  	\frametitle{Appendix III}
%
Linearisierung des Modells.
%
\begin{align}
	r_{k}^2 &= (x-x_k)^2+(y-y_k)^2+(z-z_k)^2 \nonumber \\
	&=(x-x_k+x_0-x_0)^2+(y-y_k+y_0-y_0)^2+(z-z_k+z_0-z_0)^2 \nonumber \\
%	&=((x-x_0)-(x_k-x_0))^2+((y-y_0)-(y_k-y_0))^2+((z-z_0)-(z_k-z_0))^2 \nonumber \\ 
	%2 bin. Form
	&=(x-x_0)^2-2(x-x_0)(x_k-x_0)+(x_k-x_0)^2\underbrace{+\dots{}+\dots{}}_\text{y-\& z-Terme analog}
	\label{eq:tri_temp1}
%
\end{align}
%
Durch Umstellen erhalten wir:
\begin{align}
(x-x_0)(x_k-x_0)+\dots{}+\dots{}&=\phantom{-}\frac{1}{2}[(x_k-x_0)^2 +(x-x_0)^2 +\dots{}+\dots{}-r_k^2]\nonumber
%
\end{align}
%
\begin{multline}
	(x-x_0)(x_k-x_0)+(y-y_0)(y_k-y_0)+(z-z_0)(z_k-z_0)= \\\frac{1}{2}[\underbrace{(x_k-x_0)^2+(z_k-z_0)^2+(y_k-y_0)^2}_\text{\boldmath{$d_{kj}^2$}}
	\\+\underbrace{(x-x_0)^2+(y-y_0)^2 +(z-z_0)^2}_\text{\boldmath{$r_j^2$}}-r_k^2]
\end{multline}
	
%
\end{frame}
%------------------------------------------------------
\begin{frame}[noframenumbering]
  	\frametitle{Appendix IV}
%
\begin{align*}
a_{0k} :&= \frac{1}{2}d_{kj}^2\\
a_1 :&= \frac{\lambda^2}{8}\\
a_2 :&= a_1\frac{1}{\pi}\\
a_{3kj} :&= a_1\frac{1}{(2\pi)^2}(\Theta_j^2-\Theta_k^2)
\end{align*}
\end{frame}
%------------------------------------------------------
\begin{frame}[noframenumbering]
  	\frametitle{Appendix V}
%
  \begin{center}
  \includegraphics[width=.3\textwidth]{../img/4AntennaSetup_small.png}
  \qquad
  \includegraphics[width=.3\textwidth]{../img/Calibration_Plate1.png}
%  \qquad
%  \includegraphics[width=.3\textwidth]{../img/RFID-Okto.png}
  \end{center}
\end{frame}
%------------------------------------------------------
%\begin{frame}
%  	\frametitle{Modellierung 2}
%%  
%	Wir notieren:
%%	
%	\begin{align}
%		r_1^2&= (x_1-x_{Tag} )^2 + (y_1-y_{Tag} )^2 + (z_1-z_{Tag} )^2\\
%		r_2^2&= (x_2-x_{Tag} )^2 + (y_2-y_{Tag} )^2 + (z_2-z_{Tag} )^2\\
%		r_3^2&= (x_3-x_{Tag} )^2 + (y_3-y_{Tag} )^2 + (z_3-z_{Tag} )^2\\
%		\nonumber\\
%		r_0^2&= (x_0-x_{Tag} )^2 + (y_0-y_{Tag} )^2 + (z_0-z_{Tag} )^2\\
%		\nonumber\\
%		r_{0k}^2&= (x_0-x_k )^2 + (y_0-y_k )^2 + (z_0-z_k )^2 \\
%		\nonumber\\
%		r(\Theta,n)&=\frac{\lambda}{2}\left(\Theta+n\right)
%%		
%	\end{align}
%%
%\end{frame}
%------------------------------------------------------
%\begin{frame}
%  	\frametitle{Modellierung 3}
%%--
%	Gleichung der Form:
%	\[\mathbf{A}\mathbf{x}=\mathbf{b}\]  
%%--
%	,wobei:
%	\begin{multline}
%	\mathbf{A}=\\
%	\left(
%		\begin{array}{cccccccccc}
%			x_1-x_0 & y_1-y_0 & z_1-z_0 & -a_1 & 0 & 0 & -a_2\Theta_0 & a_2\Theta_1 & 0 & 0 \\
%			x_2-x_0 & y_2-y_0 & z_2-z_0 & 0 & -a_1 & 0 & -a_2\Theta_0& 0 & a_2\Theta_2 & 0 \\
%			x_3-x_0 & y_3-y_0 & z_3-z_0 & 0 & 0 & -a_1 & -a_2\Theta_0& 0 & 0 & a_2\Theta_3
%		\end{array}
%	\right) \nonumber
%	\end{multline}
%%--
%	und
%	\begin{multline}
%	\mathbf{x}=\\
%	\left(
%		\begin{array}{cccccccccc}
%			x-x_0 & y-y_0 & z-z_0 &	n_0^2-n_1^2	& (\dots)	&	n_0^2-n_3^2 & n_0 & n_1	& (\dots) &	n_3	
%		\end{array}
%	\right)^T\nonumber
%	\end{multline}
%%--
%	\begin{multline}
%		\mathbf{b}=\\
%		\left(
%			\begin{array}{c}
%				a_{0k}-a_{3kj} 
%			\end{array}
%			\right)^T
%			= c_{kj}'\nonumber
%		\end{multline}
%%--		
%\end{frame}

\newpage
%----------------------------------------------------------------------------
% Bibliography --------------------------------------------------------------
%----------------------------------------------------------------------------

\nocite{*} % Show all Bib-entries
\bibliographystyle{plaindin}
\bibliography{../bib/mathesis_collection1}

\end{document}