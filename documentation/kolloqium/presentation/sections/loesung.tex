\subsection{Modellierung}
\begin{frame}
  \frametitle{Modellierung} %%Folientitel
%  \begin{definition} %%Definition
%    Ergebnisse hier...
%  \end{definition}
\end{frame}
%------------------------------------------------------
\begin{frame}
  \frametitle{Modellierung}
%
\begin{itemize} 
  \item Auf der Basis der Trilateration
  \pause 
  \item Lineares Modell zur Entfernungsberechnung
  \pause 
  \item Gleichzeitige Eignung für den Einsatz als Objektfunktion
\end{itemize} 
	
%
\end{frame}
%------------------------------------------------------
\begin{frame}
  \frametitle{Modellierung  1}
%
  \begin{center}
	\tiny Skizze der Szene mit einem Tag und drei Antennen. Als Referenzpunkt dient eine Landmarke
%
  	\includegraphics[width=.7\textwidth]{../img/trilaterationScene.pdf}
  \end{center}
\[
r_k^2= (x_k-x_{Tag} )^2 + (y_k-y_{Tag} )^2 + (z_k-z_{Tag} )^2
\]
\end{frame}
%------------------------------------------------------
\begin{frame}
  	\frametitle{Modellierung 2}
%  
	Wir notieren:
%	
	\begin{align}
		r_1^2&= (x_1-x_{Tag} )^2 + (y_1-y_{Tag} )^2 + (z_1-z_{Tag} )^2\\
		r_2^2&= (x_2-x_{Tag} )^2 + (y_2-y_{Tag} )^2 + (z_2-z_{Tag} )^2\\
		r_3^2&= (x_3-x_{Tag} )^2 + (y_3-y_{Tag} )^2 + (z_3-z_{Tag} )^2\\
		\nonumber\\
		r_0^2&= (x_0-x_{Tag} )^2 + (y_0-y_{Tag} )^2 + (z_0-z_{Tag} )^2\\
		\nonumber\\
		r_{0k}^2&= (x_0-x_k )^2 + (y_0-y_k )^2 + (z_0-z_k )^2
	\end{align}
%
\end{frame}
%------------------------------------------------------
\begin{frame}
  	\frametitle{Modellierung 3}
%  
	Wir notieren:
%	
	\begin{align}
		r_1^2&= (x_1-x_{Tag} )^2 + (y_1-y_{Tag} )^2 + (z_1-z_{Tag} )^2\\
		r_2^2&= (x_2-x_{Tag} )^2 + (y_2-y_{Tag} )^2 + (z_2-z_{Tag} )^2\\
		r_3^2&= (x_3-x_{Tag} )^2 + (y_3-y_{Tag} )^2 + (z_3-z_{Tag} )^2\\
		\nonumber\\
		r_0^2&= (x_0-x_{Tag} )^2 + (y_0-y_{Tag} )^2 + (z_0-z_{Tag} )^2\\
		\nonumber\\
		r_{0k}^2&= (x_0-x_k )^2 + (y_0-y_k )^2 + (z_0-z_k )^2
	\end{align}
%
\end{frame}