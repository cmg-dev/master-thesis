%% ####### PREAMBLE
%% #
\documentclass{manual}
\usepackage{graphicx}
\usepackage{listing}
\usepackage{pstricks,psfrag}
\usepackage{latexsym}
\usepackage{bbm}
\usepackage{amsmath}

\raggedbottom

%%%%%%%%%%%%%%%%%%%%%%%%%%%%%%%%%%%%%%%%%%%%%%%%%%%%%%%%%%%%%%%%%%%%%%%%%%%%%%
\def\pname{EALib}
%%%%%%%%%%%%%%%%%%%%%%%%%%%%%%%%%%%%%%%%%%%%%%%%%%%%%%%%%%%%%%%%%%%%%%%%%%%%%%

%\newenvironment{smalltext}{
%  \renewcommand{\baselinestretch}{0.8}\scriptsize}{}

\newgray{mygray}{.5}
\newcommand{\mybox}{\mbox{\mygray\rule{1ex}{1ex}}}
\newcommand{\mydaggerbox}{\mybox\hspace*{-12pt}\raisebox{-1pt}{\dag}}

\newlength{\hans}
\newenvironment{methods}{\begin{list}{\mybox}{%
  \labelwidth1ex 
  \labelsep2.5ex
  \leftmargin10pt
  \addtolength{\leftmargin}{\labelwidth}
  \addtolength{\leftmargin}{\labelsep}
  \rightmargin10pt
  \addtolength{\itemsep}{5pt}
  \def\parameters{%
    \hspace*{10pt}\textsl{Parameters}\newline%
    \hspace*{10pt}\rule[11pt]{\hans}{.3pt}%
    \vspace*{-9pt}\newline
  }
  \def\returnvalues{%
    \hspace*{10pt}\textsl{Return Values}\newline%
    \hspace*{10pt}\rule[11pt]{\hans}{.3pt}%
    \vspace*{-9pt}\newline
  }
}}{\end{list}}  


\newenvironment{smalltext}{
  \renewcommand{\baselinestretch}{0.9}\small}{}

\newenvironment{normaltext}{
  \baselineskip13pt\normalsize}{}

\newenvironment{shortlisting}{
    \tt
    \listingwidth=\columnwidth
    \advance\listingwidth-6em
    \obeyspaces
    \obeylines
    \begin{center}
    \tabcolsep3em
    \begin{tabular}{|l|}
    \hline\makebox[\listingwidth]{}\\
}{
    \\\hline
    \end{tabular}
    \end{center}
}

\newenvironment{algorithmbox}{
    \listingwidth=\columnwidth
    \advance\listingwidth-16em
    \begin{center}
    \tabcolsep8em
    \begin{tabular}{|l|}
    \hline\makebox[\listingwidth]{}\\
}{
    \\\hline
    \end{tabular}
    \end{center}
}

%%%%%%%%%%%%%%%%%%%%%%%%%%%%%%%%%%%%%%%%%%%%%%%%%%%%%%%%%%%%%%%%%%%%%%%%%%%%%%

% Ein paar Defines von mir (und wer zur Hoelle ist 'mir'?)
\def\abs#1{|{#1}|}
\def\norm#1{\|{#1}\|}
\def\trans#1{{#1}^{\!\top}}
\def\trace#1{{\hbox{tr~}}{#1}}
\def\rank#1{{\hbox{rank~}}({#1})}
\def\diag#1{{\hbox{diag}}\langle{#1}\rangle}
\def\minisqrt#1{{#1}^{^1\!/\!_2}}
\def\negsqrt#1{{#1}^{{\hbox{-}}^1\!/\!_2}}
\def\smallneg{\hbox{-}}

%%\def\genref#1#2{\mbox{#1\ \ref{#2}}}
%%\def\figref#1{\genref{Fig.}{#1}}
%%\def\eqnref#1{\genref{Eqn.}{#1}}
%%\def\tabref#1{\genref{Tab.}{#1}}
%%\def\secref#1{\genref{Sec.}{#1}}
%%\def\chapref#1{\genref{Chap.}{#1}}
%%\def\algref#1{\genref{Alg.}{#1}}
%%\def\exref#1{\genref{Ex.}{#1}}
%%\def\progref#1{\genref{Prog.}{#1}}

\def\genref#1#2{\mbox{#1\ \ref{#2}}}
\def\figref#1{\genref{Figure}{#1}}
\def\eqnref#1{\genref{Eqation}{#1}}
\def\tabref#1{\genref{Table}{#1}}
\def\secref#1{\genref{Section}{#1}}
\def\chapref#1{\genref{Chapter}{#1}}
\def\algref#1{\genref{Algorithm}{#1}}
\def\exref#1{\genref{Example}{#1}}
\def\progref#1{\genref{Program}{#1}}
\def\appref#1{\genref{Appendix}{#1}}

\def\etal{\hbox{\em et al.\/}}

\def\half{$^1\!/\!_2$}
\def\deg{$^\circ$}
\def\tim{$\times$}


\makeindex
\makeglossary

\def\myindex#1{#1\index{#1}}

\def\algbox#1#2{
    \noindent\fbox{
        \addtolength{\expandafter\columnwidth}{-1.9em}
        \parbox{\expandafter\columnwidth}{
            \begin{list}{}{\topsep1ex\parskip1ex\leftmargin#1}\item[]
                #2
            \end{list}
        }
        \addtolength{\columnwidth}{1.9em}
    }
}


\newenvironment{refcard}[4]{
    \def\refcmd{#1}
    \newpage
    \index{\refcmd@\sf\refcmd\rm|(}
    \noindent {\LARGE\bf\refcmd} \hfill
    {\bf #2} \\[0.7ex]
    \noindent #3 \hfill (#4) \\
    \rule[0.5ex]{\columnwidth}{0.5ex}

    \bigskip

    \begin{list}{}{\itemsep0em \leftmargin2em}\item[]
}{
    \end{list}
    \index{\refcmd@\sf\refcmd\rm|)}
    \clearpage
}

\def\heading#1#2{
    \hspace*{-2em}{\bf #1}\\[0.2em]
    {#2}

    \bigskip
}


%% ####### Macros
%% #
\newcommand{\hone}{\hspace*{1cm}}
\newcommand{\htwo}{\hspace*{2cm}}

\newcommand{\cpp}{%
  \mbox{\emph{\textrm{C\hspace{-1.5pt}\raisebox{1.75pt}{\scriptsize +}%
  \hspace{-2pt}\raisebox{.75pt}{\scriptsize +}}}}%
}

%% #
%% ####### end of Macro section


%% With the following line we declare all graphic extensions which
%% shall be automatically recognised by the graphicx package. Thus
%% the extensions .ps and .ps.gz may be omitted in any
%% /includegraphics... command. If a graphic shall be compressed via gzip
%% use the `gzipeps' tool to zip the .ps file and have the bounding box 
%% dimensions automatically extracted and stored in a corresponding
%% .ps.bb file.
\DeclareGraphicsExtensions{.eps,.ps,.ps.gz}
\graphicspath{{./Figures/}{../Pagode/doc/Figures/}{../SpreadCAT/doc}}

%% ####### Begin the document
%% #
\begin{document}

\author{Copyright \copyright\ 1996--2008 Martin Kreutz, Bernhard
  Sendhoff and Christian Igel\\[1em]
Institut f\"ur Neuroinformatik, Ruhr-Universit\"at Bochum\\[2em]
\normalsize\tt
http://www.neuroinformatik.ruhr-uni-bochum.de/ini/PEOPLE/kreutz/top.html\\
\normalsize\tt
Martin.Kreutz@neuroinformatik.ruhr-uni-bochum.de
}
\date{\today}
\title{{\pname}: A C++ class library for evolutionary algorithms}
%\version{1.5}
\miniabstract{The {\pname} 
C++ library is part of the Shark library. It contains classes
for the implementation of evolutionary algorithms and related techniques.
The library also includes example programs, which can be used as templates
for own programs to do evolutionary optimization.
The whole package is intended to be used in evolutionary algorithms research.
Therefore, the main emphasis has been put on providing as much flexibility
and extendability as possible.
Efficiency has been another major design issue, but has been sacrificed
where it conflicts with flexibility.
This documentation gives a short overview of how evolutionary algorithms
work, the most important operators, the programming interface, and 
examples illustrating the usage of the {\pname} classes.
}

\maketitle

\setcounter{page}{1}

\tableofcontents

%% ######################################################################
        \chapter{Evolutionary Algorithms -- Class Library}
        \label{evoalg:s:evolutionaryAlgorithmsClassLibrary}
%% ######################################################################
        \input evoalg
        \input datastruct

        \section{Genetic Operators}
        \input selection
        \input replacement
        \input recombination
        \input mutation
        \input esAdap

%% force pagebreak
        \newpage
        \input examples

%%%%%%% The Appendix
%



\cleardoublepage
\addcontentsline{toc}{chapter}{\bibname}
\bibliographystyle{abbrv}
\bibliography{manual,proceedings}

%\cleardoublepage
%\addcontentsline{toc}{chapter}{Glossary}
%\printglossary

\cleardoublepage
\addcontentsline{toc}{chapter}{\indexname}
\printindex

\end{document}
%% #
%% ####### end the document
