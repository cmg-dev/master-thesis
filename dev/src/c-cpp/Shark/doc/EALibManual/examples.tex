%
%%
%% file: examples.tex
%%
%
        \section{Examples}


% -----------------------------------------------------------------------
        \subsection{Canonical Genetic Algorithm}
% -----------------------------------------------------------------------

\begin{programlisting}{Canonical Genetic Algorithm solving the Counting Ones Problem}{
    Canonical Genetic Algorithm solving counting ones problem.}
\global\advance\numline by 5
#include <EALib/PopulationT.h>\\
\\
using namespace std;\\
\\
//=======================================================================\\
//\\
// fitness function: counting ones problem\\
//\\
double ones( const vector< bool >& x )\\
\{\\
    unsigned i;\\
    double   sum;\\
    for( sum = 0., i = 0; i < x.size( ); sum += x[ i++ ] );\\
    return sum;\\
\}\\
\\
//=======================================================================\\
//\\
// main program\\
//\\
int main( int argc, char **argv )\\
\{\\
    //\\
    // constants\\
    //\\
    const unsigned PopSize     = 20;\\
    const unsigned Dimension   = 200;\\
    const unsigned Iterations  = 1000;\\
    const unsigned Interval    = 10;\\
    const unsigned NElitists   = 1;\\
    const unsigned Omega       = 5;\\
    const unsigned CrossPoints = 2;\\
    const double   CrossProb   = 0.6;\\
    const double   FlipProb    = 1. / Dimension;\\
\\
    unsigned i, t;\\
\\
    //\\
    // initialize random number generator\\
    //\\
    Rng::seed( argc > 1 ? atoi( argv[ 1 ] ) : 1234 );\\
\\
    //\\
    // define populations\\
    //\\
    PopulationT<bool> parents   ( PopSize, ChromosomeT< bool >( Dimension ) );\\
    PopulationT<bool> offsprings( PopSize, ChromosomeT< bool >( Dimension ) );\\
\\
    //\\
    // scaling window\\
    //\\
    vector< double > window( Omega );\\
\\
    //\\
    // maximization task\\
    //\\
    parents   .setMaximize( );\\
    offsprings.setMaximize( );\\
\\
    //\\
    // initialize all chromosomes of parent population\\
    //\\
    for( i = 0; i < parents.size( ); ++i )\\
            parents[ i ][ 0 ].initialize( );\\
\\
    //\\
    // evaluate parents (only needed for elitist strategy)\\
    //\\
    if( NElitists > 0 )\\
        for( i = 0; i < parents.size( ); ++i )\\
            parents[ i ].setFitness( ones(parents[ i ][ 0 ] ) );\\
\\
    //\\
    // iterate\\
    //\\
    for( t = 0; t < Iterations; ++t ) \{\\
        //\\
        // recombine by crossing over two parents\\
        //\\
        offsprings = parents;\\
        for( i = 0; i < offsprings.size( )-1; i += 2 )\\
            if( Rng::coinToss( CrossProb ) )\\
                offsprings[ i ][ 0 ]\\
                    .crossover( offsprings[ i+1 ][ 0 ], CrossPoints );\\
        //\\
        // mutate by flipping bits\\
        //\\
        for( i = 0; i < offsprings.size( ); ++i )\\
            offsprings[ i ][ 0 ].flip( FlipProb );\\
        //\\
        // evaluate objective function\\
        //\\
        for( i = 0; i < offsprings.size( ); ++i )\\
            offsprings[ i ].setFitness( ones(offsprings[ i ][ 0 ] ) );\hspace{-10ex}\\
        //\\
        // scale fitness values and use proportional selection\\
        //\\
        offsprings.linearDynamicScaling( window, t );\\
        parents.selectProportional( offsprings, NElitists );\\
\\
        //\\
        // print out best value found so far\\
        //\\
        if( t % Interval == 0 )\\
            cout << t << "$\backslash$tbest value = "\\
                 << parents.best( ).fitnessValue( ) << endl;\\
    \}\\
\}\\
\end{programlisting}

\begin{programlisting}{Canonical Genetic Algorithm minimizing the Sphere Model}{
    Canonical Genetic Algorithm minimizing the sphere model.}
#include <SharkDefs.h>\\
#include <EALib/PopulationT.h>\\
\\
using namespace std;\\
\\
//=======================================================================\\
//\\
// fitness function: sphere model\\
//\\
double sphere( const vector< double >& x )\\
\{\\
    unsigned i;\\
    double   sum;\\
    for( sum = 0., i = 0; i < x.size( ); i++ )\\
        sum += Shark::sqr( x[ i ] );\\
    return sum;\\
\}\\
\\
//=======================================================================\\
//\\
// main program\\
//\\
int main( int argc, char **argv )\\
\{\\
    //\\
    // constants\\
    //\\
    const unsigned PopSize     = 50;\\
    const unsigned Dimension   = 20;\\
    const unsigned NumOfBits   = 10;\\
    const unsigned ChromLen    = Dimension * NumOfBits;\\
    const unsigned Iterations  = 2000;\\
    const unsigned DspInterval = 10;\\
    const unsigned NElitists   = 1;\\
    const unsigned Omega       = 5;\\
    const unsigned CrossPoints = 2;\\
    const double   CrossProb   = 0.6;\\
    const double   FlipProb    = 1. / ChromLen;\\
    const bool     UseGrayCode = true;\\
    const Interval RangeOfValues( -3, +5 );\\
\\
    unsigned i, t;\\
\\
    //\\
    // initialize random number generator\\
    //\\
    Rng::seed( argc > 1 ? atoi( argv[ 1 ] ) : 1234 );\\
\\
    //\\
    // define populations\\
    //\\
    PopulationT<bool> parents   ( PopSize, ChromosomeT< bool >( ChromLen ) );\\
    PopulationT<bool> offsprings( PopSize, ChromosomeT< bool >( ChromLen ) );\\
\\
    //\\
    // scaling window\\
    //\\
    vector< double > window( Omega );\\
\\
    //\\
    // temporary chromosome for decoding\\
    //\\
    ChromosomeT< double > dblchrom;\\
\\
    //\\
    // minimization task\\
    //\\
    parents   .setMinimize( );\\
    offsprings.setMinimize( );\\
\\
    //\\
    // initialize all chromosomes of parent population\\
    //\\
    for( i = 0; i < parents.size( ); ++i )\\
        parents[ i ][ 0 ].initialize( );\\
\\
    //\\
    // evaluate parents (only needed for elitist strategy)\\
    //\\
    if( NElitists > 0 )\\
        for( i = 0; i < parents.size( ); ++i ) \{\\
            dblchrom.decodeBinary( parents[ i ][ 0 ], RangeOfValues,\\
                                   NumOfBits, UseGrayCode );\\
            parents[ i ].setFitness( sphere( dblchrom ) );\\
        \}\\
\\
    //\\
    // iterate\\
    //\\
    for( t = 0; t < Iterations; ++t ) \{\\
        //\\
        // recombine by crossing over two parents\\
        //\\
        offsprings = parents;\\
        for( i = 0; i < offsprings.size( )-1; i += 2 )\\
            if( Rng::coinToss( CrossProb ) )\\
                offsprings[ i ][ 0 ]\\
                  .crossover( offsprings[ i+1 ][ 0 ], CrossPoints );\\
\\
        //\\
        // mutate by flipping bits\\
        //\\
        for( i = 0; i < offsprings.size( ); ++i )\\
            offsprings[ i ][ 0 ].flip( FlipProb );\\
\\
        //\\
        // evaluate objective function\\
        //\\
        for( i = 0; i < offsprings.size( ); ++i ) \{\\
            dblchrom.decodeBinary( offsprings[ i ][ 0 ], RangeOfValues,\\
                                   NumOfBits, UseGrayCode );\\
            offsprings[ i ].setFitness( sphere( dblchrom ) );\\
        \}\\
\\
        //\\
        // scale fitness values and use proportional selection\\
        //\\
        offsprings.linearDynamicScaling( window, t );\\
        parents.selectProportional( offsprings, NElitists );\\
\\
        //\\
        // print out best value found so far\\
        //\\
        if( t % DspInterval == 0 )\\
            cout << t << "$\backslash$tbest value = "\\
                 << parents.best( ).fitnessValue( ) << endl;\\
    \}\\
\}\\
\end{programlisting}

\clearpage
\subsection{Steady-State Genetic Algorithm}

\begin{programlisting}{Steady--State Genetic Algorithm solving the Counting Ones Problem}{
    Steady--State Genetic Algorithm solving the counting ones problem.}
\global\advance\numline by 3
#include <SharkDefs.h>\\
#include <EALib/PopulationT.h>\\
\\
using namespace std;\\
\\
//=======================================================================\\
//\\
// fitness function: sphere model\\
//\\
double sphere( const vector< double >& x )\\
\{\\
    unsigned i;\\
    double   sum;\\
    for( sum = 0., i = 0; i < x.size( ); i++ )\\
        sum += Shark::sqr( x[ i ] );\\
    return sum;\\
\}\\
\\
//=======================================================================\\
//\\
// main program\\
//\\
int main( int argc, char **argv )\\
\{\\
    //\\
    // constants\\
    //\\
    const unsigned PopSize     = 50;\\
    const unsigned Dimension   = 20;\\
    const unsigned NumOfBits   = 10;\\
    const unsigned Iterations  = 15000;\\
    const unsigned DspInterval = 100;\\
    const unsigned Omega       = 5;\\
    const unsigned CrossPoints = 2;\\
    const double   CrossProb   = 0.6;\\
    const double   FlipProb    = 1. / ( Dimension * NumOfBits );\\
    const bool     UseGrayCode = true;\\
    const Interval RangeOfValues( -3, +5 );\\
\\
    unsigned i, t;\\
\\
    //\\
    // initialize random number generator\\
    //\\
    Rng::seed( argc > 1 ? atoi( argv[ 1 ] ) : 1234 );\\
\\
    //\\
    // define populations\\
    //\\
    IndividualT<bool> kid( ChromosomeT< bool >( Dimension * NumOfBits ) );\\
    PopulationT<bool> pop( PopSize, kid );\\
\\
    //\\
    // scaling window\\
    //\\
    vector< double > window( Omega );\\
\\
    //\\
    // temporary chromosome for decoding\\
    //\\
    ChromosomeT< double > dblchrom;\\
\\
    //\\
    // minimization task\\
    //\\
    pop.setMinimize( );\\
\\
    //\\
    // initialize all chromosomes of the population\\
    //\\
    for( i = 0; i < pop.size( ); ++i ) \{\\
        pop[ i ][ 0 ].initialize( );\\
        dblchrom.decodeBinary( pop[ i ][ 0 ], RangeOfValues,\\
                               NumOfBits, UseGrayCode );\\
        pop[ i ].setFitness( sphere( dblchrom ) );\\
    \}\\
\\
    //\\
    // iterate\\
    //\\
    for( t = 0; t < Iterations; ++t ) \{\\
        //\\
        // scale fitness values and use proportional selection\\
        //\\
        pop.linearDynamicScaling( window, t );\\
\\
        //\\
        // recombine by crossing over two parents\\
        //\\
        if( Rng::coinToss( CrossProb ) )\\
            kid[ 0 ].crossover( pop.selectOneIndividual( )[ 0 ],\\
                                pop.selectOneIndividual( )[ 0 ],\\
                                CrossPoints );\\
        else\\
            kid = pop.selectOneIndividual( );\\
\\
        //\\
        // mutate by flipping bits\\
        //\\
        kid[ 0 ].flip( FlipProb );\\
\\
        //\\
        // evaluate objective function\\
        //\\
        dblchrom.decodeBinary( kid[ 0 ], RangeOfValues,\\
                               NumOfBits, UseGrayCode );\\
        kid.setFitness( sphere( dblchrom ) );\\
\\
        //\\
        // replace the worst individual in the population\\
        //\\
        pop.worst( ) = kid;\\
\\
        //\\
        // print out best value found so far\\
        //\\
        if( t % DspInterval == 0 )\\
            cout << t << "$\backslash$tbest value = "\\
                 << pop.best( ).fitnessValue( ) << endl;\\
    \}\\
\}\\
\end{programlisting}

\clearpage
\subsection{Canonical Evolution Strategy}

\begin{programlisting}{Canonical Evolution Strategy which Minimizes Ackley's Function}{
    Canonical Evolution Strategy which minimizes Ackley's function.
    The flag ``PlusStrategy'' is used to switch between $(\mu,\lambda)$
    and $(\mu+\lambda)$ selection.}
\global\advance\numline by 3
#include <SharkDefs.h>\\
#include <EALib/PopulationT.h>\\
\\
using namespace std;\\
\\
//=======================================================================\\
//\\
// fitness function: Ackley's function\\
//\\
double ackley( const vector< double >& x )\\
\{\\
    const double A = 20.;\\
    const double B = 0.2;\\
    const double C = M_2PI;\\
\\
    unsigned i, n;\\
    double   a, b;\\
\\
    for( a = b = 0., i = 0, n = x.size( ); i < n; ++i ) \{\\
        a += x[ i ] * x[ i ];\\
        b += cos( C * x[ i ] );\\
    \}\\
\\
    return -A * exp( -B * sqrt( a / n ) ) - exp( b / n ) + A + M_E;\\
\}\\
\\
//=======================================================================\\
//\\
// main program\\
//\\
int main( int argc, char **argv )\\
\{\\
    //\\
    // constants\\
    //\\
    const unsigned Mu           = 15;\\
    const unsigned Lambda       = 100;\\
    const unsigned Dimension    = 30;\\
    const unsigned Iterations   = 500;\\
    const unsigned Interval     = 10;\\
    const unsigned NSigma       = 1;\\
\\
    const double   MinInit      = -3;\\
    const double   MaxInit      = +15;\\
    const double   SigmaInit    = 3;\\
\\
    const bool     PlusStrategy = false;\\
\\
    unsigned       i, t;\\
\\
    //\\
    // initialize random number generator\\
    //\\
    Rng::seed( argc > 1 ? atoi( argv[ 1 ] ) : 1234 );\\
\\
    //\\
    // define populations\\
    //\\
    PopulationT<double> parents   ( Mu,     ChromosomeT< double >( Dimension ),\\
                                            ChromosomeT< double >( NSigma    ) );\\
    PopulationT<double> offsprings( Lambda, ChromosomeT< double >( Dimension ),\\
                                            ChromosomeT< double >( NSigma    ) );\\
\\
    //\\
    // minimization task\\
    //\\
    parents   .setMinimize( );\\
    offsprings.setMinimize( );\\
\\
    //\\
    // initialize parent population\\
    //\\
    for( i = 0; i < parents.size( ); ++i ) \{\\
       parents[ i ][ 0 ].initialize( MinInit,   MaxInit   );\\
       parents[ i ][ 1 ].initialize( SigmaInit, SigmaInit );\\
    \}\\
\\
    //\\
    // selection parameters (number of elitists)\\
    //\\
    unsigned numElitists = PlusStrategy ? Mu : 0;\\
\\
    //\\
    // standard deviations for mutation of sigma\\
    //\\
    double     tau0 = 1. / sqrt( 2. * Dimension );\\
    double     tau1 = 1. / sqrt( 2. * sqrt( ( double )Dimension ) );\\
\\
    //\\
    // evaluate parents (only needed for elitist strategy)\\
    //\\
    if( PlusStrategy )\\
        for( i = 0; i < parents.size( ); ++i )\\
            parents[ i ].setFitness( ackley(parents[ i ][ 0 ] ) );\\
\\
    //\\
    // iterate\\
    //\\
    for( t = 0; t < Iterations; ++t ) \{\\
        //\\
        // generate new offsprings\\
        //\\
        for( i = 0; i < offsprings.size( ); ++i ) \{\\
            //\\
            // select two random parents\\
            //\\
            Individual& mom = parents.random( );\\
            Individual& dad = parents.random( );\\
\\
            //\\
            // recombine object variables discrete,\\
            // step sizes intermediate\\
            //\\
            offsprings[ i ][ 0 ].recombineDiscrete       ( mom[ 0 ], dad[ 0 ] );\\
            offsprings[ i ][ 1 ].recombineGenIntermediate( mom[ 1 ], dad[ 1 ] );\\
\\
            //\\
            // mutate object variables normal distributed,\\
            // step sizes log normal distributed\\
            //\\
            offsprings[ i ][ 1 ].mutateLogNormal( tau0,  tau1 );\\
            offsprings[ i ][ 0 ].mutateNormal   ( offsprings[ i ][ 1 ], true );\\
        \}\\
\\
        //\\
        // evaluate objective function (parameters in chromosome #0)\\
        //\\
        for( i = 0; i < offsprings.size( ); ++i )\\
            offsprings[ i ].setFitness( ackley(offsprings[ i ][ 0 ] ) );\\
\\
        //\\
        // select (mu,lambda) or (mu+lambda)\\
        //\\
        parents.selectMuLambda( offsprings, numElitists );\\
\\
        //\\
        // print out best value found so far\\
        //\\
        if( t % Interval == 0 )\\
            cout << t << "$\backslash$tbest value = "\\
                 << parents.best( ).fitnessValue( ) << endl;\\
    \}\\
\}\\
\end{programlisting}

\clearpage
\subsection{Covariance Matrix Adaptation  Evolution Strategy}\label{ex:cma}

\begin{programlisting}{Covariance Matrix Adaptation  Evolution Strategy (CMA-ES)}{Covariance Matrix Adaptation  Evolution Strategy (CMA-ES).}
\global\advance\numline by 3
#include <EALib/CMA.h>\\
    \vdots\\
// EA parameters\\
CMA cma;\\
\\
const unsigned Iterations     = 1000;\\
const double   MinInit        = 1.;\\
const double   MaxInit        = 1.;\\
const double   GlobalStepInit = 1.;\\
\\
unsigned Lambda             = cma.suggestLambda(Dimension);\\
unsigned Mu                 = cma.suggestMu(Lambda);\\
\\
// define populations for minimization task\\
Population parents (Mu,     \\
                    ChromosomeT< double >( Dimension ),\\
                    ChromosomeT< double >( Dimension ));\\
Population offsprings (Lambda,     \\
                       ChromosomeT< double >( Dimension ),\\
                       ChromosomeT< double >( Dimension ));\\
offsprings.setMinimize( );\\
parents   .setMinimize( );\\
\\
// initialize parent populations center of gravity\\
for(i = 0; i < parents.size( ); ++i ) \\
  static_cast< ChromosomeT< double >\& >( parents[0][0] ).\\
    initialize(MinInit, MaxInit);\\
for(j = 1; j < parents.size( ); ++j ) \\
  static_cast< ChromosomeT< double >\& >(parents[j][0]) =\\
    static_cast< ChromosomeT< double >\& >(parents[0][0]);\\
\\
// strategy parameters\\
vector< double > variance( Dimension );\\
for(i = 0; i < Dimension; i++) variance[i] = 1.;\\
\\
cma.init(Dimension, variance, GlobalStepInit, parents,\\
         CMA::superlinear, CMA::rankmu); \\
// the previous three lines are equal to \\
//\\
// cma.init(Dimension, 1., parents);\\
//\\
// but demonstrates initialization with different variances\\
    \vdots\\
//\\
// iterate\\
//\\
for( unsigned t = 0; t < Iterations; ++t ) \{\\
  for(unsigned k = 0; k < offsprings.size( ); k++ ) \{\\
    cma.create(offsprings[k]);\\
    offsprings[k].setFitness(myfitness(offsprings[k]));\\
  \}\\
\\
  // select (mu,lambda) or (mu+lambda)\\
\\
  parents.selectMuLambda(offsprings, 0u);\\
\\
  // update strategy parameters\\
  cma.updateStrategyParameters(parents);\\
\}\\
    \vdots\\
\end{programlisting}








%%% Local Variables: 
%%% mode: latex
%%% TeX-master: "EALib-standalone"
%%% End: 
