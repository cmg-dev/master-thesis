%
%
%
%

\begin{appendix}
%----------------------------------------------------------------------------
%----------------------------------------------------------------------------
%----------------------------------------------------------------------------
\newpage
\huge{Anhänge}
\normalsize
%\section{Verwendete Abkürzungen}
% \dots{}
%----------------------------------------------------------------------------

\begin{landscape}
\section{Projektlaufplan}
\label{sec:projectplan}
  \begin{center}
	\scalebox{.75}{
		\begin{ganttchart}[vgrid={draw=none,*1{gray, dashed}},
				hgrid=true,
				today=42,
				title height=1,
				y unit title=0.6cm,
				y unit chart=0.8cm,
				group right shift=0,
				group top shift=.3,
				group height=.3,
				milestone width=.8,
				group peaks={}{}{.2},
				incomplete/.style={fill=black!15}, %
				bar/.style={fill=white}, %
				today label={Heute},
				today rule/.style={dashed, thick}]{44}


\gantttitle{\textbf{2013}}{44} \\
\gantttitlelist{16,...,37}{2} \\
%-------------------------------------------------------------
\ganttgroup{Projekt Evaluation}{3}{14} \\
\ganttbar[progress=100, progress label font=\small\color{black!75},
	progress label anchor/.style={right=4pt}]{Installation der Umgebungen}{3}{6} \\
	
\ganttbar[progress=100, progress label font=\small\color{black!75},
	progress label anchor/.style={right=4pt},
	bar label font=\normalsize\color{black},
	name=rech]{Recherche}{3}{7} \\
	
\ganttmilestone[name=ms1]{Vorstellung der Ergebnisse}{7} \\
	
\ganttbar[progress=100, progress label font=\small\color{black!75},
	progress label anchor/.style={right=4pt},
	bar label font=\normalsize\color{black},
	name=pflichten]
	{Pflichtenheft}{5}{8} \\
	
\ganttmilestone[name=ms2]{Pflichtenheft fertig}{8} \\

\ganttbar[progress=100, progress label font=\small\color{black!75},
	progress label anchor/.style={right=4pt},
	bar label font=\normalsize\color{black},
	name=bNumVerf]
	{Einarbeitung num. Verfahren}{5}{16} \\

\ganttbar[progress=100, progress label font=\small\color{black!75},
	progress label anchor/.style={right=34pt},
	bar label font=\normalsize\color{black},
	name=bCMAES]
	{speziell CMA-ES}{7}{10} \\

\ganttmilestone[name=ms3]{Beurteilung num. Verfahren}{16} \\

\ganttlinkedbar[progress=100, progress label font=\small\color{black!75},
	progress label anchor/.style={right=34pt},
	bar label font=\normalsize\color{black}]
	{Shark Einarbeitung}{17}{18} \\

\ganttlinkedmilestone[name=ms7]{Abschluss Evaluation}{18} \\
	
%-------------------------------------------------------------
\ganttgroup{Erstellung Prototyp}{15}{26} \\
\ganttgroup{(optional)}{15}{18} \\
\ganttbar[progress=100, progress label font=\small\color{black!75},
	progress label anchor/.style={right=4pt},
	bar label font=\normalsize\color{black}]
	{(Entwurf digi. Filter)}{15}{15} \\

\ganttlinkedbar[progress=100, progress label font=\small\color{black!75},
	progress label anchor/.style={right=4pt},
	bar label font=\normalsize\color{black},
	name=bImpFPGA]
	{(Implementation FPGA)}{16}{18} \\

\ganttmilestone[name=ms4]{(Verifikation dig. Filter)}{18} \\
	
\ganttbar[progress=100, progress label font=\small\color{black!75},
	progress label anchor/.style={right=4pt},
	bar label font=\normalsize\color{black},
	name=bImplAlgo]
	{Implementation Algorithmus}{15}{26} \\

\ganttlinkedmilestone[name=ms5]{Implementation Done}{26} \\

%-------------------------------------------------------------
\ganttgroup{Verifikation}{27}{34} \\
\ganttbar[progress=100, progress label font=\small\color{black!75},
	progress label anchor/.style={right=4pt},
	bar label font=\normalsize\color{black},
	name=bVerf]
	{Durchf\"uhrung Verifikation}{27}{34} \\

\ganttlinkedmilestone[name=ms6]{Verifikation Done}{34} \\

%-------------------------------------------------------------
\ganttgroup{Projektdokumentation}{35}{42} \\

\ganttbar[progress=100, progress label font=\small\color{black!75},
	progress label anchor/.style={right=4pt},
	bar label font=\normalsize\color{black},
	name=thesis]
	{Thesis schreiben}{35}{42} \\
	
\ganttmilestone[name=msthesis,milestone label font=\color{red}, 
	milestone/.style={fill=gray}]{Abgabe}{42}

%\ganttlink{ms7}{bImplAlgo}
\ganttlink{bImpFPGA}{ms4}
\ganttlink{bNumVerf}{ms3}
\ganttlink{bCMAES}{ms3}
\ganttlink{rech}{ms1}
\ganttlink{pflichten}{ms2}
\ganttlink{thesis}{msthesis}

	\end{ganttchart}
	}
	\end{center}
\newpage
\end{landscape}


\newpage
\section{Vorüberlegungen}
\label{sec:calculations}
Auf den folgenden Seiten befindet sich eine Modellierung des physikalischen Zusammenhangs. Ziel ist die Entwicklung eines Modells, dass sich für die Verwendung in einer Minimierungsstrategie eignet.

\begin{figure}[h]
	\begin{center}
		% Intersection of
% Author: Rasmus Pank Roulund

\begin{tikzpicture}[
    scale=10,
    axis/.style={very thick, ->, >=stealth'},
    vector/.style={thick, ->, >=stealth'},
    antenna/.style={thick},
    important line/.style={thick},
    dashed line/.style={dashed, thin},
    pile/.style={thick, ->, >=stealth', shorten <=2pt, shorten
    >=2pt},
    every node/.style={color=black}
    ]
    % axis
    \draw[axis] (-0.05,0)  -- (0.2,0) node(xline)[right] {$x$};
    \draw[axis] (0,-0.05) -- (0,0.2) node(yline)[above] {$z$};
    % Lines
    \draw[vector] (0,0) coordinate (Xor) -- (.40,.25)
        coordinate (A) node[right, text width=5em]{};
%
    \draw[antenna,rotate around={40:(A)}] (A) -- (.40,.32) coordinate (b) node[right, text width=5em] {};

    \draw[antenna,rotate around={220:(A)}] (A) -- (.40,.32);

    \draw[axis,rotate around={40:(A)},gray] (A)  -- ( .40,.30) node(xline)[above] {$x'$};
    \draw[axis,rotate around={40:(A)},gray] (A) -- ( .45,.25) node(yline)[above] {$z'$};
        
%        
%    \draw[important line] (0.9,0.5) coordinate (C) -- (D) node[right, text width=5em]
%%    \draw[important line] (D) -- (0.5,0.9) coordinate (F) node[right, text width=5em]
%%         {$\mathit{NX}=x$};
%
%	\fill[red] (-.075,-.2) coordinate (out) circle (.2pt)
%        node[below left] {$B$};

%    % Intersection of lines
%    \fill[red] (intersection cs:
%       first line={(A) -- (B)},
%       second line={(C) -- (D)}) coordinate (E) circle (.4pt)
%       node[above,] {$A$};
%    % The E point is placed more or less randomly
%    \fill[red]  (E) +(-.075cm,-.2cm) coordinate (out) circle (.4pt)
%        node[below left] {$B$};
%    % Line connecting out and ext balances
%    \draw [pile] (out) -- (intersection of A--B and out--[shift={(0:1pt)}]out)
%        coordinate (extbal);
%    \fill[red] (extbal) circle (.4pt) node[above] {$C$};
%    % line connecting  out and int balances
%    \draw [pile] (out) -- (intersection of C--D and out--[shift={(0:1pt)}]out)
%        coordinate (intbal);
%    \fill[red] (intbal) circle (.4pt) node[above] {$D$};
%    % line between out og all balanced out :)
%    \draw[pile] (out) -- (E);
\end{tikzpicture}

%%% Local Variables:
%%% mode: latex
%%% TeX-master: t
%%% End:
		 \caption[Kurzeintrag]{Entwurf des Modells für die Berechnung. Gezeigt ist, ein Tag und zwei Antennen. Die graue Linie stellt eine Iso-Fläche der Phase dar. Von diesen Flächen befinden sich $N$ viele mit einem Abstand von $\frac{\lambda}{2}$ (der Übersicht halber nicht eingezeichnet).} 
	\end{center}
\end{figure}

Aus einfachen Überlegungen ergibt sich:

\begin{equation}\label{eq:Tri1}
(l_1+\Delta)^2 = (l_1+\delta)^2+h^2
\end{equation}
\begin{equation}\label{eq:Tri2}
(l_2+\Delta)^2 = (l_2-\delta)^2+h^2
\end{equation}
%
aus \eqref{eq:Tri1}:
\begin{align}
h^2 &= (l_1+\Delta)^2 - (l_1+\delta)^2\\
	&= l_1^2 + 2l_1\Delta + \Delta^2 -[ l_1^2 + 2 l_1 \delta + \delta^2]\nonumber\\
	&= l_1^2 + 2l_1\Delta + \Delta^2 - l_1^2 - 2 l_1 \delta - \delta^2	\nonumber\\
	&= 2l_1\Delta + \Delta^2 - 2 l_1 \delta - \delta^2 \label{eq:Tri3}
\end{align}
%
analog dazu lässt sich aus~\eqref{eq:Tri2} ableiten:
%
\begin{align}
h^2 &= 2l_2\Delta + \Delta^2 + 2 l_2 \delta - \delta^2 \label{eq:Tri4}
\end{align}
%
Wir suchen einen Ausdruck um $\Delta$ zu eliminieren, \eqref{eq:Tri3}=\eqref{eq:Tri4}:
%
\begin{align}
	2l_1\Delta + \Delta^2 - 2 l_1 \delta - \delta^2 &=  2l_2\Delta + \Delta^2 + 2 l_2 \delta - \delta^2 \nonumber \\
	2l_1\Delta + \Delta^2 - 2l_2\Delta - \Delta^2&= 2 l_2 \delta - \delta^2 +2l_1\delta +\delta^2 \nonumber \\
	\Delta(l_1 - l_2  ) &= \delta (l_1+l_2) \nonumber \\
	\Delta &= \delta\frac{(l_1 + l_2)}{(l_1 - l_2)} \label{eq:Tri5}
\end{align}

\eqref{eq:Tri5} in \eqref{eq:Tri3} einsetzen:
\begin{align}
	h^2 &= 2l_1\delta\left(\frac{(l_1 + l_2)}{(l_1 - l_2)}\right) + \delta^2\left(\frac{(l_1 + l_2)}{(l_1 - l_2)}\right)^2 - 2l_1 \delta - \delta^2 \nonumber \\
	&= \underbrace{\left(\frac{(l_1 + l_2)}{(l_1 - l_2)} - 1 \right)^2}_\text{$a_0$} \delta^2 + \underbrace{\left(\frac{(l_1 + l_2)}{(l_1 - l_2)}  - 1 \right)}_\text{$a_1$} 2l_1\delta
\end{align}

Nun können wir für $\delta$ den Ausdruck $x'$ einführen. und erhalten:
\begin{equation}
	h^2 = a_0 x'^2 + a_1 x' \label{eq:TriFinal}
\end{equation}
%
Gleichung~\eqref{eq:TriFinal} drückt aus, dass sich die Berechnung noch in dem Koordinatensystem der beiden gewählten Antennen liegt. Es ist noch eine geeignete Koordinatensystem-Transformation zu wählen um auf ein vernünftiges Referenz-Koordinatensystem zu kommen.
Weiterhin gelten die Überlegungen nur für diesen Zweidimensionalen Fall. Eine Anwendung auf die dritte Dimension sollte über die Erweiterung von $h^2 = y^2+z^2$ möglich sein.

\end{appendix}
