Die Positionsbestimmung (Tracking) mittels RFID bietet gegenüber vergleichbaren
Methoden (z.B. Ultraschall, Optisch ) verschiedene Vorteile. Durch das
zugrunde liegende Messprinzip ist es möglich unabhängig von einer direkten
Sichtlinie (Line of sight:=LOS) ein Objekt zu lokalisieren. Das von dem
Messsystem der Amedo GmbH verwendete Verfahren basiert auf der Messung der
Phasenlage der Antwort eines Tags. Die Phasenlage ist direkt proportional zu
einer Entfernung. Das Problem liegt in den unbekannten, komplex zu
modellierenden Verhalten der elektromagnetischen Funkwellen in geschlossenen
Räumen. Diese Führen zu einem Fehler der Phase und damit direkt zu einem
Messfehler der Position.
\newline
[Beschreibung der Wellenzahl][Referenz auf die Dipl. Arbeit von Bernd]
\newline
Ziel dieser Arbeit ist es ein System zu
implementieren, das eine direkte Abschätzung (Ad-Hoc) der Wellenzahl erlaubt.
Dafür werden Methoden der Numerik verwendet um die Fehler der Phasenlage zu
korrigieren.